\documentclass[10pt, openany]{book}
\usepackage[FINAL]{../boilerplate/rexx} 
\usepackage{hyperref}
\usepackage{graphics}
\usepackage{fontspec}
% \fontspec
%      [ Path = /Users/rvjansen/Fonts/,
%        BoldFont       = MinionPro-Bold.otf ,
%        ItalicFont     = MinionPro-It.otf ,
%        BoldItalicFont = MinionPro-BoldIt.otf ]
%      {MinionPro-Regular.otf}
% \fontspec
%      [ Path = /Users/rvjansen/Fonts/,
%        BoldFont       = SourceCodePro-Bold.otf ]
%      {SoureCodePro-Regular.otf}

\setmainfont[Mapping=tex-text]{Minion Pro}
\setmonofont[Mapping=tex-text,Scale=0.80]{Source Code Pro}
\usepackage{tabularx}
\usepackage{booktabs}
\usepackage{makeidx}
\usepackage[all]{xy}
%\usepackage{lingmacros}
\usepackage{color}
\usepackage{xcolor}
\usepackage{listings}
\usepackage{caption}
\usepackage{longtable}
\usepackage{colortbl}
\usepackage{framed}
\usepackage{fancyvrb}
\definecolor{shadecolor}{rgb}{0.9,0.9,0.9}
\usepackage{alltt}
\DeclareCaptionFont{white}{\color{white}}
\DeclareCaptionFormat{listing}{\colorbox{gray}{\parbox{\textwidth}{#1#2#3}}}
\captionsetup[lstlisting]{format=listing,labelfont=white,textfont=white}
\usepackage{listings}
\usepackage[official]{eurosym}
\makeatletter
\lst@CCPutMacro\lst@ProcessOther {"2D}{\lst@ttfamily{-{}}{-{}}}
\@empty\z@\@empty
\makeatother
\lstdefinelanguage{NetRexx}
{morekeywords={abstract,adapter,binary,case,catch,class,constant,dependent,deprecated,digits,do,else,end,engineering,extends,final,finally,for,forever,if,implements,indirect,import,indirect,inheritable,interface,iterate,label,leave,loop,method,native,nop,numeric,options,otherwise,over,package,parent,parse,private,properties,protect,public,return,returns,rexx,say,scientific,set,digits,form,select,shared,signal,signals,sourceline,static,super,then,this,until,used,upper,volatile,when,where,while},
sensitive=false,
extendedchars=false,
morecomment=[s]={/*}{*/},
morecomment=[l]{--},
morecomment=[s]{/**}{*/},
morestring=[b]",
morestring=[d]",
morestring=[b]',
morestring=[d]'}

\lstset{language=NetRexx,
  captionpos=t,
  tabsize=3,
  alsolanguage=Rexx,
  keywordstyle=\color{blue},
  commentstyle=\color{cyan},
  stringstyle=\color{red},
  numbers=left,
  numberstyle=\tiny,
  numbersep=5pt,
  breaklines=true,
  showstringspaces=false,
  index=[1][keywords],
  columns=flexible,
  basicstyle=\fontsize{8}{8}\fontspec{Source Code Pro},emph={label}}

\usepackage{../boilerplate/rail}
\usepackage{pst-barcode,pstricks-add}
\usepackage{bashful}
\usepackage{metalogo}
\usepackage{marginnote}
\usepackage{pdfpages}
\usepackage{float}
\hyphenation{Net-Rexx Net-Rexx-A Net-Rexx-C Net-Rexx-R Mac-OSX
  infra-structure im-ple-men-ta-tion-de-pen-dent}
\makeindex
\DeclareGraphicsExtensions{.jpg,.png}
\setlength{\parskip}{4pt}
\setlength{\parindent}{0pt}
\usepackage{enumitem}
\newcommand{\nr}{Net\textsc{Rexx}}
\newcommand{\Rexx}{R\textsc{exx}}
\newcommand{\nrpackagename}{\splice{java GetPackageName}}
\newcommand{\minimalJVMversion}{1.6}
\newcommand{\keyword}[1]{\texttt{#1}}
\newcommand{\code}[1]{\texttt{#1}}
\newcommand{\thisyear}{\splice{java TexYear}}
\newcommand{\ecjjarname}{ecj-4.6.3.jar}
\newcommand{\msd}[1]{\msdhelper#1\relax}
\newcommand{\msdhelper}[1]
  {\ifx\relax#1\else
    \ifx-#1--{}\else#1\fi
    \expandafter\msdhelper\fi}
  \newcommand{\doublehyphen}{\mbox{\msd{``-~-''}}}
  \newcommand{\doublehyphenunquoted}{\mbox{\msd{-~-}}}
%%% Local Variables: 
%%% mode: latex
%%% TeX-master: t
%%% End: 

\begin{document} 
\renewcommand{\isbn}{978-90-819090-1-3}
\setcounter{tocdepth}{1}
\title{\fontspec{Bodoni URW Light}NetR\fontspec{TeX Gyre Pagella}\textsc{exx}\protect\\\fontspec{Bodoni URW Light}Language Reference}
%\title{\fontspec{Bodoni URW
%Light}NetR\textsc{exx}\protect\\\fontspec{Bodoni URW Light}Language
%\title{NetR\textsc{exx}\protect\\Language Reference}
\author{Mike Cowlishaw and R\textsc{exx}LA}
\date{\null\hfill Version \splice{java org.netrexx.process.NrVersion} of \today}
\maketitle
\pagenumbering{Roman}
\pagestyle{plain}
\frontmatter
\pagenumbering{Roman}
\pagestyle{plain}
\section*{Publication Data}
\textcopyright  Copyright The Rexx Language Association, 2011-\splice{java TexYear}
%\\

All original material in this publication is published under the Creative Commons - Share Alike 3.0 License as stated at \url{http://creativecommons.org/licenses/by-nc-sa/3.0/us/legalcode}.\\[0.5cm]
The responsible publisher of this edition is identified as \emph{IBizz IT Services and Consultancy}, Amsteldijk 14, 1074 HR Amsterdam, a registered company governed by the laws of the Kingdom of The Netherlands.\\[1cm]
This edition is registered under ISBN \isbn \\[1cm]
\psset{unit=1in}
\begin{pspicture}(3.5,1in)
  \psbarcode{\isbn}{includetext guardwhitespace}{isbn}
\end{pspicture}
\newpage
%%% Local Variables:
%%% mode: latex
%%% TeX-master: t
%%% End:

\tableofcontents
\listoftables
\newpage
\pagenumbering{arabic}
\frontmatter
\large
\chapter*{\fontspec{IBM Plex Serif}\LARGE The \nr{} Programming Series}
This book is part of a library, the \emph{\nr{} Programming Series}, documenting the \nr{} programming language and its use and applications. This section lists the other publications in this series, and their roles. These books can be ordered in convenient hardcopy and electronic formats from the Rexx Language Association.
\newline
\newline
\begin{tabularx}{\textwidth}{>{\bfseries}lX}
\toprule
%% Quick Start Guide & This guide is meant for an audience that has done some programming and wants to start quickly. It starts with a quick tour of the language, and a section on installing the \nr{} translator and how to run it. It also contains help for troubleshooting if anything in the installation does not work as designed, and states current limits and restrictions of the open source reference implementation.
%% \\\midrule
Programming Guide & The Programming Guide is the one manual that at the same time teaches programming, shows lots of examples as they occur in the real world, and explains about the internals of the translator and how to interface with it.
\\\midrule
Language Reference & Referred to as the NRL, this is meant as the formal definition for the language, documenting its syntax and semantics, and prescribing minimal functionality for language implementers.
\\\midrule
Pipelines Guide \& Reference & The Data Flow oriented companion to \nr{}, with its CMS Pipelines compatible syntax, is documented in this manual. It discusses running Pipes for \nr{} in the command shell and the Workspace, and has ample examples of defining your own stages in \nr{}.
\\\bottomrule
\end{tabularx}
%%% Local Variables: 
%%% mode: latex
%%% TeX-master: t
%%% End: 

%\chapter{Typographical conventions}
In general, the following conventions have  been observed in the NetRexx publications:
\begin{itemize}
\item Body text is in this font
\item Examples of language statements are in a \textbf{bold} type
\item Variables or strings as mentioned in source code, or things that appear on the console, are in a \texttt{typewriter} type
\item Items that are introduced, or emphasized, are in an \emph{italic} type
\item Included program fragments are listed in this fashion:
\begin{lstlisting}[label=example,caption=Example Listing]
-- salute the reader
say 'lectorem salutat'
\end{lstlisting}
\item Syntax diagrams take the form of so-called \emph{Railroad Diagrams} to convey structure, mandatory and optional items
\begin{rail}
AggregateExpression : ("AVG" |"MAX" |"MIN" |"SUM")
 (
   (
    'DISTINCT' ?  StateFieldPathExpression
   ) | 'COUNT'
   (
    'DISTINCT' ?  IdentificationVariable
                  | StateFieldPathExpression
                  | SingleValuedAssociationPathExpression
   )
 )
   ;
\end{rail}
%%% Local Variables: 
%%% mode: latex
%%% TeX-master: t
%%% End: 
\end{itemize}
\mainmatter
\chapter{Introduction}
\nr{} is a general-purpose programming language inspired by two very
different programming languages, R\textsc{exx}\textsuperscript{\texttrademark} and Java\textsuperscript{\texttrademark}. It is designed for
people, not computers. In this respect it follows R\textsc{exx} closely, with
many of the concepts and most of the syntax taken directly from R\textsc{exx}
or its object-oriented version, Object R\textsc{exx}. From Java it derives
static typing, binary arithmetic, the object model, and exception
handling. The resulting language not only provides the scripting
capabilities and decimal arithmetic of R\textsc{exx}, but also seamlessly
extends to large application development with fast binary arithmetic.

The open source reference implementation (version 3 and later) of
\nr{} produces classes for the Java Virtual Machine, and in so doing
demonstrates the value of that concrete interface between language and
machine: \nr{} classes and Java classes are entirely equivalent –
\nr{} can use any Java class (and vice versa) and inherits the
portability and robustness of the Java environment.

This document is in three parts:
\begin{enumerate}
\item The objectives of the \nr{} language and the concepts underlying its design, and acknowledgements.
\item An overview and introduction to the \nr{} language.
\item The definition of the language.
\end{enumerate}
Appendices include a sample \nr{} program, a description of an experimental feature, and some
details of the contents of the \texttt{netrexx.lang} package.
\section{Language Objectives}
This document describes a programming language, called \nr{}, which
is derived from both R\textsc{exx} and Java. \nr{} is intended as a dialect
of R\textsc{exx} that can be as efficient and portable as languages such as
Java, while preserving the low threshold to learning and the ease of
use of the original R\textsc{exx} language.
\subsection{Features of R\textsc{exx}}
The R\textsc{exx} programming language\footnote{Cowlishaw, M. F., \textbf{The REXX Language} (second edition), ISBN 0-13-780651-5, Prentice-Hall, 1990.} was designed with just one objective: to make programming easier than it was before. The design achieved this by emphasizing readability and usability, with a minimum of special notations and restrictions. It was consciously designed to make life easier for its users, rather than for its implementers.
One important feature of R\textsc{exx} syntax is \emph{keyword
  safety}. Programming languages invariably need to evolve over time
as the needs and expectations of their users change, so this is an
essential requirement for languages that are intended to be executed from source.

Keywords in R\textsc{exx} are not globally reserved but are recognized only in
context. This language attribute has allowed the language to be
extended substantially over the years without invalidating existing
programs. Even so, some areas of R\textsc{exx} have proved difficult to extend
– for example, keywords are reserved within instructions such as
\textbf{do}. Therefore, the design for \nr{} takes the concept of
keyword safety even further than in R\textsc{exx}, and also improves
extensibility in other areas.

The great strengths of R\textsc{exx} are its human-oriented features, including 
\begin{itemize}
\item simplicity
\item coherent and uncluttered syntax
\item comprehensive stringhandling
\item case-insensitivity
\item arbitrary precision decimal arithmetic.
\end{itemize}
Care has been taken to preserve these. Conversely, its interpretive
nature has always entailed a lack of efficiency: excellent R\textsc{exx}
compilers do exist, from IBM and other companies, but cannot offer the
full speed of statically-scoped languages such as
C\footnote{Kernighan, B. W., and Ritchie, D. M., \textbf{The C
    Programming Language} (second edition), ISBN 0-13-110362-8,
  Prentice- Hall, 1988.} or Java\footnote{Gosling, J. A., \emph{et
    al.} \textbf{The Java Language Specification}, ISBN 0-201-63451-1,
  Addison-Wesley, 1996.}.
\subsection{Influence of Java}
The system-independent design of R\textsc{exx} makes it an obvious and natural
fit to a system-independent execution environment such as that
provided by the Java Virtual Machine (JVM). The JVM, especially when
enhanced with “just-in-time” bytecode compilers that compile bytecodes
into native code just before execution, offers an effective and
attractive target environment for a language like R\textsc{exx}.

Choosing the JVM as a target environment does, however, place significant constraints on the design of a language suitable for that environment. For example, the semantics of method invocation are in several ways determined by the environment rather than by the source language, and, to a large extent, the object model (class structure, \emph{etc.}) of the Java environment is imposed on languages that use it.

Also, Java maintains the C concept of primitive datatypes; types (such
as \texttt{int}, a 32-bit signed integer) which allow efficient use of the underlying hardware yet do not describe true objects. These types are pervasive in classes and interfaces written in the Java language; any language intending to use Java classes effectively must provide access to these types.

Equally, the \emph{exception} (error handling) model of Java is pervasive, to
the extent that methods must check certain exceptions and declare
those that are not handled within the method. This makes it difficult
to fit an alternative exception model.

The constraints of safety, efficiency, and environment necessitated
that \nr{} would have to differ in some details of syntax and
semantics from R\textsc{exx}; unlike Object R\textsc{exx}, it could not be a fully
upwards-compatible extension of the language\footnote{Nash, S. C.,
  \textbf{Object-Oriented REXX} in Goldberg, G, and Smith, P. H. III,
  \textbf{The R\textsc{exx} Handbook}, pp115-125, ISBN 0-07-023682-8,
  McGraw-Hill, Inc., New York, 1992.}. The need for changes, however,
offered the opportunity to make some significant simplifications and
enhancements to the language, both to improve its keyword safety and to strengthen other features of the original R\textsc{exx} design\footnote{See Cowlishaw, M. F., \textbf{The Early History of REXX}, IEEE Annals of the History of Computing, ISSN 1058-6180, Vol 16, No. 4, Winter 1994, pp15-24, and Cowlishaw, M. F., \textbf{The Future of R\textsc{exx}}, Proceedings of Winter 1993 Meeting/SHARE 80, Volume II, p.2709, SHARE Inc., Chicago, 1993.}. Some additions from Object R\textsc{exx} and ANSI R\textsc{exx}\footnote{See \textbf{American National Standard for Information Technology – Programming Language REXX}, X3.274-1996, American National Standards Institute, New York, 1996.} are also included.

Similarly, the concepts and philosophy of the R\textsc{exx} design can profitably be applied to avoid many of the minor irregularities that characterize the C and Java language family, by providing suitable simplifications in the programming model. For example, the \nr{} looping construct has only one form, rather than three, and exception handling can be applied to all blocks rather than requiring an extra construct. Also, as in R\textsc{exx}, all \nr{} storage allocation and de-allocation is implicit – an explicit new operator is not required.

Further, the human-oriented design features of R\textsc{exx}
(case-insensitivity for identifiers, type deduction from context,
automatic conversions where safe, tracing, and a strong emphasis on
string representations of common values and numbers) make programming
for the Java environment especially easy in \nr{}.
\subsection{A hybrid or a whole?}
As in other mixtures, not all blends are a success; when first designing \nr{}, it was not at all obvious whether the new language would be an improvement on its parents, or would simply reflect the worst features of both.

The fulcrum of the design is perhaps the way in which datatyping is automated without losing the static typing supported by Java. Typing in \nr{} is most apparent at interfaces – where it provides most value – but within methods it is subservient and does not obscure algorithms. A simple concept, \emph{binary classes}, also lets the programmer choose between robust decimal arithmetic and less safe (but faster) binary arithmetic for advanced programming where performance is a primary consideration.

The “seamless” integration of types into what was previously an essentially typeless language does seem to have been a success, offering the advantages of strong typing while preserving the ease of use and speed of development that R\textsc{exx} programmers have enjoyed.

The end result of adding Java typing capabilities to the R\textsc{exx} language
is a single language that has both the R\textsc{exx} strengths for scripting
and for writing macros for applications and the Java strengths of
robustness, good efficiency, portability, and security for application
development.
\section{Language Concepts}
As described in the last section, \nr{} was created by applying the philosophy of the R\textsc{exx} language to the semantics required for programming the Java Virtual Machine (JVM). Despite the assumption that the JVM is a “target environment” for \nr{}, it is intended that the language not be environment-dependent; the essentials of the language do not depend on the JVM. Environment- dependent details, such as the primitive types supported, are not part of the language specification.

The primary concepts of R\textsc{exx} have been described before, in \emph{The
  R\textsc{exx} Language}, but it is worth repeating them and also indicating
where modifications and additions have been necessary to support the
concepts of statically-typed and object-oriented environments. The
remainder of this section is therefore a summary of the principal
concepts of \nr{}.
\subsection{Readability}
One concept was central to the evolution of R\textsc{exx} syntax, and hence \nr{} syntax: \emph{readability} (used here in the sense of perceived legibility). Readability in this sense is a somewhat subjective quality, but the general principle followed is that the tokens which form a program can be written much as one might write them in Western European languages (English, French, and so forth). Although \nr{} is more formal than a natural language, its syntax is lexically similar to everyday text.

The structure of the syntax means that the language is readily adapted
to a variety of programming styles and layouts. This helps satisfy
user preferences and allows a lexical familiarity that also increases
readability. Good readability leads to enhanced understandability,
thus yielding fewer errors both while writing a program and while
reading it for information, debugging, or maintenance. 

Important factors here are:
\begin{enumerate}
\item Punctuation and other special notations are required only when absolutely necessary to remove ambiguity (though punctuation may often be added according to personal preference, so long as it is syntactically correct). Where notations are used, they follow established conventions.
\item The language is essentially case-insensitive. A \nr{}
  programmer may choose a style of use of uppercase and lowercase letters that he or she finds most helpful (rather than a style chosen by some other programmer).
\item The classical constructs of structured and object-oriented
  programming are available in \nr{}, and can undoubtedly lead to
  programs that are easier to read than they might otherwise be. The
  simplicity and small number of constructs also make \nr{} an
  excellent language for teaching the concepts of good structure.
\item Loose binding between the physical lines in a program and the syntax of the language ensures that even though programs are affected by line ends, they are not irrevocably so. A clause may be spread over several lines or put on just one line; this flexibility helps a programmer lay out the program in the style felt to be most readable.
\end{enumerate}
\subsection{Natural data typing and decimal arithmetic}
“Strong typing”, in which the values that a variable may take are
tightly constrained, has been an attribute of some languages for many
years. The greatest advantage of strong typing is for the interfaces
between program modules, where errors are easy to introduce and
difficult to catch. Errors \emph{within} modules that would be
detected by strong typing (and which would not be detected from
context) are much rarer, certainly when compared with design errors,
and in the majority of cases do not justify the added program
complexity.

\nr{}, therefore, treats types as unobtrusively as possible, with a simple syntax for type description which makes it easy to make types explicit at interfaces (for example, when describing the arguments to methods).

By default, common values (identifiers, numbers, and so on) are described in the form of the symbolic notation (strings of characters) that a user would normally write to represent those values. This natural datatype for values also supports decimal arithmetic for numbers, so, from the user’s perspective, numbers look like and are manipulated as strings, just as they would be in everyday use on paper.

When dealing with values in this way, no internal or machine
representation of characters or numbers is exposed in the language,
and so the need for many data types is reduced. There are, for
example, no fundamentally different concepts of \emph{integer} and
\emph{real}; there is just the single concept of \emph{number}. The
results of all operations have a defined symbolic representation, and
will therefore act consistently and predictably for every correct
implementation.

This concept also underlies the BASIC\footnote{Kemeny, J. G. and
  Kurtz, T. E., \textbf{BASIC programming}, John Wiley \& Sons Inc.,
  New York, 1967.} language; indeed, Kemeny and Kurtz's vision for
BASIC included many of the fundamental principles that inspired
R\textsc{exx}. For example, Thomas E. Kurtz wrote:
\begin{quote}
“Regarding variable types, we felt that a distinction between ‘fixed’
and ‘floating’ was less justified in 1964 than earlier ... to our
potential audience the distinction between an integer number and a
non-integer number would seem esoteric. A number is a number is a
number.”\footnote{Kurtz, T. E., \textbf{BASIC} in Wexelblat,
  R. L. (Ed), \textbf{History of Programming Languages}, ISBN
  0-12-745040-8, Academic Press, New York 1981.}
\end{quote}
For R\textsc{exx}, intended as a scripting language, this approach was ideal; symbolic operations were all that were necessary.

For \nr{}, however, it is recognized that for some applications it
is necessary to take full advantage of the performance of the
underlying environment, and so the language allows for the use and
specification of binary arithmetic and types, if available. A very
simple mechanism (declaring a class or method to be \emph{binary}) is
provided to indicate to the language processor that binary arithmetic
and types are to be used where applicable. In this case, as in other
languages, extra care has to be taken by the programmer to avoid
exceeding limits of number size and so on.
\subsection{Emphasis on symbolic manipulation}
Many values that \nr{} manipulates are (from the user’s point of view, at least) in the form of strings of characters. Productivity is greatly enhanced if these strings can be handled as easily as manipulating words on a page or in a text editor. \nr{} therefore has a rich set of character manipulation operators and methods, which operate on values of type \texttt{R\textsc{exx}} (the name of the class of \nr{} strings).

Concatenation, the most common string operation, is treated specially
in \nr{}. In addition to a conventional concatenate operator (“||”),
the novel \emph{blank operator} from R\textsc{exx} concatenates two data
strings together with a blank in between. Furthermore, if two
syntactically distinct terms (such as a string and a variable name)
are abutted, then the data strings are concatenated directly. These
operators make it especially easy to build up complex character
strings, and may at any time be combined with the other operators.

For example, the \textbf{say} instruction consists of the keyword \textbf{say} followed
by any expression. In this instance of the instruction, if the
variable n has the value “6” then
 \begin{lstlisting}
  say 'Sorry,' n*100/50'% were rejected'
  \end{lstlisting}
would display the string
 \begin{lstlisting}
   Sorry, 12% were rejected
   \end{lstlisting}

Concatenation has a lower priority than the arithmetic operators. The order of evaluation of the expression is therefore first the multiplication, then the division, then the concatenate-with-blank, and finally the direct concatenation.
Since the concatenation operators are distinct from the arithmetic
operators, very natural coercion (automatic conversion) between
numbers and character strings is possible. Further, explicit typecasting (conversion of types) is effected by the same operators, at
the same priority, making for a very natural and consistent syntax for
changing the types of results. For example,
\begin{lstlisting}
i=int 100/7
\end{lstlisting}
would calculate the result of 100 divided by 7, convert that result to
an integer (assuming \texttt{int} describes an integer type) and then
assign it to the variable \texttt{i}.
\subsection{Nothing to declare}
Consistent with the philosophy of simplicity, \nr{} does not require
that variables within methods be declared before use. Only the
\emph{properties}\footnote{Class variables and instance variables.} of classes – which may form part of their
interface to other classes – need be listed formally.

Within methods, the type of variables is deduced statically from
context, which saves the programmer the menial task of stating the
type explicitly. Of course, if preferred, variables may be listed and
assigned a type at the start of each method.
\subsection{Environment independence}
The core \nr{} language is independent of both operating systems and hardware. \nr{} programs, though, must be able to interact with their environment, which implies some dependence on that environment (for example, binary representations of numbers may be required). Certain areas of the language are therefore described as being defined by the environment.

Where environment-independence is defined, however, there may be a
loss of efficiency – though this can usually be justified in view of
the simplicity and portability gained.

As an example, character string comparison in \nr{} is normally
independent of case and of leading and trailing blanks. (The string “
Yes ” \emph{means} the same as “yes” in most applications.) However,
the influence of underlying hardware has often subtly affected this
kind of design decision, so that many languages only allow trailing
blanks but not leading blanks, and insist on exact case matching. By
contrast, \nr{} provides the human-oriented relaxed comparison for
strings as default, with optional “strict comparison” operators.

\subsection{Limited span syntactic units}
The fundamental unit of syntax in the \nr{} language is the clause,
which is a piece of program text terminated by a semicolon (usually
implied by the end of a line). The span of syntactic units is
therefore small, usually one line or less. This means that the syntax
parser in the language processor can rapidly detect and locate errors,
which in turn means that error messages can be both precise and concise.

It is difficult to provide good diagnostics for languages (such as
Pascal and its derivatives) that have large fundamental syntactic
units. For these languages, a small error can often have a major or
distributed effect on the parser, which can lead to multiple error
messages or even misleading error messages.

\subsection{Dealing with reality}
A computer language is a tool for use by real people to do real work. Any tool must, above all, be reliable. In the case of a language this means that it should do what the user expects. User expectations are generally based on prior experience, including the use of various programming and natural languages, and on the human ability to abstract and generalize.

It is difficult to define exactly how to meet user expectations, but it helps to ask the question “Could there be a high \emph{astonishment} factor associated with this feature?”. If a feature, accidentally misused, gives apparently unpredictable results, then it has a high astonishment factor and is therefore undesirable.

Another important attribute of a reliable software tool is \emph{consistency}. A consistent language is by definition predictable and is often elegant. The danger here is to assume that because a rule is consistent and easily described, it is therefore simple to understand. Unfortunately, some of the most elegant rules can lead to effects that are completely alien to the intuition and expectations of a user who, after all, is human.

These constraints make programming language design more of an art than
a science, if the usability of the language is a primary goal. The
problems are further compounded for \nr{} because the language is
suitable for both scripting (where rapid development and ease of use
are paramount) and for application development (where some programmers
prefer extensive checking and redundant coding). These conflicting
goals are balanced in \nr{} by providing automatic handling of many
tasks (such as conversions between different representations of
strings and numbers) yet allowing for “strict” options which, for
example, may require that all types be explicit, identifiers be
identical in case as well as spelling, and so on.

\subsection{Be adaptable}
Wherever possible \nr{} allows for the extension of instructions and other language constructs, building on the experience gained with R\textsc{exx}. For example, there is a useful set of common characters available for future use, since only small set is used for the few special notations in the language.

Similarly, the rules for keyword recognition allow instructions to be added whenever required without compromising the integrity of existing programs. There are no reserved keywords in \nr{}; variable names chosen by a programmer always take precedence over recognition of keywords. This ensures that \nr{} programs may safely be executed, from source, at a time or place remote from their original writing – even if in the meantime new keywords have been added to the language.

A language needs to be adaptable because \emph{it certainly will be
  used for applications not foreseen by the designer}. Like all
programming languages, \nr{} may (indeed, probably will) prove
inadequate for certain future applications; room for expansion and
change is included to make the language more adaptable and robust.

\subsection{Keep the language small}
\nr{} is designed as a small language. It is not the sum of all the
features of R\textsc{exx} and of Java; rather, unnecessary features have been
omitted. The intention has been to keep the language as small as
possible, so that users can rapidly grasp most of the language. This
means that:
\begin{itemize}
\item the language appears less formidable to the new user
\item documentation is smaller and simpler
\item the experienced user can be aware of all the abilities of the
language, and so has the whole tool at his or her disposal
\item there are few exceptions, special cases, or rarely used embellishments
\item the language is easier to implement.
\end{itemize}
Many languages have accreted “neat” features which make certain
algorithms easier to express; analysis shows that many of these are
rarely used. As a rough rule-of-thumb, features that simply provided
alternative ways of writing code were added to R\textsc{exx} and \nr{} only
if they were likely to be used more often than once in five thousand
clauses.

\subsection{No defined size or shape limits}
The language does not define limits on the size or shape of any of its tokens or data (although there may be implementation restrictions). It does, however, define a few \emph{minimum} requirements that must be satisfied by an implementation. Wherever an implementation restriction has to be applied, it is recommended that it should be of such a magnitude that few (if any) users will be affected.

Where arbitrary implementation limits are necessary, the language
requires that the implementer use familiar and memorable decimal
values for the limits. For example 250 would be used in preference to
255, 500 to 512, and so on.

\section{Acknowledgements}
Much of \nr{} is based on earlier work, and I am indebted to the hundreds of people who contributed to the development of R\textsc{exx}, Object R\textsc{exx}, and Java.

In the 1990s I gained many insights from the deliberations of the members of the X3J18 technical committee, which, under the remarkable chairmanship of Brian Marks, led to the 1996 ANSI Standard for R\textsc{exx}. Many of the committee's suggestions are incorporated in \nr{}.

Equally important have been the comments and feedback from the pioneering users of \nr{}, and all those people who sent me comments on the language either directly or in the \nr{} mailing list or forum. I would especially like to thank Ian Brackenbury, Barry Feigenbaum, Davis Foulger, Norio Furukawa, Dion Gillard, Martin Lafaix, Max Marsiglietti, and Trevor Turton for their insightful comments and encouragement.

I also thank IBM; my appointment as an IBM Fellow made it possible to make the implementation of \nr{} a reality in months rather than years. IBM has also donated the \nr{} implementation to the R\textsc{exx} Language Association, with special thanks due to Matthew Emmons for piloting \nr{} through the convoluted legal and other processes, and to Ren\'{e} Jansen for massaging the \nr{} reference implementation into shape for its Open Source release.

Finally, this document has relied on old but trusted technology for
its creation: its GML markup was processed using macros originally
written by Bob O'Hara, and formatted using SCRIPT/VS, the IBM Document
Composition Facility. Geoff Bartlett provided critical advice on
character sets and fonts for the \nr{} book. This version uses a set
of R\textsc{exx} programs to translate that same GML markup to OpenOffice Document Text format (XML files).
\\
\\
\emph{Mike Cowlishaw, 1997 and 2009}
\chapter{Introduction to the current edition}
After the open sourcing of the \nr{} reference implementation in 2011
the \Rexx{}LA \nr{} ARB (Architecture Review
Board), in which Mike Cowlishaw takes part as Language Architect,  took responsibility for the definition of the
language. \marginnote{\color{gray}3.00} Starting from version 3.00, changes in the language
definition\footnote{This publication is traditionally known as NRL, short
    for NetRexx Language Reference. This title however, has (for
    reasons of clarity for new users) been changed in the filename of the PDF version
  of the book in favour of a longer and more descriptive name.} in this publication will be marked with the introducing
release number, in the form of a margin note.

For this version of the \nr{} Language Reference, a \nr{} program was
used to translate the original GML markup to \XeLaTeX. This edition
describes the\splice{java org.netrexx.process.NrVersion} version of the
language and supercedes all earlier versions. \marginnote{\color{gray}3.07} The previously included chapter ``A Quick
Tour of the NetRexx Language'' can now be exclusively found in the
\emph{NetRexx Quickstart Guide}.

\marginnote{\color{gray}4.01} Version 4, a new major version number of the language translator,
signifies a major milestone in the development of this
implementation. Thanks to Marc Remes, \nr{} now supports
the Java Platform Module System (JPMS), which enables it to compile (or
interpret) programs on current JDK versions.

\nr{} 4 depends on JSR 203 (NIO.2) and thus requires a minimum JDK level of Java 7,
whereas \nr{} 3 runs on Java 6.
\nr{} 4 compiles and runs on Java 7/8 (without JPMS) and on Java
9+\footnote{JDK 9,11,13,15 and 17 have been tested.} (with JPMS)
including the most recent versions.
\\
\\
\emph{René Vincent Jansen, \today}
%\chapter{A Quick Tour of NetRexx}
This chapter summarizes the main features of NetRexx, and is intended
to help you start using it quickly. It is assumed that you have some
knowledge of programming in a language such as Rexx, C, BASIC, or
Java, but extensive experience with programming is not needed.

This is not a complete tutorial, though – think of it more as a
\emph{taster}; it covers the main points of the language and shows some
examples you can try or modify. For full details of the language,
consult the NetRexx Programmer's Guide and the NetRexx Language
Definition documents.

\section{NetRexx programs}
The structure of a NetRexx program is extremely simple. This sample
program, “toast”, is complete, documented, and executable as it
stands:
\begin{lstlisting}[label=cheers,caption=Toast]
    /* This wishes you the best of health. */
    say 'Cheers!'
\end{lstlisting}
This program consists of two lines: the first is an optional comment that describes the purpose of the program, and the second is a \textbf{say} instruction. \textbf{say} simply displays the result of the expression following it – in this case just a literal string (you can use either single or double quotes around strings, as you prefer).
To run this program using the reference implementation of NetRexx,
create a file called toast.nrx and copy or paste the two lines above
into it. You can then use the NetRexxC Java program to compile it:
\begin{verbatim}
    java org.netrexx.process.NetRexxC toast
\end{verbatim}
(this should create a file called toast.class), and then use
the \texttt{java} command to run it:
\begin{verbatim}
    java toast
\end{verbatim}
You may also be able to use the netrexxc or nrc command to compile and
run the program with a single command (details may vary – see the
installation and user’s guide document for your implementation of
NetRexx):
\begin{verbatim}
    netrexxc toast –run
\end{verbatim}
Of course, NetRexx can do more than just display a character string. Although the language has a simple syntax, and has a small number of instruction types, it is powerful; the reference implementation of the language allows full access to the rapidly growing collection of Java programs known as class libraries, and allows new class libraries to be written in NetRexx.
The rest of this overview introduces most of the features of NetRexx. Since the economy, power, and clarity of expression in NetRexx is best appreciated with use, you are urged to try using the language yourself.
\section{Expressions and variables}
Like \textbf{say} in the “toast” example, many instructions in NetRexx include expressions that will be evaluated. NetRexx provides arithmetic operators (including integer division, remainder, and power operators), several concatenation operators, comparison operators, and logical operators. These can be used in any combination within a NetRexx expression (provided, of course, that the data values are valid for those operations).

All the operators act upon strings of characters (known as \emph{NetRexx
strings}), which may be of any length (typically limited only by the
amount of storage available). Quotes (either single or double) are
used to indicate literal strings, and are optional if the literal
string is just a number. For example, the expressions:
\begin{verbatim}
    '2' + '3'
    '2' + 3
     2 + 3
\end{verbatim}
would all result in '5'.

The results of expressions are often assigned to \emph{variables}, using a
conventional assignment syntax:
\begin{lstlisting}[label=assignment,caption=Assignment]
    var1=5            /* sets var1 to '5'    */
    var2=(var1+2)*10  /* sets var2 to '70' */
\end{lstlisting}
You can write the names of variables (and keywords) in whatever mixture of uppercase and lowercase that you prefer; the language is not case-sensitive.
This next sample program, “greet”, shows expressions used in various
ways:
\begin{lstlisting}[label=greet,caption=Greet]
    /* greet.nrx –– a short program to greet you.        */
    /* First display a prompt:                           */
    say 'Please type your name and then press Enter:'
    answer=ask            /* Get the reply into 'answer' */
    /* If no name was entered, then use a fixed          */
    /* greeting, otherwise echo the name politely.       */
    if answer='' then say 'Hello Stranger!'
                 else say 'Hello' answer'!'
\end{lstlisting}
After displaying a prompt, the program reads a line of text from the
user (“ask” is a keyword provided by NetRexx) and assigns it to the
variable answer. This is then tested to see if any characters were
entered, and different actions are taken accordingly; for example, if
the user typed “\texttt{Fred}” in response to the prompt, then the program
would display:
\begin{verbatim}
Hello Fred!
\end{verbatim}
As you see, the expression on the last \textbf{say} (display) instruction
concatenated the string “Hello” to the value of variable answer with a
blank in between them (the blank is here a valid operator, meaning
“concatenate with blank”). The string “!” is then directly
concatenated to the result built up so far. These unobtrusive
operators (the \emph{blank operator} and abuttal) for concatenation are very
natural and easy to use, and make building text strings simple and
clear.

The layout of instructions is very flexible. In the “greet” example,
for instance, the \textbf{if} instruction could be laid out in a number of
ways, according to personal preference. Line breaks can be added at
either side of the \textbf{then} (or following the \textbf{else}).

In general, instructions are ended by the end of a line. To continue a
instruction to a following line, you can use a hyphen (minus sign)
just as in English:
\begin{lstlisting}[label=continue,caption=Continuation]
    say 'Here we have an expression that is quite long,' –
        'so it is split over two lines'
\end{lstlisting}
This acts as though the two lines were all on one line, with the hyphen and any blanks around it being replaced by a single blank. The net result is two strings concatenated together (with a blank in between) and then displayed.
When desired, multiple instructions can be placed on one line with the
aid of the semicolon separator:
\begin{lstlisting}[label=multiple,caption=Multiple Instructions]
    if answer='Yes' then do; say 'OK!'; exit; end
\end{lstlisting}
(many people find multiple instructions on one line hard to read, but sometimes it is convenient).
\section{Control instructions}
NetRexx provides a selection of \emph{control} instructions, whose form was
chosen for readability and similarity to natural languages. The
control instructions include \textbf{if... then... else} (as in the “greet”
example) for simple conditional processing:
\begin{lstlisting}[label=Conditional,caption=Conditional]
    if ask='Yes' then say "You answered Yes"
                 else say "You didn't answer Yes"
\end{lstlisting}
\textbf{select... when... otherwise... end} for selecting from a number of
alternatives:
\begin{lstlisting}[label=selectwhenotherwise,caption=select - when - otherwise]
    select
      when a>0 then say 'greater than zero'
      when a<0 then say 'less than zero'
      otherwise say 'zero'
      end
    select case i+1
      when 1 then say 'one'
      when 1+1 then say 'two'
      when 3, 4, 5 then say 'many'
      end
\end{lstlisting}
\textbf{do... end} for grouping:
\begin{lstlisting}[label=doend,caption=do - end]
    if a>3 then do
      say 'A is greater than 3; it will be set to zero'
      a=0
      end
\end{lstlisting}
and \textbf{loop... end} for repetition:
\begin{lstlisting}[label=loopend,caption=loop - end]
    /* repeat 10 times; I changes from 1 to 10 */
    loop i=1 to 10
    say i end i
\end{lstlisting}
The \textbf{loop} instruction can be used to step a variable
\textbf{to} some limit, \textbf{by} some increment, \textbf{for} a
specified number of iterations, and \textbf{while} or \textbf{until}
some condition is satisfied. \textbf{loop forever} is also provided,
and \textbf{loop over} can be used to work through a collection of
variables.

Loop execution may be modified by \textbf{leave} and \textbf{iterate} instructions that significantly reduce the complexity of many programs.
The \textbf{select}, \textbf{do}, and \textbf{loop} constructs also have the ability to “catch”
exceptions (see \ref{exceptions} on page \pageref{exceptions}.) that occur in the body of the construct. All
three, too, can specify a \textbf{finally} instruction which introduces
instructions which are to be executed when control leaves the
construct, regardless of how the construct is ended.


\section{NetRexx arithmetic}
Character strings in NetRexx are commonly used for arithmetic
(assuming, of course, that they represent numbers). The string
representation of numbers can include integers, decimal notation,
and exponential notation; they are all treated the same way. Here are
a few:
\begin{verbatim}
    '1234'
    '12.03'
    '–12'
    '120e+7'
\end{verbatim}
The arithmetic operations in NetRexx are designed for people rather than machines, so are decimal rather than binary, do not overflow at certain values, and follow the rules that people use for arithmetic. The operations are completely defined by the ANSI X3.274 standard for Rexx, so correct implementations always give the same results.
An unusual feature of NetRexx arithmetic is the \textbf{numeric} instruction:
this may be used to select the \emph{arbitrary precision} of
calculations. You may calculate to whatever precision that you wish
(for financial calculations, perhaps), limited only by available
memory. For example:
\begin{lstlisting}[label=Digits,caption=Digits]
    numeric digits 50
    say 1/7
\end{lstlisting}
which would display
\begin{verbatim}
    0.14285714285714285714285714285714285714285714285714
\end{verbatim}
The numeric precision can be set for an entire program, or be adjusted at will within the program. The \textbf{numeric} instruction can also be used to select the notation (\emph{scientific} or \emph{engineering}) used for numbers in exponential format.
NetRexx also provides simple access to the native binary arithmetic of
computers. Using binary arithmetic offers many opportunities for
errors, but is useful when performance is paramount. You select binary
arithmetic by adding the instruction:
\begin{verbatim}
    options binary
\end{verbatim}
at the top of a NetRexx program. The language processor will then use
binary arithmetic (see page \pageref{binarith}) instead of NetRexx decimal arithmetic for calculations, if it can, throughout the program.
\section{Doing things with strings}
A character string is the fundamental datatype of NetRexx, and so, as
you might expect, NetRexx provides many useful routines for
manipulating strings. These are based on the functions of Rexx, but
use a syntax that is more like Java or other similar languages:
\begin{lstlisting}[label=strings,caption=Strings]
    phrase='Now is the time for a party'
    say phrase.word(7).pos('r')
\end{lstlisting}
The second line here can be read from left to right as:
\begin{quote}take the variable phrase, find the seventh word, and then find the position of
the first “r” in that word.\end{quote}
This would display “3” in this case, because “r” is the third character in “party”.

(In Rexx, the second line above would have been written using nested
function calls:
\begin{lstlisting}[label=nested,caption=Rexx: Nested]
    say pos('r', word(phrase, 7))
\end{lstlisting}
which is not as easy to read; you have to follow the nesting and then
backtrack from right to left to work out exactly what’s going on.)

In the NetRexx syntax, at each point in the sequence of operations
some routine is acting on the result of what has gone before. These
routines are called \emph{methods}, to make the distinction from functions
(which act in isolation). NetRexx provides (as methods) most of the
functions that were evolved for Rexx, including:
\begin{itemize}
\item \texttt{changestr} (change all occurrences of a substring to another)
\item \texttt{copies} (make multiple copies of a string)
\item \texttt{lastpos} (find rightmost occurrence)
\item \texttt{left} and \texttt{right} (return leftmost/rightmost character(s))
\item \texttt{pos} and \texttt{wordpos} (find the position of string or a word in a string)
\item \texttt{reverse} (swap end-to-end)
\item \texttt{space} (pad between words with fixed spacing)
\item \texttt{strip} (remove leading and/or trailing white space)
\item \texttt{verify} (check the contents of a string for selected characters)
\item \texttt{word}, \texttt{wordindex}, \texttt{wordlength}, and \texttt{words} (work with words).
\end{itemize}
These and the others like them, and the parsing described in the next section, make it especially easy to process text with NetRexx.
\section{Parsing strings}
The previous section described some of the string-handling facilities
available; NetRexx also provides string parsing, which is an easy way
of breaking up strings of characters using simple pattern matching.

A \textbf{parse} instruction first specifies the string to be parsed. This can be any term, but is often taken simply from a variable. The term is followed by a \emph{template} which describes how the string is to be split up, and where the pieces are to be put.
\subsection{Parsing into words}
The simplest form of parsing template consists of a list of variable
names. The string being parsed is split up into words (sequences of
characters separated by blanks), and each word from the string is
assigned (copied) to the next variable in turn, from left to
right. The final variable is treated specially in that it will be
assigned a copy of whatever is left of the original string and may
therefore contain several words. For example, in:
\begin{lstlisting}[label=parsingstrings,caption=Parsing Strings]
parse 'This is a sentence.' v1 v2 v3
\end{lstlisting}
the variable v1 would be assigned the value “This”, v2 would be assigned the value
“is”, and v3 would be assigned the value “a sentence.”.
\subsection{Literal patterns}
A literal string may be used in a template as a pattern to split up
the string. For example
\begin{lstlisting}[label=parse,caption=Parse]
    parse 'To be, or not to be?'   w1 ',' w2 w3 w4
\end{lstlisting}
would cause the string to be scanned for the comma, and then split at that point; each section is then treated in just the same way as the whole string was in the previous example.

Thus, w1 would be set to “To be”, w2 and w3 would be assigned the values “or” and “not”, and w4 would be assigned the remainder: “to be?”. Note that the pattern itself is not assigned to any variable.
The pattern may be specified as a variable, by putting the variable
name in parentheses. The following instructions:
\begin{lstlisting}[label=comma,caption=Parse with comma]
    comma=','
    parse 'To be, or not to be?'   w1 (comma) w2 w3 w4
\end{lstlisting}
therefore have the same effect as the previous example.
\subsection{Positional patterns}
The third kind of parsing mechanism is the numeric positional pattern. This allows strings to be parsed using column positions.
\section{Indexed strings}
NetRexx provides indexed strings, adapted from the compound variables of Rexx. Indexed strings form a powerful “associative lookup”, or \emph{dictionary}, mechanism which can be used with a convenient and simple syntax.

NetRexx string variables can be referred to simply by name, or also by
their name qualified by another string (the \emph{index}). When an index is
used, a value associated with that index is either set:
\begin{lstlisting}[label=index,caption=Index]
    fred=0           –– initial value
    fred[3]='abc'  –– indexed value
\end{lstlisting}
or retrieved:
\begin{lstlisting}[label=retrieving,caption=Retrieving]
    say fred[3]    –– would say "abc"
\end{lstlisting}
in the latter case, the simple (initial) value of the variable is
returned if the index has not been used to set a value. For example,
the program:
\begin{lstlisting}[label=woof,caption=Woof]
    bark='woof'
    bark['pup']='yap'
    bark['bulldog']='grrrrr'
    say bark['pup'] bark['terrier'] bark['bulldog']
\end{lstlisting}
would display
\begin{verbatim}
    yap woof grrrrr
\end{verbatim}
Note that it is not necessary to use a number as the index; any
expression may be used inside the brackets; the resulting string is
used as the index. Multiple dimensions may be used, if required:
\begin{lstlisting}[label=dimensions,caption=Multiple Dimensions]
    bark='woof'
    bark['spaniel', 'brown']='ruff'
    bark['bulldog']='grrrrr'
    animal='dog'
    say bark['spaniel', 'brown'] bark['terrier'] bark['bull'animal]
\end{lstlisting}
which would display
\begin{verbatim}
    ruff woof grrrrr
\end{verbatim}
Here’s a more complex example using indexed strings, a test program
with a function (called a \emph{static method} in NetRexx) that removes all
duplicate words from a string of words:
\begin{lstlisting}[label=justone,caption=justonetest.nrx]
    /* justonetest.nrx –– test the justone function.      */
    say justone('to be or not to be')  /* simple testcase */
    exit
    /* This removes duplicate words from a string, and    */
    /* shows the use of a variable (HADWORD) which is     */
    /* indexed by arbitrary data (words).                 */
    method justone(wordlist) static
      hadword=0         /* show all possible words as new */
      outlist=''            /* initialize the output list */
      loop while wordlist\=''  /* loop while we have data */
        /* split WORDLIST into first word and residue     */
        parse wordlist word wordlist
        if hadword[word] then iterate /* loop if had word */
        hadword[word]=1 /* remember we have had this word */
        outlist=outlist word   /* add word to output list */
        end
return outlist /* finally return the result */
\end{lstlisting}
Running this program would display just the four words “to”, “be”, “or”, and “not”.
\section{Arrays}
NetRexx also supports fixed-size \emph{arrays}. These are an ordered set of
items, indexed by integers. To use an array, you first have to
construct it; an individual item may then be selected by an index
whose value must be in the range \texttt{0} through \texttt{n–1}, where n is the number
of items in the array:
\begin{lstlisting}[label=arrays,caption=Arrays]
    array=String[3]        –– make an array of three Strings
    array[0]='String one'  –– set each array item
    array[1]='Another string'
    array[2]='foobar'
    loop i=0 to 2          –– display the items
      say array[i]
      end
\end{lstlisting}
This example also shows NetRexx \emph{line comments}; the sequence “––” (outside of literal strings or “/*” comments) indicates that the remainder of the line is not part of the program and is commentary.

NetRexx makes it easy to initialize arrays: a term which is a list of
one or more expressions, enclosed in brackets, defines an array. Each
expression initializes an element of the array. For example:
\begin{lstlisting}[label=initializingelements,caption=Initializing elements]
words=['Ogof', 'Ffynnon', 'Ddu']
\end{lstlisting}
would set words to refer to an array of three elements, each referring to a string. So, for
example, the instruction:
\begin{lstlisting}[label=addresselement,caption=Address Array Element]
say words[1]
\end{lstlisting}
would then display 
\begin{verbatim}
Ffynnon
\end{verbatim}

\section{Things that aren’t strings}
In all the examples so far, the data being manipulated (numbers, words, and so on) were expressed as a string of characters. Many things, however, can be expressed more easily in some other way, so NetRexx allows variables to refer to other collections of data, which are known as \emph{objects}.

Objects are defined by a name that lets NetRexx determine the data and methods that are associated with the object. This name identifies the type of the object, and is usually called the \emph{class} of the object.

For example, an object of class Oblong might represent an oblong to be manipulated and displayed. The oblong could be defined by two values: its width and its height. These values are called the \emph{properties} of the Oblong class.

Most methods associated with an object perform operations on the object; for example a size method might be provided to change the size of an Oblong object. Other methods are used to construct objects (just as for arrays, an object must be constructed before it can be used). In NetRexx and Java, these \emph{constructor} methods always have the same name as the class of object that they build (“Oblong”, in this case).

Here’s how an Oblong class might be written in NetRexx (by convention,
this would be written in a file called \texttt{Oblong.nrx}; implementations
often expect the name of the file to match the name of the class
inside it):
\begin{lstlisting}[label=oblong,caption=Oblong]
    /* Oblong.nrx –– simple oblong class */
    class Oblong
      width       –– size (X dimension)
      height      –– size (Y dimension)
      /* Constructor method to make a new oblong */
      method Oblong(newwidth, newheight)
        –– when we get here, a new (uninitialized) object
        –– has been created.  Copy the parameters we have
        –– been given to the properties of the object:
        width=newwidth; height=newheight
      /* Change the size of an Oblong */
      method size(newwidth, newheight) returns Oblong
        width=newwidth; height=newheight
        return this   –– return the resized object
      /* Change the size of an Oblong, relatively */
      method relsize(relwidth, relheight)–
                    returns Oblong
        width=width+relwidth; height=height+relheight
return this
      /* 'Print' what we know about the oblong */
      method print
        say 'Oblong' width 'x' height
\end{lstlisting}
To summarize:
\begin{enumerate}
\item A class is started by the \textbf{class} instruction, which names the class.
\item The \textbf{class} instruction is followed by a list of the properties of the object. These can be assigned initial values, if required.
\item The properties are followed by the methods of the object. Each
method is introduced by a \textbf{method} instruction which names the method
and describes the arguments that must be supplied to the method. The
body of the method is ended by the next method instruction (or by the
end of the file).
\end{enumerate}
The \texttt{Oblong.nrx} file is compiled just like any other NetRexx program,
and should create a \emph{class file} called \texttt{Oblong.class}. Here’s a program
to try out the Oblong class:
\begin{lstlisting}[label=tryoblong,caption=Try Oblong]
    /* tryOblong.nrx –– try the Oblong class */
    first=Oblong(5,3)        –– make an oblong
    first.print              –– show it
    first.relsize(1,1).print –– enlarge and print again
    second=Oblong(1,2)       –– make another oblong
    second.print             –– and print it
\end{lstlisting}
When tryOblong.nrx is compiled, you’ll notice (if your compiler makes a cross-reference listing available) that the variables \texttt{first} and \texttt{second} have type \texttt{Oblong}. These variables refer to Oblongs, just as the variables in earlier examples referred to NetRexx strings.

Once a variable has been assigned a type, it can only refer to objects of that type. This helps avoid errors where a variable refers to an object that it wasn’t meant to.
\subsection{Programs are classes, too}
It’s worth pointing out, here, that all the example programs in this overview are in fact classes (you may have noticed that compiling them with the reference implementation creates \texttt{xxx.class} files, where \texttt{xxx} is the name of the source file). The environment underlying the implementation will allow a class to run as a stand-alone \emph{application} if it has a static method called \texttt{main} which takes an array of strings as its argument.

If necessary (that is, if there is no class instruction) NetRexx automatically adds the necessary class and method instructions for a stand-alone application, and also an instruction to convert the array of strings (each of which holds one word from the command string) to a single NetRexx string.

The automatic additions can also be included explicitly; the “toast”
example could therefore have been written:
\begin{lstlisting}[label=toast,caption=New Toast]
    /* This wishes you the best of health. */
    class toast
      method main(argwords=String[]) static
        arg=Rexx(argwords)
        say 'Cheers!'
\end{lstlisting}
though in this program the argument string, \texttt{arg}, is not used.
\section{Extending classes}
It’s common, when dealing with objects, to take an existing class and extend it. One way to do this is to modify the source code of the original class – but this isn’t always available, and with many different people modifying a class, classes could rapidly get overcomplicated.

Languages that deal with objects, like NetRexx, therefore allow new
classes of objects to be set up which are derived from existing
classes. For example, if you wanted a different kind of Oblong in
which the Oblong had a new property that would be used when printing
the Oblong as a rectangle, you might define it thus:
\begin{lstlisting}[label=charoblong,caption=charOblong.nrx]
    /* charOblong.nrx –– an oblong class with character */
    class charOblong extends Oblong
      printchar       –– the character for display
      /* Constructor to make a new oblong with character */
      method charOblong(newwidth, newheight, newprintchar)
        super(newwidth, newheight) –– make an oblong
        printchar=newprintchar     –– and set the character
      /* 'Print' the oblong */
      method print
        loop for super.height
          say printchar.copies(super.width)
          end
\end{lstlisting}
There are several things worth noting about this example:
\begin{enumerate}
\item The “\texttt{extends Oblong}” on the class instruction means that this class is an extension of the Oblong class. The properties and methods of the Oblong class are \emph{inherited} by this class (that is, appear as though they were part of this class).
Another common way of saying this is that “\texttt{charOblong}” is a \emph{subclass} of “\texttt{Oblong}” (and “\texttt{Oblong}” is the \emph{superclass} of “\texttt{charOblong}”).
\item This class adds the \texttt{printchar} property to the properties already defined for Oblong.
\item The constructor for this class takes a width and height (just like Oblong) and adds a third argument to specify a print character. It first invokes the constructor of its superclass (Oblong) to build an Oblong, and finally sets the printchar for the new object.
\item The new charOblong object also prints differently, as a rectangle of characters, according to its dimension. The \texttt{print} method (as it has the same name and arguments – none – as that of the superclass) replaces (overrides) the \texttt{print'} method of Oblong.
\item The other methods of Oblong are not overridden, and therefore
  can be used on charOblong objects.
\end{enumerate}
The \texttt{charOblong.nrx} file is compiled just like \texttt{Oblong.nrx} was, and
should create a file called \texttt{charOblong.class}.

Here’s a program to try it out
\begin{lstlisting}[label=trycharoblong,caption=tryCharOblong.nrx]
    /* trycharOblong.nrx –– try the charOblong class */
    first=charOblong(5,3,'#')  –– make an oblong
    first.print                –– show it
    first.relsize(1,1).print   –– enlarge and print again
    second=charOblong(1,2,'*') –– make another oblong
    second.print               –– and print it
\end{lstlisting}
This should create the two charOblong objects, and print them out in a simple “character graphics” form. Note the use of the method \texttt{relsize} from Oblong to resize the charOblong object.
\subsection{Optional arguments}
All methods in NetRexx may have optional arguments (omitted from the
right) if desired. For an argument to be optional, you must supply a
default value. For example, if the charOblong constructor was to have
a default value for printchar, its method instruction could have been
written
\begin{lstlisting}[label=default,caption=Default value X]
method charOblong(newwidth, newheight, newprintchar='X')
\end{lstlisting}
which indicates that if no third argument is supplied then \texttt{'X'} should be used. A program
creating a charOblong could then simply write:
\begin{lstlisting}[label=default,caption=Default value]
first=charOblong(5,3) –– make an oblong
\end{lstlisting}
which would have exactly the same effect as if \texttt{'X'} were specified as
the third argument.


\section{Tracing}
NetRexx tracing is defined as part of the language. The flow of
execution of programs may be traced, and this trace can be viewed as
it occurs (or captured in a file). The trace can show each clause as
it is executed, and optionally show the results of expressions,
etc. For example, the \textbf{trace results} in the program “\texttt{trace1.nrx}”:
\begin{lstlisting}[label=trace,caption=Trace]
    trace results
    number=1/7
    parse number before '.' after
    say after'.'before
\end{lstlisting}
would result in:
\begin{verbatim}
       ––– trace1.nrx
     2 *=* number=1/7
       >v> number "0.142857143"
     3 *=* parse number before '.' after
       >v> before "0"
       >v> after "142857143"
     4 *=* say after'.'before
       >>> "142857143.0"
    142857143.0
\end{verbatim}
where the line marked with “\texttt{–––}” indicates the context of the trace, lines marked with “\texttt{*=*}” are the instructions in the program, lines with “\texttt{>v>}” show results assigned to local variables, and lines with “\texttt{>>>}” show results of unnamed expressions.

Further, \textbf{trace methods} lets you trace the use of all methods in a
class, along with the values of the arguments passed to each
method. Here’s the result of adding \texttt{trace methods} to the Oblong class
shown earlier and then running \texttt{tryOblong}:
\begin{verbatim}
        ––– Oblong.nrx
      8 *=*     method Oblong(newwidth, newheight)
        >a> newwidth "5"
        >a> newheight "3"
     26 *=*     method print
    Oblong 5 x 3
     20 *=*     method relsize(relwidth, relheight)–
 21 *–*
    >a> relwidth "1"
    >a> relheight "1"
 26 *=*     method print
Oblong 6 x 4
returns Oblong
     10 *=*     method Oblong(newwidth, newheight)
        >a> newwidth "1"
        >a> newheight "2"
     26 *=*     method print
    Oblong 1 x 2
\end{verbatim}
where lines with “>a>” show the names and values of the arguments.

It is often useful to be able to find out when (and where) a variable’s value is changed. The \textbf{trace var} instruction does just that; it adds names to or removes names from a list of monitored variables. If the name of a variable in the current class or method is in the list, then \textbf{trace results} is turned on for any assignment, \textbf{loop}, or \textbf{parse} instruction that assigns a new value to the named variable.

Variable names to be added to the list are specified by listing them after the \textbf{var} keyword. Any name may be optionally prefixed by a – sign., which indicates that the variable is to be removed from the list.

For example, the program “\texttt{trace2.nrx}”:
\begin{lstlisting}[label=trace2,caption=trace2.nrx]
    trace var a b
    –– now variables a and b will be traced
    a=3
    b=4
    c=5
    trace var –b c
    –– now variables a and c will be traced
    a=a+1
    b=b+1
    c=c+1
    say a b c
\end{lstlisting}
would result in:
\begin{verbatim}
        ––– trace2.nrx
  3 *=* a=3
    >v> a "3"
  4 *=* b=4
    >v> b "4"
  8 *=* a=a+1
    >v> a "4"
 10 *=* c=c+1
    >v> c "6"
4 5 6
\end{verbatim}

\section{Binary types and conversions}\label{binarith}

Most programming environments support the notion of fixed-precision
“primitive” binary types, which correspond closely to the binary
operations usually available at the hardware level in computers. For
the reference implementation, these types are:
\begin{itemize}
\item \emph{byte}, \emph{short}, \emph{int}, and \emph{long} – signed integers that will fit in 8, 16, 32, or 64 bits respectively
\item \emph{float} and \emph{double} – signed floating point numbers that will fit in 32 or 64 bits respectively.
\item \emph{char} – an unsigned 16-bit quantity, holding a Unicode character
\item \emph{boolean} – a 1-bit logical value, representing 0 or 1
    (“false” or “true”).
\end{itemize}
Objects of these types are handled specially by the implementation “under the covers” in order to achieve maximum efficiency; in particular, they cannot be constructed like other objects – their value is held directly. This distinction rarely matters to the NetRexx programmer: in the case of string literals an object is constructed automatically; in the case of an \texttt{int} literal, an object is not constructed.

Further, NetRexx automatically allows the conversion between the
various forms of character strings in implementations\footnote{In the
  reference implementation, these are String, char, char[] (an array
  of characters), and the NetRexx string type, Rexx.} and the
primitive types. The “golden rule” that is followed by NetRexx is that
any automatic conversion which is applied must not lose information:
either it can be determined before execution that the conversion is
safe (as in \texttt{int} to \texttt{String}) or it will be detected at
execution time if the conversion fails (as in \texttt{String} to
\texttt {int}).

The automatic conversions greatly simplify the writing of programs; the exact type of numeric and string-like method arguments rarely needs to be a concern of the programmer.
For certain applications where early checking or performance override
other considerations, the reference implementation of NetRexx
provides options for different treatment of the primitive types:
\begin{enumerate}
\item \textbf{options strictassign} – ensures exact type matching for all assignments. No conver- sions (including those from shorter integers to longer ones) are applied. This option provides stricter type-checking than most other languages, and ensures that all types are an exact match.
\item \textbf{options binary} – uses implementation-dependent fixed precision
arithmetic on binary types (also, literal numbers, for example, will
be treated as binary, and local variables will be given “native”
types such as \texttt{int} or \texttt{String}, where possible).
\end{enumerate}
Binary arithmetic currently gives better performance than NetRexx
decimal arithmetic, but places the burden of avoiding overflows and
loss of information on the programmer.

The options instruction (which may list more than one option) is placed before the first class instruction in a file; the \textbf{binary} keyword may also be used on a \textbf{class} or \textbf{method} instruction, to allow an individual class or method to use binary arithmetic.
\subsection{Explicit type assignment}
You may explicitly assign a type to an expression or variable:
\begin{lstlisting}[label=assigningtype,caption=Assigning Type]
i=int 3000000  –– 'i' is an 'int' with value 3000000
j=int 4000000  –– 'j' is an 'int' with value 4000000
k=int
say i*j
k=i*j
–– 'k' is an 'int', with no initial value
–– multiply and display the result
–– multiply and assign result to 'k'
\end{lstlisting}
This example also illustrates an important difference between
\textbf{options nobinary} and \textbf{options binary}. With the former
(the default) the \textbf{say} instruction would display the result
“\texttt{1.20000000E+13}” and a conversion overflow would be reported
when the same expression is assigned to the variable k.

With \textbf{options binary}, binary arithmetic would be used for the multiplications, and so no error would be detected; the say would display “–138625024” and the variable \texttt{k} takes the incorrect result.
\subsection{Binary types in practice}
In practice, explicit type assignment is only occasionally needed in
NetRexx. Those conversions that are necessary for using existing
classes (or those that use \textbf{options binary}) are generally
automatic. For example, here is an Applet for use by Java-enabled
browsers:
\begin{lstlisting}[label=asimpleapplet,caption=A Simple Applet]
    /* A simple graphics Applet */
    class Rainbow extends Applet
      method paint(g=Graphics)  –– called to repaint window
        maxx=size.width–1
        maxy=size.height–1
        loop y=0 to maxy
          col=Color.getHSBColor(y/maxy, 1, 1) –– new colour
          g.setColor(col)                     –– set it
          g.drawLine(0, y, maxx, y)           –– fill slice
       end y
\end{lstlisting}
In this example, the variable \texttt{col} will have type \texttt{Color}, and the three
arguments to the method \texttt{getHSBColor} will all automatically be
converted to type float. As no overflows are possible in this example,
\textbf{options binary} may be added to the top of the program with no other
changes being necessary.

\section{Exception and error handling}\label{exceptions}
NetRexx does not have a \textbf{goto} instruction, but a \textbf{signal} instruction is provided for abnormal transfer of control, such as when something unusual occurs. Using \textbf{signal} raises an \emph{exception}; all control instructions are then “unwound” until the exception is caught by a control instruction that specifies a suitable catch instruction for handling the exception.

Exceptions are also raised when various errors occur, such as
attempting to divide a number by zero. For example:
\begin{lstlisting}[label=exception,caption=Exception]
    say 'Please enter a number:'
    number=ask
    do
      say 'The reciprocal of' number 'is:' 1/number
    catch Exception
      say 'Sorry, could not divide "'number'" into 1'
      say 'Please try again.'
    end
\end{lstlisting}
Here, the \textbf{catch} instruction will catch any exception that is raised when the division is attempted (conversion error, divide by zero, \emph{etc.}), and any instructions that follow it are then executed. If no exception is raised, the \textbf{catch} instruction (and any instructions that follow it) are ignored.

Any of the control instructions that end with \textbf{end} (\textbf{do}, \textbf{loop}, or \textbf{select}) may be modified with one or more \textbf{catch} instructions to handle exceptions.

\section{Summary and Information Sources}
The NetRexx language, as you will have seen, allows the writing of programs for the Java environment with a minimum of overhead and “boilerplate syntax”; using NetRexx for writing Java classes could increase your productivity by 30\% or more.
Further, by simplifying the variety of numeric and string types of
Java down to a single class that follows the rules of Rexx strings,
programming is greatly simplified. Where necessary, however, full
access to all Java types and classes is available.

Other examples are available, including both stand-alone applications and samples of applets for Java-enabled browsers (for example, an applet that plays an audio clip, and another that displays the time in English). You can find these from the NetRexx web pages, at
    \url{http://www.netrexx.org}.
Also at that location, you’ll find the NetRexx language specification
and other information, and downloadable packages containing the
NetRexx software and documentation. There is a large selection of
NetRexx examples available at \url{http://www.rosettacode.org}.
The software should run on any platform that has a Java Virtual
Machine (JVM) available.
\chapter{\nr{} Language Definition}\label{refpart3}
\index{\nr{},language definition}
 
This part of the document describes the \nr{} language, version \splice{java org.netrexx.process.NrVersion}.
This version includes the original \nr{} language reference\footnote{
The \nr{} Language, M. F. Cowlishaw,
ISBN 0-13-806332-X, Prentice-Hall, 1997
}
together with additions made from 1997 through 2000 and previously
published in the \emph{\nr{} Language Supplement}.
 
The language is described first in terms of the characters from which it
is composed and its low-level syntax, and then progressively through
more complex constructions.
Finally, special sections describe the semantics of the more
complicated areas.
 
\index{Reference implementation,}
\index{Java,in reference implementation}
Some features of the language, such as \keyword{options} keywords and
binary arithmetic, are implementation-dependent.  Rather than leaving
these important aspects entirely abstract, this description includes
summaries of the treatment of such items in the \emph{reference
implementation} of \nr{}.  The reference implementation is based
on the Java environment and class libraries.
 
\emph{Paragraphs that refer to the reference implementation,
and are therefore not strictly part of the language definition, are
shown in italics, like this one.}
 %chapter 3
\section{Notations}\label{refnotat}
\index{Syntax notation,}
\index{Syntax diagrams,notation for}
\index{Notations,syntax}
\index{Notations,in text}
\index{Diagrams, of syntax,}
 
In this part of the
book,
various notations such as changes of font are used for clarity.
Within the text, a
sans-serif bold font is used to indicate \keyword{keywords}, and an italic
font is used to indicate \emph{technical terms}.
An italic font is also used to indicate a reference to a
\emph{technical term defined elsewhere} or a \emph{word} in a
syntax diagram that names a segment of syntax.
 
Similarly, in the syntax diagrams in this
book,
words (symbols) in the
sans-serif
bold font also denote keywords or
sub-keywords, and words (such as \emph{expression}) in italics
denote a token or collection of tokens defined elsewhere.
The brackets [ and ] delimit optional (and possibly
alternative) parts of the instructions, whereas the braces \{
and \} indicate that one of a number of alternatives must be
selected.
An ellipsis (\textbf{...}) following a bracket indicates that
the bracketed part of the clause may optionally be repeated.
 
Occasionally in syntax diagrams (\emph{e.g.}, for indexed references)
brackets are "real" (that is, a bracket is required in the
syntax; it is not marking an optional part).
These brackets are enclosed in single quotes, thus:
\keyword{'['} or \keyword{']'}.
 
Note that the keywords and sub-keywords in the syntax diagrams are not
case-sensitive: the symbols "IF" "If" and "iF" would
all match the keyword shown in a syntax diagram as \keyword{if}.
 
\index{Semicolons,can be omitted}
Note also that most of the clause delimiters ("\textbf{;}") shown
can usually be omitted as they will be implied by the end of a line.
 %chapter 4
\chapter{Characters and Encodings}\label{refencod}
\index{Character,encodings}
\index{Encodings, of characters,}
\index{Character,}
\index{Coded character,}
\index{Encodings,of characters}
 In the definition of a programming language it is important to
emphasize the distinction between a \emph{character} and the
\emph{coded representation}
\footnote{
These terms have the meanings as defined by the International
Organization for Standardization, in ISO 2382 :cit.Data processing
- Vocabulary:ecit..
}
(encoding) of a character.
The character "A", for example, is the first letter of the English
(Roman) alphabet, and this meaning is independent of any specific coded
representation of that character.
Different coded character sets (such as, for example, the ASCII
\footnote{
\index{ASCII,coded character set}
\index{Coded character set,ASCII}
American Standard Code for Information Interchange.
}
and EBCDIC
\footnote{
\index{EBCDIC,coded character set}
\index{Coded character set,EBCDIC}
Extended Binary Coded Decimal Interchange Code.
}
codes) use quite different encodings for this character (decimal
values 65 and 193, respectively).
 Except where stated otherwise, this
book
uses characters to convey meaning and not to imply a specific character
code (the exceptions are certain operations that specifically convert
between characters and their representations). At no time is \nr{} concerned with the glyph (actual appearance) of
a character.
\index{Character,appearance}
\index{Character,glyphs}
\index{Glyphs,}
\section{Character Sets}
\index{Character sets,}
\index{Unicode,coded character set}
\index{Coded character set,Unicode}
 Programming in the \nr{} language can be considered to involve the
use of two character sets.
The first is used for expressing the \nr{} program itself, and is the
relatively small set of characters described in the next section.
The second character set is the set of characters that can be used as
character data by a particular implementation of a \nr{} language
processor.
This character set may be limited in size (sometimes to a limit of 256
different characters, which have a convenient 8-bit representation), or
it may be much larger.  The \emph{Unicode}
\footnote{
\emph{The Unicode Standard, version 6.0}.,
The Unicode Consortium, Mountain View, 2011, ISBN 09781936213016.
}
character set, for example, allows for 1,114,112 code points, of which
currently 128,000 are defined as characters. These are represented,
depending on the serialization format, in one to four bytes.
 
Usually, most or all of the characters in the second (data) character
set are also allowed within a \nr{} program, but only within
commentary or immediate (literal) data.
 The \nr{} language explicitly defines the first character set, in
order that programs will be portable and understandable; at the same
time it avoids restrictions due to the language itself on the character
set used for data.
However, where the language itself manipulates or inspects the data (as
when carrying out arithmetic operations), there may be requirements on
the data character set (for example, numbers can only be expressed if
there are digit characters in the set).
 %chapter 5
\chapter{Structure and General Syntax}\label{refclau}
\index{Clauses,}
\index{Semicolons,}
\index{,}
 A \nr{} program is built up out of a series of \emph{clauses} that
are composed of: zero or more blanks (which are ignored); a sequence of
tokens (described in this section); zero or more blanks (again
ignored); and the delimiter "\textbf{;}" (semicolon) which may
be implied by line-ends or certain keywords.
Conceptually, each clause is scanned from left to right before
execution and the tokens composing it are resolved.
 
Identifiers (known as symbols) and numbers are recognized at this stage,
comments (described below) are removed, and multiple blanks (except
within literal strings) are reduced to single blanks.
Blanks adjacent to  operator characters (see page \pageref{refopers})  and
 special characters (see page \pageref{refspecs})  are also removed.
\section{Blanks and White Space}\label{refblank}
\index{Blank,}
\index{White space,}
 \emph{Blanks} (spaces) may be freely used in a program to
improve appearance and layout, and most are ignored.
Blanks, however, are usually significant
\begin{itemize}
\item within literal strings (see below)
\item between two tokens that are not special characters (for example,
between two symbols or keywords)
\item between the two characters forming a comment delimiter
\item immediately outside parentheses ("\textbf{(}" and
"\textbf{)}") or brackets ("\textbf{[}" and
"\textbf{]}").
\end{itemize}
 
\index{Form feed character,}
\index{Tabulation character,}
\index{Tab character,}
\index{EOF character,}
\index{End-of-file character,}
For implementations that support tabulation (tab) and form feed
characters, these characters outside of literal strings are treated as
if they were a single blank; similarly, if the last character in a
\nr{} program is the End-of-file character (EOF, encoded in ASCII as
decimal 26), that character is ignored.
\section{Comments}\label{refcomment}
\index{Comments,}
\index{Delimiters,for comments}
\index{,}
\index{,}
\index{,}
 Commentary is included in a \nr{} program by means of
\emph{comments}.  Two forms of comment notation are provided:
\emph{line comments} are ended by the end of the line on which they
start, and \emph{block comments} are typically used for more extensive
commentary.
\begin{description}
\item[Line comments]\label{reflineco}
\index{Comments,line}
\index{Line comments,}

A line comment is started by a sequence of two adjacent hyphens
(\doublehyphen{}); all characters following that sequence up to the
end of the line are then ignored by the \nr{} language processor.
 
\textbf{Example:}
\begin{alltt}
i=j+7  -- this line comment follows an assignment
\end{alltt}
\item[Block comments]\label{refblockco}
\index{Comments,block}
\index{Block comments,}
 A block comment is started by the sequence of characters
"\textbf{/*}", and is ended by the same sequence reversed,
"\textbf{*/}".
Within these delimiters any characters are allowed (including quotes,
which need not be paired).
\index{Comments,nesting}
\index{Nesting of comments,}
Block comments may be nested, which is to say that
"\textbf{/*}" and "\textbf{*/}" must pair correctly.
\index{Length,of comments}
Block comments may be anywhere, and may be of any length.
When a block comment is found, it is treated as though it were a blank
(which may then be removed, if adjacent to a special character).
 
\textbf{Example:}
\begin{alltt}
/* This is a valid block comment */
\end{alltt}
The two characters forming a comment delimiter
("\textbf{/*}" or "\textbf{*/}") must be adjacent
(that is, they may not be separated by blanks or a line-end).
\end{description}
\index{Comments,starting a program with}
\begin{shaded}\noindent
\textbf{Note: }It is recommended that \nr{} programs start with a block comment
that describes the program.
Not only is this good programming practice, but some implementations may
use this to distinguish \nr{} programs from other languages.
 \textbf{Implementation minimum:} Implementations should support
nested block comments to a depth of at least 10.
The length of a comment should not be restricted, in that it should be
possible to "comment out" an entire program.
\end{shaded}\indent
\section{Tokens}\label{reftokens}
\index{Tokens,}
 The essential components of clauses are called \emph{tokens}.
These may be of any length, unless limited by implementation
restrictions,
\footnote{
Wherever arbitrary implementation restrictions are applied, the size of
the restriction should be a number that is readily memorable in the
decimal system; that is, one of 1, 25, or 5 multiplied by a power of
ten.
500 is preferred to 512, the number 250 is more "natural" than
256, and so on.  Limits expressed in digits should be a multiple of
three.
}
and are separated by blanks, comments, ends of lines, or by the nature
of the tokens themselves.
 
The tokens are:
\begin{description}
\item[Literal strings]\label{refxstr}
\index{Strings,}
\index{Literal strings,}
\index{,}
\index{Delimiters,for strings}
\index{,}
\index{Strings,quotes in}
\index{Quotes in strings,}
\index{Double-quote,string delimiter}
\index{Single-quote,string delimiter}
\index{Strings,as literal constants}

A sequence including any characters that can safely be represented in
an implementation
\footnote{
Some implementations may not allow certain "control characters"
in literal strings.
These characters may be included in literal strings by using one of the
escape sequences provided.
}
and delimited by the single quote character (\textbf{'}) or the
double-quote (\textbf{"}).
Use \textbf{""} to include a \textbf{"} in a literal
string delimited by \textbf{"}, and similarly use two single
quotes to include a single quote in a literal string delimited by
single quotes.
A literal string is a constant and its contents will never be modified
by \nr{}.
Literal strings must be complete on a single line (this means that
unmatched quotes may be detected on the line that they occur).
\index{Strings,null}
\index{Null strings,}
 Any string with no characters (\emph{i.e.}, a string of length 0) is called
a \emph{null string}.
 
\textbf{Examples:}
\begin{alltt}
'Fred'
'Aÿ'
"Don't Panic!"
":x"
'You shouldn''t'    /* Same as "You shouldn't" */
''                  /* A null string */
\end{alltt}
 
\index{Strings,escapes in}
\index{Escape sequences in strings,}
\index{\textbackslash  backslash,escape character}
\index{Backslash character,in strings}
Within literal strings, characters that cannot safely or easily be
represented (for example "control characters") may be introduced
using an \emph{escape sequence}.  An escape sequence starts with a
\emph{backslash} ("\textbf{\textbackslash }"), which must then be
followed immediately by one of the following (letters may be in either
uppercase or lowercase) - see table \ref{table:escapecodes}.

\index{Tab character,escape sequence}
\index{Newline character,escape sequence}
\index{Line feed character,escape sequence}
\index{Carriage return character,escape sequence}
\index{Return character,escape sequence}
\index{Double-quote,escape sequence}
\index{Single-quote,escape sequence}
\index{Backslash character,escape sequence}
\index{Null character,escape sequence}
\index{Zero character,escape sequence}
\index{Hexadecimal,escape sequence}
\index{Unicode,escape sequence}
\begin{table}\caption{Escape sequences}\label{table:escapecodes}
\begin{tabularx}{\textwidth}{>{\bfseries}lX}
\toprule
\textbackslash t& the escape sequence represents a tabulation (tab)
character
\\\midrule
\textbackslash n& the escape sequence represents a new-line (line
feed) character
\\\midrule
\textbackslash r& the escape sequence represents a return (carriage
return) character
\\\midrule
\textbackslash f& the escape sequence represents a form-feed character
\\\midrule
\textbackslash "&the escape sequence represents a double-quote
character
\\\midrule
\textbackslash '&the escape sequence represents a single-quote
character
\\\midrule
\textbackslash& the escape sequence represents a backslash character
\\\midrule
\textbackslash \texttt{-}&the escape sequence represents a "null" character
(the character whose encoding equals zero), used to indicate
continuation in a \keyword{say} instruction
\\\midrule
\textbackslash 0(zero)& the escape sequence represents a "null" character
(the character whose encoding equals zero); an alternative
to \textbf{\textbackslash -}
\\\midrule
\textbackslash xhh& the escape sequence represents a character whose encoding is
given by the two hexadecimal digits ("\textbf{hh}") following the
"\textbf{x}".
If the character encoding for the implementation requires more than two
hexadecimal digits, they are padded with zero digits on the left.
\\\midrule
\textbackslash uhhhh& the escape sequence represents a character whose encoding is
given by the four hexadecimal digits ("\textbf{hhhh}") following the
"\textbf{u}".
It is an error to use this escape if the character encoding for the
implementation requires fewer than four hexadecimal digits.
\\\bottomrule
\end{tabularx}
\end{table}
\index{Hexadecimal,digits in escapes}
 Hexadecimal digits for use in the escape sequences above may be any
decimal digit (0-9) or any of the first six alphabetic
characters (a-f), in either lowercase or uppercase.
 \textbf{Examples:}
\begin{verbatim}
'You shouldn\'t'  /* Same as "You shouldn't" */
'\x6d\u0066\x63'  /* In Unicode: 'mfc' */
'\\\u005C'        /* In Unicode, two backslashes */
\end{verbatim}
 \textbf{Implementation minimum:} Implementations should support
literal strings of at least 100 characters.
(But note that the length of string expression results, \emph{etc.}, should
have a much larger minimum, normally only limited by the amount of
storage available.)
\item[Symbols]\label{refsyms}
\index{Symbols,}
\index{Symbols,valid names}
\index{Extra letters, in symbols,}
\index{Extra digits,in symbols}
\index{Extra digits,in numeric symbols}
\index{\_ underscore,in symbols}
\index{Underscore,in symbols}
\index{\$ dollar sign,in symbols}
\index{Dollar sign,in symbols}
\index{Euro character,}
\index{Euro character,in symbols}

Symbols are groups of characters selected from the Roman alphabet
in uppercase or lowercase (A-Z, a-z), the Arabic numerals
(0-9), or the characters underscore, dollar, and euro\footnote{
Note that only UTF8-encoded source files can currently use the euro
character.} ("\textunderscore \$ \euro")
Implementations may also allow other alphabetic and numeric characters
in symbols to improve the readability of programs in languages other
than English.  These additional characters are known as \emph{extra
letters} and \emph{extra digits}.
\footnote{
It is expected that implementations of \nr{} will be based on
Unicode or a similarly rich character set.
However, portability to implementations with smaller character sets may
be compromised when extra letters or extra digits are used in a program.
}
 
\textbf{Examples:}
\begin{alltt}
DanYrOgof
minx
\'{E}lan
$Virtual3D
\end{alltt}
\index{Numbers,as symbols}
\index{Numeric symbols,}
\index{Symbols,numeric}
\index{Simple number,}
\index{Hexadecimal numeric symbol,}
\index{Binary numeric symbol,}
\index{Exponential notation,in symbols}
\index{E-notation,in symbols}
\index{Extra digits,in numeric symbols}
 A symbol may include other characters only when the first character
of the symbol is a digit (0-9 or an extra digit).
In this case, it is a \emph{numeric symbol} - it may include a
period ("\textbf{.}") and it must have the syntax of a number.
This may be \emph{simple number}, which is a sequence of digits with
at most one period (which may not be the final character of the
sequence), or it may be a  \emph{hexadecimal or binary 
numeric symbol}(see page \pageref{refhexbin}) , or it may be a number expressed in \emph{exponential
notation}.
 
A number in exponential notation is a simple number followed immediately
by the sequence "\textbf{E}" (or "\textbf{e}"), followed
immediately by a sign ("\textbf{+}" or "\textbf{-}"),
\footnote{
The sign in this context is part of the symbol; it is not an
operator.
}
followed immediately by one or more digits (which may not be followed by
any other symbol characters).
 
\textbf{Examples:}
\begin{alltt}
1
1.3
12.007
17.3E-12
3e+12
0.03E+9
\end{alltt}
 
\index{Extra digits,in numeric symbols}
When \emph{extra digits} are used in numeric symbols, they must
represent values in the range zero through nine.
When numeric symbols are used as numbers, any extra digits are first
converted to the corresponding character in the range 0-9, and then the
symbol follows the usual rules for numbers in \nr{} (that is, the most
significant digit is on the left, \emph{etc.}).
 
\marginnote{\color{gray}3.03}\emph{In the reference implementation, the extra letters are those
characters (excluding A-Z, a-z, and underscore) that result
in \textbf{1} when tested with
\\\textbf{java.lang.Character.isJavaIdentifierPart}.
Similarly, the extra digits are those characters (excluding 0-9) that
result in \textbf{1} when tested with \textbf{java.lang.Character.isDigit}.
}
 
The meaning of a symbol depends on the context in which it is used.
For example, a symbol may be a constant (if a number), a keyword, the
name of a variable, or identify some algorithm.
 
It is recommended that the dollar and euro only be used in symbols in
mechanically generated programs or where otherwise essential.
 \textbf{Implementation minimum:} Implementations should support
symbols of at least 50 characters.
(But note that the length of its value, if it is a string variable,
should have a much larger limit.)
\item[Operator characters]\label{refopers}
\index{Operators,characters used for}
\index{Special characters,used for operators}
\index{Blank,adjacent to operator character}

The characters \texttt{+ \textendash \-  * \% \textbar / \& \textbackslash = < >}
are used (sometimes in combination) to indicate
 operations (see page \pageref{refops})  in expressions.
A few of these are also used in parsing templates, and the equals sign
is also used to indicate assignment.
Blanks adjacent to operator characters are removed, so, for example,
the sequences:
\begin{alltt}
345>=123
345 >=123
345 >=  123
345 > =  123
\end{alltt}
are identical in meaning.
 Some of these characters may not be available in all character sets,
and if this is the case appropriate translations may be used.
\index{++ invalid sequence,}
\index{\textbackslash \textbackslash  invalid sequence,}
\textbf{Note: }The sequences "\textbf{--}", "\textbf{/*}", and
"\textbf{*/}" are comment delimiters, as described earlier.
The sequences "\textbf{++}"
and "\textbf{\textbackslash \textbackslash }" are not valid in \nr{}
programs.
\item[Special characters]\label{refspecs}
\index{Special characters,}
\index{Parentheses,adjacent to blanks}
\index{Blank,adjacent to special character}

The characters  \textbf{.  ,  ;  )  (  ]  [}  together
with the operator characters have special significance when found
outside of literal strings, and constitute the set of \emph{special
characters}.
They all act as token delimiters, and blanks adjacent to any of these
(except the period) are removed, except that a blank adjacent to the
outside of a parenthesis or bracket is only deleted if it is also
adjacent to another special character (unless this is a parenthesis or
bracket and the blank is outside it, too).
 Some of these characters may not be available in all character sets,
and if this is the case appropriate translations may be used.
\end{description}
 To illustrate how a clause is composed out of tokens, consider this
example:
\begin{alltt}
'REPEAT'   B + 3;
\end{alltt}
This is composed of six tokens: a literal string, a blank operator
(described later), a symbol (which is probably the name of a variable),
an operator, a second symbol (a number), and a semicolon.
The blanks between the "\textbf{B}" and the "\textbf{+}"
and between the "\textbf{+}" and the "\textbf{3}" are
removed.
However one of the blanks between the \textbf{'REPEAT'} and the
"\textbf{B}" remains as an operator.
Thus the clause is treated as though written:
\begin{alltt}
'REPEAT' B+3;
\end{alltt}
\section{Implied semicolons and continuations}\label{refsemis}
\index{Implied semicolons,}
\index{Semicolons,implied}
\index{Line ends, effect of,}
\index{Clauses,continuation of}
\index{Continuation,character}
\index{Continuation,of clauses}
\index{Hyphen,as continuation character}
\index{- continuation character,}
 A semicolon (clause end) is implied at the end of each line, except
if:
\begin{enumerate}
\item The line ends in the middle of a block comment, in which case the
clause continues at the end of the block comment.
\item The last token was a hyphen.
In this case the hyphen is functionally replaced by a blank, and hence
acts as a \emph{continuation character}.
\end{enumerate}
 This means that semicolons need only be included to separate
multiple clauses on a single line.
\begin{shaded}\noindent
\textbf{Notes:}
\begin{enumerate}
\item A comment is not a token, so therefore a comment may follow the
continuation character on a line.
\item Semicolons are added automatically by \nr{} after certain
instruction keywords when in the correct context.
The keywords that may have this effect are \keyword{else},
\keyword{finally}, \keyword{otherwise}, \keyword{then}; they become
complete clauses in their own right when this occurs.
These special cases reduce program entry errors significantly.
\end{enumerate}
\end{shaded}\indent
\section{The case of names and symbols}\label{refcase}
\index{Case,of names}
\index{Uppercase,names}
\index{Lowercase,names}
\index{Mixed case,names}
\index{Symbols,case of}
\index{Names,case of}
\index{Class,names, case of}
\index{Method,names, case of}
\index{Properties,case of names}
 
In general, \nr{} is a \emph{case-insensitive} language.
That is, the names of keywords, variables, and so on, will be recognized
independently of the case used for each letter in a name; the name
"\textbf{Swildon}" would match the name
"\textbf{swilDon}".
 
\nr{}, however, uses names that may be visible outside the \nr{}
program, and these may well be referenced by case-sensitive languages.
Therefore, any name that has an external use (such as the name of a
property, method, constructor, or class) has a defined spelling, in
which each letter of the name has the case used for that letter when the
name was first defined or used.
 
Similarly, the lookup of external names is both case-preserving and
case-insensitive.  If a class, method, or property is referenced by the
name "\textbf{Foo}", for example, an exact-case match will first
be tried at each point that a search is made.
If this succeeds, the search for a matching name is complete.
If it does not succeed, a case-insensitive search in the same context
is carried out, and if one item is found, then the search is complete.
If more than one item matches then the reference is ambiguous, and an
error is reported.
 
Implementations are encouraged to offer an option that requires that all
name matches are exact (case-sensitive), for programmers or house-styles
that prefer that approach to name matching.
\section{Hexadecimal and binary numeric symbols}\label{refhexbin}
\index{Numeric symbols,hexadecimal}
\index{Hexadecimal numeric symbol,}
 
A \emph{hexadecimal numeric symbol} describes a whole number, and is
of the form \emph{n}\textbf{X}\emph{string}.  Here,
\emph{n} is a simple number with no decimal part (and optional
leading insignificant zeros) which describes the effective length of the
hexadecimal string, the \textbf{X} (which may be in lowercase) indicates
that the notation is hexadecimal, and \emph{string} is a string of
one or more hexadecimal characters (characters from the ranges
"a-f", "A-F", and the digits "0-9").
 
The \emph{string} is taken as a signed number expressed in
\emph{n} hexadecimal characters.  If necessary, \emph{string} is
padded on the left with "\textbf{0}" characters (note, not
"sign-extended") to length \emph{n} characters.
 
If the most significant (left-most) bit of the resulting string is zero
then the number is positive; otherwise it is a negative number in
twos-complement form.  In both cases it is converted to a \nr{} number
which may, therefore, be negative.  The result of the conversion is a
number comprised of the Arabic digits 0-9, with no insignificant leading
zeros but possibly with a leading "\textbf{-}".
 
The value \emph{n} may not be less than the number of characters in
\emph{string}, with the single exception that it may be zero, which
indicates that the number is always positive (as though \emph{n}
were greater than the the length of \emph{string}).
 
\textbf{Examples:}
\begin{alltt}
1x8    == -8
2x8    == 8
2x08   == 8
0x08   == 8
0x10   == 16
0x81   == 129
2x81   == -127
3x81   == 129
4x81   == 129
04x81  == 129
16x81  == 129
4xF081 == -3967
8xF081 == 61569
0Xf081 == 61569
\end{alltt}
 
\index{Numeric symbols,binary}
\index{Binary numeric symbol,}
A \emph{binary numeric symbol} describes a whole number using the
same rules, except that the identifying character is \textbf{B}
or \textbf{b}, and the digits of \emph{string} must be
either \textbf{0} or \textbf{1}, each representing a single bit.
 
\textbf{Examples:}
\begin{alltt}
1b0    == 0
1b1    == -1
0b10   == 2
0b100  == 4
4b1000 == -8
8B1000 == 8
\end{alltt}
\textbf{Note: }Hexadecimal and binary numeric symbols are a purely syntactic
device for representing decimal whole numbers.  That is, they are
recognized only within the source of a \nr{} program, and are not
equivalent to a literal string with the same characters within quotes.
 %chapter 6
\chapter{Types and Classes}\label{reftypes}
\index{Types,}
\index{Datatypes,}
\index{Data,types}
\index{,}
 
Programs written in the \nr{} language manipulate values, such as
names, numbers, and other representations of data.
All such values have an associated \emph{type} (also known as a
\emph{signature}).
 
The type of a value is a descriptor which identifies the nature of the
value and the operations that may be carried out on that value.
 
\index{Class,}
\index{Properties,}
\index{Method,}
A type is normally defined by a \emph{class}, which is a named
collection of values (called \emph{properties}) and procedures (called
\emph{methods}) for carrying out operations on the properties.
 
\index{R\textsc{exx},class/\nr{} strings}
For example, a character string in \nr{} is usually of
type \textbf{R\textsc{exx}}, which will be implemented by the class
called \textbf{R\textsc{exx}}.
The class \textbf{R\textsc{exx}} defines the properties of the string (a
sequence of characters) and the methods that work on strings.
This type of string may be the subject of arithmetic operations as well
as more conventional string operations such as concatenation, and so the
methods implement string arithmetic as well as other string operations.
 
\index{Package,}
\index{Qualified types,}
\index{Types,qualified}
The names of types can further be qualified by the name of a
\emph{package} where the class is held.  See the \texttt{package}
instruction for details of packages; in summary, a package name is a
sequence of one or more non-numeric symbols, separated by periods.
Thus, if the \textbf{R\textsc{exx}} class was part of
the \textbf{netrexx.lang} package,
\footnote{
\emph{This is in fact where it may be found in the reference
implementation.}
}
then its \emph{qualified type} would be \textbf{netrexx.lang.R\textsc{exx}}.
 
In general, only the class name need be specified to refer to a type.
However, if a class of the same name exists in more than one known
(imported) package, then the name should be qualified by the package
name.  That is, if the use of just the class name does not uniquely
identify the class then the reference is ambiguous and an error is
reported.
\subsection{Primitive types}\label{refprims}
\index{Primitive types,}
\index{Types,primitive}
 
Implementations may optionally provide \emph{primitive types} for
efficiency.
Primitive types are "built-in" types that are not necessarily
implemented as classes.
They typically represent concepts native to the underlying environment
(such as 32-bit binary integer numbers) and may exhibit semantics that
are different from other types.  \nr{}, however, makes no syntax
distinction in the names of primitive types, and assumes
 \emph{binary constructors} (see page \pageref{refbincon})  exist for primitive
values.
 
Primitive types are necessary when performance is an overriding
consideration, and so this definition will assume that primitive types
corresponding to the common binary number formats are available and will
describe how they differ from other types, where appropriate.
 
\emph{In the reference implementation, the names of the primitive types
are:}
\begin{alltt}
boolean char byte short int long float double
\end{alltt}
\emph{where the first two describe a single-bit value and Unicode
character respectively, and the remainder describe signed numbers of
various formats.
The main difference between these and other types is that the primitive
types are not a subclass of \textbf{Object}, so they cannot be
assigned to a variable of type \textbf{Object} or passed to methods
"by reference".  To use them in this way, an object must be created
to "wrap" them; Java provides classes for this (for example, the
class \textbf{Long}).
}
\subsection{Dimensioned types}\label{refdimtype}
\index{Dimensioned types,}
\index{Dimension,of arrays}
\index{Dimension,of types}
\index{Types,dimensioned}
 
Another feature that is provided for efficiency is the concept of
\emph{dimensioned types}, which are types (normal or primitive) that
have an associated dimension (in the sense of the dimensions of an
array).  Dimensioned values are described in detail in the section on
 \emph{Arrays} (see page \pageref{refarray}) .
 
The dimension of a dimensioned type is represented in \nr{} programs
by square brackets enclosing zero or more commas, where the dimension is
given by the number of commas, plus one.  A dimensioned type is distinct
from the type of the same name but with no dimensions.
 \textbf{Examples:}
\begin{alltt}
R\textsc{exx}
int
R\textsc{exx}[]
int[,,]
\end{alltt}

The examples show a normal type, a primitive type, and a dimensioned
version of each (of dimension 1 and 3 respectively).  The latter type
would result from constructing an array thus:
\begin{alltt}
myarray=int[10,10,10]
\end{alltt}
That is, the dimension of the type matches the count of indexes
defined for the array.
\subsection{Minor and Dependent classes}\label{refmindep}
 
\index{Minor classes,}
\index{Classes,minor}
\index{Classes,parent}
A \emph{minor class} in \nr{} is a class whose name is qualified by
the name of another class, called its \emph{parent}.
This qualification is indicated by the form of the name of the class:
the short name of the minor class is prefixed by the name of its parent
class (separated by a period).
For example, if the parent is called \texttt{Foo} then the full name of a
minor class \texttt{Bar} would be written \texttt{Foo.Bar}.
 
\index{Dependent classes,}
\index{Classes,dependent}
A \emph{dependent class} is a minor class that has a link to its
parent class that allows a child object simplified access to its
parent object and its properties.
 
These refinements of classes and are described in the
section  \emph{Minor and Dependent classes} (see page \pageref{refminor}) .
 %chapter 7
\section{Terms}\label{refterms}
\index{Terms,}
\index{Data,terms}
\index{Period,in terms}
\index{. (period),in terms}
\index{Parentheses,in terms}
\index{Brackets,in terms}
\index{References,in terms}
 
A \emph{term} in \nr{} is a syntactic unit which describes some
value (such as a literal string, a variable, or the result of some
computation) that can be manipulated in a \nr{} program.
 
Terms may be either \emph{simple} (consisting of a single element) or
\emph{compound} (consisting of more than one element, with a period
and no other characters between each element).
\subsection{Simple terms}\label{refsimterm}
 A simple term may be:
\index{Terms,simple}
\index{Simple terms,}
\begin{itemize}
\index{Literal strings,in terms}
\index{Strings,in terms}
\item A  \emph{literal string} (see page \pageref{refxstr})  - a character string
delimited by quotes, which is a constant.
\index{Symbols,in terms}
\index{Symbols,numeric}
\index{Numeric symbols,}
\item A  \emph{symbol} (see page \pageref{refsyms}) .
A symbol that does not begin with a digit identifies a variable, a
value, or a type.
One that does begin with a digit is a \emph{numeric symbol}, which is
a constant.
\index{Method,calls in terms}
\index{Parentheses,in method calls}
\item A  \emph{method call} (see page \pageref{refmethcon}) , which is of the form
\begin{alltt}
\emph{symbol}([\emph{expression}[,\emph{expression}]...])
\end{alltt}
\index{Indexed references,in terms}
\index{Brackets,in indexed references}
\index{Square brackets,in indexed references}
\item An  \emph{indexed reference} (see page \pageref{refinstr}) , which is of the form
\footnote{
The notations \keyword{'['} and \keyword{']'}
indicate square brackets appearing in the \nr{} program.
}
\begin{alltt}
\emph{symbol}\keyword{'['}[\emph{expression}[,\emph{expression}]...]\keyword{']'}
\end{alltt}
\index{Array initializer,in terms}
\index{Brackets,in array initializers}
\index{Square brackets,in array initializers}
\item An  \emph{array initializer} (see page \pageref{refarrin}) , which is of the form
\begin{alltt}
\keyword{'['}\emph{expression}[,\emph{expression}]...\keyword{']'}
\end{alltt}
\index{Sub-expressions, in terms,}
\item A  \emph{sub-expression} (see page \pageref{refpreced}) , which consists of any
expression enclosed within a left and a right parenthesis.
\end{itemize}

Blanks are not permitted between the symbol in a method call and the
"\textbf{(}", or between the symbol in an indexed reference and
the "\textbf{[}".
 
\index{Parentheses,omitting from method calls}
Within simple terms, method calls with no arguments (that is, with no
expressions between the parentheses) may be expressed without the
parentheses provided that they refer to a method in the current class
(or to a static method in a class \emph{used} by the current class)
and do not refer to a  constructor (see page \pageref{refcons}) .
An implementation may optionally provide a mechanism that disallows this
parenthesis omission.
\subsection{Compound terms}\label{refcomterm}
\index{Terms,compound}
\index{Compound terms,}
 
Compound terms may start with any simple term, and, in addition, may
start with a  qualified class name (see page \pageref{refpackage})  or a
 qualified constructor (see page \pageref{refmethcon}) .
These last two both start with a package name (a sequence of non-numeric
symbols separated by periods and ending in a period).
 
\index{Terms,stub of}
\index{Stub, of term,}
This first part of a compound term is known as the \emph{stub} of the
term.
 \textbf{Example stubs:}
\begin{alltt}
"A string"
Arca
12.10
paint(g)
indexedVar[i+1]
("A" "string")
java.lang.Math        -- qualified class name
netrexx.lang.R\textsc{exx}(1)  -- qualified constructor
\end{alltt}
 
All stubs are syntactically valid terms (either simple or compound) and
may optionally be followed by a \emph{continuation}, which is one or
more additional non-numeric symbols, method calls, or indexed
references, separated from each other and from the stub by
\emph{connectors} which are periods.
 \textbf{Example compound terms:}
\begin{alltt}
"A rabbit".word(2).pos('b')
Fluffy.left(3)
12.10.max(j)
paint(g).picture
indexedVar[i+1].length
("A" "string").word(1)
java.lang.Math.PI
java.lang.Math.log(10)
\end{alltt}
 
\index{Parentheses,omitting from method calls}
Within compound terms, method calls with no arguments (that is, with no
expressions between the parentheses) may be expressed without the
parentheses provided that they do not refer to a
 constructor (see page \pageref{refcons}) .
For example, the term:
\begin{lstlisting}
Thread.currentThread().suspend()
\end{lstlisting}
could be written:
\begin{lstlisting}
Thread.currentThread.suspend
\end{lstlisting}
An implementation may optionally provide a mechanism that disallows this
parenthesis omission.
\subsection{Evaluation of terms}\label{refteval}
\index{Terms,evaluation of}
\index{Evaluation,of terms}
 
Simple terms are evaluated as a whole, as described below.
Compound terms are evaluated from left to right.  First the stub is
evaluated according to the rules detailed below.
The type of the value of the stub then qualifies the next element of the
term (if any) which is then evaluated (again, the exact rules are
detailed below).
This process is then repeated for each element in the term.
 
For instance, for the example above:
\begin{lstlisting}
"A rabbit".word(2).pos('b')
\end{lstlisting}
the evaluation proceeds as follows:
\begin{enumerate}
\item The stub (\textbf{"A rabbit"}) is evaluated, resulting in a string
of type \textbf{R\textsc{exx}}.
\item 
Because that string is of type \textbf{R\textsc{exx}}, the \textbf{R\textsc{exx}} class
is then searched for the \textbf{word} method.  This is then invoked
on the string, with argument \textbf{2}.
In other words, the \textbf{word} method is invoked with the string
"\textbf{A rabbit}" as its current context (the properties of the
R\textsc{exx} class when the method is invoked reflect that value).
 
This returns a new string of type \textbf{R\textsc{exx}},
"\textbf{rabbit}" (the second word in the original string).
\item 
In the same way as before, the \textbf{pos} method of
the \textbf{R\textsc{exx}} class is then invoked on the new string, with
argument "\textbf{b}".
 
This returns a new string of type \textbf{R\textsc{exx}}, "\textbf{3}"
(the position of the first "\textbf{b}" in the previous result).
\end{enumerate}
This value, "\textbf{3}", is the final value of the term.
 
The remainder of this section gives the details of term
evaluation; it is best skipped on first reading.
\subsection{Simple term evaluation}
 
All simple terms may also be used as stubs, and are evaluated in
precisely the same way as stubs, as described below.  For example,
numeric symbols are evaluated as though they were enclosed in quotes;
their value is a string of type \textbf{R\textsc{exx}}.
 
In  binary classes (see page \pageref{refbincla}) , however, simple terms that are
strings or numeric symbols are given an implementation-defined string or
primitive type respectively, as described in the section on
 \emph{Binary values and operations} (see page \pageref{refbinary}) 
\subsection{Stub evaluation}
\index{Search order,for term evaluation}
 
A term's stub is evaluated according to the following rules:
\begin{itemize}
\item 
If the stub is a literal string, its value is the string, of
type \textbf{R\textsc{exx}}, constructed from that literal.
\item 
If the stub is a numeric symbol, its value is the string, of
type \textbf{R\textsc{exx}}, constructed from that literal (as though the
literal were enclosed in quotes).
\item 
If the stub is an unqualified method or constructor call, or a
qualified constructor call, then its value and type is the result of
invoking the method identified by the stub, as described in
 \emph{Methods and Constructors} (see page \pageref{refmethcon}) .
\item 
If the stub is a (non-numeric) symbol, then its value and type will be
determined by whichever of the following is first found:
\begin{enumerate}
\item A local variable or method argument within the current method, or a
property in the current class.
\item A method whose name matches the symbol, and takes no arguments, and
that is not a constructor, in the current class.
\footnote{
Unless parenthesis omission is disallowed by an implementation option,
such as \keyword{options strictargs}.
}
If the stub is part of a compound symbol, then it may also be in a
superclass, searching upwards from the current class.
\item A static or constant property, or a static method,
\footnote{
Unless parenthesis omission is disallowed by an implementation option,
such as \keyword{options strictargs}.
}
whose name matches the symbol (and that takes no arguments, if a method)
in a class listed in the \keyword{uses} phrase of the \keyword{class}
instruction.
Each class from the list is searched for a matching property or method, and
then its superclasses are searched upwards from the class in the same
way; this process is repeated for each of the classes, in the order
specified in the list.
\item One of the allowed special words described in
 \emph{Special words and methods} (see page \pageref{refspecial}) , such
as \textbf{this} or \textbf{version}.
\item The short name of a known class or primitive type (in which
case the stub has no value, just a type).
\end{enumerate}
\item 
If the stub is an indexed reference, then its value and type will be
determined by whichever of the following is first found:
\begin{enumerate}
\item The symbol naming the reference is an undimensioned local variable
or method argument within the current method, or a property in the
current class, which has type \textbf{R\textsc{exx}}.  In this case, the
reference is to an  \emph{indexed string} (see page \pageref{refinstr}) ;
the expressions within the brackets must be convertible to
type \textbf{R\textsc{exx}}, and the type of the result will
be \textbf{R\textsc{exx}}.
\item The symbol naming the reference is a dimensioned local variable
or method argument within the current method, or a property in the
current class.
In this case, the reference is to an
 \emph{array} (see page \pageref{refarray}) , and the expressions within the
brackets must be convertible to whole numbers allowed for array indexes.
The type of the result will have the type of the array, with dimensions
reduced by the number of dimensions specified in the stub.
 For example, if the array \textbf{foo} was of
type \textbf{Baa[,,,]} and the stub referred
to \textbf{foo[1,2]}, then the result would be of
type \textbf{Baa[,]}.
It would have been an error for the stub to have specified more than
four dimensions.
\item The symbol naming the reference is the name of a static or constant
property in a class listed in the \keyword{uses} phrase of the
\keyword{class} instruction.
Each class from the list is searched in the same way as for symbols,
above.  The reference may be to an \emph{indexed string} or an
\emph{array}, as for properties in the current class.
\item The symbol naming the reference is the name of a primitive type or
the short name of a known class, and there are no expressions within the
brackets (just optional commas).
In this case, the stub describes a  \emph{dimensioned type (see page \pageref{refdimtype})}.
\item The symbol naming the reference is the name of a primitive type or
is the short name of a known class, and there are one or more
expressions within the brackets.
In this case, the reference is to an  \emph{array constructor (see page \pageref{refarray}) }; the expressions within the brackets must
be convertible to non-negative whole numbers allowed for array indexes,
and the value is an array of the specified type, dimensions, and size.
\end{enumerate}
\item 
If the stub is a sub-expression, then its value and type will be
determined by evaluating the  \emph{expression} (see page \pageref{refexpr}) 
within the parentheses.
\item 
If the stub starts with the name of a package, then it will either
describe a qualified  type (see page \pageref{reftypes})  or a qualified
 constructor (see page \pageref{refcons}) .
Its type will be same in either case, and if a constructor then its
value will be the value returned by the constructor.
\end{itemize}
 
If the stub is not followed by further segments, the final value and
type of the term is the value and type of the stub.
\subsection{Continuation evaluation}
 
Each segment of a term's continuation is evaluated from left to right,
according to the type of the evaluation of the term so far (the
\emph{continuation type}) and the syntax of the new segment:
\begin{itemize}
\item 
If the segment is a method call, then its value and type is the result
of invoking the matching method in the class defining the continuation
type (or a superclass or subclass of that class), as described in
 \emph{Methods and Constructors} (see page \pageref{refmethcon}) .
Note that method calls in term continuations cannot be constructors.
\item 
If the stub is an indexed reference, then it will refer to a property
in the class defining the continuation type (or a superclass of that
class).
That property will either be an undimensioned \nr{} string
(type \textbf{R\textsc{exx}}) or a dimensioned array.  In either case, it is
evaluated in the same way as an indexed reference found in the stub,
except that it is not necessarily in the current class, cannot
be an array constructor, and cannot result in a dimensioned type.
\item 
If the segment is a symbol, then it refers to either a property
or a method in the class defining the continuation type (or a superclass
of that class).
\footnote{
Unless parenthesis omission is disallowed by an implementation option,
such as \keyword{options strictargs}, in which case it can only be a
property.
}
 
The search for the property or method starts with the class defining the
continuation type.  If a property name matches, it is used; if not, a
method of the same name and having no arguments (or only optional
arguments) will match.
If neither property nor method is found, then the same search is applied
to each of the continuation class's superclasses in turn, starting with
the class that it extends.
 
\index{Length,of arrays}
\index{LENGTH,special word}
As a convenient special case, if the property or method is not found,
then if the segment is the final segment of the term and is the simple
symbol \textbf{length} and the continuation type is a
single-dimensioned array, then the segment evaluates to the size of the
array.
This will be a non-negative whole number of an appropriate primitive
type (or of type \textbf{R\textsc{exx}} if there is no appropriate
type).
\end{itemize}
 
The final value and type of the term is the value and type determined by
the evaluation of the last segment of the continuation.
\subsection{Arrays in terms}\label{refsarrayp}
\index{Arrays,in terms}
 
If a partially-evaluated term results in a dimensioned
 array (see page \pageref{refarray}) , its type is treated as type
\textbf{Object} so that evaluation of the term can continue.  For
example, in
\begin{lstlisting}
ca=char[] "tosh"
say ca.toString()
\end{lstlisting}
the variable \textbf{ca} is an array of characters; in the expression
on the second line the method \textbf{toString()} of the
class \textbf{Object} will be found.
\footnote{
\emph{In the reference implementation, this would return an identifier
for the object.}
}
 %chapter 8
\chapter{Methods and Constructors}\label{refmethcon}
\index{Methods,}
\index{Methods,invocation of}
\index{Parentheses,in method calls}
\index{Comma,in method calls}
\index{,}
 
Instructions in \nr{} are grouped into \emph{methods}, which are
named routines that always belong to (are part of) a \emph{class}.
 
\index{References,to methods}
Methods are invoked by being referenced in a  term (see page \pageref{refterms}) ,
which may be part of an expression or be a clause in its own right (a
method call instruction).
In either case, the syntax used for a method invocation is:
\begin{alltt}
\emph{symbol}([\emph{expression}[,\emph{expression}]...])
\end{alltt}
\index{Arguments,of methods}
\index{Arguments,passing to methods}
 The \emph{symbol}, which must be non-numeric, is called the
\emph{name} of the method.
It is important to note that the name of the method must be followed
immediately by the "\texttt{(}", with \textbf{no} blank in
between, or the construct will not be recognized as a method call
(a \emph{blank operator} would be assumed at that point instead).
 The \emph{expression}s (separated by commas) between the
parentheses are called the \emph{arguments} to the method.
Each argument expression may include further method calls.
 The argument expressions are evaluated in turn from left to right
and the resulting values are then passed to the method (the procedure
for locating the method is described below).
The method then executes some algorithm (usually dependent on any
arguments passed, though arguments are not mandatory) and will
eventually return a value.
This value is then included in the original expression just as though
the entire method reference had been replaced by the name of a variable
whose value is that returned data.
 
For example, the \textbf{substr} method is provided for strings of
type \textbf{R\textsc{exx}} and could be used as:
\begin{alltt}
c='abcdefghijk'
a=c.substr(3,7)
/* would set A to "cdefghi" */
\end{alltt}
Here, the value of the variable \textbf{c} is a string (of
type \textbf{R\textsc{exx}}).
The \textbf{substr} (substring) method of the \textbf{R\textsc{exx}} class is
then invoked, with arguments \textbf{3} and \textbf{7}, on the value
referred to by \textbf{c}.
That is, the the properties available to (the context of)
the \textbf{substr} method are the properties constructed from the
literal string \textbf{'abcdefghijk'}.
The method returns the substring of the value, starting at the third
character and of length seven characters.
\index{Arguments,passing to methods}
 
A method may have a variable number of arguments: only those
required need be specified.
For example, \textbf{'ABCDEF'.substr(4)} would return the
string \textbf{'DEF'}, as the \textbf{substr} method will assume
that the remainder of the string is to be returned if no length is
provided.
 
Method invocations that take no arguments may omit the (empty)
parentheses in circumstances where this would not be ambiguous.
See the section on  \emph{Terms} (see page \pageref{refterms})  for details.
 
\textbf{Implementation minimum:} At least 10 argument expressions
should be allowed in a method call.
\subsection{Method call instructions}\label{refmcalli}
\index{Method call instructions,}
\index{Subroutines,calling}
 
When a clause in a method consists of just a term, and the final part of
the term is a method invocation, the clause is a \emph{method call
instruction}:
\begin{shaded}
\begin{alltt}
\emph{symbol}([\emph{expression}[,\emph{expression}]...]);
\end{alltt}
\end{shaded}
The method is being called as a subroutine of the current method, and
any returned value is discarded.
In this case (and in this case only), the method invoked need not return
a value (that is, the \texttt{return} instruction that ends it need not
specify an expression).
\footnote{
A method call instruction is equivalent to the \texttt{call} instruction
of other languages, except that no keyword is required.
}
 
A method call instruction that is the first instruction in a constructor
(see below) can only invoke the special constructors \textbf{this}
and \textbf{super}.
\subsection{Method resolution (search order)}\label{refsmeth}
\index{Resolution of methods,}
\index{Search order,for methods}
\index{Methods,resolution of}
 
Method resolution in \nr{} proceeds as follows:
\begin{itemize}
\item 
If the method invocation is the first part (stub) of a term, then:
\begin{enumerate}
\item 
The current class is searched for the method (see below for details of
searching).
\item If not found in the current class, then the superclasses of the
current class are searched, starting with the class that the current
class extends.
\item 
If still not found, then the classes listed in the \texttt{uses} phrase
of the \texttt{class} instruction are searched for the method, which in
this case must be a  static method (see page \pageref{refstatmet}) .
Each class from the list is searched for the method, and then its
superclasses are searched upwards from the class; this process is
repeated for each of the classes, in the order specified in the list.
\item 
If still not found, the method invocation must be a constructor (see
below) and so the method name, which may be qualified by a package name,
should match the name of a primitive type or a known class (type).
The specified class is then searched for a constructor that matches the
method invocation.
\end{enumerate}
\item 
If the method invocation is not the first part of the term, then the
evaluation of the parts of the term to the left of the method invocation
will have resulted in a value (or just a type), which will have a known
type (the continuation type).
Then:
\begin{enumerate}
\item 
The class that defines the continuation type is searched for the method
(see below for details of searching).
\item If not found in that class, then the superclasses of that class are
searched, starting with the class that that class extends.
\end{enumerate}
 If the search did not find a method, an error is reported.
 
If the search did find a method, that is the method which is invoked,
except in one case:
\begin{itemize}
\item If the evaluation so far has resulted in a value (an object), then
that value may have a type which is a subclass of the continuation type.
If, within that subclass, there is a method that
 exactly overrides (see page \pageref{refoverrid})  the method that was found in the
search, then the method in the subclass is invoked.
\end{itemize}

This case occurs when an object is earlier assigned to a variable of a
type which is a superclass of the type of the object.  This type
simplification hides the real type of the object from the language
processor, though it can be determined when the program is executed.
\end{itemize}
 
\index{Methods,searching for}
\index{Matching methods,}
Searching for a method in a class proceeds as follows:
\begin{enumerate}
\item 
Candidate methods in the class are selected.
To be a candidate method:
\begin{itemize}
\item 
the method must have the same name as the method invocation (independent
of the  case (see page \pageref{refcase})  of the letters of the name)
\item 
the method must have the same number of arguments as the method
invocation (or more arguments, provided that the remainder are shown as
optional in the method definition)
\item 
it must be possible to assign the result of each argument expression to
the type of the corresponding argument in the method definition (if
strict type checking is in effect, the types must match exactly).
\end{itemize}
\item 
If there are no candidate methods then the search is complete; the
method was not found.
\item If there is just one candidate method, that method is used; the
search is complete.
\item 
If there is more than one candidate method, the sum of the
 costs of the conversions (see page \pageref{refcosts})  from the type of each
argument expression to the type of the corresponding argument defined
for the method is computed for each candidate method.
\item 
The costs of those candidates (if any) whose names match the method
invocation exactly, including in case, are compared; if one has a lower
cost than all others, that method is used and the search is complete.
\item 
The costs of all the candidates are compared; if one has a lower
cost than all others, that method is used and the search is complete.
\item 
If there remain two or more candidates with the same minimum cost, the
method invocation is ambiguous, and an error is reported.
\end{enumerate}
\textbf{Note: }When a method is found in a class, superclasses of that class are
not searched for methods, even though a lower-cost method may exist in a
superclass.
\subsection{Method overriding}\label{refoverrid}
\index{Methods,overriding}
\index{Overriding methods,}
 
A method is said to \emph{exactly override} a method in another class
if
\begin{enumerate}
\item the method in the other class has the same name as the current method
\item the method in the other class is not \texttt{private}
\item the other class is a superclass of the current class, or is a class
that the current class implements (or is a superclass of one of those
classes).
\item the number and type of the arguments of the method in the other
class exactly match the number and type of the arguments of the current
method (where subsets are also checked, if either method has optional
arguments).
\end{enumerate}
For example, the \textbf{R\textsc{exx}} class includes a
 \textbf{substr} (see page \pageref{refsubstr})  method, which takes from one to
three strings of type \textbf{R\textsc{exx}}.  In the class:
\begin{alltt}
class mystring extends R\textsc{exx}
  method substr(n=R\textsc{exx}, length=R\textsc{exx})
    return this.reverse.substr(n, length)

  method substr(n=int, length=int)
    return this.reverse.substr(R\textsc{exx} n, R\textsc{exx} length)
\end{alltt}
the first method exactly overrides the \textbf{substr} method in
the \textbf{R\textsc{exx}} class, but the second does not, because the types of
the arguments do not match.
 
A method that exactly overrides a method is assumed to be an extension
of the overridden method, to be used in the same way.  For such a
method, the following rules apply:
\begin{itemize}
\item It must return a value of the same type as the overridden method (or
none, if the overridden method returns none).
\item It must be at least as visible as the overridden routine.
For example, if the overridden routine is \texttt{public} then it must
also be \texttt{public}.
\item If the overridden method is \texttt{static} then it must also
be \texttt{static}.
\item If the overridden method is not \texttt{static} then it must
not be \texttt{static}.
\item If the underlying implementation checks  exceptions (see page \pageref{refexcep}) ,
only those checked exceptions that are signalled by the overridden
method may be left uncaught in the current method.
\end{itemize}
\subsection{Constructor methods}\label{refcons}
\index{Constructors,}
\index{,}
\index{Methods,constructor}
\index{Objects,constructing}
\index{Instance, of a class,}
\index{Class,instances of}
 
As described above, methods are usually invoked in the context of an
existing value or type.
A special kind of method, called a constructor method, is used to
actually create a value of a given type (an object).
 
Constructor methods always have the same short name as the class in
which they are found, and construct and return a value of the type
defined by that class (sometimes known as an \emph{instance} of that
class).
If the class is part of a package, then the constructor call may be
qualified by the package name.
 \textbf{Example constructors:}
\begin{alltt}
File('Dan.yr.Ogof')
java.io.File('Speleogroup.letter')
R\textsc{exx}('some words')
netrexx.lang.R\textsc{exx}(1)
\end{alltt}
 
\index{Constructors,default}
There will always be at least one constructor if values can be created
for a class.  \nr{} will add a default public constructor that takes
no arguments if no constructors are provided, unless the components of
the class are all static or constant, or the class is an interface
class.
 
All constructors follow the same rules as other methods, and in
addition:
\begin{enumerate}
\item Constructor calls always include parentheses in the syntax, even
if no arguments are supplied.  This distinguishes them from a reference
to the type of the same name.
\item Constructors must call a constructor of their superclass (the class
they extend) before they carry out any initialization of their own.
This is so any initialization carried out by the superclass takes
place, and at the appropriate moment.
Only after this call is complete can they make any reference to the
special words  \textbf{this} or \textbf{super} (see page \pageref{refspecial}) .
 
Therefore, the first instruction in a constructor must be either a call
to the superclass, using the special constructor \textbf{super()}
(with optional arguments), or a call to to another constructor in the
same class, using the special constructor \textbf{this()} (with
optional arguments).
In the latter case, eventually a constructor that explicitly
calls \textbf{super()} will be invoked and the chain of local
constructor calls ends.
 
As a convenience, \nr{} will add a default call to \textbf{super()},
with no arguments, if the first instruction in a constructor is not a
call to \textbf{this()} or \textbf{super()}.
\item 
The properties of a constructed value are initialized, in the order
given in the program, after the call to \textbf{super()} (whether
implicit or explicit).
\item 
By definition, constructors create a value (object) whose type is
defined by the current class, and then return that value for use.
Therefore, the \texttt{returns} keyword on the
 \texttt{method} instruction (see page \pageref{refmethod})  that introduces the
constructor is optional (if given, the type specified must be that of
the class).
Similarly, the only possible forms of the \texttt{return} instruction
used in a constructor are either "\textbf{return this;}", which
returns the value that has just been constructed, or just
"\textbf{return;}", in which case, the "\textbf{this}" is
assumed (this form will be assumed at the end of a method, as usual, if
necessary).
\end{enumerate}
 
\index{Example,of constructors}
Here is an example of a class with two constructors, showing the use
of \textbf{this()} and \textbf{super()}, and taking advantage of
some of the assumptions:
\begin{alltt}
class MyChars extends SomeClass

  properties private
    /* the data 'in' the object */
    value=char[]

  /* construct the object from a char array */
  method MyChars(array=char[])
    /* initialize superclass */
    super()
    value=array             -- save the value

  /* construct the object from a String */
  method MyChars(s=String)
    /* convert to char[] and use the above */
    this(s.toCharArray())
\end{alltt}
 
Objects of type \textbf{MyChars} could then be created thus:
\begin{alltt}
myvar=MyChars("From a string")
\end{alltt}

or by using an argument that has type \textbf{char[]}.
 %chapter 9
\chapter{Type conversions}\label{refconv}
\index{Types,conversions}
\index{Datatypes,}
\index{Data,conversions}
\index{Primitive types,conversions}
\index{Conversion,of types}
\index{Conversion,automatic}
 
As described in the section on  \emph{Types and (see page \pageref{reftypes}) 
classes}:ea., all values that are manipulated in \nr{} have an
associated type.  On occasion, a value involved in some operation may
have a different type than is needed, and in this case conversion of
a value from one type to another is necessary.
 
\nr{} considerably simplifies the task of programming, without losing
robustness, by making many such conversions automatic.  It will
automatically convert values providing that there is no loss of
information caused by the automatic conversion (or if there is, an
exception would be raised).
 
\index{,}
Conversions can also be made explicit by  concatenating (see page \pageref{reftypeops}) 
a type:ea. to a value and in this case less robust conversions
(sometimes known as \emph{casts}) may be effected.
See below for details of the permitted automatic and explicit
conversions.
 
Almost all conversions carry some risk of failure, or have a performance
cost, and so it is expected that implementations will provide options to
either report costly conversions or require that programmers make all
conversions explicit.
\footnote{
\emph{In the reference implementation, \texttt{options strictassign} may be
used to disallow automatic conversions.}
}
Such options might be recommended for certain critical programming
tasks, but are not necessary for general programming.
\subsection{Permitted automatic conversions}
\index{Conversion,automatic}
\index{Types,simplification}
 
In general, the semantics of a type is unknown, and so conversion (from
a \emph{source type} to a \emph{target type}) is only possible in
the following cases:
\begin{itemize}
\item The target type and the source type are identical (the trivial
case).
\item 
The target type is a superclass of the source type, or is an
interface class implemented by the source type.
This is called \emph{type simplification}, and in this case the value
is not altered, no information is lost, and an explicit conversion may
be used later to revert the value to its original type.
\item 
The source type has a dimension, and the target type
is \textbf{Object}.
\item 
The source type is \textbf{null} and the target type is not primitive.
\item 
\index{Conversion,of well-known types}
\index{Well-known conversions,}
The target and source types have known semantics (that is, they are
"well-known" to the implementation) and the conversion can be
effected without loss of information (other than knowledge of the
original type).
These are called \emph{well-known conversions}.
\end{itemize}
 
Necessarily, the well-known conversions are implementation-dependent, in
that they depend on the standard classes for the implementation and on
the primitive types supported (if any).
Equally, it is this automatic conversion between strings and the
primitive types of an implementation that offer the greatest
simplifications of \nr{} programming.
 
\index{R\textsc{exx},class/conversions}
For example, if the implementation supported an \textbf{int}
binary type (perhaps a 32-bit integer) then this can safely be
converted to a \nr{} string (of type \textbf{R\textsc{exx}}).
Hence a value of type \textbf{int} can be added to a number which is a
\nr{} string (resulting in a \nr{} string) without concern over the
difference in the types of the two terms used in the operation.
 
Conversely, converting a long integer to a shorter one without checking
for truncation of significant digits could cause a loss of information
and would not be permitted.
 
\index{Strings,types of}
\index{char,as a string}
\index{Binary numbers,}
\emph{In the reference implementation, the semantics of each of the
following types is known to the language processor (the first four are
all \emph{string} types, and the remainder are known as \emph{binary
number}s):}
\begin{itemize}
\item \emph{\textbf{netrexx.lang.R\textsc{exx}} - the \nr{} string class}
\item \emph{\textbf{java.lang.String} - the Java string class}
\item \emph{\textbf{char} - the Java primitive which represents a single
character}
\item \emph{\textbf{char[]} - an array of
\textbf{char}s}
\item \emph{\textbf{boolean} - a single-bit primitive}
\item \emph{\textbf{byte}, \textbf{short}, \textbf{int}, \textbf{long},
- signed integer primitives (8, 16, 32, or 64 bits)}
\item \emph{\textbf{float}, \textbf{double} - floating-point
primitives (32 or 64 bits)}
\end{itemize}
\emph{Under the rules described above, the following well-known
conversions are permitted:}
\begin{itemize}
\item \emph{\textbf{R\textsc{exx}} \emph{to} \textbf{binary number}, \textbf{char[]}, \textbf{String},
or \textbf{char}}
\item \emph{\textbf{String} \emph{to} \textbf{binary number}, \textbf{char[]}, \textbf{R\textsc{exx}},
or \textbf{char}}
\item \emph{\textbf{char} \emph{to} \textbf{binary number}, \textbf{char[]}, \textbf{String},
or \textbf{R\textsc{exx}}}
\item \emph{\textbf{char[]} \emph{to} \textbf{binary number}, \textbf{R\textsc{exx}}, \textbf{String},
or \textbf{char}}
\item \emph{\textbf{binary number} \emph{to} \textbf{R\textsc{exx}}, \textbf{String}, \textbf{char[]},
or \textbf{char}}
\item \emph{\textbf{binary number} \emph{to} \textbf{binary number} (if no loss of
information can take place - no sign, high order digits, decimal
part, or exponent information would be lost)}
\end{itemize}
 \emph{\textbf{Notes:}}
\begin{enumerate}
\index{Exceptions,during conversions}
\item \emph{Some of the conversions can cause a run-time error (exception), as
when a string represents a number that is too large for an \textbf{int}
and a conversion to \textbf{int} is attempted, or when a string that
does not contain exactly one character is converted to a
\textbf{char}.}
\item 
\emph{The \textbf{boolean} primitive is treated as a binary number that may
only take the values 0 or 1.
\index{boolean type, value of,}
A boolean may therefore be converted to
any binary number type, as well as any of the string
(or \textbf{char}) types, as the character "\textbf{0}" or
"\textbf{1}".
Similarly, any binary number or string can be converted to boolean (and
must have a value of 0 or 1 for the conversion to succeed).}
\item 
\emph{The \textbf{char} type is a single-character string (it is not a
number that represents the encoding of the character).}
\index{char,as a string}
\end{enumerate}
\subsection{Permitted explicit conversions}
\index{Conversion,explicit}
 
Explicit conversions are permitted for all permitted automatic
conversions and, in addition, when:
\begin{itemize}
\item 
The target type is a subclass of the source type, or implements
the source type.
This conversion will fail if the value being converted was not
originally of the target type (or a subclass of the target type).
\item 
Both the source and target types are primitive and (depending on the
implementation) the conversion may fail or truncate information.
\item 
The target type is \textbf{R\textsc{exx}} or a well-known string type (all
values have an explicit string representation).
\end{itemize}
\subsection{Costs of conversions}\label{refcosts}
\index{Conversion,cost of}
 
All conversions are considered to have a cost, and, for permitted
automatic conversions, these costs are used to select a method for
execution when several possibilities arise, using the algorithm
described in  \emph{Methods and Constructors} (see page \pageref{refsmeth}) .
 
For permitted automatic conversions, the cost of a conversion from a
\emph{source type} to a \emph{target type} will range from zero
through some arbitrary positive value, constrained by the following
rules:
\begin{itemize}
\item The cost is zero only if the source and target types are the same,
or if the source type is \textbf{null} and the target type is not
primitive.
\item 
Conversions from a given primitive source type to different primitive
target types should have different costs.
For example, conversion of an 8-bit number to a 64-bit number might cost
more than conversion of a 8-bit number to a 32-bit number.
\item 
Conversions that may require the creation of a new object cost more than
those that can never require the creation of a new object.
\item 
Conversions that may fail (raise an exception) cost more than those
that may require the creation of an object but can never fail.
\end{itemize}
 
Within these constraints, exact costs are arbitrary, and (because they
mostly involve implementation-dependent primitive types) are necessarily
implementation-dependent.
The intent is that the "best performance" method be selected when
there is a possibility of more than one.
 %chapter 10
\chapter{Expressions and Operators}\label{refexpr}
\index{Terms,}
\index{Data,terms}
\index{Substitution,in expressions}
Many clauses can include \emph{expressions}.  Expressions in \nr{}
are a general mechanism for combining one or more data items in various
ways to produce a result, usually different from the original data.
\index{Dyadic operators,}
\index{Prefix operators,}
\index{Monadic (prefix) operators,}
\index{Operators,}
Expressions consist of one or more  terms (see page \pageref{refterms}) , such as
literal strings, symbols, method calls, or sub-expressions,
and zero or more \emph{operators} that denote operations to be carried
out on terms.
Most operators act on two terms, and there will be at least one of these
\emph{dyadic} operators between every pair of terms.
\footnote{
One operator, direct concatenation, is implied if two terms
 abut (see page \pageref{refabut}) .
}
There are also \emph{prefix} (monadic) operators, that act on the
term that is immediately to the right of the operator.
There may be one or more prefix operators to the left of any term,
provided that adjacent prefix operators are different.
\index{Expressions,evaluation}
\index{Expressions,results of}
\index{Evaluation,of expressions}
\index{Parentheses,in expressions}

Evaluation of an expression is left to right, modified by parentheses
and by  operator precedence (see page \pageref{refpreced})  in the usual
"algebraic" manner.
Expressions are wholly evaluated, except when an error occurs during
evaluation.
\index{Results,size of}
\index{Data,length of}
\index{Types,of values}
\index{Types,of terms}
\index{Data,type checking}
\index{Datatypes,}

As each term is used in an expression, it is evaluated as
appropriate and its value (and the type of that value) are determined.
 
The result of any operation is also a value, which may be a character
string, a data object of some other type, or (in special circumstances)
a binary representation of a character or number.  The type of the
result is well-defined, and depends on the types of any terms involved
in an operation and the operation carried out.
Consequently, the result of evaluating any expression is a value which
has a known type.
 
Note that the \nr{} language imposes no restriction on the maximum
size of results, but there will usually be some practical limitation
dependent upon the amount of storage and other resources available
during execution.
\section{Operators}\label{refops}
\index{Operators,composition of}
The operators in \nr{} are constructed from one or more
operator characters (see page \pageref{refopers}).
Blanks (and comments) adjacent to operator characters have no effect on
the operator, and so the operators constructed from more than one
character may have embedded blanks and comments.
In addition, blank characters, where they occur between tokens within
expressions but are not adjacent to another operator, also act as an
operator.
 The operators may be subdivided into five groups: concatenation,
arithmetic, comparative, logical, and type operators.  The first four
groups work with terms whose type is "well-known" (that is,
strings, or known types that can be be converted to strings without
information loss).  The operations in these groups are defined in terms
of their operations on strings.
\subsection{Concatenation}
\index{Operators,concatenation}
\index{Concatenation,of strings}
\index{Blank,as concatenation operator}
\index{Blank,operator}
\index{Operators,blank}
\index{Strings,concatenation of}
\index{|| concatenation operator,}
The concatenation operators are used to combine two strings to form
one string by appending the second string to the right-hand end of the
first string.  The concatenation may occur with or without an
intervening blank:
\begin{table}\caption{Concatenation operators}\label{table:Concatenation operators}
\begin{tabularx}{\textwidth}{>{\bfseries}lX}
\toprule
(blank)&Concatenate terms with one blank in between.
\\\midrule
||&Concatenate without an intervening blank.
\\\midrule
(abuttal)\label{refabut}&Concatenate without an intervening blank.
\\\bottomrule
\end{tabularx}
\end{table}
\index{Abuttal concatenation operator,}
 Concatenation without a blank may be forced by using
the \textbf{||} operator, but it is useful to remember that
when two terms are adjacent and are not separated by an operator,
\footnote{
This can occur when the terms are syntactically distinct (such as a
literal string and a symbol).
}
they will be concatenated in the same way.
This is the \emph{abuttal} operation.
For example, if the variable \texttt{Total} had the value
\texttt{'37.4'}, then \texttt{Total'\%'} would evaluate to \texttt{'37.4\%'}.  
 Values that are not strings are first converted to strings before
concatenation. The concatenation operators are listed in table \ref{table:Concatenation operators}.
\subsection{Arithmetic}
\index{Operators,arithmetic}
\index{Numbers,arithmetic on}
\index{Arithmetic,operators}
\index{Addition;.pi ,Subtraction;.pi /Multiplication;.pi /Division}
\index{Negation,of numbers}
\index{Remainder operator;.pi ,Integer division;.pi /Exponentiation}
\index{Power operator,}
\index{+ plus sign,addition operator}
\index{- minus sign,subtraction operator}
\index{* multiplication operator,}
\index{,}
\index{,}
\index{,}
\index{,}
\index{Prefix operators,+}
\index{Prefix operators,-}
Character strings that are  numbers (see page \pageref{refnumdef}) 
may be combined using the arithmetic operators listed in table \ref{table:Arithmetic operators}.
\begin{table}\caption{Arithmetic operators}\label{table:Arithmetic operators}
\begin{tabularx}{\textwidth}{>{\bfseries}lX}
\toprule
+&Add
\\\midrule
-&Subtract
\\\midrule
*&Multiply
\\\midrule
/&Divide
\\\midrule
\%&Integer divide. Divide and return the integer part of the result.
\\\midrule
//&Remainder. Divide and return the remainder (this is not modulo, as the result
may be negative).
\\\midrule
**&Power. Raise a number to a whole number power.
\\\midrule
Prefix -&Same as the subtraction: "\textbf{0-}number".
\\\midrule
Prefix +&Same as the addition: "\textbf{0+}number".
\\\bottomrule
\end{tabularx}
\end{table}
The section on  \emph{Numbers and Arithmetic} (see page \pageref{refnums}) 
describes numeric precision, the format of valid numbers, and the
operation rules for arithmetic.  Note that if an arithmetic result is
shown in exponential notation, then it is likely that rounding has
occurred.
 
In  binary classes (see page \pageref{refbincla}) , the arithmetic operators will use
binary arithmetic if both terms involved have values which are binary
numbers.
The section on  \emph{Binary values and operations} (see page \pageref{refbinary}) 
describes binary arithmetic.
\subsection{Comparative}\label{refcomps}
\index{Operators,comparative}
\index{Comparative operators,}
\index{Comparison,of numbers}
\index{Comparison,of strings and numbers}
\index{Strict comparative operators,}
\index{Normal comparative operators,}
\index{Numbers,comparison of}
\index{Strings,comparison of}
\index{Equality,testing of}
\index{Inequality, testing of,}
The comparative operators compare two terms and return the
value \textbf{'1'} if the result of the comparison is true,
or \textbf{'0'} otherwise.  Two sets of operators are defined: the
\emph{strict} comparisons (listed in table \ref{table:Strict
  Comparators}) and the \emph{normal} comparisons (listed in table
\ref{table:Normal Comparators}).
 The strict comparative operators all have one of the characters
defining the operator doubled.  The "\textbf{==}",
and "\textbf{\textbackslash ==}" operators
test for strict equality or inequality between two strings.
Two strings must be identical to be considered strictly equal.
Similarly, the other strict comparative operators (such as
"\textbf{>{}>}" or "\textbf{<{}<}") carry out a simple
left-to-right character-by-character comparison, with no padding of
either of the strings being compared.
If one string is shorter than, and is a leading sub-string of, another
then it is smaller (less than) the other.
Strict comparison operations are case sensitive, and the exact collating
order may depend on the character set used for the implementation.
\footnote{
For example, in an ASCII or Unicode environment, the digits 0-9
are lower than the alphabetics, and lowercase alphabetics are higher
than uppercase alphabetics.
In an EBCDIC environment, lowercase alphabetics precede uppercase, but
the digits are higher than all the alphabetics.
}
 For all the other comparative operators, if \textbf{both} the
terms involved are numeric,
\footnote{
That is, if they can be compared numerically without error.
}
a numeric comparison (in which leading zeros are ignored,
\emph{etc.}) is effected; otherwise, both terms are treated as character
strings.
For this character string comparison, leading and trailing blanks are
ignored, and then the shorter string is padded with blanks on the right.
The character comparison operation takes place from left to right, and
is \textbf{not} case sensitive (that is, "\textbf{Yes}"
compares equal to "\textbf{yes}").
As for strict comparisons, the exact collating order may depend on the
character set used for the implementation.

%  The comparative operators return true (\textbf{'1'}) if the terms
% are:
\begin{table}\caption{\textbf{Normal} comparative
    operators}\label{table:Normal Comparators}
\begin{tabularx}{\textwidth}{>{\bfseries}lX}
\toprule
=&Equal (numerically or when padded, \emph{etc.}).
\\\midrule
\textbackslash =&Not equal (inverse of =).
\\\midrule
>&Greater than.
\\\midrule
<&Less than.
\\\midrule
><, <>&Greater than or less than (same as "Not equal").
\\\midrule
>=, \textbackslash <&Greater than or equal to, not less than.
\\\midrule
<=, \textbackslash >&Less than or equal to, not greater than.
\\\bottomrule
\end{tabularx}
\end{table}
\index{= equals sign,equal operator}
\begin{table}\caption{\textbf{Strict} comparative  operators}\label{table:Strict Comparators}
\begin{tabularx}{\textwidth}{>{\bfseries}lX}
\toprule
==&Strictly equal (identical).
\\\midrule
\textbackslash ==&Strictly not equal (inverse of \textbf{==}).
\\\midrule
>{}>&Strictly greater than.
\\\midrule
<{}<&Strictly less than.
\\\midrule
>{}>=, \textbackslash <{}<&Strictly greater than or equal to, strictly not
less than.
\\\midrule
<{}<=, \textbackslash >{}>&Strictly less than or equal to, strictly not
greater than.
\\\bottomrule
\end{tabularx}
\end{table}
\index{Objects,comparing}
\index{Objects,equality}
\index{Equality,of objects}
\index{= equals sign,equal operator}
\index{\textbackslash = not equal operator,}
The equal and not equal operators ("\textbf{=}",
"\textbf{==}", "\textbf{\textbackslash =}", and
"\textbf{\textbackslash ==}") may be used to compare two objects which
are not strings for equality, if the implementation allows them to be
compared (usually they will need to be of the same type).  The strict
operators test whether the two objects are in fact the same object,
\footnote{
Note that two distinct objects compared in this way may contain values
(properties) that are identical, yet they will not compare equal as they
are not the same object.
}
and the normal operators may provide a more relaxed comparison, if
available to the implementation.
\footnote{
\emph{In the reference implementation, the \textbf{equals} method is
used for normal comparisons.  Where not provided by a type, this is
implemented by the Object class as a strict comparison.}
}
 
In  binary classes (see page \pageref{refbincla}) , all the comparative operators
will use binary arithmetic to effect the comparison if both terms
involved have values which are binary numbers.
In this case, there is no distinction between the strict and the normal
comparative operators.
The section on  \emph{Binary values and operations} (see page \pageref{refbinary}) 
describes the binary arithmetic used for comparisons.
\subsection{Boolean}
\index{Operators,logical}
\index{Boolean operations,}
\index{AND,logical operator}
\index{OR,logical inclusive}
\index{OR,logical exclusive}
\index{XOR, logical operator,}
\index{Exclusive OR,logical operator}
\index{,}
\index{NOT operator,}
\index{Negation,of logical values}
\index{Logical operations,}
\index{\& and operator,}
\index{| or operator,}
\index{\&\& exclusive or operator,}
\index{\textbackslash  backslash,not operator}
\index{Backslash character,not operator}
\index{Prefix operators,\textbackslash }
\index{True value,}
\index{False value,}
A character string is taken to have the value "false" if it
is \textbf{'0'}, and "true" if it is \textbf{'1'}.
The logical operators take one or two such values (values other
than \textbf{'0'} or \textbf{'1'} are not allowed) and
return \textbf{'0'} or \textbf{'1'} as appropriate. The Boolean
operators are listed in table \ref{table:Boolean Comparators}.
\begin{table}\caption{Boolean operators}\label{table:Boolean Comparators}
\begin{tabularx}{\textwidth}{>{\bfseries}lX}
\toprule
\&&And. Returns 1 if both terms are true.
\\\midrule
|&Inclusive or. Returns 1 if either term is true.
\\\midrule
\&\&&Exclusive or. Returns 1 if either (but not both) is true.
\\\midrule
Prefix \textbackslash &Logical not. Negates; 1 becomes 0 and
\emph{vice versa}.
\\\bottomrule
\end{tabularx}
\end{table}
 \index{Bits,binary operators}
In  binary classes (see page \pageref{refbincla}) , the logical operators will act on
all bits in the values if both terms involved have values which are
boolean or integers.
The section on  \emph{Binary values and operations (see page
  \pageref{refbinary})} describes this in more detail.
\subsection{Type}\label{reftypeops}
\index{Operators,type}
\index{Types,operations on}
\index{Types,checking instances of}
Several of the operators already described can be used to carry out
operations related to types.  Specifically:
\begin{itemize}
\index{Concatenation,of types}
\index{Casting,to type}
\index{Blank,as type conversion operator}
\index{Blank,operator}
\index{Operators,blank}
\index{Types,concatenation of}
\index{|| concatenation operator,}
\index{Abuttal concatenation operator,}
\item Any of the concatenation operators may be used for \emph{type
concatenation}, which concatenates a type to a value.  All three
operators (blank, "\textbf{||}", and abuttal) have the same
effect, which is to  convert (see page \pageref{refconv}) 
\footnote{
This is sometimes known as \emph{casting}
}
the value to the type
specified (if the conversion is not possible, an error is reported or an
exception is signalled).
The type must be on the left-hand side of the operator.
 \textbf{Examples:}
\begin{alltt}
String "abc"
int (a+1)
long 1
Exception e
InputStream myfile
\end{alltt}
\item 
\index{Prefix operators,+/with types}
\index{Prefix operators,-/with types}
\index{Prefix operators,\textbackslash /with types}
A type on the left hand side of an operator that could be a prefix
(\texttt{+,-} or \texttt{\textbackslash})
type concatenation after the prefix operator is applied to the
right-hand operand, as though an explicit concatenation operator were
placed before the prefix operator.

For example:
\begin{alltt}
x=int -y
\end{alltt}
means that \texttt{-y} is evaluated, and then the result is
converted to \texttt{int} before being assigned to \texttt{x}.
\footnote{
This could also have been written \texttt{x=int (-y)}.
}
The "less than or equal" and the "greater than or equal"
operators ("\textbf{<=}" and "\textbf{>=}") may be used
with a type on either side of the operator, or on both sides.
In this case, they test whether a value or type is a subclass of, or is
the same as, a type, or vice versa.
 \textbf{Examples:}
\begin{alltt}
if i<=Object then say 'I is an Object'
if String>=i then say 'I is a String'
if String<=Object then say 'String is an Object'
\end{alltt}
\end{itemize}
The precedence of these operators is not affected by their being
used with types as operands.
\section{Numbers}\label{refnumdef}
\index{Numbers,}
\index{Exponential notation,}
\index{E-notation,}
\index{Powers of ten in numbers,}
 The arithmetic operators above require that both terms involved be
numbers; similarly some of the comparative operators carry out a
numeric comparison if both terms are numbers.
 Numbers are introduced and defined in detail in the section
on  \emph{Numbers and arithmetic} (see page \pageref{refnums}) .
In summary, \emph{numbers} are character strings consisting of one or
more decimal digits optionally prefixed by a plus or minus sign, and
optionally including a single period ("\textbf{.}") which then
represents a decimal point.
A number may also have a power of ten suffixed in conventional
exponential notation: an "\textbf{E}" (uppercase or lowercase)
followed by a plus or minus sign then followed by one or more decimal
digits defining the power of ten.
 Numbers may have leading blanks (before and/or after the sign, if
any) and may have trailing blanks.
Blanks may not be embedded among the digits of a number or in the
exponential part.
 \textbf{Examples:}
\index{Numbers,examples of}
\begin{alltt}
'12'
'-17.9'
'127.0650'
'73e+128'
' + 7.9E-5 '
'00E+000'
\end{alltt}
 Note that the sequence \textbf{-17.9} (without quotes) in an
expression is not simply a number.
It is a minus operator (which may be prefix minus if there is no term
to the left of it) followed by a positive number.
The result of the operation will be a number.
\index{Whole numbers,}
 A  \emph{whole number} (see page \pageref{refwholed})  in \nr{} is a number
that has a zero (or no) decimal part.
 
\textbf{Implementation minimum:}
All implementations must support 9-digit arithmetic.
In unavoidable cases this may be limited to integers only, and in this
case the divide operator ("\textbf{/}") must not be supported.
If exponents are supported in an implementation, then they must be
supported for exponents whose absolute value is at least as large as the
largest number that can be expressed as an exact integer in default
precision, \emph{i.e.}, 999999999.
\section{Parentheses and operator precedence}\label{refpreced}
\index{Operators,precedence (priorities) of}
\index{Parentheses,in expressions}
\index{Precedence of operators,}
\index{Priorities of operators,}
\index{Algebraic precedence,}
 Expression evaluation is from left to right; this is modified by
parentheses and by operator precedence:
\begin{itemize}
\item 
When parentheses are encountered, other than those that identify
 method calls (see page \pageref{refmethcon}) , the entire \emph{sub-expression}
delimited by the parentheses is evaluated immediately when the term is
required.
\item When the sequence
\begin{alltt}
term\textsubscript{1} operator\textsubscript{1} term\textsubscript{2} operator\textsubscript{2} term\textsubscript{3}
\end{alltt}
is encountered, and \textbf{operator\textsubscript{2}} has a higher
precedence than \textbf{operator\textsubscript{1}}, then the
operation \textbf{(term\textsubscript{2} operator\textsubscript{2} term\textsubscript{3})} is evaluated
first.
The same rule is applied repeatedly as necessary.
 Note, however, that individual terms are evaluated from left to
right in the expression (that is, as soon as they are encountered).  It
is only the order of \textbf{operations} that is affected by the
precedence rules.
\end{itemize}
For example, "\textbf{*}" (multiply) has a higher precedence
than "\textbf{+}" (add), so \textbf{3+2*5} will evaluate to
13 (rather than the 25 that would result if strict left to right
evaluation occurred).
To force the addition to be performed before the multiplication the
expression would be written \textbf{(3+2)*5}, where the first three
tokens have been formed into a sub-expression by the addition of
parentheses.
 The order of precedence of the operators is (highest at the top) is
 listed in table \ref{table:Operator Precedence}.
\begin{table}\caption{Operator precedence}\label{table:Operator Precedence}
\begin{tabularx}{\textwidth}{>{\bfseries}lX}
\toprule
Prefix operators&\textbf{+  -  \textbackslash }
\\\midrule
Power operator&\textbf{**}
\\\midrule
Multiplication and division&\textbf{* and /}
\\\midrule
Addition and subtraction&\textbf{+  -}
\\\midrule
Concatenation&\textbf{(blank)  ||  (abuttal)}
\\\midrule
Comparative operators&\textbf{=  ==  >  <  <= >=  <{}<  \textbackslash
  >{}> } \emph{etc.}
\\\midrule
And&\textbf{\&}
\\\midrule
Or, exclusive or&\textbf{|  \&\&}
\\\bottomrule
\end{tabularx}
\end{table}

\index{Expressions,examples}
 If, for example, the symbol \textbf{a} is a variable whose value
is \textbf{'3'}, and \textbf{day} is a variable with the
value \textbf{'Monday'}, then:
\begin{lstlisting}
a+5              ==  '8'
a-4*2            ==  '-5'
a/2              ==  '1.5'
a\%2              ==  '1'
0.5**2           ==  '0.25'
(a+1)>7          ==  '0'            /* that is, False */
' '=''           ==  '1'            /* that is, True  */
' '==''          ==  '0'            /* that is, False */
' '\ ==''        ==  '1'            /* that is, True  */
(a+1)*3=12       ==  '1'            /* that is, True  */
'077'>'11'       ==  '1'            /* that is, True  */
'077'>>'11'      ==  '0'            /* that is, False */
'abc'>>'ab'      ==  '1'            /* that is, True  */
'If it is' day   ==  'If it is Monday'
day.substr(2,3)  ==  'ond'
'!'day'!'        ==  '!Monday!'
\end{lstlisting}
\begin{shaded}
\textbf{Note: }The \nr{} order of precedence usually causes no difficulty, as it
is the same as in conventional algebra and other computer languages.
There are two differences from some common notations; the prefix minus
operator always has a higher priority than the power operator, and
power operators (like other operators) are evaluated left-to-right.
Thus
\begin{alltt}
-3**2      ==  9    /* not -9  */
-(2+1)**2  ==  9    /* not -9  */
2**2**3    ==  64   /* not 256 */
\end{alltt}
 \end{shaded}
These rules were found to match the expectations of the majority of
users when the R\textsc{exx} language was first designed, and \nr{} follows the
same rules.
 %chapter 11
\section{Clauses and Instructions}\label{refclause}
  Clauses (see page \pageref{refclau})  are recognized, and can usefully be
classified, in the following order:
\begin{description}
\item[Null clauses]\label{refnullcl}
\index{Clauses,null}
\index{Null clauses,}

A clause that is empty or comprises only blanks, comments, and
continuations is a \emph{null clause} and is completely ignored by
\nr{} (except that if it includes a comment it will be traced, if
reached during execution).
\begin{shaded}
\textbf{Note: }A null clause is not an instruction, so (for example) putting an
extra semicolon after the \keyword{then} or \keyword{else} in an
\keyword{if} instruction is not equivalent to putting a dummy instruction
(as it would be in C or PL/I).
The \keyword{nop} instruction is provided for this purpose.
\end{shaded}
\item[Assignments]
\index{Instructions,assignment}
\index{Assignment,}
\index{Assignment,instruction}

Single clauses within a class and of the form
\emph{term}\textbf{=}\emph{expression}\textbf{;} are
instructions known as  \emph{assignment}s (see page \pageref{refassign}) .
An assignment gives a variable, identified by the
\emph{term}, a type or a new value.
 
In just one context, where property assignments are expected (before the
first method in a class), the "\textbf{=}" and the expression may
be omitted; in this case, the term (and hence the entire clause) will
always be a simple non-numeric symbol which names the property
\item[Method call instructions]\label{refxmeth}
\index{Method call instructions,}
\index{Instructions,method call}

A  method call instruction (see page \pageref{refmcalli})  is a clause within a
method that comprises a single term that is, or ends in, a method
invocation.
\item[Keyword instructions]\label{refkwcl}
\index{Keywords,}
\index{Keyword instructions,}
\index{Instructions,keyword}

A \emph{keyword instruction} consists of one or more clauses,
the first of which starts with a non-numeric symbol which is not the
name of a variable or property in the current class (if any) and is
immediately followed by a blank, a semicolon (which may be implied by
the end of a line), a literal string, or an operator (other than
"\textbf{=}", which would imply an assignment).
This symbol, the \emph{keyword}, identifies the instruction.
 
Keyword instructions control the external interfaces, the flow of
control, and so on.
Some  keyword instructions (see page \pageref{refkinst})  (\keyword{do}, \keyword{if},
\keyword{loop}, or \keyword{select}) can include nested instructions.
\end{description}

\chapter{Assignments and Variables}\label{refassign}
\index{Instructions,assignment}
\index{Assignment,}
\index{,}
\index{= equals sign,assignment indicator}
\index{Variables,setting new value}
\index{Variables,valid names}
\index{Variables,}
\index{Symbols,use of}
 A \emph{variable} is a named item whose value may be changed
during the course of execution of a \nr{} program.
The process of changing the value of a variable is called
\emph{assigning} a new value to it.
 
Each variable has an associated type, which cannot change during the
execution of a program; therefore, the values assigned to a given
variable must always have a type that can safely be assigned to that
variable.
\index{Symbols,assigning values to}
\index{Assignment,instruction}
\index{= equals sign,assignment indicator}
\index{Variables,setting new value}
\index{Names,of variables}
\index{Variables,valid names}
\index{Variables,names of}
 Variables may be assigned a new value by the \texttt{method} or
\texttt{parse} instructions, but the most common way of changing the
value of a variable is by using an \emph{assignment instruction}.
Any clause within a class and of the form:
\index{Assignment,}
\begin{shaded}
\begin{alltt}
\emph{assignment};

:hp0.where \emph{assignment} is::ehp0.

    \emph{term}=\emph{expression}
\end{alltt}
\end{shaded}
is taken to be an assignment instruction.
The result of the \emph{expression} becomes the new value of the
variable named by the \emph{term} to the left of the equals sign.
When the term is simply a symbol, this is called the \emph{name} of
the variable.
 \textbf{Example:}
\begin{alltt}
/* Next line gives FRED the value 'Frederic' */
fred='Frederic'
\end{alltt}
The symbol naming the variable cannot begin with a digit (0-9).
\footnote{
Without this restriction on the first character of a variable name,
it would be possible to redefine a number, in that for example the
assignment "\textbf{3=4;}" would give a variable called
"\textbf{3}" the value \textbf{'4'}.
}
 
Within a \nr{} program, variable names are not case-sensitive (for
example, the names \textbf{fred}, \textbf{Fred}, and \textbf{FRED}
refer to the same variable).
Where public names are exposed (for example, the names of properties,
classes, and methods, and in cross-reference listings) the case used for
the name will be that used when the name was first introduced
("first" is determined statically by position in a program rather
than dynamically).
 
\index{Variables,type of}
Similarly, the type of a \nr{} variable is determined by the type of
the value of the expression that is first assigned to it.
\footnote{
Since \nr{} infers the type of a variable from usage, substantial
programs can be written without introducing explicit type
declarations, although these are allowed.
}
For subsequent assignments, it is an error to assign a value to a
variable with a type mismatch unless the language processor can
determine that the value can be assigned safely to the type of the
variable.
 
In practice, this means that the types must match exactly, be a
simplification, or both be "well-known" types such
as \textbf{R\textsc{exx}}, \textbf{String}, \textbf{int}, \&, for which
safe conversions are defined.  The possibilities are described in the
section on  \emph{Conversions} (see page \pageref{refconv}) .
\footnote{
Implementations may provide for a stricter rule for assignment (where
the types must be identical), controlled by the \texttt{options}
instruction.
}
 
For example, if there are types (classes)
called \textbf{ibm.util.hex}, \textbf{RunKnown},
and \textbf{Window}, then:
\begin{alltt}
hexy=ibm.util.hex(3) -- 'hexy' has type 'ibm.util.hex'
rk=RunKnown()        -- 'rk' has type 'RunKnown'
fred=Window(10, 20)  -- 'fred' has type 'Window'
s="Los Lagos"        -- 's' has type 'R\textsc{exx}'
j=5                  -- 'j' has type 'R\textsc{exx}'
\end{alltt}
 
The first three examples invoke the \emph{constructor} method for the
type to construct a value (an object).  A constructor method always has
the same name as the class to which it belongs, and returns a new value
of that type.  Constructor methods are described in detail in
 \emph{Methods and Constructors} (see page \pageref{refmethcon}) .
 
The last two examples above illustrate that, by default, the types of
literal strings and numbers are \nr{} strings (type \textbf{R\textsc{exx}})
and so variables tend to be of type \textbf{R\textsc{exx}}.
This simplifies the language and makes it easy to learn, as many useful
programs can be written solely using the powerful \textbf{R\textsc{exx}} type.
Potentially more efficient (though less human-oriented) primitive
or built-in types for literals will be used in  binary (see page \pageref{refbincla}) 
classes:ea..
 \emph{If the examples above were in a binary class, then, in the
reference implementation, the types of \textbf{s} and \textbf{j}
would have been \textbf{java.lang.String} and \textbf{int}
respectively.
}
 
\index{Declarations,of variables}
\index{Types,declaring}
A variable may be introduced ("declared") without giving it an
initial value by simply assigning a type to it:
\begin{alltt}
i=int
r=R\textsc{exx}
f=java.io.File
\end{alltt}
Here, the expression to the right of the "\textbf{=}" simply
evaluates to a type with no value.
\subsubsection{The use and scope of variables}
 
\nr{} variables all follow the same rules of assignment, but are used
in different contexts.  These are:
\begin{description}
\item[Properties]\label{refprops}
\index{Properties,}
\index{Variables,properties}

Variables which name the values (the data) owned by an object of the
type defined by the class are called \emph{properties}.
When an object is constructed by the class, its properties are created
and are initialized to either a default value
(\textbf{null} or, for variables of primitive type, an
implementation-defined value, typically 0)
or to a value provided by the programmer.
 
The attributes of properties can be changed by the
 \texttt{properties} instruction (see page \pageref{refprop}) .
For example, properties may also be \emph{constant}, which means
that they are initialized when the class is first loaded and do not
change thereafter.
\item{Method arguments}
\index{Method,argument variables}
\index{Variables,method arguments}

When a method is invoked, arguments may be passed to it.
These \emph{method arguments} are assigned to the variables named on
the  \texttt{method} instruction (see page \pageref{refmethod})  that introduces the
method.
\item{Local variables}
\index{Local variables,}
\index{Variables,local}

Variables that are known only within a method are called \emph{local
variables}; each time a method is invoked a distinct set of local
variables is available.
Local variables are normally given an initial value by the programmer.
If they are not, they are initialized to a default value
(\textbf{null} or, for variables of primitive type, an
implementation-defined value, typically 0).
\end{description}
 
\index{Variables,scope of}
\index{Variables,visibility}
\index{Variables,static typing of}
\index{Static variable typing,}
In order for types to be determined and type-checking to be possible at
"compile-time", and easily determined by inspection, the use and
type of every variable is determined by its position in the program, not
by the order in which assignments are executed.
That is, variable typing is static.
 
The visibility of a variable depends on its use.  Properties are
visible to all methods in a class; method arguments and local variables
are only visible within the method in which they appear.  In particular:
\begin{itemize}
\item Within a class, properties have unique names (they cannot be
overridden by method arguments or by local variables within
methods); this avoids error-prone ambiguity.
\item 
Within a method, a method argument acts like a local variable (that is,
it is in the same name-space as local variables, and can be assigned new
values); it can be considered to be a local variable that is assigned a
value just before the body of the method is executed.  There cannot be
both a method argument and a local variable in a method with the same
name.
\item 
Within methods, variables can take only one type, the type assigned to
them when first encountered in the method (in a strict "physical"
sense, that is, as parsed from top to bottom of the program and from
left to right on each line).
Since methods tend to be small, there is no local scoping of variables
inside the constructs within a method.
\footnote{
Unlike the block scoping of PL/I, C, or Java.
}
 
Thus, in this example:
\begin{alltt}
method iszero(x)
  if x=0 then qualifier='is zero'
         else qualifier='is not zero'
  say 'The argument' qualifier'.'
\end{alltt}
the variable \textbf{qualifier} is known throughout the method and
hence has a known type and value when the \texttt{say} instruction is
executed.
\end{itemize}
 
To summarize: a symbol that names a variable in the current class either
refers to a property (and in any use of it within the class refers to
that property), or it refers to a variable that is unique within a
method (and any use of the name within that method refers to the same
variable).
\textbf{Note: }
A variable is just a name, or "handle" for a value.
It is possible for more than one variable to refer to the same value, as
in the program:
\begin{alltt}
first='A string'
second=first
\end{alltt}
Here, both variables refer to the same value.  If that value is
changeable then a change to the value referred to by one of the
variable names would also be seen if the value is referred to by the
other.
For example, sub-values of a \nr{} string can be changed, using
 \emph{Indexed references} (see page \pageref{refinstr}) , so a change to a
sub-value of \textbf{first} would also be seen in an identical indexed
reference to \textbf{second}.
\subsubsection{Terms on the left of assignments}
\index{Terms,in assignments}
\index{Terms,on left of =}
 
In an assignment instruction, the \emph{term} to the left of the
equals sign is most commonly a simple non-numeric symbol, which
always names a variable in the current class.
The other possibilities, as seen in the example below, are:
\begin{enumerate}
\item 
The term is an  \emph{indexed reference} (see page \pageref{refinstr}) , to an
existing variable that refers to a string of type \textbf{R\textsc{exx}} or an
 array (see page \pageref{refarray}) .
The variable may be in the current class, or be a property in a class
named in the \texttt{uses} phrase of the \texttt{class} instruction for
the current class.
\item 
The term is a  compound term (see page \pageref{refcomterm})  that ultimately refers
to a property (see above) in some class (which may be the current class).
This property cannot be a constant.
\end{enumerate}
 \textbf{Examples:}
\begin{alltt}
r=R\textsc{exx} ''
r['foo']='?'         -- indexed string assignment
s=String[3]
s[0]='test'          -- array assignment
Sample.value=1       -- property assignment
this.value=1         -- property assignment
super.value=1        -- property assignment
\end{alltt}

The last two examples show assignments to a property in the current
class or in a superclass of the current class, respectively.  Note that
references to properties in other classes must alway be qualified in
some way (for example, by the prefix \textbf{super.}).  The use of the
prefix \textbf{this.} for properties in the current class is optional.

\chapter{Indexed strings and Arrays}\label{refinstr}
\index{Indexed references,indexed strings}
\index{Indexed strings,}
\index{Strings,indexed}
\index{Variables,indexed}
\index{Variables,subscripts}
\index{Sub-values, of strings,}
\index{Strings,sub-values of}
\index{Index strings,for sub-values}
\index{References,to indexed strings}
\index{Brackets,in indexed strings}
\index{Comma,in indexed strings}
 
Any \nr{} string (that is, a value of type \textbf{R\textsc{exx}}), has the
ability to have \emph{sub-values}, values (also of type \textbf{R\textsc{exx}})
which are associated with the original string and are indexed by an
\emph{index string} which identifies the sub-value.
Any string with such sub-values is known as an \emph{indexed string}.
 
The sub-values of a \nr{} string are accessed using \emph{indexed
references}, where the name of a variable of type \textbf{R\textsc{exx}} is
followed immediately by square brackets enclosing one or more
expressions separated by commas:
\footnote{
The notations \keyword{'['} and \keyword{']'}
indicate square brackets appearing in the \nr{} program.
}
\begin{shaded}
\begin{alltt}
\emph{symbol}'['[\emph{expression}[, \emph{expression}]...]']'
\end{alltt}
\end{shaded}
It is important to note that the \emph{symbol} that names the
variable must be followed immediately by the "\keyword{[}",
with \textbf{no} blank in between, or the construct will not be
recognized as an indexed reference.
 The \emph{expression}s (separated by commas) between the
brackets are called the \emph{indexes} to the string.
These index expressions are evaluated in turn from left to right, and
each must evaluate to a value is of type \textbf{R\textsc{exx}} or that can be
converted to type \textbf{R\textsc{exx}}.
 
The resulting index strings are taken "as-is" - that is, they
must match exactly in content, case, and length for a reference to find
a previously-set item.
They may have any length (including the null string) and value (they are
not constrained to be just those strings which are numbers, for
example).
 
If a reference does not find a sub-value, then a copy of the non-indexed
value of the variable is used.
 \textbf{Example:}
\begin{lstlisting}
surname='Unknown'         -- default value
surname['Fred']='Bloggs'
surname['Davy']='Jones'
try='Fred'
say surname[try] surname['Bert']
\end{lstlisting}
would say "\textbf{Bloggs Unknown}".
 
When multiple indexes are used, they indicate accessing a hierarchy of
strings.  A single \nr{} string has a single set of indexes and
subvalues associated with it.  The sub-values, however, are also \nr{}
strings, and so may in turn have indexes and sub-values.  When more than
one index is specified in an indexed reference, the indexes are applied
in turn from left to right to each retrieved sub-value.
 
For example, in the sequence:
\begin{lstlisting}
x='?'
x['foo', 'bar']='OK'
say x['foo', 'bar']
y=x['foo']
say y['bar']
\end{lstlisting}
both \keyword{say} instructions would display the string
"\textbf{OK}".
 Indexed strings may be used to set up "associative arrays", or
dictionaries, in which the subscript is not necessarily numeric, and
thus offer great scope for the creative programmer.
A useful application is to set up a variable in which the subscripts
are taken from the value of one or more variables, so effecting a form
of associative (content addressable) memory.
%% The  \emph{justone} program (see page \pageref{refjust1})  is an example of this
%% technique.
 \textbf{Notes:}
\begin{enumerate}
\item 
A variable of type \textbf{R\textsc{exx}} must have been assigned a value
before indexing is used on it.
This is the value that is used as the default value whenever an indexed
reference finds no sub-value.
\item 
The indexes, and hence the sub-values, of a \textbf{R\textsc{exx}} object can
be retrieved in turn using the  \keyword{over} (see page \pageref{refloopov})  keyword
of the \keyword{loop} instruction.
\item 
The  \textbf{exists} method (see page \pageref{refexists})  of the \textbf{R\textsc{exx}}
class may be used to test whether an indexed reference has an
explicitly-set value.
\item 
Assigning \textbf{null} to an indexed reference (for example, the
assignment \\ \textbf{switch[7]=null;}) drops the sub-value;
until set to a new value, any reference to the sub-value (including use
of the \textbf{exists} method) will return the same result as
when it had never been set.
\end{enumerate}
\section{Arrays}\label{refarray}
\index{Arrays,}
\index{Indexed references,arrays}
\index{Brackets,in array references}
\index{Comma,in array references}
 
In addition to indexed strings, \nr{} also includes the concept of
fixed-size \emph{arrays}, which may be used for indexing values of any
type (including strings).
 
Arrays are used with the same syntax and in the same manner as
indexed strings, but with important differences that allow for
compact implementations and access to equivalent data structures
constructed using other programming languages:
\begin{enumerate}
\item 
The indexes for arrays must be whole numbers that are zero or positive.
There will usually be an implementation restriction on the maximum value
of the index (typically 999999999 or higher).
\item 
The elements of an array are considered to be \emph{ordered}; the
first element has index \textbf{0}, the second \textbf{1}, and so on.
\item 
An array is of fixed size;
\index{Fixed size, of arrays,}
it must be constructed before use.
\item 
Variables that are assigned arrays can only be assigned arrays (of the
same dimension, see below) in the future.  That is, being an array
changes the type of a value; it becomes a
 \emph{dimensioned type} (see page \pageref{refdimtype}) .
\end{enumerate}
 
\index{Arrays,references}
\index{References,to arrays}
\index{Arrays,constructors}
\index{Constructors,array}
Array references use the \nr{} \emph{indexed reference} syntax
defined above.  The same syntax is used for constructing arrays, except
that the symbol before the left bracket describes a type (and hence may
be qualified by a package name).  The expression or expressions between
the brackets indicate the size of the array in each dimension, and must
be a positive whole number or zero:
\begin{lstlisting}
arg=String[4]       -- makes an array for four Strings
arg=java.io.File[4] -- makes an array for four Files
i=int[3]            -- makes an array for three 'int's
\end{lstlisting}

(Another way of describing this is that array constructors look
just like other object constructors, except that brackets are
used instead of parentheses.)
 
Once an array has been constructed, its elements can be referred to
using brackets and expressions, as before:
\begin{lstlisting}
i[2]=3 -- sets the '2'-indexed value of 'i'
j=i[2] -- sets 'j' to the '2'-indexed value of 'i'
\end{lstlisting}
 
Regular multiple-dimensioned arrays may be constructed and referenced by
using multiple expressions within the brackets:
\begin{lstlisting}
i=int[2,3] -- makes a 2x3 array of 'int' type objects
i[1,2]=3   -- sets the '1,2'-indexed value of 'i'
j=i[1,2]   -- sets 'j' to the '1,2'-indexed value of 'i'
\end{lstlisting}
 As with indexed strings, when multiple indexes are used, they
indicate accessing a hierarchy of arrays (the underlying model is
therefore of a hierarchy of single-dimensioned arrays).
When more than one index is specified in an indexed reference to an
array, the indexes are applied in turn from left to right to each
array.
 
As described in the section on  \emph{Types} (see page \pageref{reftypes}) , the
type of a variable that refers to an array can be set (declared) by
assignment of the type with array notation that indicates the dimension
of an array without any sizes:
\begin{lstlisting}
k=int[]     -- one-dimensional array of 'int' objects
m=float[,,] -- 3-dimensional array of 'float' objects
\end{lstlisting}

The same syntax is also used when describing an array type in the
arguments of a \keyword{method} instruction or when converting types.
For example, after:
\begin{lstlisting}
gg=char[] "Horse"
\end{lstlisting}
the variable \textbf{gg} has as its value an array of
type \textbf{char[]} containing the five
characters \textbf{H}, \textbf{o}, \textbf{r}, \textbf{s},
and \textbf{e}.
\subsection{Array initializers}\label{refarrin}
\index{Arrays,initializing}
\index{Initializing arrays,}
\index{Array initializer,in terms}
\index{Brackets,in array initializers}
\index{Square brackets,in array initializers}
 
An \emph{array initializer} is a \emph{simple term} which is
recognized if it does not immediately follow (abut) a symbol, and has
the form
\footnote{
The notations \keyword{'['} and \keyword{']'}
indicate square brackets appearing in the \nr{} program.
}
\begin{alltt}
\keyword{'['}\emph{expression}[,\emph{expression}]...\keyword{']'}
\end{alltt}
 
An array initializer therefore comprises a list of one or more
expressions, separated by commas, within brackets.  When an array
initializer is evaluated, the expressions are evaluated in turn from
left to right, and all must result in a value.
An array is then constructed, with a number of elements
equal to the number of expressions in the list, with each element
initialized by being assigned the result of the corresponding
expression.
 
The type of the array is derived by adding one dimension to the type of
the result of the first expression in the list, where the type of that
expression is determined using the same rules as are used to select the
type of a variable when it is first  assigned a value(see page \pageref{refassign}).
All the other expressions in the list must have types that could be
assigned to the chosen type without error.
 
For example, in
\begin{lstlisting}
var1=['aa', 'bb', 'cc']
var2=[char 'a', 'b', 'c']
var3=[String 'a', 'bb', 'c']
var4=[1, 2, 3, 4, 5, 6]
var5=[[1,2], [3,4]]
\end{lstlisting}
the types of the variables would
be \textbf{R\textsc{exx}[]}, \textbf{char[]}, \textbf{String[]}, \textbf{R\textsc{exx}[]},
and \textbf{R\textsc{exx}[,]} respectively.
In a binary class in the reference implementation, the types would
be \textbf{String[]}, \textbf{char[]}, \textbf{String[]}, \textbf{int[]},
and \textbf{int[,]}.
 
Array initializers are most useful for initializing properties and
variables, but like other simple terms, they may start a compound term.
 
So, for example
\begin{lstlisting}
say [1,1,1,1].length
\end{lstlisting}
would display \textbf{4}.
 Note that an array of length zero cannot be constructed with an array
initializer, as its type would be undefined.  An explicitly typed array
constructor (for example, \textbf{int[0]}) must be used.

\chapter{Keyword Instructions}\label{refkinst}
\index{Instructions,}
\index{Instructions,keyword}
\index{Keywords,mixed case}
\index{Keyword instructions,}
 A \emph{keyword instruction} is one or more clauses, the first of
which starts with a keyword that identifies the instruction.
Some keyword instructions affect the flow of control; the remainder
just provide services to the programmer.
Some keyword instructions (\keyword{do}, \keyword{if}, \keyword{loop}, or
\keyword{select}) can include nested instructions.
 Appendix A (see page \pageref{refappa})  includes an example of a \nr{} program
using many of the instructions available.
 As can be deduced from the syntax rules described earlier, a keyword
instruction is recognized \textbf{only} if its keyword is the first
token in a clause, and if the second token is not an "\textbf{=}"
character (implying an assignment).
It would also not be recognized if the second token started
with "\textbf{(}", "\textbf{[}",
or "\textbf{.}" (implying that the first token starts a term).
 Further, if a current local variable, method argument, or property
has the same name as a keyword then the keyword will not be recognized.
This important rule allows \nr{} to be extended with new keywords in
the future without invalidating existing programs.
 
Thus, for example, this sequence in a program with no \textbf{say}
variable:
\begin{lstlisting}
say 'Hello'
say('1')
say=3
say 'Hello'
\end{lstlisting}
would be a \keyword{say} instruction, a call to some \textbf{say}
method, an assignment to a \textbf{say} variable, and an error.
 In \nr{}, therefore, keywords are not reserved; they may be used as
the names of variables (though this is not recommended, where known in
advance).
\index{Sub-keywords,}
 Certain other keywords, known as \emph{sub-keywords}, may be
known within the clauses of individual instructions - for
example, the symbols \keyword{to} and \keyword{while} in the \keyword{loop}
instruction.  Again, these are not reserved; if they had been used as
names of variables, they would not be recognized as sub-keywords.
 Blanks adjacent to keywords have no effect other than that of
separating the keyword from the subsequent token.
For example, this applies to the blanks next to the sub-keyword
\keyword{while} in
\begin{lstlisting}
loop  while  a=3
\end{lstlisting}
Here at least one blank was required to separate the symbols
forming the keywords and the variable name, \textbf{a}.  However the
blank following the \keyword{while} is not necessary in
\begin{lstlisting}
loop while 'Me'=a
\end{lstlisting}
though it does aid readability.

\chapter{Annotation instruction}\label{refparse}
\index{Annotate,instruction}
\index{Instructions,Annotate}
\begin{shaded}
\begin{alltt}
\textbf{@}
\end{alltt}
\end{shaded}
An \emph{annotation} starts with an \textbf{@} (commercial at sign)
and is passed through unchanged\footnote{dependent on the setting of
option -annotations, which is the default. When option -noannotations
is in effect, no annotations are passed through. In this case, no
@SuppressWarnings("unchecked") annotations are generated on methods,
which might lead to (harmless) javac warnings. } To interpret a program with an annotation is an error.
\section{Example}
\begin{lstlisting}
/* standard annotations like @Override and @Deprecated are */
/* used, as are some custom ones                           */
/* (those need to be compiled first to be used)            */
options binary
@Author(name="Class Author")
class AnnotateTest
properties private unused
propz
a = ArrayList()
test = TreeMap()

  @SuppressWarnings("unchecked")
  method main(args=String[]) static
    say 'hello annotations'
    t=AnnotateTest()
    t.old()

    @Override
  method toString() returns String
    return 'Annotations'

    @Deprecated
  method old() /* a comment with an @ in it */
    say 'do no use anymore'

    @Author(name = "Jane Doe")
    @Author(name = "John Doe")
  method repeating()
    say 'repeating annotations'

    @Author( name = "Fifi the Cat", date = "2016-01-01" )
  method parameters()
    say 'parameters are possible, but all on one line'
\end{lstlisting}
\chapter{Address instruction}\label{refparse}
\index{Address,instruction}
\index{Instructions,Address}
\begin{shaded}
\begin{alltt}
\textbf{address} \emph{[environment]} \emph{[expression]}

where \emph{environment} is one of

    \textbf{shell}
    \textbf{bash}
    \textbf{cmd}

\end{alltt}
\end{shaded}
The keyword \emph{address} temporarily or permanently changes the destination of commands. Commands are strings sent to aan external environment. You can send commands by specifying clauses consisting of only an expression or by using the ADDRESS instruction.

To send a single command to a specified environment, code an environment, a literal string or a single symbol, which is taken to be a constant, followed by an expression. The environment name is the name of an external procedure or process that can process commands. The expression is evaluated to produce a character string value, and this string is routed to the environment to be processed as a command. After execution of the command, environment is set back to its original state, thus temporarily changing the destination for a single command.
\chapter{Class instruction}\label{refclass}
\index{,}
\index{Instructions,CLASS}
\index{,}
\index{Class,starting}
\begin{shaded}
\begin{alltt}
\textbf{class} \emph{name} [\emph{visibility}] [\emph{modifier}] [\textbf{binary}] [\textbf{deprecated}]
               [\textbf{extends} \emph{classname}]
               [\textbf{uses} \emph{useslist}]
               [\textbf{implements} \emph{interfacelist}];

where \emph{visibility} is one of:

    \textbf{private}
    \textbf{public}
    \textbf{shared}

and \emph{modifier} is one of:

    \textbf{abstract}
    \textbf{adapter}
    \textbf{final}
    \textbf{interface}

and \emph{useslist} and \emph{interfacelist} are lists of one or more \emph{classname}s, separated by commas.
\end{alltt}
\end{shaded}
 The \texttt{class} instruction is used to introduce a class, as
described in the sections  \emph{Types and (see page \pageref{reftypes}) 
Classes}:ea. and  \emph{Program structure} (see page \pageref{refpstruct}) ,
and define its attributes.
The class must be given a \emph{name}, which must be different from
the name of any other classes in the program.
\index{Short name,of classes}
\index{Class,short name of}
The \emph{name}, which must be a non-numeric symbol, is known as the
\emph{short name} of the class.
 
\index{Class,name of}
A \emph{classname} can be either the short name of a class (if that is
unambiguous in the context in which it is used), or the qualified name
of the class - the name of the class prefixed by a package name and
a period, as described under the  \texttt{package} (see page \pageref{refpackage}) 
instruction:ea..
 
\index{Body,of classes}
\index{Class,body of}
The \emph{body} of the class consists of all clauses following the
class instruction (if any) until the next \texttt{class} instruction or
the end of the program.
 
The \emph{visibility}, \emph{modifier}, and \texttt{binary}
keywords, and the \texttt{extends}, \texttt{uses}, and
\texttt{implements} phrases, may appear in any order.
\subsection{Visibility}
\index{PUBLIC,on CLASS instruction}
\index{PRIVATE,on CLASS instruction}
\index{SHARED,on CLASS instruction}
\index{Visibility,of classes}
\index{Classes,private}
\index{Classes,public}
\index{Classes,shared}
 
Classes may be \texttt{public}, \texttt{private}, or
\texttt{shared}:
\begin{itemize}
\item A \emph{public class} is visible to (that is, may be used by)
all other classes.
\item A \emph{private class} is visible only within same program and to
classes in the same  package (see page \pageref{refpackage}) .
\item A \emph{shared class} is also visible only within same program and to
classes in the same package.
\footnote{
The \texttt{shared} keyword on the \texttt{class} instruction means
exactly the same as the keyword \texttt{private}, and is accepted for
consistency with the other meanings of \texttt{shared}.
}
\end{itemize}
 
A program may have only one public class, and if no class is marked
public then the first is assumed to be public (unless it is explicitly
marked private).
\subsection{Modifier}
\index{ABSTRACT,on CLASS instruction}
\index{ADAPTER,on CLASS instruction}
\index{FINAL,on CLASS instruction}
\index{INTERFACE,on CLASS instruction}
 
Most classes are collections of data (properties) and the procedures
that can act on that data (methods); they completely implement a
datatype (type), and are permitted to be subclassed.
\index{Standard classes,}
\index{Classes,standard}
These are called \emph{standard classes}.
The \emph{modifier} keywords indicate that the class is not a standard
class - it is special in some way.
Only one of the following modifier keywords is allowed:
\begin{description}
\index{Abstract classes,}
\index{Abstract methods,}
\index{Classes,abstract}
\index{Methods,abstract}
\item{abstract}

An \emph{abstract class} does not completely implement a datatype; one
or more of the methods that it defines (or which it inherits from
classes it extends or implements) is abstract - that is, the name
of the method and the types of its arguments are defined, but no
instructions to implement the method are provided.
 
Since some methods are not provided, an object cannot be constructed
from an abstract class.  Instead, the class must be extended and any
missing methods provided.  Such a subclass can then be used to construct
an object.
 
Abstract classes are useful where many subclasses can share common data
or methods, but each will have some unique attribute or attributes (data
and/or methods).  For example, some set of geometric objects might share
dimensions in X and Y, yet need unique methods for calculating the area
of the object.
\index{Adapter classes,}
\index{Classes,adapter}
\item{adapter}

An \emph{adapter class} is a class that is guaranteed to implement all
unimplemented abstract methods of its superclasses and interface classes
that it inherits or lists as implemented on the \texttt{class} instruction.
 
If any unimplemented methods are found, they will be automatically
generated by the language processor.  Methods generated in this way will
have the same visibility and signature as the abstract method they
implement, and if a return value is expected then a default value is
returned (as for the initial value of variables of the same type: that
is, \textbf{null} or, for values of primitive type, an
implementation-defined value, typically 0).  Other than possibly
returning a value, these methods are empty; that is, they have no
side-effects.
 
An adapter class provides a concrete representation of its superclasses
and the interface classes it implements.  As such, it is especially
useful for implementing event handlers and the like, where only a small
number of event-handling methods are needed but many more might be
specified in the interface class that describes the event model.
\footnote{
For example, see the "Scribble" sample in the \nr{} package.
}
 
An adapter class cannot have any abstract methods.
\index{Final classes,}
\index{Classes,final}
\item{final}

A \emph{final class} is considered to be complete; it cannot be
subclassed (extended), and all its methods are considered complete.
\footnote{
This modifier is provided for consistency with other languages, and may
allow compilers to improve the performance of classes that refer to the
final class.
In many cases it will reduce the reusability of the class, and hence
should be avoided.
}
\index{Interface classes,}
\index{Classes,interface}
\item[interface]\label{refinterf}

An \emph{interface class} is an abstract class that contains only
abstract method definitions and/or constants.  That is, it defines
neither instructions that implement methods nor modifiable properties,
and hence cannot be used to construct an object.
 
Interface classes are used by classes that claim to \emph{implement}
them (see the \texttt{implements} keyword, described below).
The difference between abstract and interface classes is that
the former may have methods which are not abstract, and hence can only
be subclassed (extended), whereas the latter are wholly abstract and
may only be implemented.
\end{description}
\subsection{Binary}\label{refbincla}
\index{BINARY,on CLASS instruction}
\index{Binary classes,}
\index{Classes,binary}
 
The keyword \texttt{binary} indicates that the class is a \emph{binary
class}.
In binary classes, literal strings and numeric symbols are assigned
native string or binary (primitive) types, rather than \nr{} types,
and native binary operations are used to implement operators where
possible.
When \texttt{binary} is not in effect (the default), terms in
expressions are converted to \nr{} types before use by operators.
The section  \emph{Binary values and operations} (see page \pageref{refbinary}) 
describes the implications of binary classes in detail.
 
Individual methods in a class which is not binary can be made into
\emph{binary methods} using the \texttt{binary} keyword on the
 \texttt{method} instruction (see page \pageref{refmethod}) .
\subsection{Deprecated}\label{refdepcla}
\index{DEPRECATED,on CLASS instruction}
 
The keyword \texttt{deprecated} indicates that the class
is \emph{deprecated}, which implies that a better alternative is
available and documented.  A compiler can use this information to warn
of out-of-date or other use that is not recommended.
\subsection{Extends}
\index{EXTENDS,on CLASS instruction}
\index{Subclass of a class,}
\index{Superclass of a class,}
\index{Classes,and subclasses}
\index{Classes,and superclasses}
 
Classes form a hierarchy, with all classes (except the top of the tree,
the \textbf{Object}
\footnote{
\emph{In the reference implementation, \textbf{java.lang.Object}.}
}
class) being a \emph{subclass} of some other class.
The \texttt{extends} keyword identifies the \emph{classname} of the
immediate \emph{superclass} of the new class - that is, the
class immediately above it in the hierarchy.
If no \texttt{extends} phrase is given, the superclass is assumed to
be \textbf{Object} (or \textbf{null}, in the case where the current
class is \textbf{Object}).
\subsection{Uses}
\index{USES,on CLASS instruction}
\index{Constants,used by classes}
\index{Functions,used by classes}
\index{Static methods,used by classes}
 
The \texttt{uses} keyword introduces a list of the names of one or
more classes that will be used as a source of constant (or static)
properties and/or methods.
 
When a  term (see page \pageref{refterms})  starts with a symbol, method call, or
indexed reference that is not known in the current context, each class
in the \emph{useslist} and its superclasses are searched (in the
order specified in the \emph{useslist}) for a constant or static
method or property that matches the item.
If found, the method or property is used just as though explicitly
qualified by the name of the class in which it was found.
 
The \texttt{uses} mechanism affects only the syntax of terms in the
current class; it is not inherited by subclasses of the current class.
\subsection{Implements}
\index{IMPLEMENTS,on CLASS instruction}
\index{Interfaces,implemented by classes}
 
The \texttt{implements} keyword introduces a list of the names of one or
more interface classes (see above).
These interface classes are then known to (inherited by) the current
class, in the order specified in the \emph{interfacelist}.
Their methods (which are all abstract) and constant properties act as
though part of the current class, unless they are overridden (hidden) by
a method or constant of the same name in the current class.
 
If the current class is not an interface class then it must implement
(provide non-abstract methods for) all the methods inherited from the
interface classes in the implements list.
 
Interface classes, therefore, can be used to:
\begin{enumerate}
\item Define a common set of methods (possibly with associated constants)
that will be implemented by other classes.
\item Conveniently package collections of constants for use by other
classes.
\end{enumerate}
 
The implements list may not include the superclass of the current class.
\index{,}

\section{Do instruction}\label{refdo}
\index{Instructions,DO}
\index{Flow control,with DO construct}
\index{Group, DO,}
\index{DO group,}
\index{Simple DO group,}
\begin{shaded}
\begin{alltt}
\textbf{do} [\textbf{label} \emph{name}] [\textbf{protect} \emph{term}] [\textbf{binary]};
        \emph{instructionlist}
    [\textbf{catch} [\emph{vare} =] \emph{exception};
        \emph{instructionlist}]...
    [\textbf{finally}[;]
        \emph{instructionlist}]
\textbf{end} [\emph{name}];

where \emph{name} is a non-numeric \emph{symbol}

and \emph{instructionlist} is zero or more \emph{instruction}s
\end{alltt}
\end{shaded}
 The \keyword{do} instruction is used to group instructions together for
execution; these are executed once.
The group may optionally be given a label, and may protect an object
while the instructions in the group are executed; exceptional conditions
can be handled with \keyword{catch} and \keyword{finally}.
 
The most common use of \keyword{do} is simply for treating a number of
instructions as group.

\textbf{Example:}
\begin{lstlisting}
/* The two instructions between DO and END will both */
/* be executed if A has the value 3.                 */
if a=3 then do
  a=a+2
  say 'Smile!'
end
\end{lstlisting}
\index{Body,of group}
Here, only the first \emph{instructionlist} is used.
This forms the \emph{body} of the group.
 
The instructions in the \emph{instructionlist}s may be any assignment,
method call, or keyword instruction, including any of the more complex
constructions such as \keyword{loop}, \keyword{if}, \keyword{select}, and
the \keyword{do} instruction itself.
\subsection{Label phrase}
\index{DO instruction,LABEL}
\index{DO group,naming of}
\index{LABEL,on DO instruction}
 
If \keyword{label} is used to specify a \emph{name} for the group,
then a \keyword{leave} which specifies that name may be used to leave the
group, and the \keyword{end} that ends the group may optionally specify
the name of the group for additional checking.

\textbf{Example:}
\begin{lstlisting}
do label sticky
  x=ask
  if x='quit' then leave sticky
  say 'x was' x
end sticky
\end{lstlisting}
\subsection{Protect phrase}
\index{PROTECT,on DO instruction}
 
If \keyword{protect} is given it must be followed by a \emph{term}
that evaluates to a value that is not just a type and is not of a
primitive type; while the \keyword{do} construct is being executed, the
value (object) is protected - that is, all the instructions in the
\keyword{do} construct have exclusive access to the object.
 
Both \keyword{label} and \keyword{protect} may be specified, in any order,
if required.
\subsection{Exceptions in do groups}
 
\index{CATCH,on DO instruction}
\index{FINALLY,on DO instruction}
Exceptions that are raised by the instructions within a do group may be
caught using one or more \keyword{catch} clauses that name the
\emph{exception} that they will catch.
When an exception is caught, the exception object that holds the details
of the exception may optionally be assigned to a variable,
\emph{vare}.
 
Similarly, a \keyword{finally} clause may be used to introduce
instructions that will always be executed at the end of the group, even
if an exception is raised (whether caught or not).
 
The  \emph{Exceptions} section (see page \pageref{refexcep})  has details and
examples of \keyword{catch} and \keyword{finally}.

\subsection{Binary}
A group of one or more statements in a \code{do binary} group will
follow the semantics of binary statements in binary classes or
methods; \marginnote{\color{gray}3.04}the scope is limited to the do binary group.
\chapter{Exit instruction}\label{refexit}
\index{EXIT instruction,}
\index{Instructions,EXIT}
\index{Return code, setting on exit,}
\index{Return string, setting on exit,}
\begin{shaded}
\begin{alltt}
\textbf{exit} [\emph{expression}];
\end{alltt}
\end{shaded}
 \keyword{exit} is used to unconditionally leave a program, and
optionally return a result to the caller.
The entire program is terminated immediately.

If an \emph{expression} is given, it is evaluated and the result
of the evaluation is then passed back to the caller in an
implementation-dependent manner when the program terminates.
Typically this value is expected to be a small whole number; most
implementations will accept values in the range 0 through 250.
If no expression is given, a default result (which depends on the
implementation, and is typically zero) is passed back to the caller.

\textbf{Example:}
\begin{lstlisting}
j=3
exit j*4
/* Would exit with the value '12' */
\end{lstlisting}
\index{Running off the end of a program,}
\index{Bottom of program, reaching during execution,}
 "Running off the end" of a program is equivalent to the
instruction \textbf{return;}.  In the case where the program is simply
a stand-alone application with no \keyword{class} or \keyword{method}
instructions, this has the same effect as \textbf{exit;}, in that it
terminates the whole program and returns a default result.

\section{If instruction}
\index{IF instruction,}
\index{Instructions,IF}
\index{THEN,following IF clause}
\index{,}
\index{Flow control,with IF construct}
\begin{shaded}
\begin{alltt}
\textbf{if} \emph{expression}[;]
     \textbf{then}[;] \emph{instruction}
    [\textbf{else}[;] \emph{instruction}]
\end{alltt}
\end{shaded}
 The \keyword{if} construct is used to conditionally execute an
instruction or group of instructions.
It can also be used to select between two alternatives.
 The expression is evaluated and must result in either 0 or 1.
If the result was 1 (true) then the instruction after the
\keyword{then} is executed.
If the result was 0 (false) and an \keyword{else} was given
then the instruction after the \keyword{else} is executed.
 \textbf{Example:}
\begin{lstlisting}
if answer='Yes' then say 'OK!'
                else say 'Why not?'
\end{lstlisting}
 Remember that if the \keyword{else} clause is on the same line as the
last clause of the \keyword{then} part, then you need a semicolon to
terminate that clause.
 \textbf{Example:}
\begin{lstlisting}
if answer='Yes' then say 'OK!';  else say 'Why not?'
\end{lstlisting}
 The \keyword{else} binds to the nearest \keyword{then} at the same level.
This means that any \keyword{if} that is used as the instruction
following the \keyword{then} in an \keyword{if} construct that has an
\keyword{else} clause, must itself have an \keyword{else} clause (which
may be followed by the dummy instruction, \keyword{nop}).
 \textbf{Example:}
\begin{lstlisting}
if answer='Yes' then if name='Fred' then say 'OK, Fred.'
                                    else say 'OK.'
                else say 'Why not?'
\end{lstlisting}
 
To include more than one instruction following \keyword{then} or
\keyword{else}, use a grouping instruction (\keyword{do}, \keyword{loop},
or \keyword{select}).
 \textbf{Example:}
\begin{lstlisting}
if answer='Yes' then do
  say 'Line one of two'
  say 'Line two of two'
end
\end{lstlisting}
In this instance, both \keyword{say} instructions are executed when
the result of the \keyword{if} expression is 1.
 
Multiple expressions, separated by commas, can be given on the
\keyword{if} clause, which then has the syntax:
\begin{shaded}
\begin{alltt}
\textbf{if} \emph{expression}[, \emph{expression}]... [;]
\end{alltt}
\end{shaded}
In this case, the expressions are evaluated in turn from left to
right, and if the result of any evaluation is 1 then the test has
succeeded and the instruction following the associated \keyword{then}
clause is executed.
If all the expressions evaluate to 0 and an \keyword{else} was given
then the instruction after the \keyword{else} is executed.
 
Note that once an expression evaluation has resulted in 1, no further
expressions in the clause are evaluated.  So, for example, in:
\begin{lstlisting}
-- assume 'name' is a string
if name=null, name='' then say 'Empty'
\end{lstlisting}
then if \texttt{name} does not refer to an object it will compare equal to
null and the \keyword{say} instruction will be executed without
evaluating the second expression in the \keyword{if} clause.
\begin{shaded}\noindent
\textbf{Notes:}
\begin{enumerate}
\item An \emph{instruction} may be any assignment, method call, or
keyword instruction, including any of the more complex constructions
such as \keyword{do}, \keyword{loop}, \keyword{select}, and the \keyword{if}
instruction itself.
A null clause is not an instruction, however, so putting an extra
semicolon after the \keyword{then} or \keyword{else} is not equivalent to
putting a dummy instruction.
The \keyword{nop} instruction is provided for this purpose.
\item The keyword \keyword{then} is treated specially, in that it need not start a
clause.
This allows the expression on the \keyword{if} clause to be terminated by
the \keyword{then}, without a "\textbf{;}" being required -
were this not so, people used to other computer languages would
be inconvenienced.
Hence the symbol \keyword{then} cannot be used as a variable name within
the expression.
\footnote{
Strictly speaking, \keyword{then} should only be recognized if not
the name of a variable.  In this special case, however, \nr{} language
processors are permitted to treat \keyword{then} as reserved in the
context of an \keyword{if} clause, to provide better performance and
more useful error reporting.
}
\end{enumerate}
\end{shaded}\indent

\chapter{Import instruction}\label{refimport}
\index{IMPORT instruction,}
\index{Instructions,IMPORT}
\begin{shaded}
\begin{alltt}
\textbf{import} \emph{name};

where \emph{name} is one or more non-numeric \emph{symbol}s separated by periods,\\ with an optional trailing period.
\end{alltt}
\end{shaded}
 
The \keyword{import} instruction is used to simplify the use of
classes from other packages.
If a class is identified by an \keyword{import} instruction, it can then
be referred to by its short name, as given on the
 \keyword{class} instruction (see page \pageref{refclass}) , as well as by its fully
qualified name.
 
There may be zero or more \keyword{import} instructions in a program.
They must precede any \keyword{class} instruction (or any instruction
that would start the default class).
 
\index{Package,name of}
In the following description, a \emph{package name} names a package
as described under the  \keyword{package} instruction (see page \pageref{refpackage}).
The import \emph{name} must be one of:
\begin{itemize}
\item A qualified class name, which is a package name immediately followed
by a period which is immediately followed by a short class name - in this case, the individual class identified is imported.
\item A package name - in this case, all the classes in the specified
package are imported.  The name may have a trailing period.
\item A partial package name (a package name with one or more parts
omitted from the right, indicated by a trailing period after the parts
that are present) - in this case, all classes in the package hierarchy
below the specified point are imported.
\end{itemize}
 \textbf{Examples:}
\begin{alltt}
import java.lang.String
import java.lang
import java.
\end{alltt}
The first example above imports a single class (which could then be
referred to simply as "\textbf{String}").
The second example imports all classes in the
"\textbf{java.lang}" package.
The third example imports all classes in all the packages whose name
starts with "\textbf{java.}".
 
\index{Imports,explicit}
When a class is imported explicitly, for example, using
\begin{alltt}
import java.awt.List
\end{alltt}
this indicates that the short name of the class (\texttt{List},
in this example) may be used to refer to the class unambiguously.
That is, using this short name will not report an ambiguous reference
warning (as it would without the \keyword{import} instruction, because
a \texttt{java.util.List} class was added in Java 1.2).
 
It follows that:
\begin{itemize}
\item Two classes imported explicitly cannot have the same short name.
\item No class in a program being compiled can have the same short name as
a class that is imported explicitly.
\end{itemize}
because in either of these situations a use of the short name would
be ambiguous.
 
Note also that an explicit import does not import the minor or dependent
classes associated with a name; they each require their own explicit
import (unless the entire package is imported).
 
\index{Imports,automatic}
\emph{In the reference implementation, the fundamental \nr{} and Java
package hierarchies are automatically imported by default, as though the
instructions:}
\begin{alltt}
import netrexx.lang.
import java.lang.
import java.io.
import java.util.
import java.net.
import java.awt.
import java.applet.
import javax.swing.
\end{alltt}
\emph{had been executed before the program begins.
In addition, classes in the current (working) directory are imported if
no \keyword{package} instruction is specified.  If a \keyword{package}
instruction is specified then all classes in that package are imported.
}

\chapter{Iterate instruction}
\index{ITERATE instruction,}
\index{Instructions,ITERATE}
\index{ITERATE instruction,use of variable on}
\index{,}
\index{Loops,modification of}
\begin{shaded}
\begin{alltt}
\textbf{iterate} [\emph{name}];

where \emph{name} is a non-numeric \emph{symbol}.
\end{alltt}
\end{shaded}
 \keyword{iterate} alters the flow of control within a \keyword{loop}
construct.
It may only be used in the body (the first \emph{instructionlist})
of the construct.

Execution of the instruction list stops, and control is passed
directly back up to the \keyword{loop} clause just as though the last
clause in the body of the construct had just been executed.
The control variable (if any) is then stepped (iterated) and termination
conditions tested as normal and the instruction list is executed again,
unless the loop is terminated by the \keyword{loop} clause.

If no \emph{name} is specified, then \keyword{iterate} will step
the innermost active loop.
 
If a \emph{name} is specified, then it must be the name of the
label, or control variable if there is no label, of a currently active
loop (which may be the innermost), and this is the loop that is
iterated.
Any active \keyword{do}, \keyword{loop}, or \keyword{select} constructs
inside the loop selected for iteration are terminated (as though by a
\keyword{leave} instruction).

\textbf{Example:}
\begin{alltt}
loop i=1 to 4
  if i=2 then iterate i
  say i
  end
/* Would display the numbers:  1, 3, 4  */
\end{alltt}
 \textbf{Notes:}
\begin{enumerate}
\index{Active constructs,}
\index{Loops,active}
\index{Names,on ITERATE instructions}
\item A loop is active if it is currently being executed.
If a method (even in the same class) is called during execution of a
loop, then the loop becomes inactive until the method has returned.
\keyword{iterate} cannot be used to step an inactive loop.
\item The \emph{name} symbol, if specified, must exactly match the
label (or the name of the control variable, if there is no label) in the
\keyword{loop} clause in all respects except case.
\end{enumerate}

\chapter{Leave instruction}\label{refleave}
\index{LEAVE instruction,}
\index{Instructions,LEAVE}
\index{LEAVE instruction,use of variable on}
\index{,}
\index{Loops,termination of}
\begin{shaded}
\begin{alltt}
\textbf{leave} [\emph{name}];

where \emph{name} is a non-numeric \emph{symbol}.
\end{alltt}
\end{shaded}
 \texttt{leave} causes immediate exit from one or more \texttt{do},
\texttt{loop}, or \texttt{select} constructs.
It may only be used in the body (the first \emph{instructionlist})
of the construct.
 
Execution of the instruction list is terminated, and control is
passed to the \texttt{end} clause of the construct, just as though the
last clause in the body of the construct had just been executed or (if
a loop) the termination condition had been met normally, except that on
exit the control variable (if any) will contain the value it had when
the \texttt{leave} instruction was executed.
 
If no \emph{name} is specified, then \texttt{leave} must be
within an active loop and will terminate the innermost active loop.
 
If a \emph{name} is specified, then it must be the name of the
label (or control variable for a loop with no label), of a currently
active \texttt{do}, \texttt{loop}, or \texttt{select} construct
(which may be the innermost).  That construct (and any active constructs
inside it) is then terminated.  Control then passes to the clause
following the \texttt{end} clause that matches the
\texttt{do}, \texttt{loop}, or \texttt{select} clause identified by the
\emph{name}.
 \textbf{Example:}
\begin{alltt}
loop i=1 to 5
  say i
  if i=3 then leave
  end i
/* Would display the numbers:  1, 2, 3  */
\end{alltt}
 \textbf{Notes:}
\begin{enumerate}
\index{FINALLY,reached by LEAVE}
\item If any construct being left includes a \texttt{finally} clause, the
\emph{instructionlist} following the \texttt{finally} will be
executed before the construct is left.
\index{Active constructs,}
\index{Loops,active}
\index{Constructs,active}
\index{Names,on LEAVE instructions}
\item 
A \texttt{do}, \texttt{loop}, or \texttt{select} construct
is active if it is currently being executed.
If a method (even in the same class) is called during execution of an
active construct, then the construct becomes inactive until the method
has returned.
\texttt{leave} cannot be used to leave an inactive construct.
\item The \emph{name} symbol, if specified, must exactly match the
label (or the name of the control variable, for a loop with no label) in
the \texttt{do}, \texttt{loop}, or \texttt{select} clause in all
respects except case.
\end{enumerate}

\chapter{Loop instruction}\label{refloop}
\index{,}
\index{Instructions,LOOP}
\index{,}
\index{FOREVER,repetitor on LOOP instruction}
\index{FOR,repetitor on LOOP instruction}
\index{OVER repetitor on LOOP instruction,}
\index{WHILE phrase of LOOP instruction,}
\index{UNTIL phrase of LOOP instruction,}
\index{BY phrase of LOOP instruction,}
\index{TO phrase of LOOP instruction,}
\index{FOR,phrase of LOOP instruction}
\index{,}
\index{,}
\index{Loops,repetitive}
\index{Conditional loops,}
\index{Infinite loops,}
\index{Numbers,in LOOP instruction}
\index{Indefinite loops,}
\index{Flow control,with LOOP construct}
\index{= equals sign,in LOOP instruction}
\begin{shaded}
\begin{alltt}
\textbf{loop} [\textbf{label} \emph{name}] [\textbf{protect} \emph{termp}] [\emph{repetitor}] [\emph{conditional}];
        \emph{instructionlist}
    [\textbf{catch} [\emph{vare} =] \emph{exception};
        \emph{instructionlist}]...
    [\textbf{finally}[;]
        \emph{instructionlist}]
    \textbf{end} [\emph{name}];

where \emph{repetitor} is one of:

    \emph{varc} = \emph{expri} [\textbf{to} \emph{exprt}] [\textbf{by} \emph{exprb}] [\textbf{for} \emph{exprf}]
    \emph{varo} \textbf{over} \emph{termo}
    \textbf{for} \emph{exprr}
    \textbf{forever}

and \emph{conditional} is either of:

    \textbf{while} \emph{exprw}
    \textbf{until} \emph{expru}

and \emph{name} is a non-numeric \emph{symbol}

and \emph{instructionlist} is zero or more \emph{instruction}s

and \emph{expri}, \emph{exprt}, \emph{exprb}, \emph{exprf}, \emph{exprr}, \emph{exprw}, and \emph{expru} are \emph{expressions}.
\end{alltt}
\end{shaded}
 The \keyword{loop} instruction is used to group instructions together
and execute them repetitively.
The loop may optionally be given a label, and may protect an object
while the instructions in the loop are executed; exceptional conditions
can be handled with \keyword{catch} and \keyword{finally}.
 \keyword{loop} is the most complicated of the \nr{} keyword
instructions.
It can be used as a simple indefinite loop, a predetermined
repetitive loop, as a loop with a bounding condition that is
recalculated on each iteration, or as a loop that steps over the
contents of a collection of values.
\subsection{Syntax notes:}
\begin{itemize}
\item 
The \keyword{label} and \keyword{protect} phrases may be in any order.
They must precede any \emph{repetitor} or \emph{conditional}.
\item 
\index{Body,of a loop}
The first \emph{instructionlist} is known as the \emph{body} of
the loop.
\item 
The \keyword{to}, \keyword{by}, and \keyword{for} phrases in the first form
of \emph{repetitor} may be in any order, if used, and will be
evaluated in the order they are written.
\item 
Any instruction allowed in a method is allowed in an
\emph{instructionlist}, including assignments, method call
instructions, and keyword instructions (including any of the more
complex constructions such as \keyword{if}, \keyword{do}, \keyword{select},
or the \keyword{loop} instruction itself).
\item 
If \keyword{for} or \keyword{forever} start the \emph{repetitor} and
are followed by an "\textbf{=}" character, they are taken as
control variable names, not keywords (as for assignment instructions).
\item 
The expressions \emph{expri}, \emph{exprt}, \emph{exprb}, or
\emph{exprf} will be ended by any of the keywords \keyword{to},
\keyword{by}, \keyword{for}, \keyword{while}, or \keyword{until} (unless
the word is the name of a variable).
\item 
The expressions \emph{exprw} or \emph{expru} will be ended by
either of the keywords \keyword{while} or \keyword{until} (unless the
word is the name of a variable).
\end{itemize}
\subsection{Indefinite loops}
\index{Indefinite loops,}
\index{FOREVER,loops}
 If neither \emph{repetitor} nor \emph{conditional} are
present, or the \emph{repetitor} is the keyword \keyword{forever},
then the loop is an \emph{indefinite loop}.
It will be ended only when some instruction in the first
\emph{instructionlist} causes control to leave the loop.
 \textbf{Example:}
\begin{alltt}
/* This displays "Go caving!" at least once */
loop forever
  say 'Go caving!'
  if ask='' then leave
  end
\end{alltt}
\subsection{Bounded loops}
\index{Bounded loop,}
\index{Repetitive loops,}
\index{Loops,repetitive}
\index{Repetitor phrase,}
\index{Conditional phrase,}
 If a \emph{repetitor} (other than \keyword{forever}) or
\emph{conditional} is given, the first \emph{instructionlist}
forms a \emph{bounded loop}, and the instruction list is executed
according to any \emph{repetitor phrase}, optionally modified by a
\emph{conditional phrase}.
\begin{description}
\item{Simple bounded loops}
\index{Simple repetitor phrase,}
\index{Bounded loop,simple}

When the \emph{repetitor} starts with the keyword \keyword{for},
the expression \emph{exprr} is evaluated immediately
(with \textbf{0} added, to effect any rounding) to give a repetition
count, which must be a whole number that is zero or positive.
The loop is then executed that many times, unless it is terminated by
some other condition.
 \textbf{Example:}
\begin{alltt}
/* This displays "Hello" five times */
loop for 5
  say 'Hello'
  end
\end{alltt}
\item{Controlled bounded loops}
\index{Bounded loop,controlled}
\index{Controlled loops,}
\index{Control variable,}
\index{Variables,controlling loops}

A \emph{controlled loop} begins with an \emph{assignment},
which can be identified by the "\textbf{=}" that follows the name
of a control variable, \emph{varc}.
The control variable is assigned an initial value (the result of
\emph{expri}, formatted as though 0 had been added)
before the first execution of the instruction list.
The control variable is then stepped (by adding the result of
\emph{exprb}) before the second and subsequent times that the
instruction list is executed.
 
The name of the control variable, \emph{varc}, must be a non-numeric
symbol that names an existing or new variable in the current method or a
property in the current class (that is, it cannot be element of an
array, the property of a superclass, or a more complex term).  It is
further restricted in that it must not already be used as the name of a
control variable or label in a loop (or \keyword{do} or \keyword{select}
construct) that encloses the new loop.
 
\index{End condition of a LOOP loop,}
The instruction list in the body of the loop is executed repeatedly
while the end condition (determined by the result of \emph{exprt})
is not met.
If \emph{exprb} is positive or zero, then the loop will be
terminated when \emph{varc} is greater than the result of
\emph{exprt}.
If negative, then the loop will be terminated when \emph{varc} is
less than the result of \emph{exprt}.
 The expressions \emph{exprt} and \emph{exprb} must result in
numbers.
They are evaluated once only (with 0 added, to effect any
rounding), in the order they appear in the instruction, and before the
loop begins and before \emph{expri} (which must also result in a
number) is evaluated and the control variable is set to its initial
value.
 
The default value for \emph{exprb} is 1.
If no \emph{exprt} is given then the loop will execute indefinitely
unless it is terminated by some other condition.
 \textbf{Example:}
\begin{alltt}
loop i=3 to -2 by -1
  say i
  end
/* Would display: 3, 2, 1, 0, -1, -2 */
\end{alltt}
Note that the numbers do not have to be whole numbers:
 \textbf{Example:}
\begin{alltt}
x=0.3
loop y=x to x+4 by 0.7
  say y
  end
/* Would display: 0.3, 1.0, 1.7, 2.4, 3.1, 3.8 */
\end{alltt}
 The control variable may be altered within the loop, and this may
affect the iteration of the loop.
Altering the value of the control variable in this way is normally
considered to be suspect programming practice, though it may be
appropriate in certain circumstances.
 Note that the end condition is tested at the start of each iteration
(and after the control variable is stepped, on the second and
subsequent iterations).  It is therefore possible for the body of the
loop to be skipped entirely if the end condition is met immediately.
 The execution of a controlled loop may further be bounded by a
\keyword{for} phrase.
In this case, \emph{exprf} must be given and must evaluate to a
non-negative whole number.
This acts just like the repetition count in a simple bounded loop, and
sets a limit to the number of iterations around the loop if it is not
terminated by some other condition.
 
\emph{exprf} is evaluated along with the expressions
\emph{exprt} and \emph{exprb}.
That is, it is evaluated once only (with \textbf{0} added), when the
\keyword{loop} instruction is first executed and before the control
variable is given its initial value; the three expressions are evaluated
in the order in which they appear.
Like the \keyword{to} condition, the \keyword{for} count is checked at the
start of each iteration, as shown in the  programmer's (see page \pageref{refloopmod}) 
model:ea..
 \textbf{Example:}
\begin{alltt}
loop y=0.3 to 4.3 by 0.7 for 3
  say y
  end
/* Would display: 0.3, 1.0, 1.7 */
\end{alltt}
 
\index{END clause,specifying control variable}
In a controlled loop, the symbol that describes the control variable may
be specified on the \keyword{end} clause (unless a label is specified,
see below).
\nr{} will then check that this symbol exactly matches the
\emph{varc} of the control variable in the \keyword{loop} clause (in
all respects except case).
If the symbol does not match, then the program is in error - this
enables the nesting of loops to be checked automatically.
 \textbf{Example:}
\begin{alltt}
loop k=1 to 10
  ...
  ...
  end k  /* Checks this is the END for K loop */
\end{alltt}
\textbf{Note: }The values taken by the control variable may be affected by the
\keyword{numeric} settings, since normal \nr{} arithmetic rules apply
to the computation of stepping the control variable.
\item[Over loops]\label{refloopov}
\index{Bounded loop,over values}
\index{Over loops,}
\index{Control variable,}

When the second token of the \emph{repetitor} is the keyword
\keyword{over}, the control variable, \emph{varo}, is used
to work through the sub-values in the collection of indexed strings
identified by \emph{termo}.
In this case, the \keyword{loop} instruction takes a "snapshot" of
the indexes that exist in the collection at the start of the loop, and
then for each iteration of the loop the control variable is set to the
next available index from the snapshot.
 
The number of iterations of the loop will be the number of indexes in
the collection, unless the loop is terminated by some other condition.
 \textbf{Example:}
\begin{alltt}
mycoll=''
mycoll['Tom']=1
mycoll['Dick']=2
mycoll['Harry']=3
loop name over mycoll
  say mycoll[name]
  end
/* might display: 3, 1, 2 */
\end{alltt}
 \textbf{Notes:}
\begin{enumerate}
\item 
The order in which the values are returned is undefined; all that
is known is that all indexes available when the loop started will be
recorded and assigned to \emph{varo} in turn as the loop iterates.
\item 
The same restrictions apply to \emph{varo} as apply to
\emph{varc}, the control variable for controlled loops (see above).
\item 
Similarly, the symbol \emph{varo} may be used as a name for the loop
and be specified on the \keyword{end} clause (unless a label is
specified, see below).
\end{enumerate}
 \emph{In the reference implementation, the \keyword{over} form of
repetitor may also be used to step though the contents of any object
that is of a type that is a subclass of \textbf{java.util.Dictionary},
such as an object of type \textbf{java.util.Hashtable}.
In this case, \emph{termo} specifies the dictionary, and a snapshot
(enumeration) of the keys to the Dictionary is taken at the start of the
loop.
Each iteration of the loop then assigns a new key to the control
variable \emph{varo} which must be (or will be given, if it is new)
the type \textbf{java.lang.Object}.
}
\item[Conditional phrases]\label{refloopwu}
\index{Conditional phrase,}

Any of the forms of loop syntax can be followed by a
\emph{conditional} phrase which may cause termination of the loop.
 
If \keyword{while} is specified, \emph{exprw} is evaluated, using the
latest values of all variables in the expression, before the instruction
list is executed on every iteration, and after the control
variable (if any) is stepped.
The expression must evaluate to either 0 or 1, and the instruction list
will be repeatedly executed while the result is 1 (that is, the loop
ends if the expression evaluates to 0).
 \textbf{Example:}
\begin{alltt}
loop i=1 to 10 by 2 while i<6
  say i
  end
/* Would display: 1, 3, 5 */
\end{alltt}
 
If \keyword{until} is specified, \emph{expru} is evaluated, using the
latest values of all variables in the expression, on the second and
subsequent iterations, and before the control variable (if any) is stepped.
\footnote{
Thus, it appears that the \keyword{until} condition is tested after the
instruction list is executed on each iteration.
However, it is the \keyword{loop} clause that carries out the evaluation.
}
The expression must evaluate to either 0 or 1, and the instruction list
will be repeatedly executed until the result is 1 (that is, the loop
ends if the expression evaluates to 1).
 \textbf{Example:}
\begin{alltt}
loop i=1 to 10 by 2 until i>6
  say i
  end
/* Would display: 1, 3, 5, 7 */
\end{alltt}
\end{description}
 Note that the execution of loops may also be modified by
using the \keyword{iterate} or \keyword{leave} instructions.
\subsection{Label phrase}
\index{Loops,label}
\index{Loops,naming of}
\index{LABEL,on LOOP instruction}
 
The \keyword{label} phrase may used to specify a \emph{name} for the
loop.  The name can then optionally be used on
\begin{itemize}
\item a \keyword{leave} instruction, to specify the name of the loop to leave
\item an \keyword{iterate} instruction, to specify the name of the loop to
be iterated
\item the \keyword{end} clause of the loop, to confirm the identity of the
loop that is being ended, for additional checking.
\end{itemize}
 \textbf{Example:}
\begin{alltt}
loop label pooks i=1 to 10
  loop label hill while j<3
    ...
    if a=b then leave pooks
    ...
    end hill
  end pooks
\end{alltt}
In this example, the \keyword{leave} instruction leaves both loops.
 
If a label is specified using the \keyword{label} keyword, it overrides
any name derived from the control variable name (if any).  That is, the
variable name cannot be used to refer to the loop if a label is
specified.
\subsection{Protect phrase}
\index{PROTECT,on LOOP instruction}
 
The \keyword{protect} phrase may used to specify a term,
\emph{termp}, that evaluates to a value that is not just a type and
is not of a primitive type;
while the \keyword{loop} construct is being executed, the value (object)
is protected - that is, all the instructions in the \keyword{loop}
construct have exclusive access to the object.
 \textbf{Example:}
\begin{alltt}
loop protect myobject while a<b
  ...
  end
\end{alltt}
 
Both \keyword{label} and \keyword{protect} may be specified, in any order,
if required.
\subsection{Exceptions in loops}
\index{CATCH,on LOOP instruction}
\index{FINALLY,on LOOP instruction}
 
Exceptions that are raised by the instructions within a \keyword{loop}
construct may be caught using one or more \keyword{catch} clauses that
name the \emph{exception} that they will catch.  When an exception is
caught, the exception object that holds the details of the exception may
optionally be assigned to a variable, \emph{vare}.
 
Similarly, a \keyword{finally} clause may be used to introduce
instructions that will always be executed when the loop ends, even if an
exception is raised (whether caught or not).
 
The  \emph{Exceptions} section (see page \pageref{refexcep})  has details and
examples of \keyword{catch} and \keyword{finally}.
\subsection{Programmer's model - how a typical loop is executed}\label{refloopmod}
 This model forms part of the definition of the \keyword{loop}
instruction.
\index{Loops,execution model}
\index{Model,of loop execution}
\index{Programmer's model of LOOP,}
 For the following loop:
\begin{alltt}
\keyword{loop} \emph{varc} \keyword{=} \emph{expri} \keyword{to} \emph{exprt} \keyword{by} \emph{exprb} \keyword{while} \emph{exprw}
  ...
  \emph{instruction list}
  ...
  \keyword{end}
\end{alltt}
 \nr{} will execute the following:
\begin{alltt}
   $tempt=exprt+0   /* ($variables are internal and   */
   $tempb=exprb+0   /*   are not accessible.)         */
   varc=expri+0
   Transfer control to the point identified as $start:

$loop:
   /* An UNTIL expression would be tested here, with: */
   /* if expru then leave                             */
   varc=varc + $tempb
$start:
   if varc > $tempt then leave  /* leave quits a loop */
   /* A FOR count would be checked here               */
   if \textbackslash exprw then leave
      ...
      instruction list
      ...
   Transfer control to the point identified as $loop:
\end{alltt}
 \textbf{Notes:}
\begin{enumerate}
\item 
This example is for \emph{exprb} \textbf{>= 0}.
For a negative \emph{exprb}, the test at the start point of the loop
would use "\textbf{<}" rather than "\textbf{>}".
\item 
The upwards transfer of control takes place at the end of the body of
the loop, immediately before the \keyword{end} clause (or any
\keyword{catch} or \keyword{finally} clause).
The \keyword{end} clause is only reached when the loop is finally
completed.
\end{enumerate}
\index{,}

\chapter{Method instruction}\label{refmethod}
\index{,}
\index{Instructions,METHOD}
\index{,}
\index{Method,starting}
\begin{shaded}
\begin{alltt}
\textbf{method} \emph{name}[([\emph{arglist}])]
               [\emph{visibility}] [\emph{modifier}] [\textbf{protect}] [\textbf{binary}] [\textbf{deprecated}]
               [\textbf{returns} \emph{termr}]
               [\textbf{signals} \emph{signallist}];

where \emph{arglist} is a list of one or more \emph{assignment}s, separated by commas

and \emph{visibility} is one of:

    \textbf{inheritable}
    \textbf{private}
    \textbf{public}
    \textbf{shared}

and \emph{modifier} is one of:

    \textbf{abstract}
    \textbf{constant}
    \textbf{final}
    \textbf{native}
    \textbf{static}

and \emph{signallist} is a list of one or more \emph{term}s, separated by commas.
\end{alltt}
\end{shaded}
 The \keyword{method} instruction is used to introduce a method within
a class, as described in  \emph{Program structure} (see page \pageref{refpstruct}), and define its attributes.
The method must be given a \emph{name}, which must be a non-numeric
symbol.
\index{Short name,of methods}
\index{Method,short name of}
This is its \emph{short name}.
 
\index{Constructors,}
\index{Constructors,method}
\index{Methods,constructor}
If the short name of a method matches the short name of the class in
which it appears, it is a \emph{constructor method}.
Constructor methods are used for constructing values (objects), and are
described in detail in  \emph{Methods and Constructors} (see page \pageref{refmethcon}).
 
\index{Body,of methods}
\index{Method,body of}
The \emph{body} of the method consists of all clauses following the
method instruction (if any) until the next \keyword{method} or
\keyword{class} instruction, or the end of the program.
 
The \emph{visibility}, \emph{modifier}, and \keyword{protect}
keywords, and the \keyword{returns} and \keyword{signals} phrases, may
appear in any order.
\section{Arguments}
\index{Arguments,on METHOD instruction}
\index{Arguments,provided by caller}
\index{Methods,arguments of}
 
The \emph{arglist} on a \keyword{method} instruction, immediately
following the method name, is optional and defines a list of the
arguments for the method.  An \emph{argument} is a value that was
provided by the caller when the method was invoked.
 
If there are no arguments, this may optionally be indicated by an
"empty" pair of parentheses.
 
In the \emph{arglist}, each argument has the syntax of an
 \emph{assignment} (see page \pageref{refassign}) , where the "\textbf{=}"
and the following \emph{expression} may be omitted.
The name in the assignment provides the name for the argument (which
must not be the same as the name of any property in the class).
Each argument is also optionally assigned a type, or type and default
value, following the usual rules of assignment.
If there is no assignment, the argument is assigned the \nr{} string
type, \textbf{R\textsc{exx}}.
 
\index{Required arguments,}
\index{Arguments,required}
If there is no assignment (that is, there is no "\textbf{=}") or
the expression to the right of the "\textbf{=}" returns just a
type, the argument is \emph{required} (that is, it must always be
specified by the caller when the method is invoked).
 
\index{Optional arguments,}
\index{Arguments,optional}
If an explicit value is given by the expression then the argument is
\emph{optional}; when the caller does not provide an argument in that
position, then the expression is evaluated when the method is invoked
and the result is provided to the method as the argument.
 
Optional arguments may be omitted "from the right" only.
That is, arguments may not be omitted to the left of arguments that are
not omitted.
 \textbf{Examples:}
\begin{alltt}
method fred
method fred()
method fred(width, height)
method fred(width=int, height=int 10)
\end{alltt}
In these examples, the first two \keyword{method} instructions are
equivalent, and take no arguments.
The third example takes two arguments, which are both strings
of type \textbf{R\textsc{exx}}.
The final example takes two arguments, both of type \textbf{int}; the
second argument is optional, and if not supplied will default to the
value 10 (note that any valid expression could be used for the default
value).
\section{Visibility}
\index{PUBLIC,on METHOD instruction}
\index{PRIVATE,on METHOD instruction}
\index{INHERITABLE,on METHOD instruction}
\index{SHARED,on METHOD instruction}
\index{Visibility,of methods}
\index{Methods,private}
\index{Methods,public}
\index{Methods,inheritable}
\index{Methods,shared}
 
Methods may be \keyword{public}, \keyword{inheritable},
\keyword{private}, or \keyword{shared}:
\begin{itemize}
\item A \emph{public method} is visible to (that is, may be used by)
all other classes to which the current class is visible.
\item An \emph{inheritable method} is visible to (that is, may be used
by) all classes in the same package and also those classes that extend
(that is, are subclasses of) the current class.
\item A \emph{private method} is visible only within the current
class.
\item 
A \emph{shared method} is visible within the current package
but is not visible outside the package.  Shared methods cannot be
inherited by classes outside the package.
\end{itemize}
 
By default (\emph{i.e.}, if no visibility keyword is specified),  methods
are public.
\section{Modifier}
\index{ABSTRACT,on METHOD instruction}
\index{CONSTANT,on METHOD instruction}
\index{FINAL,on METHOD instruction}
\index{NATIVE,on METHOD instruction}
\index{STATIC,on METHOD instruction}
 
Most methods consist of instructions that follow the \keyword{method}
instruction and implement the method; the method is associated with an
object constructed by the class.
\index{Standard methods,}
\index{Methods,standard}
These are called \emph{standard methods}.
The \emph{modifier} keywords define that the method is not a
standard method - it is special in some way.
Only one of the following modifier keywords is allowed:
\begin{description}
\index{Abstract methods,}
\index{Methods,abstract}
\item[abstract]

An \emph{abstract method} has the name of the method and the types
(but not values) of its arguments defined, but no instructions to
implement the method are provided (or permitted).
 
If a class contains any abstract methods, an object cannot be
constructed from it, and so the class itself must be abstract; the
\keyword{abstract} keyword must be present on the
 \keyword{class} instruction (see page \pageref{refclass}) .
 
Within an interface class, the \keyword{abstract} keyword is optional on
the methods of the class, as all methods must be abstract.  No other
\emph{modifier} is allowed on the methods of an interface class.
\index{Constant methods,}
\index{Methods,constant}
\item[constant]

A \emph{constant method} is a static method that cannot be
overridden by a method in a subclass of the current class.
That is, it is both \keyword{final} and \keyword{static} (see below).
\index{Final methods,}
\index{Methods,final}
\item[final]

A \emph{final method} is considered to be complete; it cannot be
overridden by a subclass of the current class.  \keyword{private} methods
are implicitly \keyword{final}.
\footnote{
This modifier may allow compilers to improve the performance of methods
that are final, but may also reduce the reusability of the class.
}
\index{Native methods,}
\index{Methods,native}
\item[native]

A \emph{native method} is a method that is implemented by the
environment, not by instructions in the current class.
Such methods have no \nr{} instructions to implement the method (and
none are permitted), and they cannot be overridden by a method in a
subclass of the current class.
 
Native methods are used for accessing primitive operations provided by
the underlying operating system or by implementation-dependent packages.
\index{Static methods,}
\index{Methods,static}
\index{,}
\index{,}
\item[static]\label{refstatmet}

A \emph{static method} is a method that is not a constructor and is
associated with the class, rather than with an object constructed by the
class.
It cannot use properties directly, except those that are also static (or
constant).
 
Static methods may be invoked in the following ways:
\begin{enumerate}
\item Within the initialization expression of a static or constant
property (such methods are invoked when the class is first loaded).
\item By qualifying the name of the method with the name of its class
(qualified by the package name if necessary), for example, using
"\textbf{Math.Sin(1.3)}" or
"\textbf{java.lang.Math.Sin(1.3)}".
Methods called in this way are in effect \emph{functions}.
\item 
By implicitly qualifying the name by including the name of its class
in the \keyword{uses} phrase in the \keyword{class} instruction for the
current class.  Static methods in classes listed in this way can be used
directly, without qualification, for example, as
"\textbf{Sin(1.3)}".
They may be still be qualified, if preferred.
\end{enumerate}
 \emph{In the reference implementation, stand-alone applications are
started by the \textbf{java} command invoking a static method
called \textbf{main} which takes a single argument (of
type \textbf{java.lang.String[]}) and returns no result.
}
\end{description}
\section{Protect}
\index{PROTECT,on METHOD instruction}
\index{Protected methods,}
\index{Methods,protected}
 
The keyword \keyword{protect} indicates that the method protects the
current object (or class, for a static method) while the instructions
in the method are executed.
That is, the instructions in the method have exclusive access to the
object; if some other method (or construct) is executing in
parallel with the invocation of the method and is protecting the same
object then the method will not start execution until the object is no
longer protected.
 
Note that if a method or construct protecting an object invokes a method
(or starts a new construct) that protects the same object then execution
continues normally.  The inner method or construct is not prevented from
executing, because it is not executing in parallel.
\section{Binary}\label{refbinme}
\index{Binary classes,binary methods}
\index{METHOD instruction,}
\index{Instructions,METHOD}
\index{BINARY,on METHOD instruction}
\index{Binary methods,}
\index{Methods,binary}
 
The keyword \keyword{binary} indicates that the method is a \emph{binary
method}.
 
In binary methods, literal strings and numeric symbols are assigned
native string or binary (primitive) types, rather than \nr{} types,
and native binary operations are used to implement operators where
possible.
When \keyword{binary} is not in effect (the default), terms in
expressions are converted to \nr{} types before use by operators.
The section  \emph{Binary values and operations (see page \pageref{refbinary}) 
operations}:ea. describes the implications of binary methods and
classes in detail.
 \textbf{Notes:}
\begin{enumerate}
\item 
Only the instructions inside the body of the method are affected by the
\keyword{binary} keyword; any arguments and expressions on the method
instruction itself are not affected (this ensures that a single rule
applies to all the method signatures in a class).
\item 
All methods in a binary class are binary methods; the \keyword{binary}
keyword on methods is provided for classes in which only the occasional
method needs to be binary (perhaps for performance reasons).
It is not an error to specify \keyword{binary} on a method in a binary
class.
\end{enumerate}
\section{Deprecated}\label{refdepme}
\index{DEPRECATED,on METHOD instruction}
 
The keyword \keyword{deprecated} indicates that the method
is \emph{deprecated}, which implies that a better alternative is
available and documented.  A compiler can use this information to warn
of out-of-date or other use that is not recommended.
 
Note that individual methods in interface classes cannot be
deprecated; the whole class should be deprecated in this case.
\section{Returns}
\index{RETURNS,on METHOD instruction}
\index{Results,of methods}
\index{Methods,return values}
 
The \keyword{returns} keyword is followed by a term, \emph{termr},
that must evaluate to a type.
This type is used to define the type of values returned by
\keyword{return} instructions within the method.
 
The \keyword{returns} phrase is only required if the method is to return
values of a type that is not \nr{} strings (class \textbf{R\textsc{exx}}).
If \keyword{returns} is specified, all
 \keyword{return} instructions (see page \pageref{refreturn})  within the method must
specify an expression.
 \textbf{Example:}
\begin{alltt}
method filer(path, name) returns File
  return File(path||name)
\end{alltt}
This method always returns a value which is a \textbf{File} object.
\section{Signals}
\index{SIGNALS,on METHOD instruction}
\index{Exceptions,listed on METHOD instruction}
 
The \keyword{signals} keyword introduces a list of terms that evaluate to
types that are  Exceptions (see page \pageref{refexcep}) .
This list enumerates and documents the exceptions that are signalled
within the method (or by a method which is called from the current
method) but are not caught by a \keyword{catch} clause in a control
construct.
 \textbf{Example:}
\begin{alltt}
method soup(cat) signals IOException, DivideByZero
\end{alltt}
 
It is considered good programming practice to use this list to document
"unusual" exceptions signalled by a method.
Implementations that support the concept of \emph{checked exceptions}(see page \pageref{refchecked}) must report as an error any checked exception that is
incorrectly included in the list (that is, if the exception is never
signalled or would always be caught).  Such implementations may also
offer an option that enforces the listing of all or some checked
exceptions.
\section{Duplicate methods}
\index{Methods,duplicate}
\index{Methods,overloading}
\index{Duplicate methods,}
\index{Overloaded methods,}
 
Methods may not duplicate properties or other methods in the same class.
Specifically:
\begin{itemize}
\item 
The short name of a method must not be the same as the name of any
property in the same class.
\item 
The number (zero or more) and types of the arguments of a method (or any
subset permitted by omitting optional arguments) must not be the same as
those of any other method of the same name in the class (also checking
any subset permitted by omitting optional arguments).
\end{itemize}
Note that the second rule does allow multiple methods with the same
name in a class, provided that the number of arguments differ or
at least one argument differs in type.

\chapter{Nop instruction}
\index{NOP instruction,}
\index{Instructions,NOP}
\index{Null instruction, NOP,}
\index{Dummy instruction, NOP,}
\begin{shaded}
\begin{alltt}
\textbf{nop};
\end{alltt}
\end{shaded}
 \keyword{nop} is a dummy instruction that has no effect.  It can be
useful as an explicit "do nothing" instruction following a
\keyword{then} or \keyword{else} clause.
 \textbf{Example:}
\begin{alltt}
select
  when a=b then nop           -- Do nothing
  when a>b then say 'A > B'
  otherwise     say 'A < B'
  end
\end{alltt}
\textbf{Note: }Putting an extra semicolon instead of the \keyword{nop} would
merely insert a null clause, which would just be ignored by \nr{}.
The second \keyword{when} clause would then immediately follow the
\keyword{then}, and hence would be reported as an error.
\keyword{nop} is a true instruction, however, and is therefore a valid
target for the \keyword{then} clause.

\chapter{Numeric instruction}\label{"id"}
\index{NUMERIC,instruction}
\index{Instructions,NUMERIC}
\index{Arithmetic,NUMERIC settings}
\index{,}
\index{FORM,option of NUMERIC instruction}
\index{SCIENTIFIC value for NUMERIC FORM,}
\index{ENGINEERING value for NUMERIC FORM,}
\index{DIGITS,on NUMERIC instruction}
\begin{shaded}
\begin{alltt}
\textbf{numeric digits} [\emph{exprd}];
                \textbf{form} [\textbf{scientific}];
                          [\textbf{engineering}];
\end{alltt}
\end{shaded}
 The \texttt{numeric} instruction is used to change the way in which
arithmetic operations are carried out by a program.
The effects of this instruction are described in detail in the
section on  \emph{Numbers and Arithmetic} (see page \pageref{refnums}) .
\begin{description}
\item[numeric digits]\label{refndigits}

controls the precision under which arithmetic operations will be
[\%book
 evaluated (see page \pageref{refndi2}) .
[\%www
evaluated.
[\%odt
evaluated.
If no expression \emph{exprd} is given then the default value of 9
is used.
Otherwise the result of the expression is rounded, if necessary,
according to the current setting of \texttt{numeric digits} before it is
used.
The value used must be a positive whole number.
 There is normally no limit to the value for \texttt{numeric digits}
(except the constraints imposed by the amount of storage and other
resources available) but note that high precisions are likely to be
expensive in processing time.
It is recommended that the default value be used wherever possible.
 
Note that small values of \texttt{numeric digits} (for example, values
less than 6) are generally only useful for very specialized applications.
The setting of \texttt{numeric digits} affects all computations, so even
the operation of loops may be affected by rounding if small values are
used.
 
If an implementation does not support a requested value for \texttt{numeric
digits} then the instruction will fail with an exception (which may,
as usual, be caught with the \texttt{catch} clause of a control
construct).
 
The current setting of \texttt{numeric digits} may be retrieved with the
 \textbf{digits} special word (see page \pageref{refswdigit}) .
\item[numeric form]\label{refnform}
\index{Exponential notation,}

controls which form of  exponential notation (see page \pageref{refnfo2})  is to
be used for the results of operations.
\index{Notation,scientific}
\index{Notation,engineering}
\index{Scientific notation,}
\index{Engineering notation,}
This may be either \emph{scientific} (in which case only one,
non-zero, digit will appear before the decimal point), or
\emph{engineering} (in which case the power of ten will always be a
multiple of three, and the part before the decimal point will be in the
range 1 through 999).
The default notation is scientific.
 The form is set directly by the sub-keywords \texttt{scientific} or
\texttt{engineering}; if neither sub-keyword is given,
\texttt{scientific} is assumed.
The current setting of \texttt{numeric form} may be retrieved with the
 \textbf{form} special word (see page \pageref{refswform}) .
 
If an implementation does not support a requested value for \texttt{numeric
form} then the instruction will fail with an exception (which may,
as usual, be caught with the \texttt{catch} clause of a control
construct).
\end{description}
 
The \texttt{numeric} instruction may be used where needed as a
dynamically executed instruction in a method.
 
It may also appear, more than once if necessary, before the first
method in a class, in which case it forms the default setting for the
initialization of subsequent properties in the class and for all methods
in the class.  In this case, any exception due to the \texttt{numeric}
instruction is raised when the class is first loaded.
 
Further, one or more \texttt{numeric} instructions may be placed
before the first \texttt{class} instruction in a program; they do
not imply the start of a class.  The numeric settings then apply
to all classes in the program (except interface classes), as
though the \texttt{numeric} instructions were placed immediately
following the \texttt{class} instruction in each class (except that
they will not be traced).

\chapter{Options instruction}\label{"id"}
\index{OPTIONS,instruction}
\index{Instructions,OPTIONS}
\begin{shaded}
\begin{alltt}
\textbf{options} \emph{wordlist};

where \emph{wordlist} is one or more \emph{symbol}s separated by blanks.
\end{alltt}
\end{shaded}
\index{System-dependent options,}
\index{Compiler options,}
\index{Interpreter options,}
\index{Language processor options,}
 
The \texttt{options} instruction is used to pass special requests to
the language processor (for example, an interpreter or compiler).
 
\index{Option words,}
Individual words, known as \emph{option words}, in the
\emph{wordlist} which are meaningful to the language processor will
be obeyed (these might control optimizations, enforce standards, enable
implementation-dependent features, \&); those which are not
recognized will be ignored (they are assumed to be instructions to a
different language processor).
Option words in the list that are known will be recognized independently
of case.
 
There may be zero or more \texttt{options} instructions in a program.
They apply to the whole program, and must come before the first
\texttt{class} instruction (or any instruction that starts a class).
 
\emph{In the reference implementation, the known option words are:}
\index{Option words,}
\begin{description}
\item{binary}
\index{BINARY,in OPTIONS instruction}
\index{NOBINARY option,}
\emph{All classes in this program will be  binary (see page \pageref{refbincla}) 
classes:ea..
In binary classes, literals are assigned binary (primitive) or native
string types, rather than NetRexx types, and native binary operations
are used to implement operators where appropriate, as described in
 "\emph{Binary values and operations}" (see page \pageref{refbinary}) .
In classes that are not binary, terms in expressions are converted to
the NetRexx string type, \textbf{Rexx}, before use by operators.}
\item{comments}
\index{COMMENTS option,}
\index{NOCOMMENTS option,}

\emph{Comments from the NetRexx source program will be passed through to the
the Java output file (which may be saved with a \textbf{.java.keep}
extension by using the \texttt{-keep} command option).}
 
\emph{Line comments become Java line comments (introduced by
"\textbf{//}").
Block comments become Java block comments (delimited by
"\textbf{/*}" and "\textbf{*/}"), with nested block
comments having their delimiters changed to "\textbf{(-}" and
"\textbf{-)}").}
\item{compact}
\index{COMPACT option,}
\index{NOCOMPACT option,}

\emph{Requests that warnings and error messages be displayed in compact
form.  This format is more easily parsed than the default format, and
is intended for use by editing environments.}
 
\emph{Each error message is presented as a single line, prefixed with the
error token identification enclosed in square brackets.
The error token identification comprises three words, with one blank
separating the words.  The words are: the source file specification, the
line number of the error token, the column in which it starts, and its
length.  For example (all on one line):}
\begin{alltt}
[D:\test\test.nrx 3 8 5] Error: The external name
'class' is a Java reserved word, so would not be
usable from Java programs
\end{alltt}
\emph{Any blanks in the file specification are replaced by a null
(:m.'\\0':em.) character.  Additional words could be added to the error
token identification later.}
\item{console}
\index{CONSOLE option,}
\index{NOCONSOLE option,}

\emph{Requests that compiler messages be written to console (the default).
Use \texttt{-noconsole} to prevent messages being written to the
console.}
 
\emph{This option only has an effect as a compiler option, and applies to all
programs being compiled.}
\item{crossref}
\index{CROSSREF option,}
\index{NOCROSSREF option,}
\emph{Requests that cross-reference listings of variables be prepared,
by class.}
\item{decimal}
\index{DECIMAL option,}
\index{NODECIMAL option,}

\emph{Decimal arithmetic may be used in the program.  If \texttt{nodecimal} is
specified, the language processor will report operations that use (or,
like normal string comparison, might use) decimal arithmetic as an
error.  This option is intended for performance-critical programs where
the overhead of inadvertent use of decimal arithmetic is
unacceptable.}
\item{diag}
\index{DIAG option,}
\index{NODIAG option,}
\emph{Requests that diagnostic information (for experimental use only)
be displayed.
The \texttt{diag} option word may also have side-effects.}
\item{explicit}
\index{EXPLICIT option,}
\index{NOEXPLICIT option,}

\emph{Requires that all local variables must be explicitly declared (by
assigning them a type but no value) before assigning any value to them.
This option is intended to permit the enforcement of "house styles"
(but note that the NetRexx compiler always checks for variables which
are referenced before their first assignment, and warns of variables
which are set but not used).}
\item{format}
\index{FORMAT,option}
\index{NOFORMAT option,}
\emph{Requests that the translator output file (Java source code) be
formatted for improved readability.
Note that if this option is in effect, line numbers from the input file
will not be preserved (so run-time errors and exception trace-backs may
show incorrect line numbers).}
\item{java}
\index{JAVA option,}
\index{NOJAVA option,}

\emph{Requests that Java source code be produced by the translator.
If \texttt{nojava} is specified, no Java source code will be produced;
this can be used to save a little time when checking of a program is
required without any compilation or Java code resulting.}
\item{logo}
\index{LOGO option,}
\index{NOLOGO option,}
\emph{Requests that the language processor display an introductory
logotype sequence (name and version of the compiler or interpreter,
\&).}
\item{replace}
\index{REPLACE option,}
\index{NOREPLACE option,}
\emph{Requests that replacement of the translator output
(\textbf{.java}) file be allowed.
The default, \texttt{noreplace}, prevents an existing \textbf{.java}
file being accidentally overwritten.}
\item{savelog}
\index{SAVELOG option,}
\index{NOSAVELOG option,}

\emph{Requests that compiler messages be written to the file NetRexxC.log in
the current directory.
The messages are also displayed on the console, unless
\texttt{-noconsole} is specified.}
 
\emph{This option only has an effect as a compiler option, and applies to all
programs being compiled.}
\item{sourcedir}
\index{SOURCEDIR option,}
\index{NOSOURCEDIR option,}

\emph{Requests that all \textbf{.class} files be placed in the same
directory as the source file from which they are compiled.  Other output
files are already placed in that directory.
Note that using this option will prevent the \texttt{-run} command
option from working unless the source directory is the current
directory.}
\item{strictargs}
\index{STRICTARGS option,}
\index{NOSTRICTARGS option,}

\emph{Requires that method invocations always specify parentheses, even
when no arguments are supplied.  Also, if \texttt{strictargs} is in
effect, method arguments are checked for usage - a warning is given
if no reference to the argument is made in the method.}
\item{strictassign}
\index{STRICTASSIGN option,}
\index{NOSTRICTASSIGN option,}
\emph{Requires that only exact type matches be allowed in assignments
(this is stronger than Java requirements).
This also applies to the matching of arguments in method calls.}
\item{strictcase}
\index{STRICTCASE option,}
\index{NOSTRICTCASE option,}
\emph{Requires that local and external name comparisons for variables,
properties, methods, classes, and special words match in case (that is,
names must be identical to match).}
\item{strictimport}
\index{STRICTIMPORT option,}
\index{NOSTRICTIMPORT option,}

\emph{Requires that all imported packages and classes be imported
explicitly using \texttt{import} instructions.  That is, if in effect,
there will be no  automatic imports (see page \pageref{refimport}) , except those
related to the \texttt{package} instruction.}
 
\emph{This option only has an effect as a compiler option, and applies to all
programs being compiled.}
\item{strictprops}
\index{STRICTPROPS option,}
\index{NOSTRICTPROPS option,}

\emph{Requires that all properties, including those local to the
current class, be qualified in references.
That is, if in effect, local properties cannot appear as simple names
but must be qualified by :m.this.:em. (or equivalent) or the class name
(for static properties).}
\item{strictsignal}
\index{STRICTSIGNAL option,}
\index{NOSTRICTSIGNAL option,}
\emph{Requires that all  checked exceptions (see page \pageref{refchecked}) 
signalled within a method but not caught by a \texttt{catch} clause be
listed in the \texttt{signals} phrase of the method instruction.}
\item{symbols}
\index{SYMBOLS option,}
\index{NOSYMBOLS option,}

\emph{Symbol table information (names of local variables, \&)
will be included in any generated \textbf{.class} file.
This option is provided to aid the production of classes that are easy
to analyse with tools that can understand the symbol table information.
The use of this option increases the size of \textbf{.class}
files.}
\item{trace, traceX}
\index{TRACE,option}
\index{NOTRACE option,}

\emph{If given as \texttt{trace}, \texttt{trace1}, or \texttt{trace2},
then \texttt{trace} instructions are accepted.  The trace output is
directed according to the option word: \texttt{trace1} requests
that trace output is written to the standard output stream,
\texttt{trace} or \texttt{trace2} imply that the output should be
written to the standard error stream (the default).}
 
\emph{If \texttt{notrace} is given, then trace instructions are ignored.
The latter can be useful to prevent tracing overheads while leaving
\texttt{trace} instructions in a program.}
\item{utf8}
\index{Unicode,UTF-8 encoding}
\index{UTF-8 encoding,}
\index{UTF8 option,}
\index{NOUTF8 option,}
\emph{If given, clauses following the \texttt{options} instruction are
expected to be encoded using UTF-8, so all Unicode characters may be
used in the source of the program.}
 
\emph{In UTF-8 encoding, Unicode characters less than \textbf{'\u0080'} are
represented using one byte (whose most-significant bit is 0), characters
in the range \textbf{'\u0080'} through \textbf{'\u07FF'} are encoded
as two bytes, in the sequence of bits:}
\begin{alltt}
110xxxxx 10xxxxxx
\end{alltt}
\emph{where the eleven digits shown as \textbf{x} are the least
significant eleven bits of the character, and characters in the
range \textbf{'\u0800'} through \textbf{'\\uFFFF'} are encoded as
three bytes, in the sequence of bits:}
\begin{alltt}
1110xxxx 10xxxxxx 10xxxxxx
\end{alltt}
\emph{where the sixteen digits shown as \textbf{x} are the
sixteen bits of the character.}
 
\emph{If \texttt{noutf8} is given, following clauses are assumed to comprise
only Unicode characters in the range \textbf{'\\x00'}
through \textbf{'\\xFF'}, with the more significant byte of the
encoding of each character being 0.}
 
\emph{:b.Note::eb.
this option only has an effect as a compiler option, and applies to
all programs being compiled.  If present on an \texttt{options}
instruction, it is checked and must match the compiler option (this
allows processing with or without \texttt{utf8} to be
enforced).}
\item{verbose, verboseX}
\index{VERBOSE option,}
\index{VERBOSEn option,}
\index{NOVERBOSE option,}
\emph{Sets the "noisiness" of the language processor.
The digit \textbf{\emph{X}} may be any of the digits \textbf{0}
through \textbf{5}; if omitted, a value of \textbf{3} is used.
The options \texttt{noverbose} and \texttt{verbose0} both suppress all
messages except errors and warnings.}
\end{description}
 
\emph{Prefixing any of the above with "\texttt{no}" turns the selected
option off.}
 \emph{Example:}
\begin{alltt}
options binary nocrossref nostrictassign strictargs
\end{alltt}
 
\emph{The default settings of the various options are:}
\begin{alltt}
nobinary nocomments nocompact console crossref decimal nodiag noexplicit
noformat java logo noreplace nosavelog nosourcedir nostrictargs
nostrictassign nostrictcase nostrictimport nostrictprops nostrictsignal
nosymbols trace2 noutf8 verbose3
\end{alltt}
 
\emph{When an option word is repeated (in the same \texttt{options}
instruction or not), or conflicting option words are specified, then the
last use determines the state of the option.}
 
\index{Command line options,}
\index{Options,on command line}
\emph{All option words may also be set as command line options when invoking
the processor, by prefixing them with "\textbf{-}":}
 \emph{Example:}
\begin{alltt}
java COM.ibm.netrexx.process.NetRexxC -format foo.nrx
\end{alltt}
\emph{In this case, any options may come before, after, or between
file specifications.}
 
\emph{With the except of the \texttt{utf8} option (see above),
options set with the \texttt{options} instruction override command-line
settings, following the "last use" rule.}
 
\emph{For more information, see the installation and user documentation for
your implementation.
}

\section{Package instruction}\label{refpackage}
\index{PACKAGE instruction,}
\index{Instructions,PACKAGE}
\index{Package,}
\begin{shaded}
\begin{alltt}
\textbf{package} \emph{name};

where \emph{name} is one or more non-numeric \emph{symbol}s separated by periods.
\end{alltt}
\end{shaded}
 
\index{Class,package of}
The \keyword{package} instruction is used to define the package to which
the class or classes in the current program belong.
 Classes that belong to the same package have privileged access to
other classes in the same package, in that each class is visible to all
other classes in the same package, even if not declared public.
Packages also conveniently group classes for use by the
 \keyword{import} instruction (see page \pageref{refimport}) .
 
\index{Package,name of}
The \emph{name} must specify a \emph{package name}, which is one
or more non-numeric symbols, separated by periods, with no
blanks.
 
There must be at most one \keyword{package} instruction in a program.
It must precede any \keyword{class} instruction (or any instruction that
would start the default class).
 
If a program contains no \keyword{package} instruction then its package
is implementation-defined.  Typically it is grouped with other programs
in some implementation-defined logical collection, such as a directory
in a file system.
 \textbf{Examples:}
\begin{lstlisting}
package testpackage
package com.ibm.venta
\end{lstlisting}
 
\index{Fully-qualified name, of classes,}
\index{Qualified name, of classes,}
\index{Class,qualified name of}
When a class is identified as belonging to a package, it has a
\emph{qualified class name}, which is its short name, as given on the
 \keyword{class} instruction (see page \pageref{refclass}) , prefixed with the package
name and a period.
For example, if the short name of a class is
"\textbf{RxLanguage}" and the package name is
"\textbf{com.ibm.venta}" then the qualified name of the class
would be "\textbf{com.ibm.venta.RxLanguage}".
 
\emph{In the reference implementation, packages are kept in a hierarchy
derived from the Java classpath, where the segments of a package name
correspond to a path in the hierarchy.
The hierarchy is typically the directories in a file system, or some
equivalent (such as a "Zip" archive file), and so package names
should be considered case-sensitive (as some Java implementations use
case-sensitive file systems).
}

\chapter{Parse instruction}\label{refparse}
\index{PARSE,instruction}
\index{Instructions,PARSE}
\index{Parsing templates,in PARSE instruction}
\index{Templates, parsing,in PARSE instruction}
\index{R\textsc{exx},class/use by PARSE}
\begin{shaded}
\begin{alltt}
\textbf{parse} \emph{term} \emph{template};

where \emph{template} is one or more non-numeric \emph{symbol}s
separated by blanks and/or \emph{pattern}s, and a \emph{pattern} is one of:

    \emph{literalstring}
    [\emph{indicator}] \emph{number}
    [\emph{indicator}] (\emph{symbol})

and \emph{indicator} is one of \textbf{+}, \textbf{-}, or \textbf{=}.
\end{alltt}
\end{shaded}
 The \keyword{parse} instruction is used to assign characters (from a
string) to one or more variables according to the rules and templates
described in the section \emph{Parsing templates} (see page \pageref{refparsing}).
\index{Variables,parsing of}
\index{Terms,parsing of}
 
The value of the \emph{term} is expected to be a string; if it is
not a string, it will be converted to a string.
 
Any variables used in the \emph{template} are named by non-numeric
\emph{symbol}s (that is, they cannot be an array reference or other
term); they refer to a variable or property in the current class.
Any values that are used in patterns during the parse are converted to
strings before use.
 
Any variables set by the \keyword{parse} instruction must have a known
string type, or are given the \nr{} string type, \textbf{R\textsc{exx}}, if
they are new.
 
The term itself is not changed unless it is a variable which also
appears in the template and whose value is changed by being in the
template.

\textbf{Example:}
\begin{lstlisting}
parse wordlist word1 wordlist
\end{lstlisting}
In this idiomatic example, the first word is removed
from \textbf{wordlist} and is assigned to the
variable \textbf{word1}, and the remainder is assigned back
to \textbf{wordlist}.

\textbf{Notes:}
\begin{enumerate}
\item 
The special words \keyword{ask}, \keyword{source}, and \keyword{version},
as described in the section  \emph{Special names and methods}(see page \pageref{refspecial}), allow:
\begin{lstlisting}
parse ask x     -- parses a line from input stream
parse source x  -- parses 'Java method \emph{filename}'
parse version x -- parses '\nr{} \emph{version} \emph{date}'
\end{lstlisting}
These special words may also be used within expressions.
\item 
Similarly, it is recommended that the initial (main) method in a
stand-alone application place the command string passed to it in a
variable called \textbf{arg}.
\footnote{
\emph{In the reference implementation, this is automatic if
the \textbf{main} method is generated by the \nr{} language
processor.}
}
 
If this is done, the instruction:
\begin{lstlisting}
parse arg template
\end{lstlisting}
will work, in a stand-alone application, in the same way as in R\textsc{exx}
(even though \textbf{arg} is not a keyword in this case).
\footnote{
Note, though, that the command string may have been edited by the
environment; certain characters may not be allowed, multiple blanks may
have been reduced to single blanks, \emph{etc.}
}
\end{enumerate}

\chapter{Properties instruction}\label{"id"}
\index{PROPERTIES instruction,}
\index{Instructions,PROPERTIES}
\index{Properties,naming}
\begin{shaded}
\begin{alltt}
\textbf{properties} [\emph{visibility}] [\emph{modifier}] [\textbf{deprecated}] [\textbf{unused}];

where \emph{visibility} is one of:

    \textbf{inheritable}
    \textbf{private}
    \textbf{public}
    \textbf{shared}

and \emph{modifier} is one of:

    \textbf{constant}
    \textbf{static}
    \textbf{transient}
    \textbf{volatile}

and there must be at least one \emph{visibility} or \emph{modifier} keyword.
\end{alltt}
\end{shaded}
\index{Properties,}
 
The \texttt{properties} instruction is used to define the attributes
of following \emph{property} variables, and therefore must precede the
first \texttt{method} instruction in a class.
A \texttt{properties} instruction replaces any previous
\texttt{properties} instruction (that is, the attributes specified on
\texttt{properties} instructions are not cumulative).
 
The \emph{visibility}, \emph{modifier},
\texttt{deprecated}, and \texttt{unused} keywords may be in any
order.
 
An example of the use of \texttt{properties} instructions may be
found in the  \emph{Program Structure} section (see page \pageref{refpstruct}) .
\subsubsection{Visibility}
\index{Properties,visibility}
\index{Properties,inheritable}
\index{Properties,public}
\index{Properties,private}
\index{Properties,shared}
\index{INHERITABLE,on PROPERTIES instruction}
\index{PUBLIC,on PROPERTIES instruction}
\index{PRIVATE,on PROPERTIES instruction}
\index{SHARED,on PROPERTIES instruction}
\index{Visibility,of properties}
 
Properties may be \texttt{public}, \texttt{inheritable},
\texttt{private}, or \texttt{shared}:
\footnote{
An experimental option for \emph{visibility}, \texttt{indirect},
is described in  Appendix B (see page \pageref{refappb}) .
}
\begin{itemize}
\item A \emph{public property} is visible to (that is, may be used by)
all other classes to which the current class is visible.
\item An \emph{inheritable property} is visible to (that is, may be used
by) all classes in the same package and also those classes that extend
(that is, are subclasses of) the current class, and which qualify the
property using an object of the subclass, or either \texttt{this}
or \texttt{super}.
\item A \emph{private property} is visible only within the current
class.
\item 
A \emph{shared property} is visible within the current package
but is not visible outside the package.  Shared properties cannot be
inherited by classes outside the package.
\end{itemize}
 
By default, if no \texttt{properties} instruction is used,
or \emph{visibility} is not specified, properties
are inheritable (but not public).
\footnote{
The default, here, was chosen to encourage the "encapsulation" of
data within classes.
}
\subsubsection{Modifier}\label{"id"}
\index{CONSTANT,on PROPERTIES instruction}
\index{STATIC,on PROPERTIES instruction}
\index{TRANSIENT,on PROPERTIES instruction}
\index{VOLATILE,on PROPERTIES instruction}
\index{Properties,modifiers}
\index{Properties,constant}
\index{Properties,static}
\index{Properties,transient}
\index{Properties,volatile}
\index{Constants,using properties}
\index{Constants,}
 
Properties may also be \texttt{constant}, \texttt{static},
\texttt{transient}, or \texttt{volatile}:
\begin{itemize}
\item 
A \emph{constant property} is associated with the class, rather
than with an instance of the class (an object).
It is initialized when the class is loaded and may not be changed
thereafter.
\item 
A \emph{static property} is associated with the class, rather
than with an instance of the class (an object).
It is initialized when the class is loaded, and may be changed
thereafter.
\item 
A \emph{transient property} is a property which should not be saved when
an instance of the class is saved (made persistent).
\item 
A \emph{volatile property} may change asynchronously, outside the
control of the class, even when no method in the class is being
executed.
If an implementation does not allow asynchronous modification of
properties, it should ignore this keyword.
\end{itemize}
 
Constant and static properties exist from when the class is first loaded
(used), even if no object is constructed by the class, and there will
only be one copy of each property.  Other properties are constructed and
initialized only when an object is constructed by the class; each object
then has its own copy of such properties.
 
By default, if no \texttt{properties} instruction is used, or
\emph{modifier} is not specified, properties are associated with an
object constructed by the class.
\subsubsection{Deprecated}\label{"id"}
\index{DEPRECATED,on PROPERTIES instruction}
\index{Properties,deprecated}
 
The keyword \texttt{deprecated} indicates that any property introduced by
this instruction is \emph{deprecated}, which implies that a
better alternative is available and documented.  A compiler can
use this information to warn of out-of-date or other use that is
not recommended.
\subsubsection{Unused}\label{"id"}
\index{UNUSED,on PROPERTIES instruction}
\index{Properties,unused}
 
The keyword \texttt{unused} indicates that the private properties
which follow are not referenced explicitly in the code for the class,
and so a language processor should not warn that they exist but have not
been used.
If a \emph{visibility} keyword is specified it must be
\texttt{private}.
 
For example:
\begin{alltt}
properties private constant unused
  -- Serialization version
  serialVersionUID=long 8245355804974198832
\end{alltt}
\subsubsection{Properties in interface classes}
\index{Properties,in interface classes}
\index{Interface classes,properties in}
 
In  interface classes (see page \pageref{refinterf}) , properties must be both
\texttt{public} and \texttt{constant}.  In such classes, these
attributes for properties are the default and the \texttt{properties}
instruction must not be used.

\section{Return instruction}\label{refreturn}
\index{RETURN instruction,}
\index{Instructions,RETURN}
\index{Functions,return from}
\index{Internal functions,return from}
\index{Subroutines,return from}
\index{Subroutines,passing back values from}
\begin{shaded}
\begin{alltt}
\textbf{return} [\emph{expression}];
\end{alltt}
\end{shaded}
\index{Results,returned by RETURN}
 \keyword{return} is used to return control (and possibly a result)
from a \nr{} program or method to the point of its invocation.
 
The expression (if any) is evaluated, active control constructs are
terminated (as though by a \keyword{leave} instruction), and the value of
the expression is passed back to the caller.
 
The result passed back to the caller is a string of type \code{Rexx},
unless a different type was specified using the \keyword{returns} keyword
on the  \keyword{method} instruction (see page \pageref{refmethod})  for the current
method.
In this case, the type of the value of the expression must match (or be
convertible to, as by the rules for assignment) the type specified by
the \keyword{returns} phrase.
 
Within a method, the use of expressions on \keyword{return} must be
consistent.  That is, either all \keyword{return} instructions must
specify a expression, or none may.
If a \keyword{returns} phrase is given on the \keyword{method} instruction
for the current method then all \keyword{return} instructions must
specify an expression.

\chapter{Say instruction}\label{refsay}
\index{SAY,instruction}
\index{Instructions,SAY}
\index{,}
\index{,}
\index{Console, writing to with SAY,}
\index{Terminal, writing to with SAY,}
\index{stdout, writing to with SAY,}
\begin{shaded}
\begin{alltt}
\textbf{say} [\emph{expression}];
\end{alltt}
\end{shaded}
 \keyword{say} writes a string to the default output character
stream.
This typically causes it to be displayed (or spoken, or typed, \emph{etc.}) to
the user.

\textbf{Example:}
\begin{alltt}
data=100
say data 'divided by 4 =>' data/4
/* would display:  "100 divided by 4 => 25"  */
\end{alltt}
 
The result of evaluating the \emph{expression} is expected to be a
string; if it is not a string, it will be converted to a string.
This result string is written from the program via an
implementation-defined output stream.
 
\index{Line, displaying,}
\begin{shaded}\noindent
By default, the result string is treated as a "line" (an
implementation-dependent mechanism for indicating line termination is
effected after the string is written).
If, however, the string ends in the NUL character
(\textbf{'\textbackslash -'} or \textbf{'\textbackslash 0'}) then that character
is removed and line termination is not indicated.
\end{shaded}\indent
The result string may be of any length.  If no expression is specified,
or the expression result is \textbf{null}, then an empty line is
written (that is, as though the expression resulted in a null string).

\section{Select instruction}
\index{,}
\index{Instructions,SELECT}
\index{THEN,following WHEN clause}
\index{,}
\index{,}
\index{,}
\index{Flow control,with SELECT construct}
\begin{shaded}
\begin{alltt}
\textbf{select} [\textbf{label} \emph{name}] [\textbf{protect} \emph{term}] [\textbf{case} \emph{expression}];
        \emph{whenlist}
        [\textbf{otherwise}[;] \emph{instructionlist}]
    [\textbf{catch} [\emph{vare} =] \emph{exception};
        \emph{instructionlist}]...
    [\textbf{finally}[;]
        \emph{instructionlist}]
    \textbf{end} [\emph{name}];

where \emph{name} is a non-numeric \emph{symbol}

and \emph{whenlist} is one or more \emph{whenconstruct}s

and \emph{whenconstruct} is:

    \textbf{when} \emph{expression}[, \emph{expression}]... [;] \textbf{then}[;] \emph{instruction}

and \emph{instructionlist} is zero or more \emph{instruction}s.
\end{alltt}
\end{shaded}
\index{Body,of select}
 \keyword{select} is used to conditionally execute one of several
alternatives.
The construct may optionally be given a label, and may protect an object
while the instructions in the construct are executed; exceptional
conditions can be handled with \keyword{catch} and \keyword{finally},
which follow the body of the construct.
 
Starting with the first \keyword{when} clause, each expression in
the clause is evaluated in turn from left to right, and if the
result of any evaluation is 1 (or equals the \keyword{case}
expression, see below) then the test has succeeded and the
instruction following the associated \keyword{then} (which may be
a complex instruction such as \keyword{if}, \keyword{do},
\keyword{loop}, or \keyword{select}) is executed and control will
then pass directly to the \keyword{end}.
 
If the result of all the expressions in a \keyword{when} clause
is 0, control will pass to the next \keyword{when} clause.
 
Note that once an expression evaluation in a \keyword{when}
clause has resulted in a successful test, no further expressions
in the clause are evaluated.
 
If none of the \keyword{when} expressions result in 1, then control will
pass to the instruction list (if any) following \keyword{otherwise}.
In this situation, the absence of an \keyword{otherwise} is a run-time
error.
\footnote{
\emph{In the reference implementation, a \textbf{NoOtherwiseException}
is raised.}
}
 \textbf{Notes:}
\begin{enumerate}
\item An \emph{instruction} may be any assignment, method call, or keyword
instruction, including any of the more complex constructions such as
\keyword{do}, \keyword{loop}, \keyword{if}, and the \keyword{select}
instruction itself.
A null clause is not an instruction, however, so putting an extra
semicolon after the \keyword{then} is not equivalent to putting a dummy
instruction (as it would be in C or PL/I).
The \keyword{nop} instruction is provided for this purpose.
\item The keyword \keyword{then} is treated specially, in that it need not
start a clause.
This allows the expression on the \keyword{when} clause to be terminated
by the \keyword{then}, without a "\textbf{;}" being required
- were this not so, people used to other computer languages would
be inconvenienced.
Hence the symbol \keyword{then} cannot be used as a variable name within
the expression.
\footnote{
Strictly speaking, \keyword{then} should only be recognized if not
the name of a variable.  In this special case, however, \nr{} language
processors are permitted to treat \keyword{then} as reserved in the
context of a \keyword{when} clause, to provide better performance and
more useful error reporting.
}
\end{enumerate}
\subsection{Label phrase}
\index{Select,label}
\index{Select,naming of}
 
\index{LABEL,on SELECT instruction}
If \keyword{label} is used to specify a \emph{name} for the select
group, then a  \keyword{leave} instruction (see page \pageref{refleave})  which
specifies that name may be used to leave the group, and the \keyword{end}
that ends the group may optionally specify the name of the group for
additional checking.
 \textbf{Example:}
\begin{lstlisting}
select label roman
  when a=b then say 'same'
  when a<b then say 'lo'
  otherwise
    say 'hi'
    if a=0 then leave roman
    say 'a non-0'
end roman
\end{lstlisting}
In this example, if the variable \textbf{a} has the value 0
and \textbf{b} is negative then just "\textbf{hi}" is
displayed.
\subsection{Protect phrase}
 
\index{PROTECT,on SELECT instruction}
If \keyword{protect} is given it must be followed by a \emph{term}
that evaluates to a value that is not just a type and is not of a
primitive type;
while the \keyword{select} construct is being executed, the value
(object) is protected - that is, all the instructions in the
\keyword{select} construct have exclusive access to the object.
 
Both \keyword{label} and \keyword{protect} may be specified, in any order,
if required.
\subsection{Case phrase}
 
\index{CASE,on SELECT instruction}
If \keyword{case} is given it must follow any \keyword{label} or
\keyword{protect} phrase, and must be followed by an
\emph{expression}.
 
When \keyword{case} is used, the expression following it is evaluated at
the start of the \keyword{select} construct.
The result of the expression is then compared, using the strict equality
operator (\texttt{==}), to the result of evaluating the expression
or expressions in each of the \keyword{when} clauses in turn until
a match is found.  As usual, if no match is found then control
will pass to the instruction list (if any) following
\keyword{otherwise}, and in this situation the absence of an
\keyword{otherwise} is a run-time error.
 For example, in:
\begin{lstlisting}
select case i+1
  when 1 then say 'one'
  when 1+1 then say 'two'
  when 3, 4, 5 then say 'many'
end
\end{lstlisting}
then if \texttt{i} had the value 1 then the message displayed would be
"\texttt{two}".
 
The third \keyword{when} clause in the example demonstrates the use of the
multiple expressions in a \keyword{when} clause in this context.
Similar to a \keyword{select} without \keyword{case}, each
expression is evaluated in turn from left to right and is then
compared to the result of the \keyword{case} expression.
As soon as one matches that result, execution of the
\keyword{when} clause stops (any further expressions are not
evaluated) and the instruction following the associated
\keyword{then} clause is executed.

\textbf{Notes:}
\begin{enumerate}
\item When \keyword{case} is used, the result of evaluating the expression
following each \keyword{when} no longer has to be 0 or 1.  Instead, it
must be possible to compare each result to the result of the
\keyword{case} expression.
\item 
The \keyword{case} expression is evaluated only on entry to the
\keyword{select} construct; it is not re-evaluated for each \keyword{when}
clause.
\item 
An exception raised during evaluation of the \keyword{case} expression
will be caught by a suitable \keyword{catch} clause in the construct, if
one is present.
Similarly, evaluation of the \keyword{case} expression is protected by
the \keyword{protect} phrase, if one is present.
\item 
\emph{In the reference implementation, a \keyword{select case} construct will
be translated into a Java \texttt{switch} construct provided that it
meets the following criteria:}
\begin{itemize}
\item 
\emph{The type of the \keyword{case} expression
is \texttt{byte}, \texttt{char}, \texttt{int}, or \texttt{short}.}
\item 
\emph{The value of all the expressions on the \keyword{when} clauses are
primitive constants (that is, they consist of only constants of
primitive types and operators valid for them and so may be evaluated at
compile time).}
\item 
\emph{No two expressions on the \keyword{when} clauses evaluate to the same
value.}
\item 
\emph{It is not subject to tracing.}
\end{itemize}
\emph{Under these conditions the semantics of the \texttt{switch} construct
match those defined for \keyword{select}.  The example shown above would
be translated to a \texttt{switch} construct if \texttt{i} had type \texttt{int}
and \keyword{options binary} were in effect.}
\end{enumerate}
\subsection{Exceptions in select constructs}
 
\index{CATCH,on SELECT instruction}
\index{FINALLY,on SELECT instruction}
Exceptions that are raised by the instructions within the body of the
group, or during evaluation of the \keyword{case} expression, may be
caught using one or more \keyword{catch} clauses that name
the \emph{exception} that they will catch.
When an exception is caught, the exception object that holds the details
of the exception may optionally be assigned to a variable,
\emph{vare}.
 
Similarly, a \keyword{finally} clause may be used to introduce
instructions that will always be executed at the end of the select
group, even if an exception is raised (whether caught or not).
 
The  \emph{Exceptions} section (see page \pageref{refexcep})  has details and
examples of \keyword{catch} and \keyword{finally}.
\index{,}

\chapter{Signal instruction}\label{refsignal}
\index{SIGNAL instruction,}
\index{Instructions,SIGNAL}
\index{Flow control,abnormal, with SIGNAL}
\index{,}
\index{,}
\index{Trapping of exceptions,}
\index{Raising exceptions,}
\index{Exceptions,raising}
\index{Exceptions,signalling}
\index{Exceptions,throwing}
\begin{shaded}
\begin{alltt}
\textbf{signal} \emph{term};
\end{alltt}
\end{shaded}
 The \keyword{signal} instruction causes an "abnormal" change
in the flow of control, by raising an \emph{exception}.
 
The exception \emph{term} may be a term that constructs or evaluates
to an exception object, or it may be expressed as the name of an
exception type (in which case the default constructor, with no
arguments, for that type is used to construct an exception object).
The exception object then represents the exception and is available, if
required, when the exception is handled.
 
The handling of exceptions is detailed in the
 \emph{Exceptions section} (see page \pageref{refexcep}).
In summary, when an exception is signalled, all active pending
\keyword{do} groups, \keyword{loop} loops, \keyword{if} constructs, and
\keyword{select} constructs may be ended.
For each one in turn, from the innermost:
\begin{enumerate}
\item No further clauses within the body of the construct will be executed
(in this respect, \keyword{signal} acts like a \keyword{leave} for the
construct).
\item The \emph{instructionlist} following the first \keyword{catch}
clause that matches the exception, if any, is executed.
\item The \emph{instructionlist} following the \keyword{finally}
clause for the construct, if any, is executed.
\end{enumerate}
If a \keyword{catch} matched the exception the exception is deemed
handled, and execution resumes as though the construct ended normally
(unless a new exception was signalled in the \keyword{catch} or
\keyword{finally} instruction lists, in which case it is processed).
Otherwise, any enclosing construct is ended in the same manner.
If there is no enclosing construct, then the current method is ended and
the exception is signalled in the caller.
 
\textbf{Examples:}
\begin{alltt}
signal RxErrorTrace
signal DivideException('Divide by zero')
\end{alltt}

\emph{In the reference implementation, the \emph{term} must
either}
\begin{itemize}
\item 
\emph{evaluate to an object that is assignable to the
type \textbf{Throwable} (for example, a subclass
of \textbf{Exception} or \textbf{RuntimeException}).}
\item 
\emph{be a type that is a subclass of \textbf{Throwable}, in which case
the default constructor (with no arguments) for the given type is used
to construct the exception object.}
\end{itemize}

\chapter{Trace instruction}\label{reftrace}
\index{TRACE,instruction}
\index{Trace setting,altering with TRACE instruction}
\index{Instructions,TRACE}
\index{Tracing,execution of programs}
\index{,}
\index{,}
\begin{shaded}
\begin{alltt}
\textbf{trace} \emph{traceoption};

where \emph{traceoption} is one of:
    \emph{tracesetting}
    \textbf{var} [\emph{varlist}]

where \emph{tracesetting} is one of:

    \textbf{all}
    \textbf{methods}
    \textbf{off}
    \textbf{results}

and \emph{varlist} is one or more variable \emph{names}, optionally prefixed with a \texttt{+} or \texttt{-}
\end{alltt}
\end{shaded}
 The \texttt{trace} instruction is used to control the tracing of the
execution of \nr{} methods, and is primarily used for debugging.
It may change either the general trace \emph{setting} or may select
or deselect the tracing of individual variables.
 
Within methods, the \texttt{trace} instruction changes the trace setting
or variables tracing when it is executed, and affects the tracing of
all clauses in the method which are then executed (until changed by a
later \texttt{trace} instruction).
 
One or more \texttt{trace} instructions may appear before the first
method in a class, one of which may set the initial trace setting
for all methods in the class (the default is \texttt{off}) and others
may set up variables tracing that applies to all the methods in the
class.
These act as though the \texttt{trace} instructions were
placed immediately following the \texttt{method} instruction in each
method (except that they will not be traced).
 
Similarly, one or more \texttt{trace} instructions may be placed
before the first \texttt{class} instruction in a program; they do not
imply the start of a class.  One of these may set the initial trace
setting and others may set up variables tracing for all classes in
the program (except interface classes) and act as though the
\texttt{trace} instructions were placed immediately following the
\texttt{class} instruction in each class.
\subsubsection{Tracing clauses}
\index{Trace setting,}
\index{Tracing,clauses}
 
The trace \emph{setting} controls the tracing of clauses in a program, and
may be one of the following:
\begin{description}
\item[all]\label{reftrall}
\index{ALL,TRACE setting}
 All clauses (except null clauses without commentary) which are in
methods and which are executed after the \texttt{trace} instruction will
be traced.
If \texttt{trace all} is placed before the first method in the current
class, the \texttt{method} instructions in the class,
together with the values of the arguments passed to each method,
will be traced when the method is invoked (that is, \texttt{trace all}
implies \texttt{trace methods}).
\item[methods]\label{reftrmeth}
\index{METHODS,TRACE setting}
 
All \texttt{method} clauses in the class will be traced when the method
they introduce is invoked, together with the values of the arguments
passed to each method; no other clauses, or results, will be traced.
The \texttt{trace methods} instruction must be placed before the first
method in the current class (as otherwise it would have no effect).
\item[off]\label{reftroff}
\index{OFF,TRACE setting}
 
Turns tracing off; no following clauses, variables, or results will be traced.
\item[results]\label{reftrres}
\index{RESULTS,TRACE setting}
 All clauses (except null clauses without commentary) which are in
methods and which are executed after the \texttt{trace} instruction will
be traced, as though \texttt{trace all} had been requested.
In addition, the results of all \emph{expression} evaluations and
any results assigned to a variable by an assignment, \texttt{loop}, or
\texttt{parse} instruction are also traced.
 
If \texttt{trace results} is placed before the first method in the
current class, the \texttt{method} instructions in the class will be
traced when the method is invoked, together with the values of the
arguments passed to each method.
\end{description}
 \textbf{Notes:}
\begin{enumerate}
\item Tracing of clauses shows the data from the source of the program,
starting at the first character of the first token of the clause and
including any commentary from that point until the end of the clause.
\item When a loop is being traced, the \texttt{loop} clause itself will be
traced on every iteration of the loop, as indicated by the
 programmer's model (see page \pageref{refloopmod}) ; the \texttt{end} clause is only
traced once, when the loop completes normally.
\item With \texttt{trace results}, an expression is not traced if it is
immediately used for an assignment (in an assignment instruction, or
when the control variable is initialized in a \texttt{loop}
instruction).
The assignment will trace the result of the expression.
\end{enumerate}
\subsubsection{Tracing variables}
\index{Tracing,variables}
 
The \texttt{var} option adds names to a list of monitored
variables; it can also remove names from the list.  If the name of a
variable in the current class or method is in the list, then \texttt{trace
results} is turned on for any assignment, \texttt{loop}, or
\texttt{parse} clause that assigns a new value to the named
variable.
 
Variable names are specified by listing them after the \texttt{var}
keyword.
Each name may be optionally prefixed by a \texttt{+} or a \texttt{-} sign.
A \texttt{+} sign indicates that the variable is to be added to the list
of monitored variables (the default), and a \texttt{-} sign indicates that
the variable is to be removed from the list.  Blanks may be added before
and after variable names and signs to separate the tokens and to improve
readability.
 For example:
\begin{alltt}
trace var a b c
-- now variables a, b, and c will be traced
trace var -b -c d
-- now variables a and d will be traced
\end{alltt}
 \textbf{Notes:}
\begin{enumerate}
\item 
Names in the list following the \texttt{var} keyword are simple symbols
that name variables in the current class or current method.
The variables may be properties, method arguments, or local variables,
and may be of any type, including arrays.
The names are not case-sensitive; any variables whose names match,
independent of case, will be monitored.
\item 
No variable name can appear more than once in the list on one
\texttt{trace var} instruction.  However, it is not an error to add the
name of a variable which does not exist or is not then assigned a value.
Similarly, it is not an error to remove a name which is not currently
being monitored.
\item 
One or more \texttt{trace var} instructions (along with one other
\texttt{trace} instruction) are allowed before the first method in a
class.  They all modify an initial list of monitored variables which
is then used for all methods in the class.  Similarly, \texttt{trace
var} instructions are allowed before the first class in a program,
in which case they apply to all classes (except interface classes).
\item 
Other \texttt{trace} instructions do not affect the list of monitored
variables.
The \texttt{trace off} instruction may be used to turn off tracing
completely; in this case \texttt{trace var} (with or without any
variable names) will then turn the tracing of variables back on, using
the current (or modified) variable list.
\item 
For a \texttt{parse} instruction, only monitored variables have their
assignments traced (unless \texttt{trace results} is already in effect).
\end{enumerate}
\subsubsection{The format of trace output}
\index{Indention during tracing,}
\index{Formatting,of output during tracing}
 
Trace output is either clauses from the program being traced, or results
(such as the results from expressions).
 
\index{Tracing,line numbers}
\index{Line numbers, in tracing,}
The first clause or result traced on any line will be preceded by its
line number in the program; this is right-justified in a space which
allows for the largest line number in the program, plus one blank.
Following clauses or results from the same line are preceded by white
space of the same width; however, any change of line number causes the
line number to be included.
 
Clauses that are traced will be displayed with the formatting
(indention) and layout used in the original source stream for the
program, starting with the first character of the first token of the
clause.
 
Results (if requested) are converted to a string for tracing if
necessary, are not indented, and have a double quote prefixed and
suffixed so that leading and trailing blanks are apparent; if, however,
the result being traced is  \textbf{null} (see page \pageref{refswnull})  then the
string "\textbf{[null]}" is shown (without quotes).
For results with an associated name (the values assigned to local
variables, method arguments, or properties in the current class), the
name of the result precedes the data, separated by a single blank.
 
For clarity, implementations may replace "control codes"
in the encoding of results (for example, EBCDIC values less
than \textbf{'\\x40'}, or Unicode values less than \textbf{'\\x20'})
by a question mark ("\textbf{?}").
\index{Tracing,data identifiers}
 All lines displayed during tracing have a three character tag to
identify the type of data being traced.  This tag follows the line
number (or the space for a line number), and is separated from the line
number by a single blank.
The traced clause or result follows the tag, after another
blank.
The identifier tags may be:
\begin{description}
\item{*=*}
\index{*=* tracing flag,}
identifies the first line of the source of a single clause, \emph{i.e.},
the data actually in the program.
\item{*-*}
\index{*-* tracing flag,}
identifies a continuation line from the source of a single clause.
Continuations may be due to the use of a  continuation (see page \pageref{refsemis}) 
character:ea. or to the use of a  block comment (see page \pageref{refblockco}) 
which spans more than one line.
\item{>a>}
\index{>a> tracing flag,}
Identifies a value assigned to a method argument of the current
method.
The name of the argument is included in the trace.
\item{>p>}
\index{>p> tracing flag,}
Identifies a value assigned to a property.
The name of the property is included in the trace if the property is in
the current class.
\item{>v>}
\index{>v> tracing flag,}
Identifies a value assigned to a local variable in the current
method.
The name of the variable is included in the trace.
\item{>>>}
\index{>>> tracing flag,}
Identifies the result of an expression evaluation that is not used
for an assignment (for example, an argument expression in a method
call).
\item{+++}
\index{+++ tracing flag,}
Reserved for error messages that are not supplied by the environment
underlying the implementation.
\end{description}

\index{Trace,context}
\index{Thread,tracing}
If a trace line is produced in a different context (program or thread)
from the preceding trace line (if any) then a \emph{trace context}
line is shown.  This shows the name of the program that produced the
trace line, and also the name of the thread (and thread group) of the
context.
 
The thread group name is not shown if it is \texttt{main}, and in this
case the thread name is then also suppressed if its name is \texttt{main}.
 \textbf{Examples:}
 If the following instructions, starting on line 53 of a 120-line
program, were executed:
\begin{alltt}
trace all
if i=1 then say 'Hello'
       else say 'i<>1'
say -
 'A continued line'
\end{alltt}
the trace output (if \emph{i} were \textbf{1}) would be:
\begin{alltt}
  54 *=* if i=1
     *=*        then
     *=*             say 'Hello'
  56 *=* say -
  57 *-*  'A continued line'
\end{alltt}
 Similarly, for the 3-line program:
\begin{alltt}
trace results
number=1/7
parse number before '.' after
\end{alltt}
the trace output would be:
\begin{alltt}
 2 *=* number=1/7
   >v> number "0.142857143"
 3 *=* parse number before '.' after
   >v> before "0"
   >v> after "142857143"
\end{alltt}
 \textbf{Notes:}
\begin{enumerate}
\item 
Trace output is written to an implementation-defined output stream
\index{stderr, used by TRACE,}
(typically the "standard error" output stream, which lets it be
redirected to a destination separate from the default destination for
output which is used by the \texttt{say} instruction).
\item In some implementations, the use of \texttt{trace} instructions
may substantially increase the size of classes and the execution time of
methods affected by tracing.
\footnote{
\emph{In the reference implementation, \texttt{options notrace} may be
used to disable all \texttt{trace} instructions and hence ensure that
tracing overhead is not accidentally incurred.}
}
\item With some implementations it may be possible to switch tracing on
externally, without requiring modification to the program.
\end{enumerate}

\chapter{Program structure}\label{refpstruct}
\index{Programs,}
\index{Programs,structure}
\index{Program,structure}
 A \nr{} \emph{program} is a collection of
 clauses (see page \pageref{refclau})  derived from a single implementation-defined
source stream (such as a file).
When a program is processed by a language processor
\footnote{
Such as a compiler or interpreter.
}
it defines one or more classes.
Classes are usually introduced by the  \keyword{class} instruction (see page \pageref{refclass}), but if the first is a standard class, intended to be
run as a stand-alone application, then the \keyword{class} instruction
can be omitted.  In this case, \nr{} defines an implied class
and initialization method that will be used.
 
The implied class and method permits the writing of "low
boilerplate" programs, with a minimum of syntax.
The simplest, documented, \nr{} program that has an effect might
therefore be:
 
\textbf{Example:}
\index{Example,Hello World}
\begin{alltt}
/* This is a very simple \nr{} program */
say 'Hello World!'
\end{alltt}
 
In more detail, a \nr{} program consists of:
\index{Program,prolog}
\index{Prolog, of a program,}
\begin{enumerate}
\item An optional \emph{prolog} (\keyword{package}, \keyword{import}, and
\keyword{options} instructions).  Only one \keyword{package} instruction
is permitted per program.
\item  One or more class definitions, each introduced by a \keyword{class}
instruction.
\end{enumerate}
 
\index{Class,definition}
A \emph{class definition} comprises:
\begin{enumerate}
\item The \keyword{class} instruction which introduces the class (which may
be inferred, see below).
\item 
Zero or more property variable assignments,
\index{Properties,initialization}
\index{Assignment,property initialization}
along with optional \keyword{properties}
instructions that can alter their attributes, and optional
\keyword{numeric} and \keyword{trace} instructions.
Property variable assignments take the form of an
 \emph{assignment} (see page \pageref{refassign}) , with an optional
"\textbf{=}" and expression, which may:
\begin{itemize}
\item just name a property (by omitting the "\textbf{=}"
and expression of the assignment), in which case it refers to a string of
type \textbf{R\textsc{exx}}
\item assign a type to the property (when the expression evaluates to just
a type)
\item 
assign a type and initial value to the property (when the expression
returns a value).
\end{itemize}
\item Zero or more method definitions, each introduced by a
\keyword{method} instruction (which may be inferred if the \keyword{class}
instruction is inferred, see below).
\end{enumerate}
 
\index{Method,definition}
A \emph{method definition} comprises:
\begin{itemize}
\item 
Any \nr{} instructions, except the \keyword{class}, \keyword{method},
and \keyword{properties} instructions and those allowed in the prolog
(the \keyword{package}, \keyword{import}, and \keyword{options}
instructions).
\end{itemize}
 \textbf{Example:}
\index{Example,of two classes}
\begin{alltt}
/* A program with two classes */
import java.applet.   -- for example

class testclass extends Applet
  properties public
    state             -- property of type 'R\textsc{exx}'
    i=int             -- property of type 'int'
  properties constant
    j=int 3           -- property initialized to '3'

  method start
    say 'I started'
    state='start'

  method stop
    say 'I stopped'
    state='stop'

class anotherclass
  method testing
    loop i=1 to 10
      say '1, 2, 3, 4...'
      if i=7 then return
     end
    return

  method anothertest
    say '1, 2, 3, 4'
\end{alltt}
This example shows a prolog (with just an \keyword{import}
instruction) followed by two classes.  The first class includes
two public properties, one constant property, and two methods.
The second class includes no properties, but also has two methods.
 
Note that a \keyword{return} instruction implies no static scoping; the
content of a method is ended by a \keyword{method} (or \keyword{class})
instruction, or by the end of the source stream.
The \keyword{return} instruction at the end of the \textbf{testing}
method is, therefore, unnecessary.
\section{Program defaults}\label{programdefaults}
 
The following defaults are provided for \nr{} programs:
\begin{enumerate}
\item If, while parsing prolog instructions, some instruction that is not
valid for the prolog and is not a \keyword{class} instruction is
encountered, then a default \keyword{class} instruction (with an
implementation-provided short name, typically derived from the name of
the source stream) is inserted.  If the instruction was not a
\keyword{method} instruction, then a default \keyword{method} instruction
(with a name and attributes appropriate for the environment, such
as \textbf{main}) is also inserted.
 
In this latter case, it is assumed that execution of the program will
begin by invocation of the default method.
In other words, a "stand-alone" application can be written without
explicitly providing the class and method instructions for the first
method to be executed.
An example of such a program is given in  Appendix A (see page \pageref{refappa}) .

 
\emph{In the reference implementation, the \textbf{main} method in a
stand-alone application is passed the words forming the command string
as an array of strings of type \textbf{java.lang.String} (one word to
each element of the array).
When the \nr{} reference implementation provides the \textbf{main}
method instruction by default, it also constructs a \nr{} string of
type \textbf{R\textsc{exx}} from this array of words, with a blank added
between words, and assigns the string to the variable
\textbf{arg}.}
 
\emph{The command string may also have been edited by the underlying
operating system environment; certain characters may not be
allowed, multiple blanks or whitespace may have been reduced to
single blanks, etc.
}
\item If a method ends and the last instruction at the outer level of the
method scope is not \keyword{return} then a \keyword{return} instruction
is added if it could be reached.
In this case, if a value is expected to be returned by the method (due
to other \keyword{return} instructions returning values, or there being a
\keyword{returns} keyword on the \keyword{method} instruction), an error
is reported.
\end{enumerate}
 
Language processors may provide options to prevent, or warn of, these
defaults being applied, as desired.

\chapter{Minor and Dependent classes}\label{refminor}
 
A \emph{minor class} in \nr{} is a class whose name is qualified by
the name of another class, called its \emph{parent}, and a
\emph{dependent class} is a minor class that has a link to its parent
class that allows a child object simplified access to its parent
object and its properties.
\section{Minor classes}\label{refsminorc}
\index{,}
\index{,}
\index{,}
\index{,}
\index{Classes,minor}
\index{Classes,parent}
\index{Parent class,}
\index{Full name,of classes}
\index{Short name,of classes}
\index{Constructors,in minor classes}
 
A \emph{minor class} in \nr{} is a class whose name is qualified by
the name of another class, called its \emph{parent}.
This qualification is indicated by the form of the name of the class:
the short name of the minor class is prefixed by the name of its parent
class (separated by a period).
For example, if the parent is called \texttt{Foo} then the full name of a
minor class \texttt{Bar} would be written \texttt{Foo.Bar}.
The short name, \texttt{Bar}, is used for the name of any constructor
method for the class; outside the class it can only be used to identify
the class in the context of the parent class (or from children of the
minor class, see below).
 
The names of minor classes may be used in exactly the same way as
other class names (types) in programs.  For example, a property might be
declared and initialized thus:
\begin{lstlisting}
abar=Foo.Bar null   -- this has type Foo.Bar
\end{lstlisting}
or, if the class has a constructor, perhaps:
\begin{lstlisting}
abar=Foo.Bar()      -- constructs a Foo.Bar object
\end{lstlisting}
 
\index{Minor classes,naming of}
Minor classes must be in the same program (and hence in the same
package) as their parent.  They are introduced by a \keyword{class}
instruction that specifies their full name, for example:
\begin{lstlisting}
class Foo.Bar extends SomeClass
\end{lstlisting}
 
Minor classes must immediately follow their parent class.
\footnote{
This allows compilers that generate Java source code to preserve line
numbering.
}
 
\index{Minor classes,nesting of}
Minor classes may have a parent which is itself a minor class,
to any depth; the name and the positioning rules are extended as
necessary.  For example, the following classes might exist in a program:
\begin{lstlisting}
class Foo
  class Foo.Bar
    class Foo.Bar.Nod
    class Foo.Bar.Pod
  class Foo.Car
\end{lstlisting}
 
As before, the children of \texttt{Foo.Bar} immediately follow their
parent.  The list of children of \texttt{Foo} can be continued after
the children of \texttt{Foo.Bar} have all been specified.
 
Note that the short name (last part of the name) of a minor class may
not be the same as the short name of any of its parents (a
class \texttt{Foo.Bar.Foo} or a class \texttt{Foo.Bar.Bar} would be in
error, for example).  This allows minor classes to refer to their parent
classes by their short name without ambiguity.
\subsection{Constructing objects in minor classes}
\index{Minor classes,constructing}
\index{Constructors,of minor classes}
 
A parent class can construct an object of a child class in the usual
manner, by simply specifying its constructor (identified by its short
name, full name, or qualified name).
For example, a method in the \texttt{Foo.Bar} class above could construct
an object of type \texttt{Foo.Bar.Nod} using:
\begin{lstlisting}
anod=Nod()
\end{lstlisting}
(assuming the \texttt{Foo.Bar.Nod} class has a constructor that takes no
arguments).
 
Similarly, minor classes can refer to the types and constructors of any
of its parents by simply using their short names.
Hence, the \texttt{Foo.Bar.Nod} class could construct objects of its
parents' types thus:
\begin{lstlisting}
abar=Bar()
afoo=Foo()
\end{lstlisting}
(again assuming the parent classes have constructors that take no
arguments).
 
Classes other than the parent or an immediate child must use the full
name (if necessary, qualified by the package name) to refer to a minor
class or its constructor.
\section{Dependent classes}\label{refsdepen}
\index{,}
\index{Classes,dependent}
\index{DEPENDENT,on CLASS instruction}
\index{,}
\index{,}
 
As described in the last section, minor classes provide an enhanced
packaging (naming) mechanism for classes, allowing classes to be
structured within packages.  A stronger link between a child class and
its parent is indicated by the modifier keyword \keyword{dependent} on the
child class, which indicates that the child is a \emph{dependent class}.
For example:
\begin{lstlisting}
class Foo.Dep dependent extends SomeClass
  method Dep   -- this is the constructor
\end{lstlisting}
 
\index{Dependent object,}
\index{Parent object,}
An object constructed from a dependent class (a \emph{dependent
object}) is linked to the context of an object of its parent
type (its \emph{parent object}).
The linkage thus provided allows the child object simplified access to
the parent object and its properties.
 
In the example, an object of type \texttt{Foo.Dep} can only be constructed
in the context of a parent object, which must be of type \texttt{Foo}.
\subsection{Constructing dependent objects}
\index{Dependent object,constructing}
\index{Constructors,of dependent objects}
\index{Parent,of dependent object}
 
A parent class can construct a dependent object in the same way as when
constructing objects of other child types; that is, by simply specifying
its constructor.  In this case, however, the current object
(\texttt{this}) becomes the parent object of the newly constructed object.
For example, a method in the \texttt{Foo} class above could construct a
dependent object of type \texttt{Foo.Dep} using:
\begin{lstlisting}
adep=Dep()
\end{lstlisting}
(assuming the \texttt{Dep} class has a constructor that takes no
arguments).
 
\index{Constructors,qualified}
In general, for a class to construct an object from a dependent class,
it must have a reference to an object of the parent class (which will
become the parent of the new object), and the constructor must be called
(by its short name) in the context of that parent object.  For example:
\begin{lstlisting}
parentObject=Foo()
adep=parentObject.Dep()
\end{lstlisting}
(In the same way, the first example could have been written:
\begin{lstlisting}
adep=this.Dep()
\end{lstlisting}
within the parent class the \texttt{this.} is implied.)
 
\index{Special methods,super}
\index{SUPER,special method}
In order to subclass a dependent class, the constructor of the dependent
class must be invoked by the subclass constructor in a similar manner.
In this case, a qualified call to the usual special
constructor \texttt{super} is used, for example:
\begin{lstlisting}
class ASub extends Foo.Dep
  method Asub(afoo=Foo)
    afoo.super()
\end{lstlisting}
 The qualifier (\texttt{afoo} in the example) must be either the name of
an argument to the constructor, or the special word \texttt{parent} (if
the classes share a common parent class), or the short name of a parent
class followed by \texttt{.this} (see below).
The call to \texttt{super} must be the first instruction in the method, as
usual, and it must be present (it will not be generated automatically by
the compiler).
\subsection{Access to parent objects and their properties}
 
Dependent classes have simplified access to their parent objects and
their properties.
In particular:
\begin{itemize}
\index{Special words,parent}
\index{PARENT,special word}
\item The special word \texttt{parent} may be used to refer to the
parent object of the current object.  It may appear alone in a term, or
at the start of a compound term.
It can only be used in non-static contexts in a dependent class.
\item 
\index{Special words,this}
\index{THIS,special word}
In general, any of the objects in the chain of parents of a dependent
object may be referred to by qualifying the special word \texttt{this}
with the short name of the parent class.
For example, extending the previous example, if the
class \texttt{Foo.Dep.Ent} was a dependent class it could contain
references to \texttt{Foo.this} (the parent of its parent)
or \texttt{Dep.this} (the latter being the same as
specifying \texttt{parent}).  If preferred, the full name or the fully
qualified name of the parent class may be used instead of the short
name.
 
Like \texttt{parent}, this construct can only be used at the start of a
term in non-static contexts in a dependent class.
\item 
\index{Properties,in minor classes}
\index{Properties,in dependent classes}
As usual, properties external to the current class must always be
qualified in some way (for example, the prefix \texttt{parent.} can be
used in a term such as \texttt{parent.aprop}).
\end{itemize}
\section{Restrictions}\label{refsminres}
\index{Minor classes,restrictions}
\index{Dependent classes,restrictions}
 
Minor classes may have any of the attributes (\keyword{public},
\keyword{interface}, \emph{etc.}) of other classes, and behave in every way
like other classes, with the following restrictions:
\begin{itemize}
\item 
If a class is a static class (that is, it contains only static or
constant properties and methods) then any children cannot be dependent
classes (because no object of the parent class can be constructed).
Similarly, interface classes and abstract classes cannot have dependent
classes.
\item 
Dependent classes may not be interfaces.
\item 
\index{Properties,in dependent classes}
Dependent classes may not contain static or constant properties (or
methods).
\footnote{
This restriction allows compilation for the Java platform.
}
These must be placed in a parent which is not a dependent class.
\item 
Minor classes may be public only if their parent is also public.
(Note that this is the only case where more than one public class is
permitted in a program.)  In general: a minor class cannot be more
visible than its parent.
\end{itemize}
\index{,}
\index{,}

\chapter{Special names and methods}\label{refspecial}
 
For convenience, \nr{} provides some \emph{special names} for naming
commonly-used concepts within terms.
These are only recognized if there is no variable of the same name
previously seen in the current scope, as described in the section on
 \emph{Terms} (see page \pageref{refterms}) .
This allows the set of special words to be expanded in the future, if
necessary, without invalidating existing variables.  Therefore, these
names are not reserved; they may be used as variable names instead, if
desired.
 
There are also two "special methods" that are used when
constructing objects.
\subsection{}\label{}
\index{Special words,}
 
The following special names are allowed in \nr{} programs, and are
recognized independently of case.
\footnote{
\emph{Unless \texttt{options strictcase} is in effect.
}
}
With the exception of \textbf{length} and \textbf{class}, these
may only be used alone as a term or at the start of a compound term.
\begin{description}
\item[ask]\label{refswask}
\index{Special words,ask}
\index{Names,special/ask}
\index{Words,special/ask}
\index{ASK special word,}
\index{stdin, reading with ASK,}
 
Returns a string of type \textbf{R\textsc{exx}}, read as a line from the
implementation-defined default input stream (often the user's
"console").
 \textbf{Example:}
\begin{alltt}
if ask='yes' then say 'OK'
\end{alltt}
 \textbf{ask} can only appear alone, or at the start of a
compound term.
\footnote{
\emph{In the reference implementation, \textbf{ask} is simply a shorthand
for \textbf{R\textsc{exx}IO.Ask()}.}
}
\item[class]\label{refswclass}
\index{Special words,class}
\index{Names, special,class}
\index{Words, special,class}
\index{CLASS,special word}
 
The object of type \textbf{Class} that describes a specific type.
This word is only recognized as the second part of a compound term,
where the evaluation of the first part of the term resulted in a
type or qualified type.
 \textbf{Example:}
\begin{alltt}
obj=String.class
say obj.isInterface /* would say '0' */
\end{alltt}
\item[digits]\label{refswdigit}
\index{Special words,digits}
\index{Names,special/digits}
\index{Words,special/digits}
\index{DIGITS,special word}
 
The current setting of  \texttt{numeric digits} (see page \pageref{refndigits}) ,
returned as a string of type \textbf{R\textsc{exx}}.
This will be one or more Arabic numerals, with no leading blanks, zeros,
or sign, and no trailing blanks or exponent.
 \textbf{digits} can only appear alone, or at the start of a
compound term.
\item[form]\label{refswform}
\index{Special words,form}
\index{Names,special/form}
\index{Words,special/form}
\index{FORM,special word}
 
The current setting of  \texttt{numeric form} (see page \pageref{refnform}) ,
returned as a string of type \textbf{R\textsc{exx}}.
This will have either the value "\textbf{scientific}" or the
value "\textbf{engineering}".
 \textbf{form} can only appear alone, or at the start of a
compound term.
\item[length]\label{refswleng}
\index{Special words,length}
\index{Names,special/length}
\index{Words,special/length}
\index{LENGTH,special word}
 
The length of an  array (see page \pageref{refarray}) , returned as an
implementation-dependent binary type or string.
This word is only recognized as the last part of a compound term,
where the evaluation of the rest of the term resulted in an array of
dimension 1.
 \textbf{Example:}
\begin{alltt}
foo=char[7]
say foo.length     /* would say '7' */
\end{alltt}
 
Note that you can get the length of a \nr{} string with the
same syntax.
\footnote{
\emph{Unless \texttt{options strictargs} is in effect.
}
}
In that case, however, a \textbf{length()} method is being invoked.
\item[null]\label{refswnull}
\index{Special words,null}
\index{Names,special/null}
\index{Words,special/null}
\index{NULL special word,}
\index{Empty reference, null,}
\index{References,null}
 
The \emph{empty reference}.  This is a special value that represents
"no value" and may be assigned to variables (or returned from
methods) except those whose type is both primitive and undimensioned.
It may also be be used in a comparison for equality (or inequality) with
values of suitable type, and may be given a type.
 \textbf{Examples:}
\begin{alltt}
blob=int[3]   -- 'blob' refers to array of 3 ints
blob=null     -- 'blob' is still of type int[],
              -- but refers to no real object
mob=Mark null -- 'mob' is type 'Mark'
\end{alltt}
 The \textbf{null} value may be considered to represent the state of
being uninitialized.  It can only appear as simple symbol, not as a part
of a compound term.
\item[source]\label{refswsourc}
\index{Special words,source}
\index{Names,special/source}
\index{Words,special/source}
\index{SOURCE special word,}
\index{Program,filename of}
\index{Class,filename of}
 
Returns a string of type \textbf{R\textsc{exx}} identifying the source of the
current class.
The string consists of the following words, with a single blank between
the words and no trailing or leading blanks:
\begin{enumerate}
\item the name of the underlying environment (\emph{e.g.}, \textbf{Java})
\item either \textbf{method} (if the term is being used within a method)
or \textbf{class} (if the term is being used within a property
assignment, before the first method in a class)
\item 
an implementation-dependent representation of the name of the
source stream for the class (\emph{e.g.}, \textbf{Fred.nrx}).
\end{enumerate}
 \textbf{source} can only appear alone, or at the start of a
compound term.
\item[sourceline]\label{refswsourl}
\index{Special words,sourceline}
\index{Names, special,sourceline}
\index{Words, special,sourceline}
\index{SOURCELINE,special word}
 
The line number of the first token of the current clause in the
\nr{} program, returned as a string of type \textbf{R\textsc{exx}}.
This will be one or more Arabic numerals, with no leading blanks, zeros,
or sign, and no trailing blanks or exponent.
 \textbf{sourceline} can only appear alone, or at the start of a
compound term.
\item[super]\label{refswsuper}
\index{Special words,super}
\index{Names,special/super}
\index{Words,special/super}
\index{SUPER,special word}
\index{References,to current object}
 
Returns a reference to the current object, with a type that is the
type of the class that the current object's class extends.
This means that a search for methods or properties
which \textbf{super} qualifies will start from the superclass rather
than in the current class.
This is used for invoking a method or property (in the superclass or one
of its superclasses) that has been overridden in the current class.
 \textbf{Example:}
\begin{alltt}
method printit(x)
  say 'it'          -- modification
  super.printit(x)  -- now the usual processing
\end{alltt}
 
If a property being referenced is in fact defined by a superclass of
the current class, then the prefix "\textbf{super.}" is perhaps
the clearest way to indicate that name refers to a property of a
superclass rather than to a local variable.
(You could also qualify it by the name of the superclass.)
 \textbf{super} can only appear alone, or at the start of a
compound term.
\item[this]\label{refswthis}
\index{Special words,this}
\index{Names,special/this}
\index{Words,special/this}
\index{THIS,special word}
\index{References,to current object}
 
Returns a reference to the current object.
When a method is invoked, for example in:
\begin{alltt}
word=R\textsc{exx} "hello"  -- 'word' refers to "hello"
say word.substr(3) -- invokes substr on "hello"
\end{alltt}
then the method \textbf{substr} in the class \textbf{R\textsc{exx}} is
invoked, with argument \textbf{'3'}, and with the properties of the
value (object) \textbf{"hello"} available to it.
These properties may be accessed simply by name, or (more explicitly) by
prefixing the name with "\textbf{this.}".
Using "\textbf{this.}" can make a method more readable,
especially when several objects of the same type are being manipulated
in the method.
 \textbf{this} can only appear alone, or at the start of a
compound term.
\item[trace]\label{refswtrace}
\index{Special words,trace}
\index{Names,special/trace}
\index{Words,special/trace}
\index{TRACE,special word}

The current  \texttt{trace} (see page \pageref{reftrace})  setting,
returned as a \nr{} string.
This will be one of the words:
\begin{alltt}
off var methods all results
\end{alltt}

(\texttt{var} is returned when clause tracing is off but variable
tracing has then been turned on using a \texttt{trace var} instruction.)
 \textbf{trace} can only appear alone, or at the start of a
compound term.
\item[version]\label{refswvers}
\index{Special words,version}
\index{Names,special/version}
\index{Words,special/version}
\index{VERSION special word,}
 
Returns a string of type \textbf{R\textsc{exx}} identifying the version of the
\nr{} language in effect when the current class was processed.
The string consists of the following words, with a single blank between
the words and no trailing or leading blanks:
\begin{enumerate}
\item A word describing the language.  The first seven letters will be the
characters \textbf{\nr{}}, and the remainder may be used to identify
a particular implementation or language processor.
This word may not include any periods.
\item 
The language level description, which must be a number with no sign or
exponential part.
For example, "\textbf{\nrversion{}}" is the language level of this
definition.
\item 
Three words describing the language processor release date in
the same format as the default for the R\textsc{exx} "\textbf{date()}"
function.
\footnote{
As defined in :cit.American National Standard for Information
Technology - Programming Language REXX, X3.274-1996:ecit., American
National Standards Institute, New York, 1996.
}
For example, "\textbf{22 May 2009}".
\end{enumerate}
 \textbf{version} can only appear alone, or at the start of a
compound term.
\end{description}
\subsection{Special methods}\label{refspecm}
\index{Special methods,}
\index{Methods,special}
\index{Constructors,special}
 
Constructors (methods used for constructing objects) in \nr{}
must invoke a constructor of their superclass before making any
modifications to the current object (or invoke another constructor in
the current class).
 
\index{SUPER,special method}
\index{THIS,special method}
\index{Special methods,this}
\index{Special methods,super}
This is simplified and made explicit by the provision of the special
method names \textbf{super} and \textbf{this}, which refer to
constructors of the superclass and current class respectively.  These
special methods are only recognized when used as the first, method call,
instruction in a constructor, as described in
 \emph{Methods and constructors} (see page \pageref{refmethcon}) .
Their names will be recognized independently of case.
\footnote{
\emph{Unless \texttt{options strictcase} is in effect.
}
}
 
In addition, \nr{} provides special constructor methods for the
primitive types that allow binary construction of primitives.
These are described in  \emph{Binary values and (see page \pageref{refbincon}) 
arithmetic}:ea..

\chapter{Appendix B - JavaBean Support}\label{refappb}
\index{Experimental feature,}
 
This appendix describes an experimental feature, \emph{indirect
properties}, which is supported by the \nr{} reference
implementation.
 
\index{JavaBean properties,}
\index{Properties,for JavaBeans}
The intention of the feature is to make it easier to write a certain
kind of class known as a \emph{JavaBean}.
Almost all JavaBeans will have \emph{properties}, which are data items
that a user of a JavaBean is expected to be able to customize (for
example, the text on a pushbutton).  The names and types of the
properties of a JavaBean are inferred from "\emph{design
patterns}" (in this context, conventions for naming methods) or
from PropertyDescriptor objects associated with the JavaBean.
 
The JavaBean properties do not necessarily correspond to instance
variables in the class - although very often they do.  The
JavaBean specification does not guarantee that JavaBean properties
that can be set can also be inspected, nor does it describe how
ambiguities of naming and method signatures are to be handled.
 
The \nr{}C compiler
allows a more rigorous
treatment of JavaBean properties, by allowing an optional attribute of
properties in a class that declares them to be \emph{indirect
properties}.  Indirect properties are properties of a known type
that are private to the class, but which are expected to be publicly
accessible indirectly, though certain conventional method calls.
 
Declaring properties to be indirect offers the following advantages:
\begin{itemize}
\item For many simple cases, the access methods for the properties can be
generated automatically; there is no need to explicitly code them in the
source file for the class.  This is especially helpful for Indexed
Properties (where four methods are needed, in general).
\item Where access methods are explicitly provided in the class, they can
be checked for correct form, signature and accessibility.  This detects
errors at compile time that otherwise would only be determined by
testing.
\item Similarly, attention can be drawn to the presence of methods that
may be intended to be an access method for an indirect property, but
will not be recognized as such by builders.
\end{itemize}
 The next section describes the use of indirect properties in more
detail.
\chapter{Indirect properties}\label{refindprop}
\index{PROPERTIES instruction,}
\index{Instructions,PROPERTIES}
\index{INDIRECT,on PROPERTIES instruction}
\index{Indirect properties,}
\index{Properties,indirect}
 
The  \texttt{properties} instruction (see page \pageref{refprop})  is used to
define the attributes of following \emph{property} variables.
The \emph{visibility} of properties may include a new alternative:
\texttt{indirect}.
Properties with this form of visibility are known as \emph{indirect
properties}.  These are properties of a known type that are private
to the class, but which are expected to be publicly accessible
indirectly, though certain conventional method calls.
 
For example, consider the simple program:
\begin{alltt}
class Sandwich extends Canvas implements Serializable
  properties indirect
    slices=Color.gray
    filling=Color.red

  method Sandwich
    resize(100,30)

  method paint(g=Graphics)
    g.setColor(slices)
    g.fillRect(0, 0, size.width, size.height)
    g.setColor(filling)
    g.fillRect(12, 12, size.width-12, size.height-12)
\end{alltt}
This declares the \textbf{Sandwich} class as having two indirect
properties, called \textbf{slices} and \textbf{filling}, both being
of type \textbf{java.awt.Color}.
 
In the example, no access methods are provided for the properties, so
the compiler will add them.  By implementation-dependent convention, the
names are prefixed with verbs such as \textbf{get} and \textbf{set},
\&, and have the first character of their name uppercased to form the
method names.
Hence, in this Java-based example, the following four methods are added:
\begin{alltt}
method getSlices  returns java.awt.Color
  return slices
method getFilling returns java.awt.Color
  return filling
method setSlices(\$1=java.awt.Color)
  slices=\$1
method setFilling(\$2=java.awt.Color)
  filling=\$2
\end{alltt}
(where \textbf{\$1} and \textbf{\$2} are "hidden" names used for
accessing the method arguments).
 
Note that the \texttt{indirect} attribute for a property is an
alternative to the \texttt{public}, \texttt{private}, and
\texttt{inheritable}, and \texttt{shared} attributes.
Like private properties, indirect properties can only be accessed
directly by name from within the class in which they occur; other
classes can only access them using the access methods (or other methods
that may use, or have a side-effect on, the properties).
 
Indirect properties may be \texttt{constant} (implying that only
a \texttt{get} method is generated or allowed, though the private property
may be changed by methods within the class)
or  \texttt{transient} (see page \pageref{refpropmod}) .
They may not be \texttt{static} or \texttt{volatile}.
 
In detail, the rules used for generating automatic methods for a
property whose name is \textbf{xxxx} are as follows:
\begin{enumerate}
\item A method called \textbf{getXxxx} which returns the value of the
property is generated.  The returned value will have the same type
as \textbf{xxxx}.
\item If the type of \textbf{xxxx} is \textbf{boolean} then the generated
method will be called \textbf{isXxxx} instead of \textbf{getXxxx}.
\item If the property is not \texttt{constant} then a method for setting the
property will also be generated.  This will be called \textbf{setXxxx},
and take a single argument of the same type as \textbf{xxxx}.  This
assigns the argument to the property and returns no value.
\end{enumerate}
 
If the property has an array type (for example, \textbf{char[]}),
then it must only have a single dimension.
Two further methods may then be generated, according to the rules:
\begin{enumerate}
\item A method called \textbf{getXxxx} which takes a single \textbf{int}
as an argument and which returns an item from the property array is
generated. The returned value will have the same type as \textbf{xxxx},
without the \textbf{[]}.  The integer argument is used to
index into the array.
\item As before, if the result type of the method would be \textbf{boolean}
then the name of the method will be \textbf{isXxxx} instead
of \textbf{getXxxx}.
\item If the property is not \texttt{constant} then a method for setting an
item in the property array will also be generated.
This will be called \textbf{setXxxx}, and take two arguments: the
first is an \textbf{int} that is used to select the item to be
changed, and the second is an undimensioned argument of the same type
as \textbf{xxxx}.  It assigns the second argument to the item in the
property array indexed by the first argument, and returns no value.
\end{enumerate}
 For example, for an indirect property declared thus:
\begin{alltt}
properties indirect
  fred=foo.Bar[]
\end{alltt}
the four methods generated would be:
\begin{alltt}
method getFred returns foo.Bar[]; return fred
method getFred($1=int) returns foo.Bar; return fred[$1]
method setFred($2=foo.Bar[]); fred=$2
method setFred($3=int, $4=foo.Bar); fred[$3]=$4
\end{alltt}
 
Note that in all cases a method will only be generated if it would not
exactly
match a method explicitly coded in the current class.
\subsubsection{Explicit provision of access methods}
 
Often, for example when an indirect property has an on-screen
representation, it is desirable to redraw the property when the property
is changed (and in more complicated cases, there may be interactions
between properties).
These and other actions will require extra processing which will not be
carried out by automatically generated methods.  To add this processing
the access methods will have to be coded explicitly.  In the
"Sandwich" example, we only need to supply the \textbf{set}
methods, perhaps by adding the following to the example class above:
\begin{alltt}
method setSlices(col=Color)
  slices=col      -- update the property
  this.repaint    -- redraw the component

method setFilling(col=Color)
  filling=col
  this.repaint
\end{alltt}
If we add these two methods, they will no longer be added
automatically (the two \textbf{get} methods will continue to be
provided automatically, however).  Further, since the names match
possible access methods for properties that are declared to be indirect,
the compiler will check the method declaration: the method signatures
and return type (if any) must be correct, for example.  Also, since the
names of access methods are case-sensitive (in a Java environment), you
will be warned if a method appears to be intended to be an access method
but the case of one or more letters is wrong.
 
Specifically, the checks carried out are as follows:
\begin{enumerate}
\item For methods whose names exactly match a potential access method for
an indirect property (that is, start with \textbf{is}, \textbf{get},
or \textbf{set}, which is then followed by the name of an indirect
property with the first character of the name uppercased):
\begin{itemize}
\item The argument list for (signature of) the method must match one of
those that could possibly be automatically generated for the property.
\item The returns type (if any) must match the expected returns type for
that method.
\item If the returns type is simply \textbf{boolean}, then the method name
must start with \textbf{is}.
Conversely, if the method name starts with \textbf{is} then the returns
type must be just \textbf{boolean}.
\item If the property is \texttt{constant} then the name of the method
cannot start with \textbf{set}.
\item A warning is given if the method is not \texttt{public} (the default).
\end{itemize}
\item For methods whose names match a potential access method, as above,
except in case:
\begin{itemize}
\item A warning is given that the method in question may be intended to
be an indirect property access method, but will not be recognized as
such by builders.
\end{itemize}
\end{enumerate}
 These checks detect a wide variety of errors at compile time, hence
speeding the development of classes that use indirect properties.

\chapter{Parsing templates}\label{refparsing}
\index{PARSE,parsing rules}
\index{Parsing,general rules}
\index{Templates, parsing,general rules}
 The \keyword{parse} instruction allows a selected string to
be parsed (split up) and assigned to variables, under the control of a
\emph{template}.
 
The various mechanisms in the template allow a string to be split up by
explicit matching of strings (called \emph{patterns}), or by
specifying numeric positions (\emph{positional patterns} - for
example, to extract data from particular columns of a line read from a
character stream).
Once split into parts, each segment of the string can then be assigned
to variables as a whole or by words (delimited by blanks).
 
This section first gives some informal examples of how the parsing
template can be used, and then defines the algorithms in detail.
\index{Parsing,introduction}
\section{Introduction to parsing}\label{parseintro}
 The simplest form of parsing template consists of a list of variable
names.
The string being parsed is split up into words (characters delimited by
blanks), and each word from the string is assigned to a
variable in sequence from left to right.
The final variable is treated specially in that it will be assigned
whatever is left of the original string and may therefore contain
several words.
For example, in the \keyword{parse} instruction:
\begin{lstlisting}
parse 'This is a sentence.' v1 v2 v3
\end{lstlisting}
the term (in this case a literal string) following the instruction
keyword is parsed, and then:  the variable \textbf{\emph{v1}}
would be assigned the value "\textbf{This}", \textbf{\emph{v2}}
would be assigned the value "\textbf{is}",
and \textbf{\emph{v3}} would be assigned the
value "\textbf{a sentence.}".
 
Leading blanks are removed from each word in the string before it is
assigned to a variable, as is the blank that delimits the end of the
word.
Thus, variables set in this manner (\textbf{\emph{v1}}
and \textbf{\emph{v2}} in the example) will never have leading or
trailing blanks, though \textbf{\emph{v3}} could have both leading
and trailing blanks.
 Note that the variables assigned values in a template are always
given a new value and so if there are fewer words in the string than
variables in the template then the unused variables will be set to the
null string.
 The second parsing mechanism uses a literal string in a template as a
pattern, to split up the string.
For example:
\begin{lstlisting}
parse 'To be, or not to be?' w1 ',' w2
\end{lstlisting}
would cause the string to be scanned for the comma, and then split
at that point; the variable \textbf{\emph{w1}} would be set
to "\textbf{To be}", and \textbf{\emph{w2}} is set to
"\textbf{ or not to be?}".
Note that the pattern itself (and \textbf{only} the pattern) is
removed from the string.
Each section of the string is treated in just the same way as the whole
string was in the previous example, and so either section could be split
up into words.
 Thus, in:
\begin{lstlisting}
parse 'To be, or not to be?' w1 ',' w2 w3 w4
\end{lstlisting}
\textbf{\emph{w2}} and \textbf{\emph{w3}} would be
assigned the values "\textbf{or}" and "\textbf{not}",
and \textbf{\emph{w4}} would be assigned the remainder:
"\textbf{to be?}".
 
If the string in the last example did not contain a comma, then
the pattern would effectively "match" the end of the string, so
the variable to the left of the pattern would get the entire input
string, and the variables to the right would be set to a null string.
 The pattern may be specified as a variable, by putting the variable
name in parentheses.  The following instructions therefore have the
same effect as the last example:
\begin{lstlisting}
c=','
parse 'To be, or not to be?' w1 (c) w2 w3 w4
\end{lstlisting}
 The third parsing mechanism is the numeric positional
pattern.
This works in the same way as the string pattern except that it
specifies a column number.  So:
\begin{lstlisting}
parse 'Flying pigs have wings' x1 5 x2
\end{lstlisting}
would split the string at the fifth column,
so \textbf{\emph{x1}} would be "\textbf{Flyi}"
and \textbf{\emph{x2}} would start at column 5 and
so be "\textbf{ng pigs have wings}".
 More than one pattern is allowed, so for example:
\begin{lstlisting}
parse 'Flying pigs have wings' x1 5 x2 10 x3
\end{lstlisting}
would split the string at columns 5 and 10,
so \textbf{\emph{x2}} would be
"\textbf{ng pi}" and \textbf{\emph{x3}} would be
"\textbf{gs have wings}".
 The numbers can be relative to the last number used, so:
\begin{lstlisting}
parse 'Flying pigs have wings' x1 5 x2 +5 x3
\end{lstlisting}
would have exactly the same effect as the last example; here
the \textbf{+5} may be thought of as specifying the length of the
string to be assigned to \textbf{\emph{x2}}.
 As with literal string patterns, the positional patterns can
be specified as a variable by putting the name of a variable, in
parentheses, in place of the number.
An absolute column number should then be indicated by using an equals
sign ("\textbf{=}") instead of a plus or minus sign.
The last example could therefore be written:
\begin{lstlisting}
start=5
length=5
data='Flying pigs have wings'
parse data  x1 =(start) x2 +(length) x3
\end{lstlisting}
 String patterns and positional patterns can be mixed (in effect the
beginning of a string pattern just specifies a variable column number)
and some very powerful things can be done with templates.
The next section describes in more detail how the various mechanisms
interact.
\index{Parsing,definition}
\section{Parsing definition}\label{}
 This section describes the rules that govern parsing.
\index{Parsing,general rules}
 In its most general form, a template consists of alternating pattern
specifications and variable names.  Blanks may be added between
patterns and variable names to separate the tokens and to improve
readability.  The patterns and variable names are used strictly in
sequence from left to right, and are used once only.  In practice,
various simpler forms are used in which either variable names or
patterns may be omitted; we can therefore have variable names without
patterns in between, and patterns without intervening variable names.
 In general, the value assigned to a variable is that sequence of
characters in the input string between the point that is matched by the
pattern on its left and the point that is matched by the pattern on its
right.
 If the first item in a template is a variable, then there is an
implicit pattern on the left that matches the start of the string, and
similarly if the last item in a template is a variable then there is an
implicit pattern on the right that matches the end of the string.
Hence the simplest template consists of a single variable name which in
this case is assigned the entire input string.
 Setting a variable during parsing is identical in effect to setting a
variable in an assignment.
 The constructs that may appear as patterns fall into two categories;
patterns that act by searching for a matching string (literal
patterns), and numeric patterns that specify an absolute or relative
position in the string (positional patterns).
Either of these can be specified explicitly in the template, or
alternatively by a reference to a variable whose value is to be used
as the pattern.
 For the following examples, assume that the following sample string
is being parsed; note that all blanks are significant - there are
two blanks after the first word "\textbf{is}" and also after the
second comma:
\begin{lstlisting}
'This is  the text which, I think,  is scanned.'
\end{lstlisting}
\subsection{Parsing with literal patterns}
\index{,}
\index{,}
\index{Parsing,literal patterns}
\index{Literal patterns,}
 Literal patterns cause scanning of the data string to find a
sequence that matches the value of the literal.  Literals are expressed
as a quoted string.  The null string matches the end of the data.
 The template:
\begin{lstlisting}
w1 ',' w2 ',' w3
\end{lstlisting}
when parsing the sample string, results in:
\begin{alltt}
\emph{w1} \emph{has the value} "This is  the text which"
\emph{w2} \emph{has the value} " I think"
\emph{w3} \emph{has the value} "  is scanned."
\end{alltt}
 Here the string is parsed using a template that asks that each of
the variables receive a value corresponding to a portion of the
original string between commas; the commas are given as quoted strings.
Note that the patterns themselves are removed from the data being
parsed.
 A different parse would result with the template:
\begin{lstlisting}
w1 ',' w2 ',' w3 ',' w4
\end{lstlisting}
which would result in:
\begin{alltt}
\emph{w1} \emph{has the value} "This is  the text which"
\emph{w2} \emph{has the value} " I think"
\emph{w3} \emph{has the value} "  is scanned."
\emph{w4} \emph{has the value} ""  (null string)
\end{alltt}
 This illustrates an important rule.  When a match for a pattern
cannot be found in the input string, it instead "matches" the end
of the string.  Thus, no match was found for the third \textbf{','} in
the template, and so \textbf{\emph{w3}} was assigned the rest of
the string. \textbf{\emph{w4}} was assigned a null string
because the pattern on its left had already reached the end of the
string.
 Note that \textbf{all} variables that appear in a template in this
way are assigned a new value.
\subsection{Parsing strings into words}
\index{Words,in parsing}
\index{Parsing,selecting words}
 If a variable is directly followed by one or more other variables,
then the string selected by the patterns is assigned to the variables
in the following manner.
Each blank-delimited word in the string is
assigned to each variable in turn, except for the last variable in the
group (which is assigned the remainder of the string).
The values of the variables which are assigned words will have neither
leading nor trailing blanks.
 Thus the template:
\begin{lstlisting}
w1 w2 w3 w4 ','
\end{lstlisting}
would result in:
\begin{alltt}
\emph{w1} \emph{has the value} "This'
\emph{w2} \emph{has the value} "is"
\emph{w3} \emph{has the value} "the"
\emph{w4} \emph{has the value} "text which"
\end{alltt}
Note that the final variable (\textbf{\emph{w4}} in this
example) could have had both leading blanks and trailing blanks, since
only the blank that delimits the previous word is removed from the data.
 Also observe that this example is not the same as specifying
explicit blanks as patterns, as the template:
\begin{lstlisting}
w1 ' ' w2 ' ' w3 ' ' w4 ','
\end{lstlisting}
would in fact result in:
\begin{alltt}
\emph{w1} \emph{has the value} "This'
\emph{w2} \emph{has the value} "is"
\emph{w3} \emph{has the value} ""  (null string)
\emph{w4} \emph{has the value} "the text which"
\end{alltt}
since the third pattern would match the third blank in the data.
 In general, when a variable is followed by another variable then
parsing of the input into individual words is implied.
The parsing process may be thought of as first splitting the original
string up into other strings using the various kinds of patterns, and
then assigning each of these new strings to (zero or more) variables.
\subsection{Use of the period as a placeholder}\label{refplaceh}
\index{Period,as placeholder in parsing}
\index{. (period),as placeholder in parsing}
 A period (separated from any symbols by at least one blank) acts as a
placeholder in a template.
It has exactly the same effect as a variable name, except that no
variable is set.
It is especially useful as a "dummy variable" in a list of
variables, or to collect (ignore) unwanted information at the end of a
string.  Thus the template:
\begin{lstlisting}
 . . . word4 .
\end{lstlisting}
would extract the fourth word ("\textbf{text}") from the sample
string and place it in the variable \textbf{\emph{word4}}.
Blanks between successive periods in templates may be omitted, so the
template:
\begin{lstlisting}
 ... word4 .
\end{lstlisting}
would have the same result as the last template.
\subsection{Parsing with positional patterns}
\index{Parsing,positional patterns}
\index{Positional patterns,}
 Positional patterns may be used to cause the parsing to occur on the
basis of position within the string, rather than on its contents.
\index{Signs in parsing templates,}
They take the form of whole numbers, optionally preceded by a plus,
minus, or equals sign which indicate relative or absolute positioning.
These may cause the matching operation to "back up" to an earlier
position in the data string, which can only occur when positional
patterns are used.
\index{Absolute,column specification in parsing}
\index{Column specification in parsing,}
 \textbf{Absolute positional patterns:}
A number in a template that is \textbf{not} preceded by a sign
refers to a particular (absolute)
character column in the input, with 1 referring to the first
column.
For example, the template:
\begin{lstlisting}
s1 10 s2 20 s3
\end{lstlisting}
results in:
\begin{alltt}
\emph{s1} \emph{has the value} "This is  "
\emph{s2} \emph{has the value} "the text w"
\emph{s3} \emph{has the value} "hich, I think,  is scanned."
\end{alltt}
 Here \textbf{\emph{s1}} is assigned characters from the first
through the ninth character, and \textbf{\emph{s2}} receives input
characters 10 through 19.
As usual the final variable, \textbf{\emph{s3}}, is assigned the
remainder of the input.
 
\index{= equals sign,in parsing template}
\index{Parsing,absolute columns}
An equals sign ("\textbf{=}") may be placed before the number
to indicate explicitly that it is to be used as an absolute column
position; the last template could have been written:
\begin{lstlisting}
s1 =10 s2 =20 s3
\end{lstlisting}
 A positional pattern that has no sign or is preceded by the
equals sign is known as an \emph{absolute positional pattern}.
\index{Absolute,positional pattern}
\index{Relative column specification in parsing,}
\index{+ plus sign,in parsing template}
\index{- minus sign,in parsing template}
 \textbf{Relative positional patterns:}
A number in a template that is preceded by a plus or minus sign
indicates
movement relative to the character position at which the previous
pattern match occurred.
This is a \emph{relative positional pattern}.
\index{Relative positional pattern,}
 If a plus or minus
is specified, then the position used for the next
match is calculated by adding (or subtracting) the number given to the
last matched position.
The last matched position is the position of the first character of the
last match, whether specified numerically or by a string.
 
For example, the instructions:
\begin{lstlisting}
parse '123456789'  3 w1 +3 w2 3 w3
\end{lstlisting}
result in
\begin{alltt}
\emph{w1} \emph{has the value} "345"
\emph{w2} \emph{has the value} "6789"
\emph{w3} \emph{has the value} "3456789"
\end{alltt}
The \textbf{+3} in this case is equivalent to the absolute
number \textbf{6} in the same position, and may also be considered to
be specifying the length of the data string to be assigned to the
variable \textbf{\emph{w1}}.
 This example also illustrates the effects of a positional pattern
that implies movement to a character position to the left of (or to)
the point at which the last match occurred.
The variable on the left is assigned characters through the end of the
input, and the variable on the right is, as usual, assigned characters
starting at the position dictated by the pattern.
 A useful effect of this is that multiple assignments can be made:
\begin{lstlisting}
parse x 1 w1 1 w2 1 w3
\end{lstlisting}
This results in assigning the (entire) value
of \textbf{\emph{x}}
to \textbf{\emph{w1}}, \textbf{\emph{w2}},
and \textbf{\emph{w3}}.
(The first "\textbf{1}" here could be omitted as it is
effectively the same as the implicit starting pattern described at the
beginning of this section.)
 If a positional pattern specifies a column that is greater than the
length of the data, it is equivalent to specifying the end of the data
(\emph{i.e.}, no padding takes place).
Similarly, if a pattern specifies a column to the left of the first
column of the data, this is not an error but instead is taken to
specify the first column of the data.
 Any pattern match sets the "last position" in a string to which
a relative positional pattern can refer.
The "last position" set by a literal pattern is the position at
which the match occurred, that is, the position in the data of the
\emph{first} character in the pattern.
The literal pattern in this case is \textbf{not} removed from the
parsed data.
Thus the template:
\begin{lstlisting}
',' -1 x +1
\end{lstlisting}
 will:
\begin{enumerate}
\item Find the first comma in the input (or the end of the string if
there is no comma).
\item Back up one position.
\item Assign one character (the character immediately preceding the comma
or end of string) to the variable \textbf{\emph{x}}.
\end{enumerate}
 One possible application of this is looking for abbreviations in a
string. Thus the instruction:
\begin{lstlisting}
/* Ensure options have a leading blank and are
   in uppercase before parsing. */
parse (' 'opts).upper ' PR' +1 prword ' '
\end{lstlisting}
will set the variable \textbf{\emph{prword}} to the first word
in \textbf{\emph{opts}} that starts with "\textbf{PR}" (in
any case), or will set it to the null string if no such word exists.
 \textbf{Notes:}
\begin{enumerate}
\item The positional patterns \textbf{+0} and \textbf{-0} are valid,
have the same effect, and may be used to include the whole of a previous
literal (or variable) pattern within the data string to be parsed into
any following variables.
\item As illustrated in the last example, patterns may follow each other
in the template without intervening variable names.  In this case each
pattern is obeyed in turn from left to right, as usual.
\item There may be blanks between the sign in a positional pattern and
the number, because \nr{} defines that blanks adjacent to special
characters are removed.
\end{enumerate}
\subsection{Parsing with variable patterns}
\index{Variables,in parsing patterns}
\index{Parsing,variable patterns}
\index{Parentheses,in parsing templates}
 It is sometimes desirable to be able to specify a pattern by using
the value of a variable instead of a fixed string or number.
This may be achieved by placing the name of the variable to be used as
the pattern in parentheses (blanks are not necessary either inside or
outside the parentheses, but may be added if desired).
\index{Variable reference,in parsing template}
This is called a \emph{variable reference}; the value of the variable
is converted to string before use, if necessary.
 If the parenthesis to the left of the variable name is not preceded
by an equals, plus, or minus sign ("\textbf{=}",
"\textbf{+}", or "\textbf{-}")
the value of the variable is then used as though it were a literal
(string) pattern.
The variable may be one that has been set earlier in the parsing
process, so for example:
\begin{lstlisting}
input="L/look for/1 10"
parse input  verb 2 delim +1 string (delim) rest
\end{lstlisting}
will set:
\begin{alltt}
\emph{verb} \emph{to} 'L'
\emph{delim} \emph{to} '/'
\emph{string} \emph{to} 'look for'
\emph{rest} \emph{to} '1 10'
\end{alltt}
 If the left parenthesis \textbf{is} preceded by an equals, plus,
or minus sign then the value of the variable is used as an absolute or
relative positional pattern (instead of as a literal string pattern).
In this case the value of the variable must be a non-negative
whole number, and (as before) it may have been set earlier in the
parsing process.
\index{,}
\index{,}
\index{,}
\index{,}

\chapter{Numbers and Arithmetic}\label{refnums}
\index{Arbitrary precision arithmetic,}
\index{Precision,arbitrary}
\index{R\textsc{exx},arithmetic}
 \nr{} arithmetic attempts to carry out the usual operations
(including addition, subtraction, multiplication, and division) in as
"natural" a way as possible.
What this really means is that the rules followed are those that are
conventionally taught in schools and colleges.
However, it was found that unfortunately the rules used vary
considerably (indeed much more than generally appreciated) from person
to person and from application to application and in ways that are not
always predictable.
The \nr{} arithmetic described here is therefore a compromise which
(although not the simplest) should provide acceptable results in most
applications.
\section{Introduction}\label{arithintro}
 
\index{Numbers,}
Numbers can be expressed in \nr{} very flexibly (leading and trailing
blanks are permitted, exponential notation may be used) and follow
conventional syntax.
Some valid numbers are:
\begin{alltt}
     12          /* A whole number               */
   '-76'         /* A signed whole number        */
     12.76       /* Some decimal places          */
 ' +  0.003 '    /* Blanks around the sign, etc. */
     17.         /* Equal to 17                  */
      '.5'       /* Equal to 0.5                 */
     4E+9        /* Exponential notation         */
      0.73e-7    /* Exponential notation         */
\end{alltt}
\index{Exponential notation,}
(Exponential notation means that the number includes a sign and a power
of ten following an "\textbf{E}" that indicates how the decimal
point will be shifted.  Thus \textbf{4E+9} above is just a short way
of writing \textbf{4000000000}, and \textbf{0.73e-7} is short
for \textbf{0.000000073}.)
\index{Operators,arithmetic}
\index{Numbers,arithmetic on}
\index{Arithmetic,operators}
\index{Integer division,}
\index{Division,integer}
\index{Remainder operator,}
 The arithmetic operators include
addition (indicated by a "\textbf{+}"),
subtraction ("\textbf{-}"),
multiplication ("\textbf{*}"),
power ("\textbf{**}"), and
division ("\textbf{/}").
There are also two further division operators:
integer divide ("\textbf{\%}") which divides and returns the integer part, and
remainder ("\textbf{//}") which divides and returns the remainder.
Prefix plus ("\textbf{+}") and
prefix minus ("\textbf{-}") operators are also provided.
\index{Rounding,}
 When two numbers are combined by an operation, \nr{} uses a set of
rules to define what the result will be (and how the result is to be
represented as a character string).
These rules are defined in the next section, but in summary:
\begin{itemize}
\item Results will be calculated with up to some maximum number of
significant digits.
That is, if a result required more than 9 digits it would normally be
rounded to 9 digits.
For instance, the division of 2 by 3 would result in 0.666666667 (it
would require an infinite number of digits for perfect accuracy).
 
You can change the default of 9 significant digits by using the
\keyword{numeric digits} instruction.  This lets you calculate using
as many digits as you need - thousands, if necessary.
\item Except for the division and power operators, trailing zeros are
preserved (this is in contrast to most electronic calculators, which
remove all trailing zeros in the decimal part of results).
So, for example:
\begin{alltt}
2.40 + 2  =>  4.40
2.40 - 2  =>  0.40
2.40 * 2  =>  4.80
2.40 / 2  =>  1.2
\end{alltt}
This preservation of trailing zeros is desirable for most
calculations (and especially financial calculations).
 If necessary, trailing zeros may be easily removed with the
 \textbf{strip} method (see page \pageref{refstrip}) , or by division by 1.
\item A zero result is always expressed as the single
digit \textbf{'0'}.
\item 
Exponential form is used for a result depending on its value and
the setting of \keyword{numeric digits} (the default is 9 digits).
If the number of places needed before the decimal point exceeds this
setting, or the absolute value of the number is less
than \textbf{0.000001}, then the number will be expressed in
exponential notation; thus
\begin{alltt}
1e+6 * 1e+6
\end{alltt}
results in "\textbf{1E+12}" instead of
"\textbf{1000000000000}", and
\begin{alltt}
1 / 3E+10
\end{alltt}
results in "\textbf{3.33333333E-11}" instead of
"\textbf{0.0000000000333333333}".
\item 
Any mixture of Arabic numerals (0-9) and  Extra digits (see page \pageref{refsyms}) 
can be used for the digits in numbers used in calculations.  The results
are expressed using Arabic numerals.
\end{itemize}
\section{Definition}\label{arithdefinition}
 This definition describes arithmetic for \nr{} strings
(type \textbf{R\textsc{exx}}).
The arithmetic operations are identical to those defined in the ANSI
standard for R\textsc{exx}.
\index{ANSI standard,arithmetic definition}
\footnote{
:cit.American National Standard for Information Technology -
Programming Language REXX, X3.274-1996:ecit., American National
Standards Institute, New York, 1996.
}
\subsection{Numbers}\label{refdefnum}
\index{Numbers,definition}
 A \emph{number} in \nr{} is a character string that includes one or
more decimal digits, with an optional decimal point.
The decimal point may be embedded in the digits, or may be prefixed or
suffixed to them.
The group of digits (and optional point) thus constructed may have
leading or trailing blanks, and an optional sign ("\textbf{+}"
or "\textbf{-}") which must come before any digits or decimal
point.
The sign may also have leading or trailing blanks.
Thus:
\index{Numeric,part of a number}
\index{Digits,in numbers}
\begin{alltt}
sign    ::=  + | -
digit   ::=  0 | 1 | 2 | 3 | 4 | 5 | 6 | 7 | 8 | 9
digits  ::=  digit [digit]...
numeric ::=  digits . [digits]
             | [.] digits
number  ::=  [blank]... [sign [blank]...]
             numeric [blank]...
\end{alltt}

\index{Extra digits,in numbers}
where if the implementation supports  extra digits (see page \pageref{refsyms}) 
these are also accepted as \emph{digit}s, providing that they
represent values in the range zero through nine.
In this case each extra digit is treated as though it were
the corresponding character in the range 0-9.
\index{Period,in numbers}
\index{. (period),in numbers}
 Note that a single period alone is not a valid number.
\subsection{Precision}\label{refndi2}
\index{Arithmetic,precision}
\index{Precision,of arithmetic}
\index{Significant digits, in arithmetic,}
\index{DIGITS,on NUMERIC instruction}
\index{NUMERIC,DIGITS}
 The maximum number of significant digits that can result from an
arithmetic operation is controlled by the \keyword{digits} keyword on the
 \keyword{numeric} instruction (see page \pageref{refnumeric}) :
\begin{alltt}
\keyword{numeric digits} [\emph{expression}];
\end{alltt}
The expression is evaluated and must result in a positive whole
number.
This defines the precision (number of significant digits) to which
arithmetic calculations will be carried out; results will be rounded to
that precision,
if necessary.
 If no expression is specified, then the default precision is used.
The default precision is 9, that is, all implementations must support
at least nine digits of precision.  An implementation-dependent maximum
(equal to or larger than 9) may apply: an attempt to exceed this will
cause execution of the instruction to terminate with an exception.
Thus if an algorithm is defined to use more than 9 digits then if
the \keyword{numeric digits} instruction succeeds then the computation
will proceed and produce identical results to any other implementation.
 Note that \keyword{numeric digits} may set values below the default of
nine.
Small values, however, should be used with care - the loss of
precision and rounding thus requested will affect all \nr{}
computations, including (for example) the computation of new values for
the control variable in loops.
 
In the remainder of this section, the notation \textbf{digits} refers
to the current setting of \keyword{numeric digits}.
This setting may also be referred to in expressions in programs by using
the \textbf{digits}  special word (see page \pageref{refspecial}) .
\subsection{Arithmetic operators}
\index{Operators,arithmetic}
\index{Numbers,arithmetic on}
\index{Arithmetic,operators}
 
\nr{} arithmetic is effected by the operators "\textbf{+}",
"\textbf{-}", "\textbf{*}", "\textbf{/}",
"\textbf{\%}", "\textbf{//}", and "\textbf{**}"
(add, subtract, multiply, divide, integer divide, remainder, and power)
which all act upon two terms, together with the prefix operators
"\textbf{+}" and "\textbf{-}" (plus and minus)
which both act on a single term.
The result of all these operations is a \nr{} string, of
type \textbf{R\textsc{exx}}.
This section describes the way in which these operations are carried
out.
 Before every arithmetic operation, the term or terms being operated
upon have any extra digits converted to the corresponding Arabic numeral
(the digits 0-9).  They then have leading zeros removed (noting the
position of any decimal point, and leaving just one zero if all the
digits in the number are zeros) and are then truncated
to \textbf{digits+1} significant digits
\footnote{
\index{Guard digit in arithmetic,}
That is, to the precision set by \keyword{numeric digits}, plus one extra
"guard" digit.
}
(if necessary) before being used in the computation.
The operation is then carried out under up to double that precision, as
described under the individual operations below.
When the operation is completed, the result is rounded if necessary to
the precision specified by the \keyword{numeric digits} instruction.
\index{Rounding,definition}
 Rounding is done in the "traditional" manner, in that the extra
(guard) digit is inspected and values of 5 through 9 are rounded up,
and values of 0 through 4 are rounded down.
\footnote{
\index{,}
Even/odd rounding would require the ability to calculate to arbitrary
precision (that is, to a precision not governed by the setting of
\keyword{numeric digits}) at any time and is therefore not the mechanism
defined for \nr{}.
}
 A conventional zero is supplied preceding a decimal point if
otherwise there would be no digit before it.  Trailing zeros are
retained for addition, subtraction, and multiplication, according to
the rules given below, except that a result of zero is always expressed
as the single character \textbf{'0'}.  For division, insignificant
trailing zeros are removed after rounding.
 
The  \textbf{format} method (see page \pageref{refformat})  is defined to allow a
number to be represented in a particular format if the standard result
provided by \nr{} does not meet requirements.
\subsection{Arithmetic operation rules - basic operators}\label{}
\index{Trailing zeros,}
\index{Arithmetic,operation rules}
 The basic operators (addition, subtraction, multiplication, and
division) operate on numbers as follows:
\begin{description}
\item[Addition and subtraction]
\index{Addition,definition}
\index{+ plus sign,addition operator}
\index{Subtraction,definition}
\index{- minus sign,subtraction operator}
\index{Prefix operators,arithmetic}

If either number is zero then the other number, rounded
to \textbf{digits} digits if necessary, is used as the result (with
sign adjustment as appropriate).
Otherwise, the two numbers are extended on the right and left as
necessary up to a total maximum of \textbf{digits+1} digits.
 
The number with smaller absolute value may therefore lose some or
all of its digits on the right.
\footnote{
In the example, the number \textbf{yy.yyyyy} would have three digits
truncated if \textbf{digits} were \textbf{5}.
}
The numbers are then added or subtracted as appropriate.  For example:
\begin{alltt}
xxxx.xxx + yy.yyyyy
\end{alltt}
becomes:
\begin{alltt}
  xxxx.xxx00
+ 00yy.yyyyy
------------
  zzzz.zzzzz
\end{alltt}
.sumadd
The result is then rounded to \textbf{digits} digits if necessary,
taking into account any extra (carry) digit on the left after an
addition, but otherwise counting from the position corresponding to the
most significant digit of the terms being added or subtracted.
Finally, any insignificant leading zeros are removed.
 The \emph{prefix operators} are evaluated using the same rules;
the operations "\textbf{+number}" and "\textbf{-number}"
are calculated as "\textbf{0+number}" and
"\textbf{0-number}", respectively.
\item[Multiplication]
\index{Multiplication,definition}
\index{* multiplication operator,}

The numbers are multiplied together ("long multiplication")
resulting in a number which may be as long as the sum of the lengths of
the two operands.  For example:
\begin{alltt}
xxx.xxx * yy.yyyyy
\end{alltt}
becomes:
\begin{alltt}
zzzzz.zzzzzzzz
\end{alltt}
and the result is then rounded to \textbf{digits} digits if
necessary, counting from the first significant digit of the result.
\item[Division]
\index{Division,definition}

For the division:
\begin{alltt}
yyy / xxxxx
\end{alltt}
the following steps are taken: first, the number
"\textbf{yyy}" is extended
with zeros on the right until it is larger than
the number "\textbf{xxxxx}" (with note being taken of the change
in the power of ten that this implies).  Thus in this example,
"\textbf{yyy}"
might become
"\textbf{yyy00}".
Traditional long division then takes place, which can be written:
\begin{alltt}
         zzzz
      .------
xxxxx | yyy00
\end{alltt}

The length of the result ("\textbf{zzzz}") is such that the
rightmost "\textbf{z}" will be at least as far right as the
rightmost digit of the (extended) "\textbf{y}" number in the
example.  During the division, the "\textbf{y}" number will be
extended further as necessary, and the "\textbf{z}" number
(which will not include any leading zeros) may increase up
to \textbf{digits+1} digits, at which point the division stops and the
result is rounded.
Following completion of the division (and rounding if necessary),
insignificant trailing zeros are removed.
\end{description}
 \textbf{Examples:}
\begin{alltt}
/* With 'numeric digits 5' */
12+7.00     ==  19.00
1.3-1.07    ==  0.23
1.3-2.07    ==  -0.77
1.20*3      ==  3.60
7*3         ==  21
0.9*0.8     ==  0.72
1/3         ==  0.33333
2/3         ==  0.66667
5/2         ==  2.5
1/10        ==  0.1
12/12       ==  1
8.0/2       ==  4
\end{alltt}
\textbf{Note: }With all the basic operators, the position of the decimal point
in the terms being operated upon is arbitrary.
The operations may be carried out as integer operations with the
exponent being calculated and applied afterwards.
Therefore the significant digits of a result are not in any way
dependent on the position of the decimal point in either of the terms
involved in the operation.
\subsection{Arithmetic operation rules - additional operators}
 The operation rules for the power ("\textbf{**}"),
integer division ("\textbf{\%}"), and remainder
("\textbf{//}") operators are as follows:
\begin{description}
\item[Power]\label{refpower}

\index{Exponentiation,definition}
\index{Power operator,definition}
The "\textbf{**}" (power) operator raises a number (on the
left of the operator) to a power (on the right of the operator).
The term on the right is rounded to \textbf{digits} digits (if
necessary), and must, after any rounding, be a whole number, which may
be positive, negative, or zero.
If negative, the absolute value of the power is used, and then the
result is inverted (divided into 1).
 
For calculating the power, the number is effectively multiplied by
itself for the number of times expressed by the power, and finally
trailing zeros are removed (as though the result were divided by one).
 In practice (see note below for the reasons), the power is
calculated by the process of left-to-right binary reduction.
For "\textbf{x**n}": "\textbf{n}" is converted to
binary, and a temporary accumulator is set to 1.
If "\textbf{n}" has the value 0 then the initial calculation is
complete.
Otherwise each bit (starting at the first non-zero bit) is inspected
from left to right.
If the current bit is 1 then the accumulator is multiplied by
"\textbf{x}".
If all bits have now been inspected then the initial calculation is
complete, otherwise the accumulator is squared by multiplication and the
next bit is inspected.
When the initial calculation is complete, the temporary result is
divided into 1 if the power was negative.
 
The multiplications and division are done under the normal
arithmetic operation rules, detailed earlier in this section, using a
precision of \textbf{digits+elength+1} digits.
Here, \textbf{elength} is the length in decimal digits of the integer
part of the whole number "\textbf{n}" (\emph{i.e.}, excluding any sign,
decimal part, decimal point, or insignificant leading zeros, as though
the operation \textbf{n\%1} had been carried out and any sign removed).
Finally, the result is rounded to \textbf{digits} digits, if
necessary, and insignificant trailing zeros are removed.
\item[Integer division]

\index{Integer division,definition}

The "\textbf{\%}" (integer divide) operator divides two numbers
and returns the integer part of the result.
The result returned is defined to be that which would result from
repeatedly subtracting the divisor from the dividend while the dividend
is larger than the divisor.  During this subtraction, the absolute
values of both the dividend and the divisor are used: the sign of the
final result is the same as that which would result if normal division
were used.
 The result returned will have no fractional part (that is, no
decimal point or zeros following it).
If the result cannot be expressed exactly within \textbf{digits}
digits, the operation is in error and will fail - that is, the
result cannot have more digits than the current setting of \keyword{numeric
digits}.
For example, \textbf{10000000000\%3} requires ten digits to express the
result exactly (\textbf{3333333333}) and would therefore fail
if \textbf{digits} were \textbf{9} or smaller.
\item[Remainder]

\index{Remainder operator,definition}
The "\textbf{//}" (remainder) operator will return the remainder
from integer division, and is defined
as being the residue of the dividend after the operation of calculating
integer division as just described.
The sign of the remainder, if non-zero, is the same as that of the
original dividend.
 This operation will fail under the same conditions as integer
division (that is, if integer division on the same two terms would
fail, the remainder cannot be calculated).
\end{description}
 \textbf{Examples:}
\begin{alltt}
/* Again with 'numeric digits 5' */
2**3        ==  8
2**-3       ==  0.125
1.7**8      ==  69.758
2\%3         ==  0
2.1//3      ==  2.1
10\%3        ==  3
10//3       ==  1
-10//3      ==  -1
10.2//1     ==  0.2
10//0.3     ==  0.1
3.6//1.3    ==  1.0
\end{alltt}
 \textbf{Notes:}
\begin{enumerate}
\item A particular algorithm for calculating powers is described, since
it is efficient (though not optimal) and considerably reduces the
number of actual multiplications performed.
It therefore gives better performance than the simpler definition of
repeated multiplication.
Since results could possibly differ from those of repeated
multiplication, the algorithm must be defined here so that different
implementations will give identical results for the same operation on
the same values.
Other algorithms for this (and other) operations may always be used, so
long as they give identical results to those described here.
\item The integer divide and remainder operators are defined so that they
may be calculated as a by-product of the standard division operation
(described above).  The division process is ended as soon as the
integer result is available; the residue of the dividend is the
remainder.
\end{enumerate}
\subsection{Numeric comparisons}\label{arithnumericcomparisons}
\index{Arithmetic,comparisons}
\index{Operators,comparative}
\index{Comparison,of numbers}
\index{Numbers,comparison of}
 Any of the  comparative operators (see page \pageref{refcomps})  may be used
for comparing numeric strings.
However, the strict comparisons (for example, "\textbf{==}" and
"\textbf{>>}") are not numeric comparative operators and should
not normally be used for comparing numbers, since they compare from left
to right and leading and trailing blanks (and leading zeros) are
significant for these operators.
 Numeric comparison, using the normal comparative operators, is
effected by subtracting the two numbers (calculating the difference) and
then comparing the result with \textbf{'0'} - that is, the
operation:
\begin{alltt}
A ? B
\end{alltt}
where "\textbf{?}" is any normal comparative operator, is
identical to:
\begin{alltt}
(A - B) ? '0'
\end{alltt}
It is therefore the \emph{difference} between two numbers, when
subtracted under \nr{} subtraction rules, that determines their equality.
\subsection{Exponential notation}
\index{Ten, powers of,}
\index{Pure numbers,}
 The definition of numbers  above (see page \pageref{refdefnum}) 
describes "pure" numbers, in the sense that the character strings
that describe numbers can be very long.
 \textbf{Examples:}
\begin{alltt}
say  10000000000 * 10000000000
/* would display: 100000000000000000000 */

say  0.00000000001 * 0.00000000001
/* would display: 0.0000000000000000000001 */
\end{alltt}
For both large and small numbers some form of exponential notation
is useful, both to make such long numbers more readable and to make
evaluation possible in extreme cases.  In addition, exponential notation
is used whenever the "pure" form would give misleading
information.  For example:
\begin{alltt}
numeric digits 5
say 54321*54321
\end{alltt}
would display "\textbf{2950800000}" if long form were to be
used.
This is misleading, as it appears that the result is an exact multiple
of 100000, and so \nr{} would express the result in exponential
notation, in this case "\textbf{2.9508E+9}".
\index{Numbers,definition}
\index{Mantissa of exponential numbers,}
\index{Significand of exponential numbers,}
\index{Numeric,part of a number}
\index{Powers of ten in numbers,}
\index{Exponential notation,definition}
\index{E-notation,definition}
 The definition of \emph{number} (see above) is therefore extended
by replacing the description of \textbf{numeric} by the following:
\begin{alltt}
mantissa ::=  digits . [digits]
              | [.] digits
numeric  ::=  mantissa [E sign digits]
\end{alltt}
In other words, the numeric part of a number may be followed by an
"\textbf{E}" (indicating an exponential part), a sign,
and an integer following the sign that represents a power of ten that is
to be applied.
The "\textbf{E}" may be in uppercase or lowercase.
Note that no blanks are permitted within this part of a number, but the
integer may have leading zeros.
 \textbf{Examples:}
\begin{alltt}
12E+11  =  1200000000000
12E-5   =  0.00012
 12e+4  =  120000
\end{alltt}
 All valid numbers may be used as data for arithmetic.  The results
of calculations will be returned in exponential form depending on the
setting of \keyword{numeric digits}.
If the number of places needed before the decimal point
exceeds \textbf{digits}, or if the absolute value of the result is
less than \textbf{0.000001}, then exponential form will be used.
The exponential form generated by \nr{} always has a sign following the
"\textbf{E}".
If the exponent is 0 then the exponential part is omitted - that
is, an exponential part of "\textbf{E+0}" will never be
generated.
 If the default format for a number is not satisfactory for a
particular application, then the \textbf{format} method may be used to
control its format.  Using this, numbers may be explicitly converted to
exponential form or even forced to be returned in "pure" form.
\index{Notation,scientific}
\index{Notation,engineering}
\index{Exponential notation,}
\index{E-notation,}
\index{Scientific notation,}
\index{Engineering notation,}
\index{FORM,option of NUMERIC instruction}
\index{NUMERIC,FORM}
\label{refnfo2}
 Different exponential notations may be selected with the
 \keyword{numeric form} instruction (see page \pageref{refnform}) .
This instruction allows the selection of either scientific or
engineering notation.
\emph{Scientific notation} adjusts the power of ten so there is a
single non-zero digit to the left of the decimal point.
\emph{Engineering notation} causes powers of ten to be expressed as a
multiple of three - the integer part may therefore range
from \textbf{1} through \textbf{999}.
 \textbf{Examples:}
\begin{alltt}
numeric form scientific
say 123.45 * 1e11
/* would display: 1.2345E+13 */

numeric form engineering
say 123.45  * 1e11
/* would display: 12.345E+12 */
\end{alltt}
 The default exponential notation is scientific.
\subsection{Whole numbers}\label{refwholed}
\index{Whole numbers,definition}
\index{DIGITS,effect on whole numbers}
 Within the set of numbers understood by \nr{} it is useful to
distinguish the subset defined as \emph{whole numbers}.
 
A \emph{whole number} in \nr{} is a number that has a decimal part
which is all zeros (or that has no decimal part).
\subsection{Numbers used directly by \nr{}}\label{refnumuse}
\index{Numbers,use of by \nr{}}
\index{Functions,numeric arguments of}
\index{Rounding,when numbers used}
\index{DIGITS,rounding when numbers used}
 As discussed above, the result of any arithmetic operation is
rounded (if necessary) according to the setting of \keyword{numeric digits}.
Similarly, when a number (which has not necessarily been involved in an
arithmetic operation) is used directly by \nr{} then the same rounding
is also applied, just as though the operation of adding the number
to \textbf{0} had been carried out.
After this operation, the integer part of the number must have no more
digits than the current setting of \keyword{numeric digits}.
 
In the following cases, the number used must be a whole number and
an implementation restriction on the largest number that can be used
may apply:
\begin{itemize}
\item positional patterns, including variable positional patterns,
in  parsing templates (see page \pageref{refparsing}) 
\item the power value (right hand operand) of the power operator (see page \pageref{refpower}).
\item the values of \emph{exprr} and \emph{exprf} (following the
\keyword{for} keyword) in the  \keyword{loop} instruction (see page \pageref{refloop}) 
\item the value of \emph{exprd} (following the \keyword{digits}
keyword) in the  \keyword{numeric} instruction (see page \pageref{refnumeric}) .
\end{itemize}
 \textbf{Implementation minimum:} A minimum length of 9 digits must
be supported for these uses of whole numbers by a \nr{} language
processor.
\subsection{Implementation independence}
\index{Arithmetic,implementation independence}
 The \nr{} arithmetic rules are defined in detail, so that when a
given program is run the results of all computations are sufficiently
defined that the same answer will result for all correct
implementations.  Differences due to the underlying machine
architecture will not affect computations.
 This contrasts with most other programming languages, and with
 binary arithmetic (see page \pageref{refbinary})  in \nr{}, where the
result obtained may depend on the implementation because the precision
and algorithms used by the language processor are defined by the
implementation rather than by the language.
\subsection{Exceptions and errors}
\index{Exceptions,during arithmetic}
\index{Errors during arithmetic,}
\index{Arithmetic,exceptions}
\index{Arithmetic,errors}
\index{Arithmetic,overflow}
\index{Arithmetic,underflow}
\index{Overflow, arithmetic,}
\index{Underflow, arithmetic,}
 The following exceptions and errors may be signalled during arithmetic:
\begin{itemize}
\item Divide exception
 This exception will be signalled if division by zero was attempted,
or if the integer result of an integer divide or remainder operation had
too many digits.
\item Overflow/Underflow exception
 This exception will be signalled if the exponential part of a result
(from an operation that is not an attempt to divide by zero) would
exceed the range that can be handled by the language processor, when the
result is formatted according to the current settings of \keyword{numeric
digits} and \keyword{numeric form}.
The language defines a minimum capability for the exponential part,
namely exponents whose absolute value is at least as large as the
largest number that can be expressed as an exact integer in default
precision.
Thus, since the default precision is nine, implementations must support
exponents in the range \textbf{-999999999}
through \textbf{999999999}.
\item Insufficient storage
 Storage is needed for calculations and intermediate results, and on
occasion an arithmetic operation may fail due to lack of storage.
This is considered an operating environment error as usual, rather
than an arithmetical exception.
\end{itemize}
 \emph{In the reference implementation, the exceptions and error types
used for these three cases
are \textbf{DivideException}, \textbf{ExponentOverflowException},
and \textbf{OutOfMemoryError}, respectively.}
\index{,}
\index{,}
\index{,}

\chapter{Binary values and operations}\label{"id"}
\index{Binary,arithmetic}
\index{Binary,operations}
\index{Binary,values}
 
By default, arithmetic and string operations in NetRexx are carried out
using the NetRexx string class, \textbf{Rexx}, which offers the robust
set of operators described in  \emph{Expressions and (see page \pageref{refexpr}) 
operators}:ea..
 
\index{Primitive types,}
\index{Types,primitive}
NetRexx implementations, however, may also provide \emph{primitive}
datatypes, as described in  \emph{Types and (see page \pageref{refprims}) 
Classes}:ea..  These primitive types are used for compact storage
of numbers and for fast binary arithmetic, features which are built-in
to the hardware of most computers.
 
\index{Binary classes,}
To make use of binary arithmetic, a class is declared to be a
 \emph{binary class} (see page \pageref{refbincla})  by using the \texttt{binary}
keyword on the \texttt{class} instruction.
In such a class, literal strings and numeric symbols are assigned native
string or primitive types, rather than NetRexx types, where appropriate,
and native binary operations are used to implement operators where
possible, as detailed below.
Implementations may also provide a keyword on the
 \texttt{options} (see page \pageref{refoptions})  instruction that indicates that
all classes in a program are binary classes.
\footnote{
\emph{In the reference implementation, \texttt{options binary} is
used}.
}
 
\index{Binary methods,}
Alternatively, individual methods within a class may be declared to
be a  \emph{binary method} (see page \pageref{refbinme})  by using the \texttt{binary}
keyword on the \texttt{method} instruction.
 
\index{Integers, binary,}
\index{Floating-point numbers, binary,}
\index{Real numbers, binary,}
\index{Binary numbers,}
Binary classes and methods should be used with care.  Although binary
arithmetic can have a considerable performance advantage over
arithmetic that is not implemented in hardware, it can give incorrect
or unexpected results.
In particular, whole numbers (integers) are often held in fixed-sized
data areas (of 8, 16, 32, or 64 bits), and overflowing the data area
during a calculation can result in a positive number becoming negative
and vice versa.
Similarly, binary numbers that are not whole numbers (floating-point
numbers) cannot exactly represent common numbers in the decimal system
(\textbf{0.1}, \textbf{0.2}, \&), and hence can give unexpected
results.
\subsubsection{Operations in binary classes and methods}\label{"id"}
 
In a binary class or method, the following (and only the following)
rules differ from the usual rules:
\begin{description}
\item{Dyadic operations in expressions}
\index{Binary operations,dyadic}
 
If the operands of a dyadic operator both have primitive numeric types
\footnote{
\emph{In the reference implementation, \textbf{boolean} is considered to
be a numeric type (having the values \textbf{0} or \textbf{1})
but \textbf{char} is not.
Characters, and strings or arrays of characters, always use the rules
defined for NetRexx strings.}
}
then binary operations are carried out.  The type of the result is
implementation defined, and is typically the type of the more precise of
the two operands, or of some minimum precision.
\footnote{
\emph{In the reference implementation, the minimum precision is 32 bits,
so an \textbf{int} is returned for results that would otherwise
be \textbf{byte} or \textbf{short}.
If both operands are \textbf{boolean}, however, and the operation is a
logical operation, then the type of the result
is \textbf{boolean}.}
}
Arithmetic operations follow the usual rules of binary arithmetic, as
defined for the underlying environment of the implementation.
 
Note that NetRexx provides both divide and integer divide operators; in
a binary class or method, the divide operator ("\textbf{/}")
converts its operands to floating-point types and returns a
floating-point result, whereas the integer divide operator
("\textbf{\%}") converts its operands to integer types and
returns an integer result.
The remainder operator must accept both integer and floating-point
types.
 
Logical operations (\emph{and}, \emph{or}, and \emph{exclusive
or}) apply to all the bits of the operands, and are not permitted
on floating-point types.
\item{Prefix operations in expressions}
\index{Binary operations,prefix}
\index{Binary operations,monadic}
 
If the operand of a prefix operator has a primitive numeric type, then
the type of the result is the type of the operand, subject to
the same minimum as dyadic operations.
Prefix plus and minus follow the rules of dyadic operators (because they
are defined as being zero plus or minus the operand) with the additional
rule that if acting on a literal number (a constant in the program) then
the result is also considered to be a literal constant.
Logical not (prefix "\textbf{\textbackslash }") does not apply to all
the bits of its operand; instead, it changes a \textbf{0}
to \textbf{1} and vice versa.
\item{Assignments}
\index{Binary classes,assignment}
\index{Binary methods,assignment}
\index{Binary literals,}
\index{Literals, binary,}
\index{Assignment,of literals}
\index{Assignment,binary}
 
In assignments where the value being assigned is the result of an
expression which comprises a string or number literal constant, the
type of the result is defined as follows:
\begin{enumerate}
\item Strings are given the native string type, even for a
single-character literal.
\footnote{
\emph{In the reference implementation, this type is \textbf{java.lang.String}}.
}
\item Numbers are given the smallest possible primitive numeric type that
will contain the literal without loss of information (or minimal loss of
information for numbers with decimal or exponential parts).
If this is smaller than the implementation-defined minimum precision
used for the result of adding the literal to \textbf{0}, then the type
of that minimum precision is used.
 
If the constant is an integer, and no primitive integer binary type has
sufficient precision to hold the number without loss of information,
then the number is treated as a literal string (that is, as though it
were enclosed in quotes).  NetRexx arithmetic would then be used if it
were involved in an arithmetic operation.
\end{enumerate}
 
These rules can apply in assignment instructions, the initial assignment
to the control variable in the \texttt{loop} instruction, or the
assignment of a default value to the argument of a method; the result
type may define the type of the variable (if new, or a method argument).
\item{Control variables in loops}
\index{Binary classes,control variables}
\index{Binary classes,LOOP instruction}
\index{Binary methods,control variables}
\index{Binary methods,LOOP instruction}
\index{Loops,in binary classes and methods}
 
In the \texttt{loop} instruction, if the control variable has a
primitive integer type, and the increment (\texttt{by} value) has a
primitive integer type, then binary arithmetic will be used for stepping
the control variable, following the rules for binary arithmetic in
expressions described above.
 
Similarly, if the control variable has a primitive integer type, and the
end (\texttt{to}) value has a primitive integer type, then binary
arithmetic will be used for the comparison that tests for loop
termination.
\item{Numeric instruction}
\index{Binary classes,NUMERIC instruction}
\index{Binary methods,NUMERIC instruction}
\index{NUMERIC,in binary classes and methods}
 
The \texttt{numeric} instruction does not affect binary operations.
It has the usual effects on operations carried out using NetRexx
arithmetic.
\end{description}
\textbf{Note: }
At all times (whether in binary classes, binary methods, or anywhere
else) implementations may use primitive types and operations, and
techniques such as late binding of types, as an optimization
providing that the results obtained are identical to those defined in
this language definition.
\subsubsection{Binary constructors}\label{"id"}
\index{Binary constructors,}
\index{Constructors,binary}
\index{Conversion,binary constructors}
\index{Conversion,of characters}
\index{Character,encodings}
\index{Character,converting to binary}
\index{Encodings,binary}
 
NetRexx provides special constructors for implementation-defined
primitive types that allow bit-wise construction of primitives.
These \emph{binary constructors} are especially useful for
manipulating the binary encodings of individual characters.
 
The binary constructors follow the same syntax as other constructors,
with the name being that of a primitive type.  All binary constructors
take one argument, which must have a primitive type.
 
The bits of the value of the argument are extended or truncated on the
left to the same length as the bits required for the type of the
constructor (following the usual binary rules of sign extension if the
argument type is a signed numeric type), and a value with the type of
the constructor is then constructed directly from those bits and
returned.
 \textbf{Example:}
 This example illustrates types from the reference implementation,
with 32-bit signed integers of type \textbf{int} and 16-bit Unicode
characters of type \textbf{char}.
\begin{alltt}
i=int 77   -- i is now the integer 77
c=char(i)  -- c is now the character 'M'
j=int(c)   -- j is now the integer 77
\end{alltt}
Note that the conversion
\begin{alltt}
j=int c
\end{alltt}
would have failed, as "\textbf{M}" is not a number.

\chapter{Exceptions}\label{"id"}
\index{Exceptions,}
\index{Signals,}
\index{Caught exceptions,}
 
Exceptional conditions, including errors, in NetRexx are handled by a
mechanism called \emph{Exceptions}.
When an exceptional condition occurs, a \emph{signal} takes place
which may optionally be \emph{caught} by an enclosing control
construct, as detailed below.
 
An exception can be signalled by:
\begin{enumerate}
\item the program's environment, when some processing error occurs (such
as running out of memory, or a problem discovered when reading or
writing a file)
\item a method called by a NetRexx program (if, for example, it is passed
incorrect arguments)
\item the  \texttt{signal} instruction (see page \pageref{refsignal}) .
\end{enumerate}
In all cases, the signal is handled in exactly the same way.
First, execution of the current clause ceases; no further operations
within the clause will be carried out.
\footnote{
This is the only case in which an expression will not be wholly
evaluated, for example.
}
Next, an object that represents the exception is constructed.  The type
of the exception object is implementation-dependent, as described for
the  \texttt{signal} instruction (see page \pageref{refsignal}) , and defines the
type of the exception.  The object constructed usually contains
information about the Exception (such as a descriptive string).
 
Once the object has been constructed, all active \texttt{do} groups,
\texttt{loop} loops, \texttt{if} constructs, and \texttt{select}
constructs in the active method are "unwound", starting with the
innermost, until the exception is caught by a control construct that
specifies a suitable \texttt{catch} clause (see below) for handling the
exception.
 
This unwinding takes place as follows:
\begin{enumerate}
\item No further clauses within the body of the construct will be executed
(in this respect, the signal acts like a \texttt{leave} for the
construct).
\item If a \texttt{catch} clause specifies a type to which
the exception object can be assigned (that is, it matches or is a
superclass of the type of exception object), then the
\emph{instructionlist} following that clause is executed, and the
exception is considered to be handled (no further control constructs
will be unwound).
If more than one \texttt{catch} clause specifies a suitable type, the
first is used.
\item The \emph{instructionlist} following the \texttt{finally} clause
for the construct, if any, is executed.
\item The \texttt{end} clause is executed, hence completing execution of
the construct.
(The only effect of this is that it is seen when tracing.)
\item 
If the exception was handled, then execution resumes as though the
construct completed normally.  If it was not handled, then the process
is repeated for any enclosing constructs.
\end{enumerate}
 
If the exception is not caught by any of the control constructs
enclosing the original point of the exception signal, then
the current active method is terminated, without returning any data, and
the exception is then signalled at the point where the method was
invoked (that is, in the caller).
 
The process of unwinding control constructs and terminating the method
is then repeated in each calling method until the exception is caught or
the initial program invocation method (the main method) is terminated,
in which case the program ends and the environment receives the signal
(it would usually then display diagnostic information).
\subsubsection{Syntax and example}
\index{CATCH,use of}
\index{FINALLY,use of}
 
The constructs that may be used to handle (catch) an exception are
\texttt{do} groups, \texttt{loop} loops, and \texttt{select} constructs.
Specifically, as shown in the syntax diagrams (\emph{q.v.}), where the
\texttt{end} clause can appear in these constructs, zero or more
\texttt{catch} clauses can be used to define exception handlers,
followed by zero or one \texttt{finally} clauses that describe
"clean-up" code for the construct.
The whole construct continues to be ended by an \texttt{end} clause.
 
The syntax of a \texttt{catch} clause is shown in the syntax diagrams.
It always specifies an \emph{exception} type, which may be
qualified.  It may optionally specify a symbol, \emph{vare}, which
is followed by an equals sign.  This indicates that when the exception
is caught then the object representing the exception will be assigned to
the variable \emph{vare}.  If new, the type of the variable will be
\emph{exception}.
 Here is an example of a program that handles some of the exceptions
signalled by methods in the \textbf{Rexx} class; the \texttt{trace
results} instruction is included to show the flow of execution:
\index{Example,of exception handling}
\begin{alltt}
trace results
do                -- could be LOOP i=1 to 10, etc.
  say 1/arg
catch DivideException
  say 'Divide exception'
catch ex=NumberFormatException
  /* 'ex' is assigned the exception object */
  say 'Bad number for division:' ex.getMessage
finally
  say 'Done!'
end
\end{alltt}
In this example, if the argument passed to the program (and hence
placed in \textbf{arg}) is a valid number, then its inverse is
displayed.  If the argument is 0, then "\textbf{Divide
exception}" would be displayed.  If the argument were an invalid
number, the message describing the bad number would be displayed.
For any other exception (such as an \textbf{ExponentOverflowException}),
the program would end and the environment would normally report the
exception.
 
In \textbf{all} cases, the message "\textbf{Done!}" would be
displayed; this would be true even if the body of the \texttt{do}
construct executed a \texttt{return}, \texttt{leave}, or
\texttt{iterate} instruction.  Only an  \texttt{exit} (see page \pageref{refexit}) 
instruction:ea. would cause immediate termination of the construct (and
the program).
\textbf{Note: }The \texttt{finally} keyword, like \texttt{otherwise} in the
\texttt{select} construct, implies a semicolon after it, so the last
\texttt{say} instruction in the example could have appeared on the same
line as the \texttt{finally} without an intervening semicolon.
\subsubsection{Exceptions after catch and finally clauses}
\index{Exceptions,after CATCH clause}
\index{Exceptions,after FINALLY clause}
 
If an exception is signalled in the \emph{instructionlist} following
a \texttt{catch} or \texttt{finally} clause, then the current exception
is considered handled, the \emph{instructionlist} is terminated, and
the new exception is signalled.  It will not be caught by \texttt{catch}
clauses in the current construct.  If it occurs in the
\emph{instructionlist} following a \texttt{catch} clause, then any
\texttt{finally} instructions will be executed, as usual.
 
Similarly, executing a \texttt{return}
or \texttt{exit} instruction within either of the
\emph{instructionlists} completes the handling of any pending
signal.
\subsubsection{Checked exceptions}\label{"id"}
\index{Exceptions,checked}
\index{Checked exceptions,}
 
NetRexx implementations may define certain exceptions as \emph{checked
exceptions}.  These are exceptions that the implementation considers
it useful to check; the checked exceptions that each method may signal
are recorded.  Within \texttt{do} groups, \texttt{loop} loops, and
\texttt{select} constructs, for example, it is then possible to report
if a \texttt{catch} clause tries to catch a checked exception that is
not signalled within the body of the construct.
 
Checked exceptions that are signalled within a method (by a
\texttt{signal} instruction or a method invocation) but not caught by a
\texttt{catch} clause in the method are automatically added to the
\texttt{signals} list for a method.  Implementations that support
checked exceptions are encouraged to provide options that list the
uncaught checked exceptions for methods or enforce the explicit
inclusion of some or all checked exceptions in the \texttt{signals} list
on the method instruction.
 
\emph{In the reference implementation, all exceptions are checked except
those that are subclasses of \textbf{java.lang.RuntimeException}
or \textbf{java.lang.Error}.  These latter are considered so
ubiquitous that almost all methods would signal them.}
 
\emph{Expressions assigned as the initial values of properties must not invoke
methods that may signal checked exceptions.}
 
\emph{The \texttt{strictsignal} option on the \texttt{options} instruction may
be used to enforce the inclusion of all uncaught checked exceptions in
methods' \texttt{signals} lists; this may be used to assure that
any uncaught checked exceptions are intentional.}

\chapter{Thread Pool Support}\label{refthreads}
\index{Threads}

 Support for thread pooling is built into the \nr{} language. 


\begin{description}
\item[rtp=RexxTaskPool(size,number)]
\emph{size} is the number of parallel threads desired - default is the current number of available processors
\emph{number} is the number of the threadpool if using multiple pools - default is pool number 0

\item[rtp.start(runtask)]
runtask needs to be a class that implements the \emph{runnable} interface

\item[rtp.start(maintask,mainparm)]
\emph{maintask} is a NetRexx Java class with a standard "main" method
\emph{mainparm} is a string parm to pass to the main method at startup

\item[rtp.start("taskname",mainparm)]
\emph{taskname} is a string identifying a Java class with a standard main method
\emph{mainparm} is a string parm to pass to the main method at
startup\footnote{the start method returns the RexxTaskPool instance it
  is called on so that multiple calls can be stacked. Due to reflection use when starting "main" methods that call format cannot be interpreted - runnables interpret ok}

\item[rtp.waituntildone]
Blocks until all threads in the pool are finished

\item[rtp.waitforallpools]
Blocks until all threads in all task pools are complete
\end{description}

\section{Examples}
\begin{lstlisting}[label=threadpool,caption=ThreadPool example]
RexxTaskPool(3,1).start(Test(7)).start(Test(8)).start("TestMain","9").start("enviroscan")
RexxTaskPool(9).start(Test(1)).start(Test(2)).start(TestMain("3"),"3").start(enviroscan.class)
RexxTaskPool().start(Test(4)).start(Test(5)).start(TestMain("6"),"6").waituntildone
RexxTaskPool().start(Test(4)).start(Test(5)).start(TestMain("6"),"6").waitforallpools
\end{lstlisting}
\section{Structured Lists Interface}\label{reflists}
\index{lists,}
\index{Structured Lists}
 
A Structured List class is used to create objects which can contain
structured or string encoded lists with associated methods for easy
access. A Structured List class is a subclass of the native NetRexx
"Rexx" class data type which implements the StructuredList interface
as described below and can be used in many cases as a normal NetRexx
data item. Assuming a class named "AStructuredList" is available to
support structured lists:

\begin{itemize}
\item A StructuredList class should have a constructor that sets the
  encoded list value
\end{itemize}
 
This creates an unexamined (unstructured) list object. The object is
an object of type Rexx with additional methods for processing lists. You
must pass a Ruleset to the list object (see the "buildlist" method
below) in order to obtain a structured version of the list object
which can be accessed with list methods. Note that due to the
immutability rule for base Rexx object values, all methods which alter
the content of a structured list return a new structured list
object. This rule also means that if a list is altered by a Java List
view, the new structured list object must be obtained from the
"currentlist" property of the Java List view. 

Structured List formats
known to \nr{} include 
\begin{itemize}
\item WORDLIST
\item DSV
\item CSV
\item TSV
\item XML
\item JSON
\item PYTHON
\end{itemize}

\subsection{Essential List Processing Methods}
This section lists the methods provided by the \emph{StructuredList}
class.

\begin{description}
\item[buildlist(ruleset)]

Returns a structured list form of the encoded list contained in the object's string value after examination with the provided ruleset.

\item[join(anotherStructuredList)]

Returns a structured list containing the elements of the structured list with the elements from another structured list appended to it.

\item[islist]

Returns 1 if the item is a structured list, otherwise returns 0. If a structured list is empty, index "elements" will have a 0 value. 

\item[elementcount]

Returns the count of elements in a structured list.
    
\item[getelement(index)]

Returns the element at the specified location in a structured list.

\item[getJavaList]

Returns a Java List interface view of the structured list. 

\item[index(element,start)]

Returns the numeric index of the first occurence of the given element in the list with index equal to or higher than start or 0 if not found.

\item[insertelement(index, element)]

Adds an element to a list at the specified location. Returns modified list.

\item[replaceanelement(index, element)]

Replaces an element of a list at the specified location. Returns modified list.


\item[deleteanelement(index)]

Removes an element from a list at the specified location. Returns modified list.

\item[sublist(fromIndex,toIndex)]

Creates a list which is a subset of the list containing the elements starting at the from index and ending at the to index. 
\end{description}

\subsection{Convenience Methods}

\begin{description}
\item[append(element)]

Adds an element to the end of a list. Returns modified list.

\item[head]

Returns the first element in a list.

\item[tail]

Returns a list containing all elements except the first in a list.

\item[count(value)]

Returns a count of how many elements have the provided value.

\item[minval]

Returns the lowest value in a list. This is a runtime error if not all list elements are numeric.

\item[maxval]

Returns the highest value in a list. This is a runtime error if not all list elements are numeric.

\item[sum]

Returns the sum of all list elements. This is a runtime error if not all list elements are numeric.

\item[reverselist]

Returns a structured list with the order of the elements reversed.

\item[flatlist]

Returns a structured list with only the list elements (all metadata is
removed). Nested sublists are also flattened. 

\end{description}


\subsection{NetRexx Structured List Format}
A structured list is an ordered sequence of items stored in a NetRexx
native data object (class Rexx) along with meta data describing the
list. Each element of the list can be accessed with a whole number
ranging from 1 to n where n is the number of elements in the list. The
string encoded (aka "serialized") form of the string is the base value
of the Rexx object. If an element of a list is itself a list (ie a
nested list), then elements of the sublist can also be accessed by
whole number indexes so that for example the 3rd element of the nested
list which is the second element of the first list could be found at
myList[2,3] where myList is the object containing the structured
list. Note that although a structured list is a Rexx object, changing
the list directly rather than through the StructuredList interface
methods will cause loss of metadata and unpredictable results for
further list access.

\subsection{Accessing structured lists with NetRexx facilities}
Assuming myList is a Rexx object containing a structured list, then:
\begin{description}


\item[myList["elements"]]

Provides a count of the number of elements in a structured list. This is meaningless if the item is not actually in the structured list format (ie if islist returns 0). 

\item[myList[n,m]]

A list element can be accessed by numeric index using NetRexx indexed variable syntax. If the element is also a list, sub-elements can be accessed using the multiple index syntax.
\end{description}

\subsubsection{Additional indexed values}
\begin{description}
\item["rules"]
The rules used to structure this list (the rules are also a structured list).
The following indexes must be prefixed with "\#" to use if they are whole numbers according to the NetRexx datatype BIF - this avoids conflict with list index numbers.
\item[elementname]
If an element has a name, using the name as an index will return the associated element value.

\item[element]
The element string will return the index of the first location of that element in the list.
elementvalue
If an element has a name the named value string will return the index of the first location of that value in the list.
\end{description}

\subsection{Structured List Ruleset Description}

A list ruleset specifies a set of delimiters and options and can be provided in one of four ways:
\begin{enumerate}
\item A null ruleset (or the string "default") signifies a default ruleset. Default rules are start/end delimiters "(" and ")", a separator " " (a blank), an escape character " (double quote), and the option "escape is quoted string mode".

\item A string such as "CSV", which is a well known list format name, selects a built-in ruleset. For example, a list in CSV format could be decoded like this: inputlist=inputstring.getlist("CSV"). Formats known to a Structured List include "WORDLIST", "DSV", "CSV", "TSV", "XML", "JSON", "PYTHON".

\item A string which provides a human readable custom set of list rules that is itself a decomposable list according to the default ruleset. Rulesets contain two sections: delimiters and options which are specified as separate sublists as in the following example: 

\begin{verbatim}
'delimiters(start("<") end(">") separator(",") meta("/") escape("\\")
 nameseparator("=") ) 
 options(separators-must-follow-sublists "adjacent separators reduce to one")'
\end{verbatim}
\item A ruleset string that is itself an encoded list according to a
  known ruleset can simply be preparsed before use like in this
  example: 
\begin{verbatim}
inputlist=inputstring.buildlist(rulesetstring.buildlist("default"))
\end{verbatim}
\end{enumerate}

\subsubsection{Delimiters}

Any type of delimiter can be specified but the following (along with the ruleset options) define the structure of a list. Other delimiters can be provided and will be recognized and recorded as list elements when they occur which means they are "defacto" separators for the list elements.

\begin{description}
\item[Start]

A delimiter which signals the start of a sublist. Example: start("{")

\item[End]

A delimiter which signals the end of a sublist. Example: End("}")

\item[Escape] 

An escape character used to include delimiters in the list data elements. Example: escape("||")

\item[Separator]

A delimiter used to separate list elements. Example: Separator(",")

\item[NameSeparator]

A delimiter used to associate element names with element values. Example: Namesep("=")
\end{description}

\subsubsection{Options}
The following options are recognized (A "0" in front of the option indicates a default value and a "1" indicates an option which overrides the default value. The additional descriptions in parenthesis are not part of the option. Options can be specified as quoted text or with dashes substituted for spaces. Options can be abbreviated as long as they are unique. Options are not case-sensitive.
\begin{description}
\item[option 1]
\begin{verbatim}
0 = separators follow sublists
1 = sublists are separators
\end{verbatim}
\item[option 2]
\begin{verbatim}
0 = adjacent separators reduce to one
1 = produce empty elements for adjacent separators
\end{verbatim}
\item[option 3]
\begin{verbatim}
0 = translate escape sequences
1 = do not translate escape sequences
\end{verbatim}
\item[option 4]
\begin{verbatim}
0 = whitespace is translated to blank (TAB,FF,LF,CR,VT)
1 = treat whitespace as text
\end{verbatim}
\item[option 5]
\begin{verbatim}
0 = escape is quoted string mode (ie "text, delimiters or double escape like this: "" more text")
1 = single character escape (ie \x)
\end{verbatim}
\item[option 6]
\begin{verbatim}
0 = attribute names are implicit (ie fun(x,y) )
1 = explicit attribute names (ie with delimiter as in fun=(x,y) or
fun:(x,y) )
\end{verbatim}
\item[option 7]
\begin{verbatim}
0 = delimiters are implicit (do not record structural delimiters)
1 = tokens are delimiters (save delimiters as separate elements)
\end{verbatim}
\item[option 8]
\begin{verbatim}
0 = keep leading and trailing whitespace
1 = strip leading and trailing whitespace
\end{verbatim}
\end{description}

\chapter{Methods for \nr{} strings}\label{refbmeth}
\index{R\textsc{exx},class/methods of}
 This section describes the set of methods defined for the \nr{}
string class, \textbf{R\textsc{exx}}.  These are called \emph{built-in
methods}, and include character manipulation, word manipulation,
conversion, and arithmetic methods.
 
Implementations will also provide other methods for the \textbf{R\textsc{exx}}
class (for example, to implement the \nr{} operators or to provide
constructors with primitive arguments), but these are not part of the
\nr{} language.
\footnote{
\emph{Details of the methods provided in the reference implementation are
included in  Appendix C (see page \pageref{refappc}) .}
}
\section{General notes on the built-in methods:}
\begin{enumerate}
\item All methods work on a \nr{} string of type \textbf{R\textsc{exx}}; this
is referred to by the name \emph{string} in the descriptions of the
methods.  For example, if the \textbf{word} method were invoked using
the term:
\begin{alltt}
"Three word phrase".word(2)
\end{alltt}
then in the description of \textbf{word} the name
\emph{string} refers to the string "\textbf{Three word
phrase}", and the name \emph{n} refers to the string
"\textbf{2}".
\item All method arguments are of type \textbf{R\textsc{exx}} and all methods
return a string of type \textbf{R\textsc{exx}}; if a number is returned, it
will be formatted as though 0 had been added with no rounding.
\item 
The first parenthesis in a method call must immediately follow the name
of the method, with no space in between.
\item The parentheses in a method call can be omitted if no
arguments are required and the method call is part of a
 \emph{compound term} (see page \pageref{refcomterm}) .
\footnote{
Unless an implementation-provided option to disallow parenthesis
omission is in force.
}
\item A position in a string is the number of a character in the string,
where the first character is at position 1, \emph{etc.}
\item Where arguments are optional, commas may only be included between
arguments that are present (that is, trailing commas in argument lists
are not permitted).
\item A \emph{pad} argument, if specified, must be exactly one
character long.
\item If a method has a sub-option selected by the first character of a
string, that character may be in upper or lowercase.
\item Conversion between character encodings and decimal or hexadecimal
is dependent on the machine representation (encoding) of characters
and hence will return appropriately different results for Unicode,
ASCII, EBCDIC, and other implementations.
\end{enumerate}
\section{The built-in methods}\label{builtinmethods}
\begin{description}
\item[abbrev(info [,length{]})]\label{refabbrev}
\index{ABBREV method,}
\index{Method, built-in,ABBREV}
\index{Abbreviations,testing with ABBREV method}
returns 1 if \emph{info} is equal to the leading characters of
\emph{string} and \emph{info} is not less than
the minimum length, \emph{length}; 0 is returned
if either of these conditions is not met.
\emph{length} must be a non-negative whole number; the default is
the length of \emph{info}.
 \textbf{Examples:}
\begin{alltt}
'Print'.abbrev('Pri')   == 1
'PRINT'.abbrev('Pri')   == 0
'PRINT'.abbrev('PRI',4) == 0
'PRINT'.abbrev('PRY')   == 0
'PRINT'.abbrev('')      == 1
'PRINT'.abbrev('',1)    == 0
\end{alltt}
\textbf{Note: }A null string will always match if a length of 0 (or the default)
is used.
This allows a default keyword to be selected automatically if desired.
 \textbf{Example:}
\begin{alltt}
say 'Enter option:';  option=ask
select  /* keyword1 is to be the default */
  when 'keyword1'.abbrev(option) then ...
  when 'keyword2'.abbrev(option) then ...
     ...
  otherwise ...
  end
\end{alltt}
\item[abs()]\label{refabs}
\index{ABS method,}
\index{Method, built-in,ABS}
\index{Mathematical method,ABS}
\index{Absolute,value, finding using ABS method}
returns the absolute value of \emph{string}, which must be a
number.
 Any sign is removed from the number, and it is then formatted by adding
zero with a digits setting that is either nine or, if greater, the
number of digits in the mantissa of the number (excluding leading
insignificant zeros).
Scientific notation is used, if necessary.
 
\textbf{Examples:}
\begin{alltt}
'12.3'.abs              == 12.3
' -0.307'.abs           == 0.307
'123.45E+16'.abs        == 1.2345E+18
'- 1234567.7654321'.abs == 1234567.7654321
\end{alltt}

\item[b2d([n{]})]\label{refb2x}
\index{B2D method}
\index{Packing a string,with B2D}
\index{Method, built-in,B2D}
\index{Conversion,binary to decimal}
\index{Binary,conversion to decimal}
\marginnote{\color{gray}3.02} Binary to decimal.
Converts \emph{string}, a string of at least one binary
(\textbf{0} and/or \textbf{1}) digits, to an equivalent string of
decimal characters (a number), without rounding.
The returned string will use digits,
and will not include any blanks.
 If the number of binary digits in the string is not a multiple of four,
then up to three \textbf{'0'} digits will be added on the left before
conversion to make a total that is a multiple of four.
 If \emph{string} is the null string, 0 is returned. If n is not specified, \emph{string} is taken to be an unsigned number.

\textbf{Examples:}
\begin{alltt}
'01110'.b2d == 14 
'10000001'.b2d == 129 
'111110000001'.b2d == 3969 
'1111111110000001'.b2d == 65409 
'1100011011110000'.b2d == 50928 
\end{alltt}
If n is specified, string is taken as a signed number expressed in n binary characters. If the most significant (left-most) bit is zero then the number is positive; otherwise it is a negative number in twos-complement form. In both cases it is converted to a NetRexx number which may, therefore, be negative. If n is 0, 0 is always returned.

If necessary, string is padded on the left with '0' characters (note, not “signextended”), or truncated on the left, to length n characters; (that is, as though string.right(n, '0') had been executed.)

\textbf{Examples:}
\begin{alltt}
'10000001'.b2d(8) == -127 
'10000001'.b2d(16) == 129 
'1111000010000001'.b2d(16) == -3967 
'1111000010000001'.b2d(12) == 129 
'1111000010000001'.b2d(8) == -127 
'1111000010000001'.b2d(4) == 1 
'0000000000110001'.b2d(0) == 0
\end{alltt} 
\item[b2x()]\label{refb2x}
\index{B2X method}
\index{Packing a string,with B2X}
\index{Method, built-in,B2X}
\index{Conversion,binary to hexadecimal}
\index{Binary,conversion to hexadecimal}
Binary to hexadecimal.
Converts \emph{string}, a string of at least one binary
(\textbf{0} and/or \textbf{1}) digits, to an equivalent string of
hexadecimal characters.
The returned string will use uppercase Roman letters for the values A-F,
and will not include any blanks.
 If the number of binary digits in the string is not a multiple of four,
then up to three \textbf{'0'} digits will be added on the left before
conversion to make a total that is a multiple of four.
 
\textbf{Examples:}
\begin{alltt}
'11000011'.b2x  == 'C3'
'10111'.b2x     == '17'
'0101'.b2x      == '5'
'101'.b2x       == '5'
'111110000'.b2x == '1F0'
\end{alltt}
  
\item[center(length [,pad{]})]\label{refcenter}

\emph{or}
\item[centre(length [,pad{]})]\label{refcentre}
\index{CENTRE method,}
\index{CENTER method,}
\index{Method, built-in,CENTRE}
\index{Method, built-in,CENTER}
\index{Formatting,text centering}

returns a string of length \emph{length} with \emph{string}
centered in it, with \emph{pad} characters added as necessary to
make up the required length.
\emph{length} must be a non-negative whole number.
The default \emph{pad} character is blank.
If the string is longer than \emph{length}, it will be truncated at
both ends to fit.
If an odd number of characters are truncated or added, the right hand
end loses or gains one more character than the left hand end.
 
\textbf{Examples:}
\begin{alltt}
'ABC'.centre(7)          == '  ABC  '
'ABC'.center(8,'-')      == '--ABC---'
'The blue sky'.centre(8) == 'e blue s'
'The blue sky'.center(7) == 'e blue '
\end{alltt}
\textbf{Note: }This method may be called either \textbf{centre} or \textbf{center},
which avoids difficulties due to the difference between the British and
American spellings.
\item[changestr(needle, new)]\label{refchastr}
\index{CHANGESTR method,}
\index{Method, built-in,CHANGESTR}
\index{Replacing strings,using CHANGESTR}
\index{Changing strings,using CHANGESTR}

returns a copy of \emph{string} in which each occurrence of the
\emph{needle} string is replaced by the \emph{new} string.
Each unique (non-overlapping) occurrence of the \emph{needle} string
is changed, searching from left to right and starting from the first
(leftmost) position in \emph{string}.
Only the original \emph{string} is searched for the
\emph{needle}, and each character in \emph{string} can only be
included in one match of the \emph{needle}.
 
If the \emph{needle} is the null string, the result is a copy of
\emph{string}, unchanged.
 
\textbf{Examples:}
\begin{alltt}
'elephant'.changestr('e','X')    == 'XlXphant'
'elephant'.changestr('ph','X')   == 'eleXant'
'elephant'.changestr('ph','hph') == 'elehphant'
'elephant'.changestr('e','')     == 'lphant'
'elephant'.changestr('','!!')    == 'elephant'
\end{alltt}
 The  \textbf{countstr} method (see page \pageref{refcoustr})  can be used to
count the number of changes that could be made to a string in this
fashion.
\item[compare(target [,pad{]})]\label{refcompar}
\index{COMPARE method,}
\index{Method, built-in,COMPARE}
\index{Finding a mismatch using COMPARE,}
\index{Comparison,of strings/using COMPARE}

returns 0 if \emph{string} and \emph{target}
are the same.
If they are not, the returned number is positive and is the position of
the first character that is not the same in both strings.
If one string is shorter than the other, one or more \emph{pad}
characters are added on the right to make it the same length for the
comparison.
The default \emph{pad} character is a blank.
 
\textbf{Examples:}
\begin{alltt}
'abc'.compare('abc')      == 0
'abc'.compare('ak')       == 2
'ab '.compare('ab')       == 0
'ab '.compare('ab',' ')   == 0
'ab '.compare('ab','x')   == 3
'ab-- '.compare('ab','-') == 5
\end{alltt}
\item[copies(n)]\label{refcopies}
\index{COPIES method,}
\index{Method, built-in,COPIES}
\index{Copying a string using COPIES,}
\index{Repeating a string with COPIES,}
returns \emph{n} directly concatenated copies of
\emph{string}.
\emph{n} must be positive or 0; if 0, the null string is returned.
 
\textbf{Examples:}
\begin{alltt}
'abc'.copies(3) == 'abcabcabc'
'abc'.copies(0) == ''
''.copies(2)    == ''
\end{alltt}
\item[copyindexed(sub)]\label{refcopyind}
\index{COPYINDEXED method,}
\index{Method, built-in,COPYINDEXED}
\index{Indexed strings,merging}
\index{Indexed strings,copying}
\index{Copying indexed variables,}
\index{Merging indexed variables,}
copies the collection of indexed  sub-values (see page \pageref{refinstr}) 
of \emph{sub} into the collection associated with
\emph{string}, and returns the modified \emph{string}.  The
resulting collection is the union of the two collections (that is,
it contains the indexes and their values from both collections).
If a given index exists in both collections then the sub-value of
\emph{string} for that index is replaced by the sub-value from
\emph{sub}.
 
The non-indexed value of \emph{string} is not affected.
 
\textbf{Example:}
 Following the instructions:
\begin{alltt}
foo='def'
foo['a']=1
foo['b']=2
bar='ghi'
bar['b']='B'
bar['c']='C'
merged=foo.copyIndexed(bar)
\end{alltt}
then:
\begin{alltt}
merged['a'] == '1'
merged['b'] == 'B'
merged['c'] == 'C'
merged['d'] == 'def'
\end{alltt}


\index{COUNTSTR method,}
\index{Method, built-in,COUNTSTR}
\index{Counting,strings, using COUNTSTR}
\item[countstr(needle)]\label{refcoustr}
returns the count of non-overlapping occurrences of the
\emph{needle} string in \emph{string}, searching from left to
right and starting from the first (leftmost) position in
\emph{string}.
 
If the \emph{needle} is the null string, \textbf{0} is returned.
 
\textbf{Examples:}
\begin{alltt}
'elephant'.countstr('e')  == '2'
'elephant'.countstr('ph') == '1'
'elephant'.countstr('')   == '0'
\end{alltt}
 The  \textbf{changestr} method (see page \pageref{refchastr})  can be used to
change occurrences of \emph{needle} to some other string.
\item[c2d()]\label{refc2d}
\index{C2D method,}
\index{Method, built-in,C2D}
\index{Conversion,coded character to decimal}
\index{Conversion,character to decimal}
\index{Character,conversion to decimal}
\index{Coded character,conversion to decimal}

Coded character to decimal.
Converts the encoding of the character in \emph{string} (which must be
exactly one character) to its decimal representation.
The returned string will be a non-negative number that represents
the encoding of the character and will not include any sign, blanks,
insignificant leading zeros, or decimal part.
 
\textbf{Examples:}
\begin{alltt}
'M'.c2d  == '77'  -- ASCII or Unicode
'7'.c2d  == '247' -- EBCDIC
'\textbackslash{}r'.c2d == '13'  -- ASCII or Unicode
'\textbackslash{}0'.c2d == '0'
\end{alltt}
 The  \textbf{c2x} method (see page \pageref{refc2x})  can be used to
convert the encoding of a character to a hexadecimal representation.
\item[c2x()]\label{refc2x}
\index{C2X method,}
\index{Unpacking a string,with C2X}
\index{Method, built-in,C2X}
\index{Conversion,coded character to hexadecimal}
\index{Conversion,character to hexadecimal}
\index{Character,conversion to hexadecimal}
\index{Coded character,conversion to hexadecimal}

Coded character to hexadecimal.
Converts the encoding of the character in \emph{string} (which must be
exactly one character) to its hexadecimal representation (unpacks).
The returned string will use uppercase Roman letters for the values A-F,
and will not include any blanks.
Insignificant leading zeros are removed.
 
\textbf{Examples:}
\begin{alltt}
'M'.c2x  == '4D' -- ASCII or Unicode
'7'.c2x  == 'F7' -- EBCDIC
'\textbackslash{}r'.c2x == 'D'  -- ASCII or Unicode
'\textbackslash{}0'.c2x == '0'
\end{alltt}
 The  \textbf{c2d} method (see page \pageref{refc2d})  can be used to
convert the encoding of a character to a decimal number.
\item[datatype(option)]\label{refdataty}
\index{DATATYPE method,}
\index{Method, built-in,DATATYPE}
\index{Mathematical method,DATATYPE options}
\index{Types,checking with DATATYPE}
\index{Numbers,checking with DATATYPE}
\index{Whole numbers,checking with DATATYPE}
\index{Alphanumerics,checking with DATATYPE}
\index{Alphabetics,checking with DATATYPE}
\index{Letters,checking with DATATYPE}
\index{Bits,checking with DATATYPE}
\index{Binary,checking with DATATYPE}
\index{Digits,checking with DATATYPE}
\index{Lowercase,checking with DATATYPE}
\index{Mixed case,checking with DATATYPE}
\index{Whole numbers,checking with DATATYPE}
\index{Numbers,checking with DATATYPE}
\index{Symbol characters,checking with DATATYPE}
\index{Uppercase,checking with DATATYPE}
\index{Hexadecimal,checking with DATATYPE}
\index{,}
\index{,}
\index{,}

returns 1 if \emph{string} matches the description requested with
the \emph{option}, or 0 otherwise.
If \emph{string} is the null string, 0 is always returned.
 
Only the first character of \emph{option} is significant, and it may
be in either uppercase or lowercase.
The following \emph{option} characters are recognized:
\begin{description}
\item[A]\label{refdta}
(Alphanumeric); returns 1 if \emph{string} only contains
characters from the ranges "a-z", "A-Z", and "0-9".
\item[B]\label{refdtb}
(Binary); returns 1 if \emph{string} only contains the
characters "0" and/or "1".
\item[D]\label{refdtd}
(Digits); returns 1 if \emph{string} only contains
characters from the range "0-9".
\item[L]\label{refdtl}
(Lowercase); returns 1 if \emph{string} only contains
characters from the range "a-z".
\item[M]\label{refdtm}
(Mixed case); returns 1 if \emph{string} only contains
characters from the ranges "a-z" and "A-Z".
\item[N]\label{refdtn}
(Number); returns 1 if \emph{string} is a syntactically valid
\nr{} number that could be added to \textbf{'0'} without error,
\item[S]\label{refdts}
(Symbol); returns 1 if \emph{string} only contains characters
that are valid in non-numeric symbols (the alphanumeric characters and
underscore), and does not start with a digit.  Note that both uppercase
and lowercase letters are permitted.
\item[U]\label{refdtu}
(Uppercase); returns 1 if \emph{string} only contains
characters from the range "A-Z".
\item[W]\label{refdtw}
(Whole Number); returns 1 if \emph{string} is a syntactically valid
\nr{} number that can be added to \textbf{'0'} without error, and
whose decimal part after that addition, with no rounding, is zero.
\item[X]\label{refdtx}
(heXadecimal); returns 1 if \emph{string} only contains
characters from the ranges "a-f", "A-F", and "0-9".
\end{description}
 
\textbf{Examples:}
\begin{alltt}
'101'.datatype('B')    == 1
'12.3'.datatype('D')   == 0
'12.3'.datatype('N')   == 1
'12.3'.datatype('W')   == 0
'LaArca'.datatype('M') == 1
''.datatype('M')       == 0
'Llanes'.datatype('L') == 0
'3 d'.datatype('s')    == 0
'BCd3'.datatype('X')   == 1
'BCgd3'.datatype('X')  == 0
\end{alltt}
\textbf{Note: }The \textbf{datatype} method tests the meaning of the characters
in a string, independent of the encoding of those characters.  Extra
letters and Extra digits cause \textbf{datatype} to return 0 except
for the number tests ("\textbf{N}" and "\textbf{W}"),
which treat extra digits whose value is in the range 0-9 as though they
were the corresponding Arabic numeral.
\item[delstr(n [,length{]})]\label{refdelstr}
\index{DELSTR method,}
\index{Method, built-in,DELSTR}
\index{Deleting,part of a string}

returns a copy of \emph{string} with the sub-string of
\emph{string} that begins at the \emph{n}\emph{th} character, and is
of length \emph{length} characters, deleted.
If \emph{length} is not specified, or is greater than the number of
characters from \emph{n} to the end of the string, the rest of the
string is deleted (including the \emph{n}\emph{th} character).
\emph{length} must be a non-negative whole number, and \emph{n}
must be a positive whole number.  If \emph{n} is greater than the
length of \emph{string}, the string is returned unchanged.
 
\textbf{Examples:}
\begin{alltt}
'abcd'.delstr(3)    == 'ab'
'abcde'.delstr(3,2) == 'abe'
'abcde'.delstr(6)   == 'abcde'
\end{alltt}
\item[delword(n [,length])]\label{refdelword}
\index{DELWORD method,}
\index{Method, built-in,DELWORD}
\index{Deleting,words from a string}
\index{Words,deleting from a string}

returns a copy of \emph{string} with the sub-string of
\emph{string} that starts at the \emph{n}\emph{th} word, and is of
length \emph{length} blank-delimited words, deleted.
If \emph{length} is not specified, or is greater than number of
remaining words in the string, it defaults to be the remaining words
in the string (including the \emph{n}\emph{th} word).
\emph{length} must be a non-negative whole number, and \emph{n}
must be a positive whole number.  If \emph{n} is greater than the
number of words in \emph{string}, the string is returned unchanged.
The string deleted includes any blanks following the final word
involved, but none of the blanks preceding the first word involved.
 
\textbf{Examples:}
\begin{alltt}
'Now is the  time'.delword(2,2) == 'Now time'
'Now is the time '.delword(3)   == 'Now is '
'Now  time'.delword(5)          == 'Now  time'
\end{alltt}

\index{D2B method,}
\index{Method, built-in,D2B}
\index{Conversion,decimal to binary}
\index{Decimal,conversion to binary}
\index{Binary,from decimal}
\index{Binary,from decimal}
\item[d2b([n{]})]\label{refd2b}
\marginnote{\color{gray}3.02} Decimal to binary.
Returns a string of binary characters of length as needed or of length
n, which is the binary representation of the decimal number. The
returned string will use 0 and 1 characters for binary values. string
must be a whole number, and must be non-negative unless n is
specified, or an error will result. If n is not specified, the length
of the result returned is such that there are no leading 0 characters,
unless string was equal to 0 (in which case '0' is returned).

If n is specified it is the length of the final result in characters;
that is, after conversion the input string will be sign-extended to
the required length (negative numbers are converted assuming
twos-complement form). If the number is too big to fit into n
characters, it will be truncated on the left. n must be a nonnegative
whole number.

 \textbf{Examples:}
\begin{alltt}
'0'.d2b == 0 
'9'.d2b == 1001 
'19'.d2b == 10011 
'129'.d2b == 10000001 
'129'.d2b(1) == 1 
'129'.d2b(8) == 10000001 
'127'.d2b(12) == 000001111111 
'129'.d2b(16) == 0000000010000001 
'257'.d2b(8) == 00000001 
'-127'.d2b(8) == 10000001 
'-127'.d2b(16) == 1111111110000001 
'12'.d2b(0) == 
\end{alltt}

\index{D2C method,}
\index{Method, built-in,D2C}
\index{Conversion,decimal to character}
\index{Decimal,conversion to character}
\index{Coded character,from decimal}
\index{Character,from decimal}
\index{Character,from a number}
\index{Numbers,conversion to character}
\item[d2c()]\label{refd2c}
Decimal to coded character.
Converts the \emph{string} (a \nr{} \emph{number}) to a
single character, where the number is used as the encoding of the
character.
 
\emph{string} must be a non-negative whole number.
An error results if the encoding described does not produce a valid
character for the implementation (for example, if it has more
significant bits than the implementation's encoding for characters).
 
\textbf{Examples:}
\begin{alltt}
'77'.d2c  == 'M' -- ASCII or Unicode
'+77'.d2c == 'M' -- ASCII or Unicode
'247'.d2c == '7' -- EBCDIC
'0'.d2c   == '\textbackslash 0'
\end{alltt}
\item[d2x([n])]\label{refd2x}
\index{D2X method,}
\index{Method, built-in,D2X}
\index{Conversion,decimal to hexadecimal}
\index{Decimal,conversion to hexadecimal}
\index{Numbers,conversion to hexadecimal}

Decimal to hexadecimal.
Returns a string of hexadecimal characters of length as needed or of
length \emph{n}, which is the hexadecimal (unpacked) representation
of the decimal number.  The returned string will use uppercase
Roman letters for the values A-F, and will not include any blanks.
 \emph{string} must be a whole number, and must be non-negative
unless \emph{n} is specified, or an error will result.
If \emph{n} is not specified, the length of the result returned is
such that there are no leading 0 characters, unless \emph{string}
was equal to 0 (in which case \textbf{'0'} is returned).
 
If \emph{n} is specified it is the length of the final result in
characters; that is, after conversion the input string will be
sign-extended to the required length (negative numbers are converted
assuming twos-complement form).
If the number is too big to fit into \emph{n} characters, it will be
truncated on the left.
\emph{n} must be a non-negative whole number.
 
\textbf{Examples:}
\begin{alltt}
'9'.d2x       == '9'
'129'.d2x     == '81'
'129'.d2x(1)  == '1'
'129'.d2x(2)  == '81'
'127'.d2x(3)  == '07F'
'129'.d2x(4)  == '0081'
'257'.d2x(2)  == '01'
'-127'.d2x(2) == '81'
'-127'.d2x(4) == 'FF81'
'12'.d2x(0)   == ''
\end{alltt}
\item[exists(index)]\label{refexists}
\index{EXISTS method,}
\index{Method, built-in,EXISTS}
\index{Index strings,testing for}
\index{Indexed strings,testing for}
\index{Testing for indexed variables,}

returns 1 if \emph{index} names a  sub-value (see page \pageref{refinstr})  of
\emph{string} that has explicitly been assigned a value, or 0
otherwise.
 
\textbf{Example:}
 Following the instructions:
\begin{alltt}
vowel=0
vowel['a']=1
vowel['b']=1
vowel['b']=null -- drops previous assignment
\end{alltt}
then:
\begin{alltt}
vowel.exists('a') == '1'
vowel.exists('b') == '0'
vowel.exists('c') == '0'
\end{alltt}
\item[format([before [,after]])]\label{refformat}
\index{FORMAT,method}
\index{Method, built-in,FORMAT}
\index{Mathematical method,FORMAT}
\index{Formatting,numbers for display}
\index{Numbers,formatting for display}
\index{Numbers,rounding}
\index{Conversion,formatting numbers}

formats (lays out) \emph{string}, which must be a number.
 
The number, \emph{string}, is first formatted by adding zero with a
digits setting that is either nine or, if greater, the number of digits
in the mantissa of the number (excluding leading insignificant zeros).
If no arguments are given, the result is precisely that of this
operation.
 
The arguments \emph{before} and \emph{after} may be specified to
control the number of characters to be used for the integer part and
decimal part of the result respectively.  If either of these is omitted
(with no arguments specified to its right), or is \textbf{null}, the
number of characters used will be as many as are needed for that part.
 
\emph{before} must be a positive number; if it is larger than is
needed to contain the integer part, that part is padded on the left with
blanks to the requested length.
If \emph{before} is not large enough to contain the integer part
of the number (including the sign, for negative numbers), an error
results.
 
\emph{after} must be a non-negative number; if it is not the same
size as the decimal part of the number, the number will be rounded (or
extended with zeros) to fit.  Specifying 0 for \emph{after} will
cause the number to be rounded to an integer (that is, it will have no
decimal part or decimal point).
 
\textbf{Examples:}
\begin{alltt}
' - 12.73'.format         == '-12.73'
'0.000'.format            == '0'
'3'.format(4)             == '   3'
'1.73'.format(4,0)        == '   2'
'1.73'.format(4,3)        == '   1.730'
'-.76'.format(4,1)        == '  -0.8'
'3.03'.format(4)          == '   3.03'
' - 12.73'.format(null,4) == '-12.7300'
\end{alltt}
 
Further arguments may be passed to the \keyword{format} method to control
the use of exponential notation.
The full syntax of the method is then:
 
\keyword{format([before[,after[,explaces[,exdigits[,exform]]]]])}
 The first two arguments are as already described.  The other three
(\emph{explaces}, \emph{exdigits}, and \emph{exform})
control the exponent part of the result.  The default for any of the
arguments may be selected by omitting them (if there are no arguments to
be specified to their right) or by using the value \textbf{null}.
 
\emph{explaces} must be a positive number; it sets the number of
places (digits after the sign of the exponent) to be used for any
exponent part, the default being to use as many as are needed.
If \emph{explaces} is specified and is not large enough to contain
the exponent, an error results.
If \emph{explaces} is specified and the exponent will be 0,
then \emph{explaces}+2 blanks are supplied for the exponent
part of the result.
 
\emph{exdigits} sets the trigger point for use of exponential
notation.
If, after the first formatting, the number of places needed before the
decimal point exceeds \emph{exdigits}, or if the absolute value of
the result is less than \textbf{0.000001}, then exponential form will
be used, provided that \emph{exdigits} was specified.
When \emph{exdigits} is not specified, exponential notation
will never be used.
The current setting of \keyword{numeric digits} may be used for
\emph{exdigits} by specifying the special word
 \textbf{digits} (see page \pageref{refswdigit}) .
If 0 is specified for \emph{exdigits}, exponential
notation is always used unless the exponent would be 0.
 
\emph{exform} sets the form for exponential notation (if needed).
\emph{exform} may be either \textbf{'Scientific'} (the default)
or \textbf{'Engineering'}.  Only the first character of
\emph{exform} is significant and it may be in uppercase or in
lowercase.
The current setting of \keyword{numeric form} may be used by specifying
the special word  \textbf{form} (see page \pageref{refswform}) .
If engineering form is in effect, up to three digits (plus sign) may be
needed for the integer part of the result (\emph{before}).
 
\textbf{Examples:}
\begin{alltt}
'12345.73'.format(null,null,2,2) == '1.234573E+04'
'12345.73'.format(null,3,null,0) == '1.235E+4'
'1.234573'.format(null,3,null,0) == '1.235'
'123.45'.format(null,3,2,0)      == '1.235E+02'
'1234.5'.format(null,3,2,0,'e')  == '1.235E+03'
'1.2345'.format(null,3,2,0)      == '1.235    '
'12345.73'.format(null,null,3,6) == '12345.73     '
'12345e+5'.format(null,3)        == '1234500000.000'
\end{alltt}
 \textbf{Implementation minimum:} If exponents are supported in an
implementation, then they must be supported for exponents whose
absolute value is at least as large as the largest number that can be
expressed as an exact integer in default precision, \emph{i.e.}, 999999999.
Therefore, values for \emph{explaces} of up to 9 should also be
supported.
\item[insert(new [,n [,length [,pad{]]]})]\label{refinsert}
\index{INSERT method,}
\index{Method, built-in,INSERT}
\index{Inserting a string into another,}

inserts the string \emph{new}, padded or truncated to length
\emph{length}, into a copy of the target \emph{string} after the
\emph{n}\emph{th} character; the string with any inserts is returned.
\emph{length} and \emph{n} must be a non-negative whole numbers.
If \emph{n} is greater than the length of the target string,
padding is added before the \emph{new} string also.
The default value for \emph{n} is 0, which means insert before the
beginning of the string.  The default value for \emph{length} is
the length of \emph{new}.  The default \emph{pad} character is
a blank.
 
\textbf{Examples:}
\begin{alltt}
'abc'.insert('123')         == '123abc'
'abcdef'.insert(' ',3)      == 'abc def'
'abc'.insert('123',5,6)     == 'abc  123   '
'abc'.insert('123',5,6,'+') == 'abc++123+++'
'abc'.insert('123',0,5,'-') == '123--abc'
\end{alltt}

\index{LASTPOS method,}
\index{Method, built-in,LASTPOS}
\index{Finding a string in another string,}
\index{Locating,a string in another string}
\item[lastpos(needle [,start{]})]\label{reflastpos}
returns the position of the last occurrence of the string
\emph{needle} in \emph{string} (the "haystack"), searching
from right to left.
If the string \emph{needle} is not found, or is the null string,
0 is returned.
By default the search starts at the last character of
\emph{string} and scans backwards.
This may be overridden by specifying \emph{start}, the point at
which to start the backwards scan.
\emph{start} must be a positive whole number, and defaults to the
value \emph{string}\textbf{.length} if larger than that
value or if not specified (with a minimum default value of one).
 
\textbf{Examples:}
\begin{alltt}
'abc def ghi'.lastpos(' ')   == 8
'abc def ghi'.lastpos(' ',7) == 4
'abcdefghi'.lastpos(' ')     == 0
'abcdefghi'.lastpos('cd')    == 3
''.lastpos('?')              == 0
\end{alltt}

\index{LEFT method,}
\index{Method, built-in,LEFT}
\index{Formatting,text left justification}
\item[left(length [,pad{]})]\label{refleft}
returns a string of length \emph{length} containing the
left-most \emph{length} characters of \emph{string}.
The string is padded with \emph{pad} characters (or truncated) on
the right as needed.
The default \emph{pad} character is a blank.
\emph{length} must be a non-negative whole number.
This method is exactly equivalent to
\emph{string}\textbf{.substr(1}, \emph{length}
[, \emph{pad}]\textbf{)}.
 
\textbf{Examples:}
\begin{alltt}
'abc d'.left(8)     == 'abc d   '
'abc d'.left(8,'.') == 'abc d...'
'abc defg'.left(6)  == 'abc de'
\end{alltt}

\index{LENGTH,method}
\index{Method, built-in,LENGTH}
\index{Strings,length of}
\index{Data,length of}
\item[length()]\label{reflength}
returns the number of characters in \emph{string}.
 
\textbf{Examples:}
\begin{alltt}
'abcdefgh'.length == 8
''.length         == 0
\end{alltt}
\item[lower([n [,length{]}])]\label{reflower}
\index{LOWER method,}
\index{Method, built-in,LOWER}
\index{Strings,lowercasing}
\index{Lowercasing strings,}

returns a copy of \emph{string} with any uppercase characters in
the sub-string of \emph{string} that begins at the \emph{n}\emph{th}
character, and is of length \emph{length} characters, replaced by
their lowercase equivalent.
 
\emph{n} must be a positive whole number, and defaults to 1 (the
first character in \emph{string}).  If \emph{n} is greater than
the length of \emph{string}, the string is returned unchanged.
 
\emph{length} must be a non-negative whole number.
If \emph{length} is not specified, or is greater than the number of
characters from \emph{n} to the end of the string, the rest of the
string (including the \emph{n}\emph{th} character) is assumed.
 
\textbf{Examples:}
\begin{alltt}
'SumA'.lower      == 'suma'
'SumA'.lower(2)   == 'Suma'
'SuMB'.lower(1,1) == 'suMB'
'SUMB'.lower(2,2) == 'SumB'
''.lower          == ''
\end{alltt}
\item[max(number)]\label{refmax}
\index{MAX method,}
\index{Method, built-in,MAX}
\index{Mathematical method,MAX}

returns the larger of \emph{string} and \emph{number}, which
must both be numbers.  If they compare equal (that is, when subtracted,
the result is 0), then \emph{string} is selected for the result.
 
The comparison is effected using a numerical comparison with a digits
setting that is either nine or, if greater, the larger of the number of
digits in the mantissas of the two numbers (excluding leading
insignificant zeros).
 
The selected result is formatted by adding zero to the selected number
with a digits setting that is either nine or, if greater, the number of
digits in the mantissa of the number (excluding leading insignificant
zeros).
Scientific notation is used, if necessary.
 
\textbf{Examples:}
\begin{alltt}
0.max(1)          ==1
'-1'.max(1)       ==1
'+1'.max(-1)      ==1
'1.0'.max(1.00)   =='1.0'
'1.00'.max(1.0)   =='1.00'
'123456700000'.max(1234567E+5)   == '123456700000'
'1234567E+5'.max('123456700000') == '1.234567E+11'
\end{alltt}
\item[min(number)]\label{refmin}
\index{MIN method,}
\index{Method, built-in,MIN}
\index{Mathematical method,MIN}

returns the smaller of \emph{string} and \emph{number}, which
must both be numbers.  If they compare equal (that is, when subtracted,
the result is 0), then \emph{string} is selected for the result.
 
The comparison is effected using a numerical comparison with a digits
setting that is either nine or, if greater, the larger of the number of
digits in the mantissas of the two numbers (excluding leading
insignificant zeros).
 
The selected result is formatted by adding zero to the selected number
with a digits setting that is either nine or, if greater, the number of
digits in the mantissa of the number (excluding leading insignificant
zeros).
Scientific notation is used, if necessary.
 
\textbf{Examples:}
\begin{alltt}
0.min(1)          ==0
'-1'.min(1)       =='-1'
'+1'.min(-1)      =='-1'
'1.0'.min(1.00)   =='1.0'
'1.00'.min(1.0)   =='1.00'
'123456700000'.min(1234567E+5)   == '123456700000'
'1234567E+5'.min('123456700000') == '1.234567E+11'
\end{alltt}

\index{OVERLAY method,}
\index{Method, built-in,OVERLAY}
\index{Overlaying a string onto another,}
\item[overlay(new [,n [,length [,pad{]]]})]\label{refoverlay}
overlays the string \emph{new}, padded or truncated to length
\emph{length}, onto a copy of the target \emph{string} starting
at the \emph{n}\emph{th} character; the string with any overlays is
returned.  Overlays may extend beyond the end of the original
\emph{string}.
If \emph{length} is specified it must be a non-negative whole
number.
If \emph{n} is greater than the length of
the target string, padding is added before the \emph{new} string
also.
The default \emph{pad} character is a blank, and the default value
for \emph{n} is 1.
\emph{n} must be greater than 0.
The default value for \emph{length} is the length of \emph{new}.
 
\textbf{Examples:}
\begin{alltt}
'abcdef'.overlay(' ',3)      == 'ab def'
'abcdef'.overlay('.',3,2)    == 'ab. ef'
'abcd'.overlay('qq')         == 'qqcd'
'abcd'.overlay('qq',4)       == 'abcqq'
'abc'.overlay('123',5,6,'+') == 'abc+123+++'
\end{alltt}

\index{POS position method,}
\index{Method, built-in,POS}
\index{Finding a string in another string,}
\index{Locating,a string in another string}
\index{Searching a string for a word or phrase,}
\item[pos(needle [,start{]})]\label{refpos}
returns the position of the string \emph{needle}, in
\emph{string} (the "haystack"), searching from left to right.
If the string \emph{needle} is not found, or is the null string,
0 is returned.
By default the search starts at the first character of
\emph{string} (that is, \emph{start} has the value 1).
This may be overridden by specifying \emph{start} (which must be a
positive whole number), the point at which to start the search; if
\emph{start} is greater than the length of \emph{string} then 0
is returned.
 \textbf{Examples:}
\begin{alltt}
'Saturday'.pos('day')    == 6
'abc def ghi'.pos('x')   == 0
'abc def ghi'.pos(' ')   == 4
'abc def ghi'.pos(' ',5) == 8
\end{alltt}

\index{REVERSE method,}
\index{Method, built-in,REVERSE}
\item[reverse()]\label{refreverse}
returns a copy of \emph{string}, swapped end for end.
 
\textbf{Examples:}
\begin{alltt}
'ABc.'.reverse        == '.cBA'
'XYZ '.reverse        == ' ZYX'
'Tranquility'.reverse == 'ytiliuqnarT'
\end{alltt}

\item[right(length [,pad{]})]\label{refright}
\index{RIGHT method,}
\index{Method, built-in,RIGHT}
\index{Formatting,text right justification}
\index{Leading zeros,adding with the RIGHT method}
\index{Zeros,adding on the left}
\index{Zeros,padding}

returns a string of length \emph{length} containing the
right-most \emph{length} characters of \emph{string} -
that is, padded with \emph{pad} characters (or truncated) on the
left as needed.  The default \emph{pad} character is a blank.
\emph{length} must be a non-negative whole number.
 
\textbf{Examples:}
\begin{alltt}
'abc  d'.right(8)  == '  abc  d'
'abc def'.right(5) == 'c def'
'12'.right(5,'0')  == '00012'
\end{alltt}

\index{SEQUENCE method,}
\index{Method, built-in,SEQUENCE}
\index{Collating sequence, using SEQUENCE,}
\item[sequence(final)]\label{refsequen}
 returns a string of all characters, in ascending order of encoding,
between and including the character in \emph{string} and the
character in \emph{final}.
\emph{string} and \emph{final} must be single characters;
if \emph{string} is greater than \emph{final}, an error is
reported.
 
\textbf{Examples:}
\begin{alltt}
'a'.sequence('f')           == 'abcdef'
'\\0'.sequence('\\x03')       == '\\x00\\x01\\x02\\x03'
'\\ufffe'.sequence('\\uffff') == '\\ufffe\\uffff'
\end{alltt}

\index{SIGN method,}
\index{Method, built-in,SIGN}
\index{Mathematical method,SIGN}
\item[sign()]\label{refsign}
returns a number that indicates the sign of \emph{string}, which
must be a number.
\emph{string} is first formatted, just as though the operation
"\textbf{string+0}" had been carried out with sufficient digits
to avoid rounding.
If the number then starts with \textbf{'-'} then \textbf{'-1'} is
returned; if it is \textbf{'0'} then \textbf{'0'} is returned; and
otherwise \textbf{'1'} is returned.
 
\textbf{Examples:}
\begin{alltt}
'12.3'.sign    ==  1
'0.0'.sign     ==  0
' -0.307'.sign == -1
\end{alltt}

\index{SPACE method,}
\index{Method, built-in,SPACE}
\index{Formatting,text spacing}
\index{Blank,removal with SPACE method}
\item[space([n [,pad{]]})]\label{refspace}
returns a copy of \emph{string} with the blank-delimited words in
\emph{string} formatted with \emph{n} (and only \emph{n})
\emph{pad} characters between each word.
\emph{n} must be a non-negative whole number.
If \emph{n} is 0, all blanks are removed.
Leading and trailing blanks are always removed.
The default for \emph{n} is 1, and the default \emph{pad}
character is a blank.
 
\textbf{Examples:}
\begin{alltt}
'abc  def  '.space        == 'abc def'
'  abc def '.space(3)     == 'abc   def'
'abc  def  '.space(1)     == 'abc def'
'abc  def  '.space(0)     == 'abcdef'
'abc  def  '.space(2,'+') == 'abc++def'
\end{alltt}

\index{STRIP method,}
\index{Method, built-in,STRIP}
\index{Leading blanks,removal with STRIP method}
\index{Blank,removal with STRIP method}
\index{Zeros,removal with STRIP method}
\index{Leading zeros,removal with STRIP method}
\index{Character,removal with STRIP method}
\index{Trailing blanks,removal with STRIP method}
\item[strip([option [,char{]]}])]\label{refstrip}
returns a copy of \emph{string} with Leading, Trailing, or Both
leading and trailing characters removed, when the first character of
\emph{option} is L, T, or B respectively (these may be given in
either uppercase or lowercase).  The default is B.
The second argument, \emph{char}, specifies the character to be
removed, with the default being a blank.
If given, \emph{char} must be exactly one character long.
 
\textbf{Examples:}
\begin{alltt}
'  ab c  '.strip        == 'ab c'
'  ab c  '.strip('L')   == 'ab c  '
'  ab c  '.strip('t')   == '  ab c'
'12.70000'.strip('t',0) == '12.7'
'0012.700'.strip('b',0) == '12.7'
\end{alltt}

\index{SUBSTR method,}
\index{Method, built-in,SUBSTR}
\index{Extracting,a sub-string}
\index{Sub-string, extracting,}
\item[substr(n [,length [,pad{]]})]\label{refsubstr}
returns the sub-string of \emph{string} that begins at the
\emph{n}\emph{th} character, and is of length \emph{length}, padded
with \emph{pad} characters if necessary.
\emph{n} must be a positive whole number, and \emph{length} must
be a non-negative whole number.
If \emph{n} is greater than \emph{string}\textbf{.length},
then only pad characters can be returned.
 If \emph{length} is omitted it defaults to be the rest of the
string (or 0 if \emph{n} is greater than the length of the string).
The default \emph{pad} character is a blank.
 
\textbf{Examples:}
\begin{alltt}
'abc'.substr(2)       == 'bc'
'abc'.substr(2,4)     == 'bc  '
'abc'.substr(5,4)     == '    '
'abc'.substr(2,6,'.') == 'bc....'
'abc'.substr(5,6,'.') == '......'
\end{alltt}
\textbf{Note: }In some situations the positional (numeric) patterns of parsing
templates are more convenient for selecting sub-strings, especially if
more than one sub-string is to be extracted from a string.

\index{SUBWORD method,}
\index{Method, built-in,SUBWORD}
\index{Extracting,words from a string}
\index{Words,extracting from a string}
\item[subword(n [,length{]})]\label{refsubword}
returns the sub-string of \emph{string} that starts at the
\emph{n}\emph{th} word, and is up to \emph{length} blank-delimited
words long.
\emph{n} must be a positive whole number; if greater than the number
of words in the string then the null string is returned.
\emph{length} must be a non-negative whole number.
If \emph{length} is omitted it defaults to be the remaining words
in the string.
The returned string will never have leading or trailing blanks, but
will include all blanks between the selected words.
 
\textbf{Examples:}
\begin{alltt}
'Now is the  time'.subword(2,2) == 'is the'
'Now is the  time'.subword(3)   == 'the  time'
'Now is the  time'.subword(5)   == ''
\end{alltt}

\index{TRANSLATE method,}
\index{Method, built-in,TRANSLATE}
\index{Translation,with TRANSLATE method}
\index{Re-ordering characters,with TRANSLATE method}
\index{Moving characters, with TRANSLATE method,}
\index{Strings,moving with TRANSLATE method}
\index{,}
\index{Replacing strings,using TRANSLATE}
\index{Changing strings,using TRANSLATE}
\item[translate(tableo, tablei [,pad{]})]\label{reftrans}
returns a copy of \emph{string} with each character in
\emph{string} either unchanged or translated to another character.
 
The \textbf{translate} method acts by searching the input translate
table, \emph{tablei}, for each character in \emph{string}.
If the character is found in \emph{tablei} (the first, leftmost,
occurrence being used if there are duplicates) then the corresponding
character in the same position in the output translate table,
\emph{tableo}, is used in the result string; otherwise the original
character found in \emph{string} is used.
The result string is always the same length as \emph{string}.
 
The translate tables may be of any length, including the null string.
The output table, \emph{tableo}, is padded with \emph{pad} or
truncated on the right as necessary to be the same length as
\emph{tablei}.
The default \emph{pad} is a blank.
 
\textbf{Examples:}
\begin{alltt}
'abbc'.translate('\&','b')           == 'a\&\&c'
'abcdef'.translate('12','ec')       == 'ab2d1f'
'abcdef'.translate('12','abcd','.') == '12..ef'
'4123'.translate('abcd','1234')     == 'dabc'
'4123'.translate('hods','1234')     == 'shod'
\end{alltt}
\textbf{Note: }The last two examples show how the \textbf{translate} method
may be used to move around the characters in a string.
In these examples, any 4-character string could be specified as the
first argument and its last character would be moved to the beginning of
the string.
Similarly, the term:
\begin{alltt}
'gh.ef.abcd'.translate(19970827,'abcdefgh')
\end{alltt}
(which returns "\textbf{27.08.1997}") shows how a string (in
this case perhaps a date) might be re-formatted and merged with other
characters using the \textbf{translate} method.

\index{TRUNC method,}
\index{Method, built-in,TRUNC}
\index{Truncating numbers,}
\index{Numbers,truncating}
\index{Formatting,numbers with TRUNC}
\item[trunc([n{]})]\label{reftrunc}
returns the integer part of \emph{string}, which must be a
number, with \emph{n} decimal places (digits after the decimal
point).
\emph{n} must be a non-negative whole number, and defaults to zero.
 
The number \emph{string} is formatted by adding zero with a digits
setting that is either nine or, if greater, the number of digits in the
mantissa of the number (excluding leading insignificant zeros).
It is then truncated to \emph{n} decimal places (or trailing zeros
are added if needed to make up the specified length).
If \emph{n} is 0 (the default) then an integer with no decimal
point is returned.
The result will never be in exponential form.
 
\textbf{Examples:}
\begin{alltt}
'12.3'.trunc         == 12
'127.09782'.trunc(3) == 127.097
'127.1'.trunc(3)     == 127.100
'127'.trunc(2)       == 127.00
'0'.trunc(2)         == 0.00
\end{alltt}

\index{UPPER method,}
\index{Method, built-in,UPPER}
\index{Strings,uppercasing}
\index{Uppercasing strings,}
\item[upper([n [,length{]]})]\label{refupper}
returns a copy of \emph{string} with any lowercase characters in
the sub-string of \emph{string} that begins at the \emph{n}\emph{th}
character, and is of length \emph{length} characters, replaced by
their uppercase equivalent.
 
\emph{n} must be a positive whole number, and defaults to 1 (the
first character in \emph{string}).  If \emph{n} is greater than
the length of \emph{string}, the string is returned unchanged.
 
\emph{length} must be a non-negative whole number.
If \emph{length} is not specified, or is greater than the number of
characters from \emph{n} to the end of the string, the rest of the
string (including the \emph{n}\emph{th} character) is assumed.
 
\textbf{Examples:}
\begin{alltt}
'Fou-Baa'.upper        == 'FOU-BAA'
'Mad Sheep'.upper      == 'MAD SHEEP'
'Mad sheep'.upper(5)   == 'Mad SHEEP'
'Mad sheep'.upper(5,1) == 'Mad Sheep'
'Mad sheep'.upper(5,4) == 'Mad SHEEp'
'tinganon'.upper(1,1)  == 'Tinganon'
''.upper               == ''
\end{alltt}

\index{VERIFY method,}
\index{Method, built-in,VERIFY}
\index{Strings,verifying contents of}
\item[verify(reference [,option [,start{]]})]\label{refverify}
verifies that \emph{string} is composed only of characters
from \emph{reference}, by returning the position of the first
character in \emph{string} that is not also in
\emph{reference}.  If all the characters were found in
\emph{reference}, 0 is returned.
 The \emph{option} may be either \textbf{'Nomatch'} (the
default) or \textbf{'Match'}.  Only the first character of
\emph{option} is significant and it may be in uppercase or in
lowercase.
If \textbf{'Match'} is specified, the position of the first character
in \emph{string} that \textbf{is} in \emph{reference} is
returned, or 0 is returned if none of the characters were found.
 The default for \emph{start} is 1 (that is, the search starts at
the first character of \emph{string}).
This can be overridden by giving a different \emph{start} point,
which must be positive.
 If \emph{string} is the null string, the method returns 0,
regardless of the value of the \emph{option}.
Similarly if \emph{start} is greater than
\emph{string}\textbf{.length}, 0 is returned.
 If \emph{reference} is the null string, then the returned value
is the same as the value used for \emph{start},
unless \textbf{'Match'} is specified as the \emph{option}, in
which case 0 is returned.
 
\textbf{Examples:}
\begin{alltt}
'123'.verify('1234567890')          == 0
'1Z3'.verify('1234567890')          == 2
'AB4T'.verify('1234567890','M')     == 3
'1P3Q4'.verify('1234567890','N',3)  == 4
'ABCDE'.verify('','n',3)            == 3
'AB3CD5'.verify('1234567890','m',4) == 6
\end{alltt}

\index{WORD method,}
\index{Method, built-in,WORD}
\index{Words,extracting from a string}
\item[word(n)]\label{refword}
returns the \emph{n}\emph{th} blank-delimited word in
\emph{string}.
\emph{n} must be positive.
If there are fewer than \emph{n} words in \emph{string}, the
null string is returned.
This method is exactly equivalent to
\emph{string}\textbf{.subword(}\emph{n},\textbf{1)}.
 
\textbf{Examples:}
\begin{alltt}
'Now is the time'.word(3) == 'the'
'Now is the time'.word(5) == ''
\end{alltt}

\index{WORDINDEX method,}
\index{Method, built-in,WORDINDEX}
\index{Words,locating in a string}
\item[wordindex(n)]\label{refwordind}
returns the character position of the \emph{n}\emph{th}
blank-delimited word in \emph{string}.
\emph{n} must be positive.
If there are fewer than \emph{n} words in the string, 0 is returned.
 
\textbf{Examples:}
\begin{alltt}
'Now is the time'.wordindex(3) == 8
'Now is the time'.wordindex(6) == 0
\end{alltt}

\index{WORDLENGTH method,}
\index{Method, built-in,WORDLENGTH}
\index{Words,finding length of}
\item[wordlength(n)]\label{refwordlen}
returns the length of the \emph{n}\emph{th} blank-delimited word in
\emph{string}.
\emph{n} must be positive.
If there are fewer than \emph{n} words in the string, 0 is returned.
 
\textbf{Examples:}
\begin{alltt}
'Now is the time'.wordlength(2)    == 2
'Now comes the time'.wordlength(2) == 5
'Now is the time'.wordlength(6)    == 0
\end{alltt}

\index{WORDPOS method,}
\index{Method, built-in,WORDPOS}
\index{Searching a string for a word or phrase,}
\index{Locating,a word or phrase in a string}
\index{Words,finding in a string}
\item[wordpos(phrase [,start{]})]\label{refwordpos}
searches \emph{string} for the first occurrence of the sequence
of blank-delimited words \emph{phrase}, and returns the word number
of the first word of \emph{phrase} in \emph{string}.  Multiple
blanks between words in either \emph{phrase} or \emph{string}
are treated as a single blank for the comparison, but otherwise the
words must match exactly.  Similarly, leading or trailing blanks on
either string are ignored.
If \emph{phrase} is not found, or contains no words, 0 is returned.
 By default the search starts at the first word in \emph{string}.
This may be overridden by specifying \emph{start} (which must be
positive), the word at which to start the search.
 
\textbf{Examples:}
\begin{alltt}
'now is the time'.wordpos('the')       == 3
'now is the time'.wordpos('The')       == 0
'now is the time'.wordpos('is the')    == 2
'now is the time'.wordpos('is    the') == 2
'now is the time'.wordpos('is  time')  == 0
'To be or not to be'.wordpos('be')     == 2
'To be or not to be'.wordpos('be',3)   == 6
\end{alltt}

\index{WORDS method,}
\index{Method, built-in,WORDS}
\index{Words,counting, using WORDS}
\index{Counting,words, using WORDS}
\item[words()]\label{refwords}
returns the number of blank-delimited words in \emph{string}.
 
\textbf{Examples:}
\begin{alltt}
'Now is the time'.words == 4
' '.words               == 0
''.words                == 0
\end{alltt}

\index{X2B method,}
\index{Unpacking a string,with X2B}
\index{Method, built-in,X2B}
\index{Conversion,hexadecimal to binary}
\index{Hexadecimal,conversion to binary}
\item[x2b()]\label{refx2b}
Hexadecimal to binary.
Converts \emph{string} (a string of at least one hexadecimal
characters) to an equivalent string of binary digits.
Hexadecimal characters may be any decimal digit character (0-9) or any
of the first six alphabetic characters (a-f), in either lowercase or
uppercase.
 \emph{string} may be of any length; each hexadecimal character
with be converted to a string of four binary digits.
The returned string will have a length that is a multiple of four, and
will not include any blanks.
 
\textbf{Examples:}
\begin{alltt}
'C3'.x2b  == '11000011'
'7'.x2b   == '0111'
'1C1'.x2b == '000111000001'
\end{alltt}

\index{X2C method,}
\index{Packing a string,with X2C}
\index{Method, built-in,X2C}
\index{Conversion,hexadecimal to character}
\index{Hexadecimal,conversion to character}
\index{Coded character,from hexadecimal}
\index{Character,from hexadecimal}
\index{Character,from a number}
\index{Numbers,conversion to character}
\item[x2c()]\label{refx2c}
Hexadecimal to coded character.
Converts the \emph{string} (a string of hexadecimal characters) to
a single character (packs).
Hexadecimal characters may be any decimal digit character (0-9) or any
of the first six alphabetic characters (a-f), in either lowercase or
uppercase.
 
\emph{string} must contain at least one hexadecimal character;
insignificant leading zeros are removed, and the string is then padded
with leading zeros if necessary to make a sufficient number of
hexadecimal digits to describe a character encoding for the
implementation.
 
An error results if the encoding described does not produce a valid
character for the implementation (for example, if it has more
significant bits than the implementation's encoding for characters).
 \textbf{Examples:}
\begin{alltt}
'004D'.x2c == 'M' -- ASCII or Unicode
'4d'.x2c   == 'M' -- ASCII or Unicode
'A2'.x2c   == 's' -- EBCDIC
'0'.x2c    == '\textbackslash 0'
\end{alltt}
 The  \textbf{d2c} method (see page \pageref{refd2c})  can be used to
convert a \nr{} number to the encoding of a character.

\index{X2D method,}
\index{Method, built-in,X2D}
\index{Conversion,hexadecimal to decimal}
\index{Hexadecimal,conversion to decimal}
\item[x2d([n{]})]\label{refx2d}
Hexadecimal to decimal.
Converts the \emph{string} (a string of hexadecimal characters) to
a decimal number, without rounding.
If \emph{string} is the null string, 0 is returned.
 
If \emph{n} is not specified, \emph{string} is taken to
be an unsigned number.
 
\textbf{Examples:}
\begin{alltt}
'0E'.x2d    == 14
'81'.x2d    == 129
'F81'.x2d   == 3969
'FF81'.x2d  == 65409
'c6f0'.x2d  == 50928
\end{alltt}
 
If \emph{n} is specified, \emph{string} is taken as a signed
number expressed in \emph{n} hexadecimal characters.
If the most significant (left-most) bit is zero then the number is
positive; otherwise it is a negative number in twos-complement form.
In both cases it is converted to a \nr{} number which may,
therefore, be negative.
If \emph{n} is 0, 0 is always returned.
 
If necessary, \emph{string} is padded on the left
with \textbf{'0'} characters (note, not "sign-extended"), or
truncated on the left, to length \emph{n} characters; (that is, as
though \emph{string}\textbf{.right(}\emph{n}, \textbf{'0')}
had been executed.)
 
\textbf{Examples:}
\begin{alltt}
'81'.x2d(2)   == -127
'81'.x2d(4)   == 129
'F081'.x2d(4) == -3967
'F081'.x2d(3) == 129
'F081'.x2d(2) == -127
'F081'.x2d(1) == 1
'0031'.x2d(0) == 0
\end{alltt}
 The  \textbf{c2d} method (see page \pageref{refc2d})  can be used to convert
a character to a decimal representation of its encoding.
\end{description}


\chapter{Appendix A - A Sample \nr{} Program}\label{refappa}
 
This appendix includes a short program, called \textbf{qtime}, which
is an example of a "real" \nr{} program.  The programs included
elsewhere in this book have been contrived to illustrate specific
points.  By contrast, \textbf{qtime} is a simple but useful tool that
genuinely improves the human factors of computer systems.  People
frequently wish to know the time of day, and this program presents this
information in a natural way.
 
The style used for this example is the same as that used throughout
the
book,
with all symbols except those describing classes being written
in lower case.  Other \nr{} programming styles are possible, of
course; \nr{} syntax is designed to permit a wide variety of styles
with a minimum of punctuation.
 
The \textbf{qtime} program is a modification of one of the first R\textsc{exx}
programs ever written (much of the program is identical).  The main
changes are:
\begin{itemize}
\item Indexed variables (brackets notation) are used instead of R\textsc{exx}
stems.
\item The \textbf{word} method from the \textbf{R\textsc{exx}} class is used
instead of the \textbf{word} R\textsc{exx} built-in function.
\item The Java \textbf{Date} class is used to determine the current
time.
\end{itemize}
 
\index{,}
\index{Programs,examples}
\index{Example,program}
\index{qtime example program,}
\textbf{qtime.nrx - Query Time}
\lstinputlisting[label=qtime,caption=qtime.nrx]{../../examples/ibm-historic/qtime.nrx}

\chapter{Appendix B - The netrexx.lang Package}\label{refappc}
\index{Package,netrexx.lang}
\index{netrexx.lang package,}
 
This appendix documents the \textbf{netrexx.lang}
package, which includes the classes used for creating string objects of
type \textbf{R\textsc{exx}} along with several classes that are often used
while running \nr{} programs.
 
This appendix describes the public methods and properties of these
classes, as implemented by the reference implementation.  It does not
include those "built-in"  Methods for \nr{} (see page \pageref{refbmeth}) 
strings:ea. in the \textbf{R\textsc{exx}} class that form part of the
\nr{} language, or those classes and methods that are internal
"helper" components (which, for example, are used as
repositories for rarely-executed code).
 
The classes in the netrexx.lang package are:
\begin{itemize}
\item The  Exception classes (see page \pageref{refnlexcep}) 
\item  R\textsc{exx} (see page \pageref{refnlrexx}) 
\item R\textsc{exx}IO (helper class, for \keyword{say} and \keyword{ask})
\item R\textsc{exx}Node (helper class, for indexed strings)
\item  R\textsc{exx}Operators interface (see page \pageref{refnlrops}) 
\item R\textsc{exx}Parse (helper class, for \keyword{parse})
\item  R\textsc{exx}Set (see page \pageref{refnlrset}) 
\item R\textsc{exx}Trace (helper class, for \keyword{trace})
\item R\textsc{exx}Util (helper class, for the R\textsc{exx} class)
\item R\textsc{exx}Words (helper class, for the R\textsc{exx} class)
\end{itemize}

\chapter{Exception classes}\label{refnlexcep}
\index{netrexx.lang,Exceptions}
 
The classes provided for exceptions in the \textbf{netrexx.lang} package
are all subclasses of \textbf{java.lang.RuntimeException} and all have
the same content.  Each has two constructors: one taking no argument and
the other taking a string of type \textbf{java.lang.String}, which is
used for additional detail describing the exception.
 
The Exceptions are signalled as follows.
\begin{description}
\item[BadArgumentException]\label{refexpbae}
\index{BadArgumentException,}
\index{Exception,BadArgumentException}
 signalled when an argument to a method is incorrect.
\item[BadColumnException]\label{refexpbce}
\index{BadColumnException,}
\index{Exception,BadColumnException}
 signalled when a column number in a parsing template is not valid
(for example, not a number).
\item[BadNumericException]\label{refexpbne}
\index{BadNumericException,}
\index{Exception,BadNumericException}
 signalled when a \keyword{numeric digits} instruction tries to set
a value that is not a whole number, or is not positive, or is more than
nine digits.
\item[DivideException]\label{refexpdve}
\index{DivideException,}
\index{Exception,DivideException}
 signalled when an error occurs during a division.  This may be due
to an attempt to divide by zero, or when the intermediate result of an
integer divide or remainder operation is not valid.
\item[ExponentOverflowException]\label{refexpeoe}
\index{ExponentOverflowException,}
\index{Exception,ExponentOverflowException}
 signalled when the exponent resulting from an operation would
require more than nine digits.
\item[NoOtherwiseException]\label{refexpnoe}
\index{NoOtherwiseException,}
\index{Exception,NoOtherwiseException}
 signalled when a \keyword{select} construct does not supply an
\keyword{otherwise} clause and none of expressions on the \keyword{when}
clauses resulted in \textbf{'1'}.
\item[NotCharacterException]\label{refexpnce}
\index{NotCharacterException,}
\index{Exception,NotCharacterException}
 signalled when a conversion from a string to a single character was
attempted but the string was not exactly one character long.
\item[NotLogicException]\label{refexpnle}
\index{NotLogicException,}
\index{Exception,NotLogicException}
 signalled when a conversion from a string to a boolean was
attempted but the string was neither the string \textbf{'0'} nor the
string \textbf{'1'}.
\end{description}
 
\index{NumberFormatException,}
\index{Exception,NumberFormatException}
\index{NullPointerException,}
\index{Exception,NullPointerException}
Other exceptions, from the \textbf{java.lang} package, may also be
signalled, for example \textbf{NumberFormatException}
or \textbf{NullPointerException}.

\chapter{The R\textsc{exx} class}\label{refnlrexx}
\index{netrexx.lang,R\textsc{exx} class}
 
The class \textbf{netrexx.lang.R\textsc{exx}} implements the \nr{} string
class, and includes the "built-in"  Methods for (see page \pageref{refbmeth}) 
\nr{} strings:ea..
 
Described here are the platform-dependent methods as provided in
the reference implementation:  constructors (see page \pageref{refrexxcon})  for the
class, the methods for  arithmetic operations (see page \pageref{refrexxops}) , and
 miscellaneous methods (see page \pageref{refrexxmis})  intended for general
use.
 
The class \textbf{netrexx.lang.R\textsc{exx}} is serializable.
\section{R\textsc{exx} constructors}\label{refrexxcon}
\index{netrexx.lang,R\textsc{exx} constructors}
 These constructors all create a string of type \textbf{netrexx.lang.R\textsc{exx}}.
\begin{description}
\item{R\textsc{exx}(arg=boolean)}
\index{R\textsc{exx}(boolean) constructor,}
\index{Constructor,R\textsc{exx}(boolean)}

Constructs a string which will have
the value \textbf{'1'} if \emph{arg} is 1 (\emph{true})
or the value \textbf{'0'} if \emph{arg} is 0 (\emph{false}).
\item{R\textsc{exx}(arg=byte)}
\index{R\textsc{exx}(byte) constructor,}
\index{Constructor,R\textsc{exx}(byte)}

Constructs a string which is the decimal representation of
the 8-bit signed binary integer \emph{arg}.
The string will contain only decimal digits, prefixed with a
leading minus sign (hyphen) if \emph{arg} is negative.
A leading zero will be present only if \emph{arg} is zero.
\item{R\textsc{exx}(arg=char)}
\index{R\textsc{exx}(char) constructor,}
\index{Constructor,R\textsc{exx}(char)}

Constructs a string of length 1 whose first and only character is a
copy of \emph{arg}.
\item{R\textsc{exx}(arg=char[])}
\index{R\textsc{exx}(char[]) constructor,}
\index{Constructor,R\textsc{exx}(char[])}

Constructs a string by copying the characters of the character array
\emph{arg} in sequence.
The length of the string is the number of elements in the character
array (that is, \textbf{arg.length}).
\item{R\textsc{exx}(arg=int)}
\index{R\textsc{exx}(int) constructor,}
\index{Constructor,R\textsc{exx}(int)}

Constructs a string which is the decimal representation of
the 32-bit signed binary integer \emph{arg}.
The string will contain only decimal digits, prefixed with a
leading minus sign (hyphen) if \emph{arg} is negative.
A leading zero will be present only if \emph{arg} is zero.
\item{R\textsc{exx}(arg=double)}
\index{R\textsc{exx}(double) constructor,}
\index{Constructor,R\textsc{exx}(double)}

Constructs a string which is the decimal representation of
the 64-bit signed binary floating point number \emph{arg}.
 \emph{(The precise format of the result may change and
will be defined later.)}
\item{R\textsc{exx}(arg=float)}
\index{R\textsc{exx}(float) constructor,}
\index{Constructor,R\textsc{exx}(float)}

Constructs a string which is the decimal representation of
the 32-bit signed binary floating point number \emph{arg}.
 \emph{(The precise format of the result may change and
will be defined later.)}
\item{R\textsc{exx}(arg=long)}
\index{R\textsc{exx}(long) constructor,}
\index{Constructor,R\textsc{exx}(long)}

Constructs a string which is the decimal representation of
the 64-bit signed binary integer \emph{arg}.
The string will contain only decimal digits, prefixed with a
leading minus sign (hyphen) if \emph{arg} is negative.
A leading zero will be present only if \emph{arg} is zero.
\item{R\textsc{exx}(arg=R\textsc{exx})}
\index{R\textsc{exx}(R\textsc{exx}) constructor,}
\index{Constructor,R\textsc{exx}(R\textsc{exx})}

Constructs a string which is copy of \emph{arg}, which is of
type \textbf{netrexx.lang.R\textsc{exx}}.
\emph{arg} must not be \textbf{null}.
Any  sub-values (see page \pageref{refinstr})  are ignored (that is, they are not present in
the object returned by the constructor).
\item{R\textsc{exx}(arg=short)}
\index{R\textsc{exx}(short) constructor,}
\index{Constructor,R\textsc{exx}(short)}

Constructs a string which is the decimal representation of
the 16-bit signed binary integer \emph{arg}.
The string will contain only decimal digits, prefixed with a
leading minus sign (hyphen) if \emph{arg} is negative.
A leading zero will be present only if \emph{arg} is zero.
\item{R\textsc{exx}(arg=String)}
\index{R\textsc{exx}(String) constructor,}
\index{Constructor,R\textsc{exx}(String)}

Constructs a \nr{} string by copying the characters of \emph{arg},
which is of type \textbf{java.lang.String}, in sequence.
The length of the string is same as the length of \emph{arg}
(that is, \textbf{arg.length()}).
\emph{arg} must not be \textbf{null}.
\item{R\textsc{exx}(arg=String[])}
\index{R\textsc{exx}(String[]) constructor,}
\index{Constructor,R\textsc{exx}(String[])}

Constructs a \nr{} string by concatenating the elements of
the \textbf{java.lang.String} array \emph{arg} together in
sequence with a blank between each pair of elements.
This may be used for converting the argument word array passed to
the \textbf{main} method of a Java application into a single string.
 
If the number of elements of \emph{arg} is zero then an empty string
(of length 0) is returned.  Otherwise, the length of the string is the
sum of the lengths of the elements of \emph{arg}, plus the number of
elements of \emph{arg}, less one.
 
\emph{arg} must not be \textbf{null}.
\end{description}
\section{R\textsc{exx} arithmetic methods}\label{refrexxops}
\index{netrexx.lang,R\textsc{exx} arithmetic methods}
 These methods implement the \nr{} arithmetic operators, as
described in the section on  \emph{Numbers and (see page \pageref{refnums}) 
arithmetic}:ea..
Each corresponds to and implements a
method in the  R\textsc{exx}Operators interface class (see page \pageref{refnlrops}) .
 
Each of the methods here takes a  R\textsc{exx}Set (see page \pageref{refnlrset})  object as
an argument.  This argument provides the \texttt{numeric} settings for
the operation; if \textbf{null} is provided for the argument then the
default settings are used (\texttt{digits}=\textbf{9},
\texttt{form}=\texttt{scientific}).
 
For monadic operators, only the \textbf{R\textsc{exx}Set} argument is present;
the operation acts upon the current object.
For dyadic operators, the \textbf{R\textsc{exx}Set} argument and
a \textbf{R\textsc{exx}} argument are present; the operation acts with the
current object being the left-hand operand and the second argument being
the right-hand operand.  For example, under default numeric settings,
the expression:
\begin{alltt}
award+extra
\end{alltt}
(where \emph{award} and \emph{extra} are references to objects
of type \textbf{R\textsc{exx}}) could be written as:
\begin{alltt}
award.OpAdd(null, extra)
\end{alltt}
which would return the result of adding \emph{award} and
\emph{extra} under the default numeric settings.
\begin{description}
\item{OpAdd(set=R\textsc{exx}Set, rhs=R\textsc{exx})}
\index{OpAdd method,}
\index{Method,OpAdd}

Implements the \nr{} \textbf{\textbf{+}} (Add) operator,
and returns the result as a string of type \textbf{R\textsc{exx}}.
\item{OpAnd(set=R\textsc{exx}Set, rhs=R\textsc{exx})}
\index{OpAnd method,}
\index{Method,OpAnd}

Implements the \nr{} \textbf{\textbf{\&}} (And)
operator,
and returns a result (0 or 1) of type \textbf{boolean}.
\item{OpCc(set=R\textsc{exx}Set, rhs=R\textsc{exx})}
\index{OpCc method,}
\index{Method,OpCc}

Implements the \nr{} \textbf{\textbf{||}} or
\emph{abuttal} (Concatenate without blank) operator, and
returns the result as a string of type \textbf{R\textsc{exx}}.
\item{OpCcblank(set=R\textsc{exx}Set, rhs=R\textsc{exx})}
\index{OpCcblank method,}
\index{Method,OpCcblank}

Implements the \nr{} \emph{blank} (Concatenate with blank)
operator, and returns the result as a string of type \textbf{R\textsc{exx}}.
\item{OpDiv(set=R\textsc{exx}Set, rhs=R\textsc{exx})}
\index{OpDiv method,}
\index{Method,OpDiv}

Implements the \nr{} \textbf{\textbf{/}} (Divide) operator,
and returns the result as a string of type \textbf{R\textsc{exx}}.
\item{OpDivI(set=R\textsc{exx}Set, rhs=R\textsc{exx})}
\index{OpDivI method,}
\index{Method,OpDivI}

Implements the \nr{} \textbf{\textbf{\%}} (Integer divide) operator
, and returns the result as a string of type \textbf{R\textsc{exx}}.
\item{OpEq(set=R\textsc{exx}Set, rhs=R\textsc{exx})}
\index{OpEq method,}
\index{Method,OpEq}

Implements the \nr{} \textbf{\textbf{=}} (Equal) operator,
and returns a result (0 or 1) of type \textbf{boolean}.
\item[OpEqS(set=R\textsc{exx}Set, rhs=R\textsc{exx})]\label{refopeqs}
\index{OpEqS method,}
\index{Method,OpEqS}

Implements the \nr{} \textbf{\textbf{==}} (Strictly equal)
operator, and returns a result (0 or 1) of type \textbf{boolean}.
\item{OpGt(set=R\textsc{exx}Set, rhs=R\textsc{exx})}
\index{OpGt method,}
\index{Method,OpGt}

Implements the \nr{} \textbf{\textbf{>}} (Greater than)
operator, and returns a result (0 or 1) of type \textbf{boolean}.
\item{OpGtEq(set=R\textsc{exx}Set, rhs=R\textsc{exx})}
\index{OpGtEq method,}
\index{Method,OpGtEq}

Implements the \nr{} \textbf{\textbf{>=}} (Greater than or equal)
operator, and returns a result (0 or 1) of type \textbf{boolean}.
\item{OpGtEqS(set=R\textsc{exx}Set, rhs=R\textsc{exx})}
\index{OpGtEqS method,}
\index{Method,OpGtEqS}

Implements the \nr{} \textbf{\textbf{>>=}} (Strictly greater than
or equal) operator, and returns a result (0 or 1) of
type \textbf{boolean}.
\item{OpGtS(set=R\textsc{exx}Set, rhs=R\textsc{exx})}
\index{OpGtS method,}
\index{Method,OpGtS}

Implements the \nr{} \textbf{\textbf{>>}} (Strictly greater than)
operator, and returns a result (0 or 1) of type \textbf{boolean}.
\item{OpLt(set=R\textsc{exx}Set, rhs=R\textsc{exx})}
\index{OpLt method,}
\index{Method,OpLt}

Implements the \nr{} \textbf{\textbf{<}} (Less than)
operator, and returns a result (0 or 1) of type \textbf{boolean}.
\item{OpLtEq(set=R\textsc{exx}Set, rhs=R\textsc{exx})}
\index{OpLtEq method,}
\index{Method,OpLtEq}

Implements the \nr{} \textbf{\textbf{<=}} (Less than or equal)
operator, and returns a result (0 or 1) of type \textbf{boolean}.
\item{OpLtEqS(set=R\textsc{exx}Set, rhs=R\textsc{exx})}
\index{OpLtEqS method,}
\index{Method,OpLtEqS}

Implements the \nr{} \textbf{\textbf{<<=}} (Strictly less than
or equal) operator, and returns a result (0 or 1) of
type \textbf{boolean}.
\item{OpLtS(set=R\textsc{exx}Set, rhs=R\textsc{exx})}
\index{OpLtS method,}
\index{Method,OpLtS}

Implements the \nr{} \textbf{\textbf{<<}} (Strictly less than)
operator, and returns a result (0 or 1) of type \textbf{boolean}.
\item{OpMinus(set=R\textsc{exx}Set)}
\index{OpMinus method,}
\index{Method,OpMinus}

Implements the \nr{} \textbf{\textbf{Prefix -}} (Minus) operator
, and returns the result as a string of type \textbf{R\textsc{exx}}.
\item{OpMult(set=R\textsc{exx}Set, rhs=R\textsc{exx})}
\index{OpMult method,}
\index{Method,OpMult}

Implements the \nr{} \textbf{\textbf{*}} (Multiply) operator
, and returns the result as a string of type \textbf{R\textsc{exx}}.
\item{OpNot(set=R\textsc{exx}Set)}
\index{OpNot method,}
\index{Method,OpNot}

Implements the \nr{} \textbf{\textbf{Prefix \textbackslash }} (Not)
operator, and returns a result (0 or 1) of type \textbf{boolean}.
\item{OpNotEq(set=R\textsc{exx}Set, rhs=R\textsc{exx})}
\index{NotEq method,}
\index{Method,NotEq}

Implements the \nr{} \textbf{\textbf{\textbackslash =}} (Not equal)
operator, and returns a result (0 or 1) of type \textbf{boolean}.
\item{OpNotEqS(set=R\textsc{exx}Set, rhs=R\textsc{exx})}
\index{NotEqS method,}
\index{Method,NotEqS}

Implements the \nr{} \textbf{\textbf{\textbackslash ==}} (Strictly not
equal) operator, and returns a result (0 or 1) of
type \textbf{boolean}.
\item{OpOr(set=R\textsc{exx}Set, rhs=R\textsc{exx})}
\index{OpOr method,}
\index{Method,OpOr}

Implements the \nr{} \textbf{\textbf{|}} (Inclusive or)
operator, and returns a result (0 or 1) of type \textbf{boolean}.
\item{OpPlus(set=R\textsc{exx}Set)}
\index{OpPlus method,}
\index{Method,OpPlus}

Implements the \nr{} \textbf{\textbf{Prefix +}} (Plus) operator
, and returns the result as a string of type \textbf{R\textsc{exx}}.
\item{OpPow(set=R\textsc{exx}Set, rhs=R\textsc{exx})}
\index{OpPow method,}
\index{Method,OpPow}

Implements the \nr{} \textbf{\textbf{**}} (Power) operator
, and returns the result as a string of type \textbf{R\textsc{exx}}.
\item{OpRem(set=R\textsc{exx}Set, rhs=R\textsc{exx})}
\index{OpRem method,}
\index{Method,OpRem}

Implements the \nr{} \textbf{\textbf{//}} (Remainder) operator
, and returns the result as a string of type \textbf{R\textsc{exx}}.
\item{OpSub(set=R\textsc{exx}Set, rhs=R\textsc{exx})}
\index{OpSub method,}
\index{Method,OpSub}

Implements the \nr{} \textbf{\textbf{-}} (Subtract) operator,
and returns the result as a string of type \textbf{R\textsc{exx}}.
\item{OpXor(set=R\textsc{exx}Set, rhs=R\textsc{exx})}
\index{OpXor method,}
\index{Method,OpXor}

Implements the \nr{} \textbf{\textbf{\&\&}} (Exclusive or)
operator, and returns a result (0 or 1) of
type \textbf{boolean}.
\end{description}
\section{R\textsc{exx} miscellaneous methods}\label{refrexxmis}
\index{netrexx.lang,R\textsc{exx} miscellaneous methods}
 These methods provide standard Java methods for the class, together with
various conversions.
\begin{description}
\item{charAt(offset=int)}
\index{charAt method,}
\index{Method,charAt}

Returns the character from the string at \emph{offset} (that is, if
\emph{offset} is 0 then the first character is returned, \&).
The character is returned as type \textbf{char}.
 
If \emph{offset} is negative, or is greater than or equal to the
length of the string, an exception is signalled.
\item{equals(item=Object)}
\index{equals method,}
\index{Method,equals}

Compares the string with the value of \emph{item}, using a strict
character-by-character comparison, and returns a result of
type \textbf{boolean}.
 
If \emph{item} is \textbf{null} or is not an instance of one of
the types \textbf{R\textsc{exx}}, \textbf{java.lang.String}, or \textbf{char[]},
then 0 is returned.
Otherwise, \emph{item} is converted to type \textbf{R\textsc{exx}} and the
 OpEqS (see page \pageref{refopeqs})  method (or equivalent) is used to compare the
current string with the converted string, and its result is returned.
\item{hashCode()}
\index{hashCode method,}
\index{Method,hashCode}

Returns a hashcode of type \textbf{int} for the string.
This hashcode is suitable for use by the \textbf{java.util.Hashtable}
class.
\item{toboolean()}
\index{toboolean method,}
\index{Method,toboolean}

Converts the string to type \textbf{boolean}.  If the string is
neither \textbf{"0"} nor \textbf{"1"} then
a  \textbf{NotLogicException} (see page \pageref{refexpnle})  is signalled.
\item{tobyte()}
\index{tobyte method,}
\index{Method,tobyte}

Converts the string to type \textbf{byte}.  If the string is
not a number, has a non-zero decimal part, or is out of the possible
range for a \textbf{byte} (8-bit signed integer) result then
a \textbf{NumberFormatException} is signalled.
\item{tochar()}
\index{tochar method,}
\index{Method,tochar}

Converts the string to type \textbf{char}.  If the string is
not exactly one character in length then
a  \textbf{NotCharacterException} (see page \pageref{refexpnce})  is signalled.
\item{toCharArray()}
\index{tochar method,}
\index{Method,tochar}

Converts the string to type \textbf{char[]}.  A character array object
of the same length as the string is created, and the characters of the
string are copied to the array in sequence.  The character array is then
returned.
\item{todouble()}
\index{todouble method,}
\index{Method,todouble}

Converts the string to type \textbf{double}.  If the string is
not a number, or is out of the possible range for a \textbf{double}
(64-bit signed floating point) result then a \textbf{NumberFormatException}
is signalled.
\item{tofloat()}
\index{tofloat method,}
\index{Method,tofloat}

Converts the string to type \textbf{float}.  If the string is
not a number, or is out of the possible range for a \textbf{float}
(32-bit signed floating point) result then a \textbf{NumberFormatException}
is signalled.
\item{toint()}
\index{toint method,}
\index{Method,toint}

Converts the string to type \textbf{int}.  If the string is
not a number, has a non-zero decimal part, or is out of the possible
range for an \textbf{int} (32-bit signed integer) result then
a \textbf{NumberFormatException} is signalled.
\item{tolong()}
\index{tolong method,}
\index{Method,tolong}

Converts the string to type \textbf{long}.  If the string is
not a number, has a non-zero decimal part, or is out of the possible
range for a \textbf{long} (64-bit signed integer) result then
a \textbf{NumberFormatException} is signalled.
[\%hide
\item{toR\textsc{exx}(arg=char[]) static}
\index{toR\textsc{exx} method,}
\index{Method,toR\textsc{exx}}

Takes \emph{arg}, an array of characters, and returns a copy
of it as a string of type \textbf{netrexx.lang.R\textsc{exx}}.
If the argument is \textbf{null}, then \textbf{null} is returned
(not a null string).
This is a static method (a function).
\item{toR\textsc{exx}(arg=String) static}
\index{toR\textsc{exx} method,}
\index{Method,toR\textsc{exx}}

 
Takes \emph{arg}, a \textbf{java.lang.String}, and returns a copy
of it as a string of type \textbf{netrexx.lang.R\textsc{exx}}.
If the argument is \textbf{null}, then \textbf{null} is returned
(not a null string).
This is a static method (a function).
\item{toshort()}
\index{toshort method,}
\index{Method,toshort}

Converts the string to type \textbf{short}.  If the string is
not a number, has a non-zero decimal part, or is out of the possible
range for a \textbf{short} (16-bit signed) result then
a \textbf{NumberFormatException} is signalled.
\item{toString()}
\index{toString method,}
\index{Method,toString}

Converts the string to type \textbf{java.lang.String}.  A String
object of the same length as the string is created, and the characters
of the string are copied to the new string in sequence.  The String is
then returned.
\end{description}
   %The Rexx Class
\section{The R\textsc{exx}IO class}\label{refrexxio}
\index{netrexx.lang,R\textsc{exx}IO class}
 
The \texttt{R\textsc{exx}IO} class implements a number of helper methods, for example \texttt{RexxIO.say}, a call to which is generated when a program contains a \texttt{say} statement. 

\begin{description}
  \item[Ask() static returns R\textsc{exx}]
\index{Ask method,}
\index{Method,Ask}

get a line of text from the console
  \item [AskOne() static returns R\textsc{exx}]
\index{toboolean AskOne,}
\index{Method,AskOne}
\index{AskOne method,}

get one character from the console (still requires a return)
 \item[Say(obj=Object) static returns boolean]
\index{Say method,}
\index{Method,Say}

put a line out to the console
    If the line ends in the NUL character ('\\-' or '\\0') then no
    line termination is provided (and the NUL is deleted).
    If the write succeeds 0 is returned, otherwise 1 is returned.
 
  \item[Say(str=String) static returns boolean]

put a line out to the console
  \item[Say(line=R\textsc{exx}) static returns boolean]

put a line out to the console
  \item[Say(c=char) static returns boolean]

put a line out to the console
  \item[Say(n=long) static returns boolean]

put a line out to the console
  \item[Say(f=float)   static returns boolean]

put a line out to the console
  \item[Say(d=double)  static returns boolean]

put a line out to the console
  \item[Say(b=boolean) static returns boolean]

put a line out to the console
  \item[Say(aline=char[]) static returns boolean]

put a line out to the console
  \item[setOutputStream(out=OutputStream) static]
\index{setOutputStream method,}

\marginnote{\color{gray}3.07}change the outputstream for say to use  
 
\item[pushOutputStream(out=OutputStream) static]
\index{pushOutputStream method,}
\index{Method,pushOutputStream}


\marginnote{\color{gray}3.07}push an outputstream on the decque, for say to use
  \item[popOutputStream() static]
\index{popOutputStream method,}
\index{Method,popOutputStream}

\marginnote{\color{gray}3.07}remove an outputstream from the decque, will not be used anymore
  \item[File(nm) returns R\textsc{exx}IO]
\index{File method,}
\index{Method,File}

\marginnote{\color{gray}3.07}define a file to the RexxXIO instance
  \item[forEachline(c=LineHandler)]
\index{forEachline method,}
\index{Method,forEachline}

\marginnote{\color{gray}3.07}define a callback that calls an instance of the LineHandler interface
  \item[forEachline(c=LineHandler,numLines)]

\marginnote{\color{gray}3.07}define a callback that calls an instance of the LineHandler interface that is only called a number of times as specified in numLines
\end{description}

 %RexxIO runtime class
\section{The R\textsc{exx}Date class}\label{refrexxdate}
\index{netrexx.lang,R\textsc{exx}Date class}
 
The \texttt{R\textsc{exx}Date} class inherits from
\texttt{R\textsc{exx}Time} which implements a superset of the Classic
Rexx \texttt{Date()} and \texttt{Time()} functions\footnote{At the
  4.02 level, including the input and conversion functions}, including
some of the options that were available in Rexx/VM but left out of the
Rexx ANSI/ISO/INCITS standard.

\begin{figure}[h]
  \begin{shaded}
\begin{rail}
  DATE : 'DATE' '('   outputDateFormat? GROUP1?  ')'
  ;
  GROUP1 : ','  inputDate GROUP2?
  | ','  ','  outputSeparatorChar
  ;
  GROUP2 : ','  inputDateFormat? GROUP3?
  ;
  GROUP3 : ','  outputSeparatorChar? (',' inputSeparatorChar?)?
  ;
  
\end{rail}
\end{shaded}
\end{figure}

The \texttt{date()} function can be called standalone when using a \texttt{uses RexxDate} option on the \texttt{class} statement. You can use the following options to obtain specific date formats. (Only the capitalized letter is needed; all characters following it are ignored.)
\subsection{Options}
\begin{description}
\item[Base]
  the number of complete days (that is, not including the current day) since and including the base date, 1 January 0001, in the format: dddddd (no leading zeros or blanks). The expression DATE('B')//7 returns a number in the range 0–6 that corresponds to the current day of the week, where 0 is Monday and 6 is Sunday.
Thus, this function can be used to determine the day of the week independent of the national language in which you are working.
Note: The base date of 1 January 0001 is determined by extending the current Gregorian calendar backward (365 days each year, with an extra day every year that is divisible by 4 except century years that are not divisible by 400). It does not take into account any errors in the calendar system that created the Gregorian calendar originally.
\item[Century]
the number of days, including the current day, since and including January 1 of the last year that is a multiple of 100 in the form: ddddd (no leading zeros). Example: A call to DATE('C') on March 13 1992 returns 33675, the number of days from 1 January 1900 to 13 March 1992. Similarly, a call to DATE('C') on 2 January 2000 returns 2, the number of days from 1 January 2000 to 2 January 2000.
Note: When the Century option is used for input, the output may change, depending on the current century. For example, if DATE('S','1','C') was entered on any day between 1 January 1900 and 31 December 1999, the result would be 19000101. However, if DATE('S','1','C') was entered on any day between 1 January 2000 and 31 December 2099, the result would be 20000101. It is important to understand the above, and code accordingly.
\item[Days]
the number of days, including the current day, so far in the current year in the format: ddd (no leading zeros or blanks).
\item[Julian]
  date in the format: yyyyddd (yy and ddd must have leading zeros).
 \item[European]
    date in the format: dd/mm/yy (dd, mm, and yy must have leading zeros).
\item[Month]
full name of the current month. Only valid for OutputDateFormat.
\item[Normal]
date in the format: dd mon yyyy. This is the default (dd cannot have any leading zeros or blanks; yyyy must have leading zeros but cannot have any leading blanks). If Normal is specified for input\_date\_format, the input\_date must have the month (mon) specified in English (for example, Jan, Feb, Mar, and so on).
\item[Ordered]
date in the format: yy/mm/dd (suitable for sorting, and so forth; yy, mm, and dd must have leading zeros).
\item[Standard]
date in the format: yyyymmdd (suitable for sorting, and so forth; yyyy, mm, and dd must have leading zeros).
\item[Usa]
date in the format: mm/dd/yy (mm, dd, and yy must have leading zeros).
\item[Weekday]
the name for the day of the week.
\end{description}
\subsection{Examples}

\begin{lstlisting}[label=datessexample,caption=Example of using Date()]
  class TestDate uses RexxDate

  method main(args=String[]) static

    say 'Date input/conversion options'
    say "date('b','10 Mar 1962')      ==> 716308      :" date('b','10 Mar 1962')
    say "date('w','10 Mar 1962','n')  ==> Saturday    :" date('w','10 Mar 1962','n')
    say "date('w','716308','b')       ==> Saturday    :" date('w','716308','b')
    say "date('s','716308','b')       ==> 19620310    :" date('s','716308','b')


    say 'with separators specified'
    say "date('s','716308','b','/')   ==> 1962/03/10  :" date('s','716308','b','/')
    say "date('s','716308','b','-')   ==> 1962-03-10  :" date('s','716308','b','-')
    say "date('w',7688,'c')           ==> Sunday      :" date('w',7688,'c')
    say "date('c','1 Feb 2021')       ==> 7703        :" date('c','1 Feb 2021')
    say "date('J','18 Jan 2021')      ==> 2021018     :" date('j','18 Jan 2021')
    say "date('J','10 Mar 1962')      ==> 1962069     :" date('j','10 Mar 1962')
  \end{lstlisting}


\section{The R\textsc{exx}Time class}\label{refrexxtime}
\index{netrexx.lang,R\textsc{exx}Time class}
 
The \texttt{R\textsc{exx}Time} class implements the Classic Rexx \texttt{Time()} function.
\begin{figure}[h]
   \begin{shaded}
\begin{rail}
  TIME:  'TIME' '('   option?  ')'
  ;
\end{rail}
 \end{shaded}
\end{figure}

The \texttt{time()} function can be called standalone when using a \texttt{uses RexxTime} option on the \texttt{class} statement. You can use the following options to obtain specific time formats. (Only the capitalized letter is needed; all characters following it are ignored.)
\subsection{Options}
\begin{description}
\item[Civil]
  \index{Time,Civil}
returns the time in Civil format: hh:mmxx. The hours may take the values 1 through 12, and the minutes the values 00 through 59. The minutes are followed immediately by the letters am or pm. This distinguishes times in the morning (12 midnight through 11:59 a.m.—appearing as 12:00am through 11:59am) from noon and afternoon (12 noon through 11:59 p.m.—appearing as 12:00pm through 11:59pm). The hour has no leading zero. The minute field shows the current minute (rather than the nearest minute) for consistency with other TIME results.
\item[Elapsed]
    \index{Time,Elapsed}
returns sssssssss.uuuuuu, the number of seconds.microseconds since the elapsed-time clock (described later) was started or reset. The number has no leading zeros or blanks, and the setting of NUMERIC DIGITS does not affect the number. The fractional part always has six digits.
\item[Hours]
    \index{Time,Hours}
returns up to two characters giving the number of hours since midnight in the format: hh (no leading zeros or blanks, except for a result of 0).
\item[Long]
    \index{Time,Long}
returns time in the format: hh:mm:ss.uuuuuu (uuuuuu is the fraction of seconds, in microseconds). The first eight characters of the result follow the same rules as for the Normal form, and the fractional part is always six digits.
\item[Minutes]
    \index{Time,Minutes}
returns up to four characters giving the number of minutes since midnight in the format: mmmm (no leading zeros or blanks, except for a result of 0).
\item[Normal]
    \index{Time,Normal}
returns the time in the default format hh:mm:ss, as described previously. The hours can have the values 00 through 23, and minutes and seconds, 00 through 59. All these are always two digits. Any fractions of seconds are ignored (times are never rounded up). This is the default.
\item[Reset]
    \index{Time,Reset}
returns sssssssss.uuuuuu, the number of seconds.microseconds since the elapsed-time clock (described later) was started or reset and also resets the elapsed-time clock to zero. The number has no leading zeros or blanks, and the setting of NUMERIC DIGITS does not affect the number. The fractional part always has six digits.
\item[Seconds]
    \index{Time,Seconds}
returns up to five characters giving the number of seconds since midnight in the format: sssss (no leading zeros or blanks, except for a result of 0).
\end{description}
\subsection{Examples}

\begin{lstlisting}[label=timeexample,caption=Example of using Time()]
  method main(args=String[]) static
    say time()       -- 22:16:33
    say time('C')    -- 10:16pm
    say time('E')    -- 0.000000
    say time('R')    -- 0
    say time('H')    -- 22
    say time('L')    -- 22:16:33.836725
    say time('M')    -- 1336
    say time('N')    -- 22:16:33
    say time('O')    -- 22:16:33
    say time('R')    -- 0.001204
    say time('E')    -- 0.000271
    say time('S')    -- 80193
  \end{lstlisting}


\chapter{The RexxOperators interface class}\label{"id"}
\index{netrexx.lang,RexxOperators class}
 
The \textbf{RexxOperators} interface class defines the signatures of
the methods that implement the NetRexx (and Rexx) operators.  These
methods are described in the section :cit. Rexx (see page \pageref{refrexxops}) 
arithmetic methods:ea.:ecit..
 
In the future this interface may be used to allow the overloading of
operators for objects of types other than \textbf{Rexx}.  The current
NetRexx language definition does not permit operator overloading.

\chapter{The R\textsc{exx}Set class}\label{refnlrset}
\index{netrexx.lang,R\textsc{exx}Set class}
 
The \textbf{R\textsc{exx}Set} class is used to provide the numeric settings for
the methods described in the section :cit. R\textsc{exx} (see page \pageref{refrexxops}) 
arithmetic methods:ea.:ecit..
When provided, a R\textsc{exx}Set Object supplies the \keyword{numeric} settings
for the operation; when \textbf{null} is provided then the default
settings are used (\keyword{digits}=\textbf{9},
\keyword{form}=\keyword{SCIENTIFIC}).
\section{}\label{}
\index{netrexx.lang,R\textsc{exx}Set properties}
 
These properties supply the numeric settings and certain values they may
take.  After construction, the \keyword{digits} and \keyword{form} values
should only be changed by using the \keyword{setDigits} and
\keyword{setForm} methods.
\begin{description}
\item{DEFAULT\_DIGITS}
\index{DEFAULT\_DIGITS property,}
\index{Property,DEFAULT\_DIGITS}

A constant of type \textbf{int} that describes the default number of
digits for a numeric operation (9).
\item{DEFAULT\_FORM}
\index{DEFAULT\_FORM property,}
\index{Property,DEFAULT\_FORM}

A constant of type \textbf{byte} that describes the default exponential
format for a numeric operation (\keyword{SCIENTIFIC}).
\item{digits}
\index{digits property,}
\index{Property,digits}

A value of type \textbf{int} that describes the numeric digits to be
used for a numeric operation.  The  R\textsc{exx} arithmetic (see page \pageref{refrexxops}) 
methods:ea. use this value to determine the significance of results.
\keyword{digits} must always be greater than zero.
\item{ENGINEERING}
\index{ENGINEERING property,}
\index{Property,ENGINEERING}

A constant of type \textbf{byte} that signifies that engineering
exponential formatting should be used for a numeric operation.
\item{form}
\index{form property,}
\index{Property,form}

A value of type \textbf{byte} that describes the exponential format to
be used for a numeric operation.  The  R\textsc{exx} arithmetic (see page \pageref{refrexxops}) 
methods:ea. use this value to determine the formatting of results that
require an exponent.
\keyword{form} must be either \keyword{ENGINEERING} or \keyword{SCIENTIFIC}.
\item{SCIENTIFIC}
\index{SCIENTIFIC property,}
\index{Property,SCIENTIFIC}

A constant of type \textbf{byte} that signifies that scientific
exponential formatting should be used for a numeric operation.
\end{description}
\section{}\label{}
\index{netrexx.lang,R\textsc{exx}Set constructors}
 
These constructors are used to set the initial values of a R\textsc{exx}Set
object.
\begin{description}
\item{R\textsc{exx}Set()}
\index{R\textsc{exx}Set() constructor,}
\index{Constructor,R\textsc{exx}Set()}

Constructs a R\textsc{exx}Set object which has default \keyword{digits} and
\keyword{form} properties.
\item{R\textsc{exx}Set(newdigits=int)}
\index{R\textsc{exx}Set(int) constructor,}
\index{Constructor,R\textsc{exx}Set(int)}

Constructs a R\textsc{exx}Set object which has its \keyword{digits} property set
to \emph{newdigits} and its \keyword{form} property set
to \keyword{DEFAULT\_DIGITS}.
\item{R\textsc{exx}Set(newdigits=int, newform=byte)}
\index{R\textsc{exx}Set(int,byte) constructor,}
\index{Constructor,R\textsc{exx}Set(int,byte)}

Constructs a R\textsc{exx}Set object which has its \keyword{digits} property set
to \emph{newdigits} and its \keyword{form} property set to
\emph{newform}.
\item{R\textsc{exx}Set(arg=R\textsc{exx}Set)}
\index{R\textsc{exx}Set(R\textsc{exx}Set) constructor,}
\index{Constructor,R\textsc{exx}Set(R\textsc{exx}Set)}

Constructs a R\textsc{exx}Set object which is copy of \emph{arg}, which is of
type \textbf{netrexx.lang.R\textsc{exx}Set}.
\emph{arg} must not be \textbf{null}.
\end{description}
\section{}\label{}
\index{netrexx.lang,R\textsc{exx}Set methods}
 
The R\textsc{exx}Set class has the following additional methods:
\begin{description}
\item{formword()}
\index{formword() method,}
\index{Method,formword()}

Returns a string of type \textbf{netrexx.lang.R\textsc{exx}} that describes the
\keyword{form} property.  This will either be the string \textbf{'engineering'}
or the string \textbf{'scientific'}, corresponding to the \keyword{form}
value \keyword{ENGINEERING} or \keyword{SCIENTIFIC}, respectively.
\item{setDigits(newdigits=R\textsc{exx})}
\index{setDigits(R\textsc{exx}) method,}
\index{Method,setDigits(R\textsc{exx})}

Sets the \keyword{digits} value for the \textbf{R\textsc{exx}Set} object, from
\emph{newdigits}, after rounding and checking as defined for the
\keyword{numeric} instruction; \emph{newdigits} must be a positive
whole number with no more than nine digits.
No value is returned.
\item{setForm(newformword=R\textsc{exx})}
\index{setForm(R\textsc{exx}) method,}
\index{Method,setForm(R\textsc{exx})}

Sets the \keyword{form} value for the \textbf{R\textsc{exx}Set} object, from
\emph{newformword}.
This must equal either the string \textbf{'engineering'} or the
string \textbf{'scientific'}, corresponding to the \keyword{form}
value \keyword{ENGINEERING} or \keyword{SCIENTIFIC}, respectively.
No value is returned.
\end{description}

There are a number of options for the translator, some of which can be
specified on the translator command line, and others also in the
program source on the \textbf{option} statement. In the following
table, c stands for \emph{commandline only}, s stands for
\emph{source} and b stands for \emph{both}. On the commandline,
options are prefixed with a \emph{dash} (``-''), while in
programsource they are not - there they are preceded by the
\keyword{option} statement.
\begin{longtable}[l]{|l|p{10cm}|l|}
\caption{ Options } \\
\hline
\rowcolor[gray]{0.8} \bfseries Option & \bfseries Meaning & \bfseries Place   \
\endfirsthead
\multicolumn{3}{r}%
{{\tablename\ \thetable{} -- \emph{continued from previous page}}} \\
\endhead
\hline \multicolumn{3}{r}{\emph{Continued on next page}}
\endfoot

\endlastfoot
\rowcolor[gray]{0.8} \bfseries \huge   & \normalsize  &  \\
\hline
arg words & interpret; remaining words are arguments & c \\
\hline
binary &  classes are binary classes & b \\
\hline
 classpath  & specify a classpath & c \\
\hline
 compile  & compile (default; -nocompile implies -keep) & c \\
\hline
 comments     & copy comments across to generated .java &b \\
\hline
 compact      & display error messages in compact form &b \\
\hline
 console   & display messages on console (default) &c \\
\hline
 crossref     & generate cross-reference listing &b \\
\hline
 decimal      & allow implicit decimal arithmetic &b \\
\hline
 diag         & show diagnostic messages &b \\
\hline
 exec        & interpret with no argument words &c \\
\hline
explicit     & local variables must be explicitly declared &b \\
\hline
format       & format output file (pretty-print) &b \\
\hline
java         & generate Java source code for this program &b \\
\hline
 keep         & keep any completed .java file (as xxx.java.keep) &c \\
\hline
keepasjava   & keep any completed .java file (as xxx.java) &c \\
\hline
 logo         & display logo (banner) after starting &b \\
\hline
prompt       & prompt for new request after processing &c \\
\hline
 savelog      & save messages in NetRexxC.log &c \\
\hline
 replace      & replace .java file even if it exists &b \\
\hline
 sourcedir    & force output files to source directory &b \\
\hline
 strictargs   & empty argument lists must be specified as () &b \\
\hline
 strictassign & assignment must be cost-free &b \\
\hline
 strictcase   & names must match in case &b \\
\hline
 strictimport & all imports must be explicit &b \\
\hline
 strictmethods & superclass methods are not compared to local methods for best match &b \\
\hline
 strictprops  & even local properties must be qualified &b \\
\hline
 strictsignal & signals list must be explicit &b \\
\hline
 symbols      & include symbols table in generated .class files &b \\
\hline
 time         & display timings &c \\
\hline
 trace[n]     & trace stream [1 or 2], or 0 for NOTRACE &b \\
\hline
 utf8         & source file is in UTF8 encoding &b \\
\hline
 verbose[n]   & verbosity of progress reports [0-5] &b \\
\hline
 warnexit0    & exit with a zero returncode on warnings &c \\
\hline
\end{longtable}

\subsubsection{Options valid for the options statement and on the commandline}
These are the options that can be used on the \textbf{options} statement:
\begin{description}
\index{option, binary}
\index{flag, binary}
\index{binary option}
\item[binary]
All classes in this program will be binary classes. In binary classes, literals are assigned binary (primitive) or native string types, rather than \nr{} types, and native binary operations are used to implement operators where appropriate, as described in “Binary values and operations”. In classes that are not binary, terms in expressions are converted to the \nr{} string type, Rexx, before use by operators.

\index{option,comments}
\index{flag,comments}
\index{comments option}
\item[comments]
Comments from the \nr{} source program will be passed through to the Java output file (which may be saved with a .java.keep or .java extension by using the -keep and -keepasjava command options, respectively).

\index{option,compact}
\index{flag,compact}
\index{compact option}
\item[compact]
Requests that warnings and error messages be displayed in compact form. This format is more easily parsed than the default format, and is intended for use by editing environments.
Each error message is presented as a single line, prefixed with the error token identification enclosed in square brackets. The error token identification comprises three words, with one blank separating the words. The words are: the source file specification, the line number of the error token, the column in which it starts, and its length. For example (all on one line):
\begin{verbatim}
  [D:\test\test.nrx 3 8 5] Error: The external name
  'class' is a Java reserved word, so would not be
  usable from Java programs
\end{verbatim}
Any blanks in the file specification are replaced by a null ('\textbackslash 0') character. Additional words could be added to the error token identification later.

\index{option,crossref}
\index{flag,crossref}
\index{crossref option}
\item[crossref]
Requests that cross-reference listings of variables be prepared, by class.
\index{option,decimal}
\index{flag,decimal}
\index{decimal option}
\item[decimal]
Decimal arithmetic may be used in the program. If nodecimal is specified, the language processor will report operations that use (or, like normal string comparison, might use) decimal arithmetic as an error. This option is intended for performance-critical programs where the overhead of inadvertent use of decimal arithmetic is unacceptable.
\index{option,diag}
\index{flag,diag}
\index{diag option}
\item[diag]
Requests that diagnostic information (for experimental use only) be displayed. The diag option word may also have side-effects.
\index{option,explicit}
\index{flag,explicit}
\index{explicit option}
\item[explicit]
Requires that all local variables must be explicitly declared (by assigning them a type but no value) before assigning any value to them. This option is intended to permit the enforcement of “house styles” (but note that the \nr{} compiler always checks for variables which are referenced before their first assignment, and warns of variables which are set but not used).
\index{option,format}
\index{flag,format}
\index{format option}
\item[format]
Requests that the translator output file (Java source code) be formatted for improved readability. Note that if this option is in effect, line numbers from the input file will not be preserved (so run-time errors and exception trace-backs may show incorrect line numbers).
\index{option,java}
\index{flag,java}
\index{java option}
\item[java]
Requests that Java source code be produced by the translator. If nojava is specified, no Java source code will be produced; this can be used to save a little time when checking of a program is required without any compilation or Java code resulting.
\index{option,logo}
\index{flag,logo}
\index{logo option}
\item[logo]
Requests that the language processor display an introductory logotype sequence (name and version of the compiler or interpreter, etc.).
\index{option,sourcedir}
\index{flag,sourcedir}
\index{sourcedir option}
\item[sourcedir]
Requests that all .class files be placed in the same directory as the source file from which they are compiled. Other output files are already placed in that directory. Note that using this option will prevent the -run command option from working unless the source directory is the current directory.
\index{option,strictargs}
\index{flag,strictargs}
\index{strictargs option}
\item[strictargs]
Requires that method invocations always specify parentheses, even when no arguments are supplied. Also, if strictargs is in effect, method arguments are checked for usage – a warning is given if no reference to the argument is made in the method.
\index{option,strictassign}
\index{flag,strictassign}
\index{strictassign option}
\item[strictassign]
Requires that only exact type matches be allowed in assignments (this is stronger than Java requirements). This also applies to the matching of arguments in method calls.
\index{option,strictcase}
\index{flag,strictcase}
\index{strictcase option}
\item[strictcase]
Requires that local and external name comparisons for variables, properties, methods, classes, and special words match in case (that is, names must be identical to match).
\index{option,strictimport}
\index{flag,strictimport}
\index{strictimport option}
\item[strictimport]
Requires that all imported packages and classes be imported explicitly using import instructions. That is, if in effect, there will be no automatic imports, except those related to the package instruction.
\index{option,strictmethods}
\index{flag,strictmethods}
\index{strictmethods option}
\item[strictmethods]
Superclass methods are not compared to local methods for best match.
\index{option,strictprops}
\index{flag,strictprops}
\index{strictprops option}
\item[strictprops]
Requires that all properties, including those local to the current class, be qualified in references. That is, if in effect, local properties cannot appear as simple names but must be qualified by this. (or equivalent) or the class name (for static properties).
\index{option,strictsignal}
\index{flag,strictsignal}
\index{strictsignal option}
\item[strictsignal]
Requires that all checked exceptions signalled within a method but not caught by a catch clause be listed in the signals phrase of the method instruction.
\index{option,symbols}
\index{flag,symbols}
\index{symbols option}
\item[symbols]
Symbol table information (names of local variables, etc.) will be included in any generated .class file. This option is provided to aid the production of classes that are easy to analyse with tools that can understand the symbol table information. The use of this option increases the size of .class files.
\index{option,trace, traceX}
\index{flag,trace, traceX}
\index{trace, traceX option}
\item[trace, traceX]
If given as \textbf{-trace}, \textbf{-trace1}, or \textbf{-trace2}, then trace instructions are accepted. The trace output is directed according to the option word: \textbf{-trace1} requests that trace output is written to the standard output stream, \textbf{-trace} or \textbf{-trace2} imply that the output should be written to the standard error stream (the default).
\index{option,utf8}
\index{flag,utf8}
\index{utf8 option}
\item[utf8]
If given, clauses following the options instruction are expected to be encoded using UTF-8, so all Unicode characters may be used in the source of the program.
In UTF-8 encoding, Unicode characters less than '\textbackslash u0080' are represented using one byte (whose most-significant bit is 0), characters in the range '\textbackslash u0080' through '\textbackslash u07FF' are encoded as two bytes, in the sequence of bits:
\begin{verbatim}
  110xxxxx 10xxxxxx
\end{verbatim}
where the eleven digits shown as x are the least significant eleven bits of the character, and characters in the range '\textbackslash u0800' through '\textbackslash uFFFF' are encoded as three bytes, in the sequence of bits:
\begin{verbatim}
  1110xxxx 10xxxxxx 10xxxxxx
\end{verbatim}
where the sixteen digits shown as x are the sixteen bits of the character.
If noutf8 is given, following clauses are assumed to comprise only Unicode characters in the range '\textbackslash x00' through '\textbackslash xFF', with the more significant byte of the encoding of each character being 0.
Note: this option only has an effect as a compiler option, and applies to all programs being compiled. If present on an options instruction, it is checked and must match the compiler option (this allows processing with or without utf8 to be enforced).
\index{option,verbose, verboseX}
\index{flag,verbose, verboseX}
\index{verbose, verboseX option}
\item[verbose, verboseX]
Sets the “noisiness” of the language processor. The digit X may be any of the digits 0 through 5; if omitted, a value of 3 is used. The options \textbf{-noverbose} and \textbf{verbose0} both suppress all messages except errors and warnings
\end{description}

\subsubsection{Options valid on the commandline}
The translator also implements some additional option words, which
control compilation features.  These cannot be used on the
\textbf{options} instruction\footnote{Although at the moment, there will be no indication of this}, and are:
\begin{description}
\index{option,arg words}
\index{flag,arg words}
\index{arg words option}
\item[arg]
The \textbf{-arg} \emph{words} option is used when interpreting
programs, it indicates that after the \textbf{-arg} statement,
commandline arguments for ther interpreted program follow

\index{option,classpath}
\index{flag,classpath}
\index{classpath option}
\item[classpath]
The -classpath option allows dynamic specification of the classpath
used by the \nr{}C compiler without having to depend on the
CLASSPATH environment variable. (since: \nr{} 3.02)
.
\index{option,exec}
\index{flag,exec}
\index{exec option}
\item[exec]
The \textbf{-exec} \emph{words} option is used when interpreting programs. With this option, no commandline arguments are possible.
\index{option,keep}
\index{flag,keep}
\index{keep option}
\item[keep]
keep the intermediate \emph{.java} file for each program.  It is kept in
the same directory as the \nr{} source file as \emph{xxx.java.keep},
where \emph{xxx} is the source file name.  The file will also be kept
automatically if the \emph{javac} compilation fails for any reason.
\index{option,keepasjava}
\index{flag,keepasjava}
\index{keepasjava option}
\item[keepasjava]
keep the intermediate \emph{.java} file for each program.  It is kept in
the same directory as the \nr{} source file as \emph{xxx.java},
where \emph{xxx} is the source file name.  Implies -replace. Note: use this option carefully in mixed-source projects where you might have .java source files around.
\item[nocompile]
\index{option, nocompile}
\index{flag, nocompile}
\index{nocompile option}
do not compile (just translate).  Use this option when you want to use a
different Java compiler.  The \emph{.java} file for each program is kept
in the same directory as the \nr{} source file, as the
file \emph{xxx.java.keep} (where \emph{xxx} is the source file name).
\item[noconsole]
\index{option, noconsole}
\index{flag, noconsole}
\index{noconsole option}
do not display compiler messages on the console (command display
screen).  This is usually used with the \emph{savelog} option.
\item[savelog]
\index{option, savelog}
\index{flag, savelog}
\index{savelog option}
write compiler messages to the file \emph{\nr{}C.log}, in the current
directory.
This is often used with the \emph{noconsole} option.
\item[time]
\index{option, time}
\index{flag, time}
\index{time option}
display translation, \emph{javac} or \emph{ecj} compile, and total times (for the sum
of all programs processed).
\item[run]
\index{option, run}
\index{flag, run}
\index{run option}
run the resulting Java class as a stand-alone application, provided that
the compilation had no errors.
\index{option,warnexit0}
\index{flag,warnexit0}
\index{warnexit0 option}
\item[warnexit0]
Exit the translator with returncode 0 even if warnings are issued. Useful with build tools that would otherwise exit a build.
\end{description}


\backmatter
%\listoffigures
%\lstlistoflistings
\printindex
\clearpage
\psset{unit=1in}
\begin{pspicture}(3.5,1in)
  \psbarcode{\isbn}{includetext guardwhitespace}{isbn}
\end{pspicture}
\end{document} 
