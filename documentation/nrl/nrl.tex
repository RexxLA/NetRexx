\documentclass[10pt, openany]{book}
\usepackage[FINAL]{../boilerplate/rexx} 
\usepackage{hyperref}
\usepackage{graphics}
\usepackage{fontspec}
% \fontspec
%      [ Path = /Users/rvjansen/Fonts/,
%        BoldFont       = MinionPro-Bold.otf ,
%        ItalicFont     = MinionPro-It.otf ,
%        BoldItalicFont = MinionPro-BoldIt.otf ]
%      {MinionPro-Regular.otf}
% \fontspec
%      [ Path = /Users/rvjansen/Fonts/,
%        BoldFont       = SourceCodePro-Bold.otf ]
%      {SoureCodePro-Regular.otf}

\setmainfont[Mapping=tex-text]{Minion Pro}
\setmonofont[Mapping=tex-text,Scale=0.80]{Source Code Pro}
\usepackage{tabularx}
\usepackage{booktabs}
\usepackage{makeidx}
\usepackage[all]{xy}
%\usepackage{lingmacros}
\usepackage{color}
\usepackage{xcolor}
\usepackage{listings}
\usepackage{caption}
\usepackage{longtable}
\usepackage{colortbl}
\usepackage{framed}
\usepackage{fancyvrb}
\definecolor{shadecolor}{rgb}{0.9,0.9,0.9}
\usepackage{alltt}
\DeclareCaptionFont{white}{\color{white}}
\DeclareCaptionFormat{listing}{\colorbox{gray}{\parbox{\textwidth}{#1#2#3}}}
\captionsetup[lstlisting]{format=listing,labelfont=white,textfont=white}
\usepackage{listings}
\usepackage[official]{eurosym}
\makeatletter
\lst@CCPutMacro\lst@ProcessOther {"2D}{\lst@ttfamily{-{}}{-{}}}
\@empty\z@\@empty
\makeatother
\lstdefinelanguage{NetRexx}
{morekeywords={abstract,adapter,binary,case,catch,class,constant,dependent,deprecated,digits,do,else,end,engineering,extends,final,finally,for,forever,if,implements,indirect,import,indirect,inheritable,interface,iterate,label,leave,loop,method,native,nop,numeric,options,otherwise,over,package,parent,parse,private,properties,protect,public,return,returns,rexx,say,scientific,set,digits,form,select,shared,signal,signals,sourceline,static,super,then,this,until,used,upper,volatile,when,where,while},
sensitive=false,
extendedchars=false,
morecomment=[s]={/*}{*/},
morecomment=[l]{--},
morecomment=[s]{/**}{*/},
morestring=[b]",
morestring=[d]",
morestring=[b]',
morestring=[d]'}

\lstset{language=NetRexx,
  captionpos=t,
  tabsize=3,
  alsolanguage=Rexx,
  keywordstyle=\color{blue},
  commentstyle=\color{cyan},
  stringstyle=\color{red},
  numbers=left,
  numberstyle=\tiny,
  numbersep=5pt,
  breaklines=true,
  showstringspaces=false,
  index=[1][keywords],
  columns=flexible,
  basicstyle=\fontsize{8}{8}\fontspec{Source Code Pro},emph={label}}

\usepackage{../boilerplate/rail}
\usepackage{pst-barcode,pstricks-add}
\usepackage{bashful}
\usepackage{metalogo}
\usepackage{marginnote}
\usepackage{pdfpages}
\usepackage{float}
\hyphenation{Net-Rexx Net-Rexx-A Net-Rexx-C Net-Rexx-R Mac-OSX
  infra-structure im-ple-men-ta-tion-de-pen-dent}
\makeindex
\DeclareGraphicsExtensions{.jpg,.png}
\setlength{\parskip}{4pt}
\setlength{\parindent}{0pt}
\usepackage{enumitem}
\newcommand{\nr}{Net\textsc{Rexx}}
\newcommand{\Rexx}{R\textsc{exx}}
\newcommand{\nrpackagename}{\splice{java GetPackageName}}
\newcommand{\minimalJVMversion}{1.6}
\newcommand{\keyword}[1]{\texttt{#1}}
\newcommand{\code}[1]{\texttt{#1}}
\newcommand{\thisyear}{\splice{java TexYear}}
\newcommand{\ecjjarname}{ecj-4.6.3.jar}
\newcommand{\msd}[1]{\msdhelper#1\relax}
\newcommand{\msdhelper}[1]
  {\ifx\relax#1\else
    \ifx-#1--{}\else#1\fi
    \expandafter\msdhelper\fi}
  \newcommand{\doublehyphen}{\mbox{\msd{``-~-''}}}
  \newcommand{\doublehyphenunquoted}{\mbox{\msd{-~-}}}
%%% Local Variables: 
%%% mode: latex
%%% TeX-master: t
%%% End: 

\begin{document} 
\renewcommand{\isbn}{978-90-819090-1-3}
\setcounter{tocdepth}{1} 
\title{\nr{}\protect\\Language Reference}
\author{Mike Cowlishaw and RexxLA}
\date{Version \nrversion{} of \today}
\maketitle
\pagenumbering{Roman}
\pagestyle{plain}
\frontmatter
\pagenumbering{Roman}
\pagestyle{plain}
\section*{Publication Data}
\textcopyright  Copyright The Rexx Language Association, 2011-\splice{java TexYear}
%\\

All original material in this publication is published under the Creative Commons - Share Alike 3.0 License as stated at \url{http://creativecommons.org/licenses/by-nc-sa/3.0/us/legalcode}.\\[0.5cm]
The responsible publisher of this edition is identified as \emph{IBizz IT Services and Consultancy}, Amsteldijk 14, 1074 HR Amsterdam, a registered company governed by the laws of the Kingdom of The Netherlands.\\[1cm]
This edition is registered under ISBN \isbn \\[1cm]
\psset{unit=1in}
\begin{pspicture}(3.5,1in)
  \psbarcode{\isbn}{includetext guardwhitespace}{isbn}
\end{pspicture}
\newpage
%%% Local Variables:
%%% mode: latex
%%% TeX-master: t
%%% End:

\tableofcontents
\newpage
\pagenumbering{arabic}
\frontmatter
\large
\chapter*{\fontspec{IBM Plex Serif}\LARGE The \nr{} Programming Series}
This book is part of a library, the \emph{\nr{} Programming Series}, documenting the \nr{} programming language and its use and applications. This section lists the other publications in this series, and their roles. These books can be ordered in convenient hardcopy and electronic formats from the Rexx Language Association.
\newline
\newline
\begin{tabularx}{\textwidth}{>{\bfseries}lX}
\toprule
%% Quick Start Guide & This guide is meant for an audience that has done some programming and wants to start quickly. It starts with a quick tour of the language, and a section on installing the \nr{} translator and how to run it. It also contains help for troubleshooting if anything in the installation does not work as designed, and states current limits and restrictions of the open source reference implementation.
%% \\\midrule
Programming Guide & The Programming Guide is the one manual that at the same time teaches programming, shows lots of examples as they occur in the real world, and explains about the internals of the translator and how to interface with it.
\\\midrule
Language Reference & Referred to as the NRL, this is meant as the formal definition for the language, documenting its syntax and semantics, and prescribing minimal functionality for language implementers.
\\\midrule
Pipelines Guide \& Reference & The Data Flow oriented companion to \nr{}, with its CMS Pipelines compatible syntax, is documented in this manual. It discusses running Pipes for \nr{} in the command shell and the Workspace, and has ample examples of defining your own stages in \nr{}.
\\\bottomrule
\end{tabularx}
%%% Local Variables: 
%%% mode: latex
%%% TeX-master: t
%%% End: 

\chapter{Typographical conventions}
In general, the following conventions have  been observed in the NetRexx publications:
\begin{itemize}
\item Body text is in this font
\item Examples of language statements are in a \textbf{bold} type
\item Variables or strings as mentioned in source code, or things that appear on the console, are in a \texttt{typewriter} type
\item Items that are introduced, or emphasized, are in an \emph{italic} type
\item Included program fragments are listed in this fashion:
\begin{lstlisting}[label=example,caption=Example Listing]
-- salute the reader
say 'lectorem salutat'
\end{lstlisting}
\item Syntax diagrams take the form of so-called \emph{Railroad Diagrams} to convey structure, mandatory and optional items
\begin{rail}
AggregateExpression : ("AVG" |"MAX" |"MIN" |"SUM")
 (
   (
    'DISTINCT' ?  StateFieldPathExpression
   ) | 'COUNT'
   (
    'DISTINCT' ?  IdentificationVariable
                  | StateFieldPathExpression
                  | SingleValuedAssociationPathExpression
   )
 )
   ;
\end{rail}
%%% Local Variables: 
%%% mode: latex
%%% TeX-master: t
%%% End: 
\end{itemize}
\chapter{Introduction}
This document is the \emph{Quick Start Guide} for the reference implementation of
\nr{}. \nr{} is a \emph{human-oriented} programming language which makes
writing and using Java\footnote{Java is a trademark of Oracle, Inc.}
classes quicker and easier than writing in Java. It is part of the Rexx
language family, under the governance of the Rexx Language
Association.\footnote{\url{http.www.rexxla.org}} \nr{} has been
developed and was made available as a free download by IBM since 1995
and is free and open source since June 8, 2011.

In this Quick Start Guide, you’ll find information on
\begin{enumerate} 
\item How easy it is to write for the JVM: A Quick Tour of \nr{}
\item Installing \nr{} 
\item Using the \nr{} translator as a compiler, interpreter, or
  syntax checker 
\item Troubleshooting when things do not work as expected
\item Current restrictions.
\end{enumerate} 
The \nr{} documentation and software are distributed
by The Rexx Language Association under the \textsc{ICU} license. For
the terms of this license, see the included \textsc{LICENSE} file in
this package.

For details of the \nr{} language, and the latest news, downloads,
etc., please see the \nr{} documentation included with the package
or available at: \url{http://www.netrexx.org}.

\begin{shaded}\noindent
The highest Java version that is supported in this version, 3.09, is
\emph{Java 8}. Higher versions are not yet supported due to changes in
Java, including incompatibilities introduced with the Java module system.
\end{shaded}\indent

\mainmatter
\chapter{Introduction}
NetRexx is a general-purpose programming language inspired by two very
different programming languages, Rexx\textsuperscript{\texttrademark} and Java\textsuperscript{\texttrademark}. It is designed for
people, not computers. In this respect it follows Rexx closely, with
many of the concepts and most of the syntax taken directly from Rexx
or its object- oriented version, Object Rexx. From Java it derives
static typing, binary arithmetic, the object model, and exception
handling. The resulting language not only provides the scripting
capabilities and decimal arithmetic of Rexx, but also seamlessly
extends to large application development with fast binary arithmetic.

The open source reference implementation (version 3 and later) of
NetRexx produces classes for the Java Virtual Machine, and in so doing
demonstrates the value of that concrete interface between language and
machine: NetRexx classes and Java classes are entirely equivalent –
NetRexx can use any Java class (and vice versa) and inherits the
portability and robustness of the Java environment.

This document is in three parts:
\begin{enumerate}
\item The objectives of the NetRexx language and the concepts underlying its design, and acknowledgements.
\item An overview and introduction to the NetRexx language.
\item The definition of the language.
\end{enumerate}
Appendices include a sample NetRexx program, a description of an experimental feature, and some
details of the contents of the \texttt{netrexx.lang} package.
\section{Language Objectives}
This document describes a programming language, called NetRexx, which
is derived from both Rexx and Java. NetRexx is intended as a dialect
of Rexx that can be as efficient and portable as languages such as
Java, while preserving the low threshold to learning and the ease of
use of the original Rexx language.
\subsection{Features of Rexx}
The Rexx programming language\footnote{Cowlishaw, M. F., \textbf{The REXX Language} (second edition), ISBN 0-13-780651-5, Prentice-Hall, 1990.} was designed with just one objective: to make programming easier than it was before. The design achieved this by emphasizing readability and usability, with a minimum of special notations and restrictions. It was consciously designed to make life easier for its users, rather than for its implementers.
One important feature of Rexx syntax is \emph{keyword
  safety}. Programming languages invariably need to evolve over time
as the needs and expectations of their users change, so this is an
essential requirement for languages that are intended to be executed from source.

Keywords in Rexx are not globally reserved but are recognized only in
context. This language attribute has allowed the language to be
extended substantially over the years without invalidating existing
programs. Even so, some areas of Rexx have proved difficult to extend
– for example, keywords are reserved within instructions such as
\textbf{do}. Therefore, the design for NetRexx takes the concept of
keyword safety even further than in Rexx, and also improves
extensibility in other areas.

The great strengths of Rexx are its human-oriented features, including 
\begin{itemize}
\item simplicity
\item coherent and uncluttered syntax
\item comprehensive stringhandling
\item case-insensitivity
\item arbitrary precision decimal arithmetic.
\end{itemize}
Care has been taken to preserve these. Conversely, its interpretive
nature has always entailed a lack of efficiency: excellent Rexx
compilers do exist, from IBM and other companies, but cannot offer the
full speed of statically-scoped languages such as
C\footnote{Kernighan, B. W., and Ritchie, D. M., \textbf{The C
    Programming Language} (second edition), ISBN 0-13-110362-8,
  Prentice- Hall, 1988.} or Java\footnote{Gosling, J. A., \emph{et
    al.} \textbf{The Java Language Specification}, ISBN 0-201-63451-1,
  Addison-Wesley, 1996.}.
\subsection{Influence of Java}
The system-independent design of Rexx makes it an obvious and natural
fit to a system-independent execution environment such as that
provided by the Java Virtual Machine (JVM). The JVM, especially when
enhanced with “just-in-time” bytecode compilers that compile bytecodes
into native code just before execution, offers an effective and
attractive target environment for a language like Rexx.

Choosing the JVM as a target environment does, however, place significant constraints on the design of a language suitable for that environment. For example, the semantics of method invocation are in several ways determined by the environment rather than by the source language, and, to a large extent, the object model (class structure, \emph{etc.}) of the Java environment is imposed on languages that use it.

Also, Java maintains the C concept of primitive datatypes; types (such
as \texttt{int}, a 32-bit signed integer) which allow efficient use of the underlying hardware yet do not describe true objects. These types are pervasive in classes and interfaces written in the Java language; any language intending to use Java classes effectively must provide access to these types.

Equally, the \emph{exception} (error handling) model of Java is pervasive, to
the extent that methods must check certain exceptions and declare
those that are not handled within the method. This makes it difficult
to fit an alternative exception model.

The constraints of safety, efficiency, and environment necessitated
that NetRexx would have to differ in some details of syntax and
semantics from Rexx; unlike Object Rexx, it could not be a fully
upwards-compatible extension of the language\footnote{Nash, S. C.,
  \textbf{Object-Oriented REXX} in Goldberg, G, and Smith, P. H. III,
  \textbf{The Rexx Handbook}, pp115-125, ISBN 0-07-023682-8,
  McGraw-Hill, Inc., New York, 1992.}. The need for changes, however,
offered the opportunity to make some significant simplifications and
enhancements to the language, both to improve its keyword safety and to strengthen other features of the original Rexx design\footnote{See Cowlishaw, M. F., \textbf{The Early History of REXX}, IEEE Annals of the History of Computing, ISSN 1058-6180, Vol 16, No. 4, Winter 1994, pp15-24, and Cowlishaw, M. F., \textbf{The Future of Rexx}, Proceedings of Winter 1993 Meeting/SHARE 80, Volume II, p.2709, SHARE Inc., Chicago, 1993.}. Some additions from Object Rexx and ANSI Rexx\footnote{See \textbf{American National Standard for Information Technology – Programming Language REXX}, X3.274-1996, American National Standards Institute, New York, 1996.} are also included.

Similarly, the concepts and philosophy of the Rexx design can profitably be applied to avoid many of the minor irregularities that characterize the C and Java language family, by providing suitable simplifications in the programming model. For example, the NetRexx looping construct has only one form, rather than three, and exception handling can be applied to all blocks rather than requiring an extra construct. Also, as in Rexx, all NetRexx storage allocation and de-allocation is implicit – an explicit new operator is not required.

Further, the human-oriented design features of Rexx
(case-insensitivity for identifiers, type deduction from context,
automatic conversions where safe, tracing, and a strong emphasis on
string representations of common values and numbers) make programming
for the Java environment especially easy in NetRexx.
\subsection{A hybrid or a whole?}
As in other mixtures, not all blends are a success; when first designing NetRexx, it was not at all obvious whether the new language would be an improvement on its parents, or would simply reflect the worst features of both.

The fulcrum of the design is perhaps the way in which datatyping is automated without losing the static typing supported by Java. Typing in NetRexx is most apparent at interfaces – where it provides most value – but within methods it is subservient and does not obscure algorithms. A simple concept, \emph{binary classes}, also lets the programmer choose between robust decimal arithmetic and less safe (but faster) binary arithmetic for advanced programming where performance is a primary consideration.

The “seamless” integration of types into what was previously an essentially typeless language does seem to have been a success, offering the advantages of strong typing while preserving the ease of use and speed of development that Rexx programmers have enjoyed.

The end result of adding Java typing capabilities to the Rexx language
is a single language that has both the Rexx strengths for scripting
and for writing macros for applications and the Java strengths of
robustness, good efficiency, portability, and security for application
development.
\section{Language Concepts}
As described in the last section, NetRexx was created by applying the philosophy of the Rexx language to the semantics required for programming the Java Virtual Machine (JVM). Despite the assumption that the JVM is a “target environment” for NetRexx, it is intended that the language not be environment-dependent; the essentials of the language do not depend on the JVM. Environment- dependent details, such as the primitive types supported, are not part of the language specification.

The primary concepts of Rexx have been described before, in \emph{The
  Rexx Language}, but it is worth repeating them and also indicating
where modifications and additions have been necessary to support the
concepts of statically-typed and object-oriented environments. The
remainder of this section is therefore a summary of the principal
concepts of NetRexx.
\subsection{Readability}
One concept was central to the evolution of Rexx syntax, and hence NetRexx syntax: \emph{readability} (used here in the sense of perceived legibility). Readability in this sense is a somewhat subjective quality, but the general principle followed is that the tokens which form a program can be written much as one might write them in Western European languages (English, French, and so forth). Although NetRexx is more formal than a natural language, its syntax is lexically similar to everyday text.

The structure of the syntax means that the language is readily adapted
to a variety of programming styles and layouts. This helps satisfy
user preferences and allows a lexical familiarity that also increases
readability. Good readability leads to enhanced understandability,
thus yielding fewer errors both while writing a program and while
reading it for information, debugging, or maintenance. 

Important factors here are:
\begin{enumerate}
\item Punctuation and other special notations are required only when absolutely necessary to remove ambiguity (though punctuation may often be added according to personal preference, so long as it is syntactically correct). Where notations are used, they follow established conventions.
\item The language is essentially case-insensitive. A NetRexx
  programmer may choose a style of use of uppercase and lowercase letters that he or she finds most helpful (rather than a style chosen by some other programmer).
\item The classical constructs of structured and object-oriented
  programming are available in NetRexx, and can undoubtedly lead to
  programs that are easier to read than they might otherwise be. The
  simplicity and small number of constructs also make NetRexx an
  excellent language for teaching the concepts of good structure.
\item Loose binding between the physical lines in a program and the syntax of the language ensures that even though programs are affected by line ends, they are not irrevocably so. A clause may be spread over several lines or put on just one line; this flexibility helps a programmer lay out the program in the style felt to be most readable.
\end{enumerate}
\subsection{Natural data typing and decimal arithmetic}
“Strong typing”, in which the values that a variable may take are
tightly constrained, has been an attribute of some languages for many
years. The greatest advantage of strong typing is for the interfaces
between program modules, where errors are easy to introduce and
difficult to catch. Errors \emph{within} modules that would be
detected by strong typing (and which would not be detected from
context) are much rarer, certainly when compared with design errors,
and in the majority of cases do not justify the added program
complexity.

NetRexx, therefore, treats types as unobtrusively as possible, with a simple syntax for type description which makes it easy to make types explicit at interfaces (for example, when describing the arguments to methods).

By default, common values (identifiers, numbers, and so on) are described in the form of the symbolic notation (strings of characters) that a user would normally write to represent those values. This natural datatype for values also supports decimal arithmetic for numbers, so, from the user’s perspective, numbers look like and are manipulated as strings, just as they would be in everyday use on paper.

When dealing with values in this way, no internal or machine
representation of characters or numbers is exposed in the language,
and so the need for many data types is reduced. There are, for
example, no fundamentally different concepts of \emph{integer} and
\emph{real}; there is just the single concept of \emph{number}. The
results of all operations have a defined symbolic representation, and
will therefore act consistently and predictably for every correct
implementation.

This concept also underlies the BASIC\footnote{Kemeny, J. G. and
  Kurtz, T. E., \textbf{BASIC programming}, John Wiley \& Sons Inc.,
  New York, 1967.} language; indeed, Kemeny and Kurtz's vision for
BASIC included many of the fundamental principles that inspired
Rexx. For example, Thomas E. Kurtz wrote:
\begin{quote}
“Regarding variable types, we felt that a distinction between ‘fixed’
and ‘floating’ was less justified in 1964 than earlier ... to our
potential audience the distinction between an integer number and a
non-integer number would seem esoteric. A number is a number is a
number.”\footnote{Kurtz, T. E., \textbf{BASIC} in Wexelblat,
  R. L. (Ed), \textbf{History of Programming Languages}, ISBN
  0-12-745040-8, Academic Press, New York 1981.}
\end{quote}
For Rexx, intended as a scripting language, this approach was ideal; symbolic operations were all that were necessary.

For NetRexx, however, it is recognized that for some applications it
is necessary to take full advantage of the performance of the
underlying environment, and so the language allows for the use and
specification of binary arithmetic and types, if available. A very
simple mechanism (declaring a class or method to be \emph{binary}) is
provided to indicate to the language processor that binary arithmetic
and types are to be used where applicable. In this case, as in other
languages, extra care has to be taken by the programmer to avoid
exceeding limits of number size and so on.
\subsection{Emphasis on symbolic manipulation}
Many values that NetRexx manipulates are (from the user’s point of view, at least) in the form of strings of characters. Productivity is greatly enhanced if these strings can be handled as easily as manipulating words on a page or in a text editor. NetRexx therefore has a rich set of character manipulation operators and methods, which operate on values of type \texttt{Rexx} (the name of the class of NetRexx strings).

Concatenation, the most common string operation, is treated specially
in NetRexx. In addition to a conventional concatenate operator (“||”),
the novel \emph{blank operator} from Rexx concatenates two data
strings together with a blank in between. Furthermore, if two
syntactically distinct terms (such as a string and a variable name)
are abutted, then the data strings are concatenated directly. These
operators make it especially easy to build up complex character
strings, and may at any time be combined with the other operators.

For example, the \textbf{say} instruction consists of the keyword \textbf{say} followed
by any expression. In this instance of the instruction, if the
variable n has the value “6” then
\begin{verbatim}
  say 'Sorry,' n*100/50'% were rejected'
\end{verbatim}
would display the string
\begin{verbatim}
  Sorry, 12% were rejected
\end{verbatim}
Concatenation has a lower priority than the arithmetic operators. The order of evaluation of the expression is therefore first the multiplication, then the division, then the concatenate-with-blank, and finally the direct concatenation.
Since the concatenation operators are distinct from the arithmetic
operators, very natural coercion (automatic conversion) between
numbers and character strings is possible. Further, explicit type-
casting (conversion of types) is effected by the same operators, at
the same priority, making for a very natural and consistent syntax for
changing the types of results. For example,
\begin{verbatim}
i=int 100/7
\end{verbatim}
would calculate the result of 100 divided by 7, convert that result to
an integer (assuming \texttt{int} describes an integer type) and then
assign it to the variable \texttt{i}.
\subsection{Nothing to declare}
Consistent with the philosophy of simplicity, NetRexx does not require
that variables within methods be declared before use. Only the
\emph{properties}\footnote{Class variables and instance variables.} of classes – which may form part of their
interface to other classes – need be listed formally.

Within methods, the type of variables is deduced statically from
context, which saves the programmer the menial task of stating the
type explicitly. Of course, if preferred, variables may be listed and
assigned a type at the start of each method.
\subsection{Environment independence}
The core NetRexx language is independent of both operating systems and hardware. NetRexx programs, though, must be able to interact with their environment, which implies some dependence on that environment (for example, binary representations of numbers may be required). Certain areas of the language are therefore described as being defined by the environment.

Where environment-independence is defined, however, there may be a
loss of efficiency – though this can usually be justified in view of
the simplicity and portability gained.

As an example, character string comparison in NetRexx is normally
independent of case and of leading and trailing blanks. (The string “
Yes ” \emph{means} the same as “yes” in most applications.) However,
the influence of underlying hardware has often subtly affected this
kind of design decision, so that many languages only allow trailing
blanks but not leading blanks, and insist on exact case matching. By
contrast, NetRexx provides the human-oriented relaxed comparison for
strings as default, with optional “strict comparison” operators.

\subsection{Limited span syntactic units}

\backmatter
% \listoffigures
% \listoftables
% \lstlistoflistings
\printindex
\clearpage
\psset{unit=1in}
\begin{pspicture}(3.5,1in)
  \psbarcode{\isbn}{includetext guardwhitespace}{isbn}
\end{pspicture}
\end{document} 
