\section{If instruction}
\index{IF instruction,}
\index{Instructions,IF}
\index{THEN,following IF clause}
\index{,}
\index{Flow control,with IF construct}
\begin{shaded}
\begin{alltt}
\textbf{if} \emph{expression}[;]
     \textbf{then}[;] \emph{instruction}
    [\textbf{else}[;] \emph{instruction}]
\end{alltt}
\end{shaded}
 The \keyword{if} construct is used to conditionally execute an
instruction or group of instructions.
It can also be used to select between two alternatives.
 The expression is evaluated and must result in either 0 or 1.
If the result was 1 (true) then the instruction after the
\keyword{then} is executed.
If the result was 0 (false) and an \keyword{else} was given
then the instruction after the \keyword{else} is executed.
 \textbf{Example:}
\begin{lstlisting}
if answer='Yes' then say 'OK!'
                else say 'Why not?'
\end{lstlisting}
 Remember that if the \keyword{else} clause is on the same line as the
last clause of the \keyword{then} part, then you need a semicolon to
terminate that clause.
 \textbf{Example:}
\begin{lstlisting}
if answer='Yes' then say 'OK!';  else say 'Why not?'
\end{lstlisting}
 The \keyword{else} binds to the nearest \keyword{then} at the same level.
This means that any \keyword{if} that is used as the instruction
following the \keyword{then} in an \keyword{if} construct that has an
\keyword{else} clause, must itself have an \keyword{else} clause (which
may be followed by the dummy instruction, \keyword{nop}).
 \textbf{Example:}
\begin{lstlisting}
if answer='Yes' then if name='Fred' then say 'OK, Fred.'
                                    else say 'OK.'
                else say 'Why not?'
\end{lstlisting}
 
To include more than one instruction following \keyword{then} or
\keyword{else}, use a grouping instruction (\keyword{do}, \keyword{loop},
or \keyword{select}).
 \textbf{Example:}
\begin{lstlisting}
if answer='Yes' then do
  say 'Line one of two'
  say 'Line two of two'
end
\end{lstlisting}
In this instance, both \keyword{say} instructions are executed when
the result of the \keyword{if} expression is 1.
\subsection{Short circuit evaluation} 
Multiple expressions, separated by commas, can be given on the
\keyword{if} clause, which then has the syntax:
\begin{shaded}
\begin{alltt}
\textbf{if} \emph{expression}[, \emph{expression}]... [;]
\end{alltt}
\end{shaded}
In this case, the expressions are evaluated in turn from left to
right, and if the result of any evaluation is 1 then the test has
succeeded and the instruction following the associated \keyword{then}
clause is executed.
If all the expressions evaluate to 0 and an \keyword{else} was given
then the instruction after the \keyword{else} is executed.
 
Note that once an expression evaluation has resulted in 1, no further
expressions in the clause are evaluated.  So, for example, in:
\begin{lstlisting}
-- assume 'name' is a string
if name=null, name='' then say 'Empty'
\end{lstlisting}
then if \texttt{name} does not refer to an object it will compare equal to
null and the \keyword{say} instruction will be executed without
evaluating the second expression in the \keyword{if} clause.
\begin{shaded}\noindent
\textbf{Notes:}
\begin{enumerate}
\item An \emph{instruction} may be any assignment, method call, or
keyword instruction, including any of the more complex constructions
such as \keyword{do}, \keyword{loop}, \keyword{select}, and the \keyword{if}
instruction itself.
A null clause is not an instruction, however, so putting an extra
semicolon after the \keyword{then} or \keyword{else} is not equivalent to
putting a dummy instruction.
The \keyword{nop} instruction is provided for this purpose.
\item The keyword \keyword{then} is treated specially, in that it need not start a
clause.
This allows the expression on the \keyword{if} clause to be terminated by
the \keyword{then}, without a "\textbf{;}" being required -
were this not so, people used to other computer languages would
be inconvenienced.
Hence the symbol \keyword{then} cannot be used as a variable name within
the expression.
\footnote{
Strictly speaking, \keyword{then} should only be recognized if not
the name of a variable.  In this special case, however, \nr{} language
processors are permitted to treat \keyword{then} as reserved in the
context of an \keyword{if} clause, to provide better performance and
more useful error reporting.
}
\end{enumerate}
\end{shaded}\indent
