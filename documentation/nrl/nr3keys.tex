\chapter{Keyword Instructions}\label{refkinst}
\index{Instructions,}
\index{Instructions,keyword}
\index{Keywords,mixed case}
\index{Keyword instructions,}
 A \emph{keyword instruction} is one or more clauses, the first of
which starts with a keyword that identifies the instruction.
Some keyword instructions affect the flow of control; the remainder
just provide services to the programmer.
Some keyword instructions (\texttt{do}, \texttt{if}, \texttt{loop}, or
\texttt{select}) can include nested instructions.
 Appendix A (see page \pageref{refappa})  includes an example of a \nr{} program
using many of the instructions available.
 As can be deduced from the syntax rules described earlier, a keyword
instruction is recognized \textbf{only} if its keyword is the first
token in a clause, and if the second token is not an "\textbf{=}"
character (implying an assignment).
It would also not be recognized if the second token started
with "\textbf{(}", "\textbf{[}",
or "\textbf{.}" (implying that the first token starts a term).
 Further, if a current local variable, method argument, or property
has the same name as a keyword then the keyword will not be recognized.
This important rule allows \nr{} to be extended with new keywords in
the future without invalidating existing programs.
 
Thus, for example, this sequence in a program with no \textbf{say}
variable:
\begin{alltt}
say 'Hello'
say('1')
say=3
say 'Hello'
\end{alltt}
would be a \texttt{say} instruction, a call to some \textbf{say}
method, an assignment to a \textbf{say} variable, and an error.
 In \nr{}, therefore, keywords are not reserved; they may be used as
the names of variables (though this is not recommended, where known in
advance).
\index{Sub-keywords,}
 Certain other keywords, known as \emph{sub-keywords}, may be
known within the clauses of individual instructions - for
example, the symbols \texttt{to} and \texttt{while} in the \texttt{loop}
instruction.  Again, these are not reserved; if they had been used as
names of variables, they would not be recognized as sub-keywords.
 Blanks adjacent to keywords have no effect other than that of
separating the keyword from the subsequent token.
For example, this applies to the blanks next to the sub-keyword
\texttt{while} in
\begin{alltt}
loop  while  a=3
\end{alltt}
Here at least one blank was required to separate the symbols
forming the keywords and the variable name, \textbf{a}.  However the
blank following the \texttt{while} is not necessary in
\begin{alltt}
loop while 'Me'=a
\end{alltt}
though it does aid readability.
