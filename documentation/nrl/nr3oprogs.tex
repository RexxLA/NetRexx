\subsection{NetRexx programs}\label{refoprogs}
\index{Overview,programs}
\index{Programs,overview}
:p.
The structure of a NetRexx program is extremely simple.  This sample
program, "toast", is complete, documented, and executable as it
stands:
\index{Example,program}
\begin{verbatim}
/* This wishes you the best of health. */
say 'Cheers!'
\end{verbatim}
:p.
This program consists of two lines: the first is an optional
comment that describes the purpose of the program, and the second is a
\texttt{say} instruction.  \texttt{say} simply displays the result of
the expression following it - in this case just a literal string
(you can use either single or double quotes around strings, as you
prefer).
:p.
To run this program using the reference implementation of NetRexx,
create a file called \textbf{toast.nrx} and copy or paste the two
lines above into it.  You can then use the \textbf{NetRexxC} Java
program to compile it:
\begin{verbatim}
java COM.ibm.netrexx.process.NetRexxC toast
\end{verbatim}
:pc.(this should create a file called \textbf{toast.class}),
and then use the \textbf{java} command to run it:
\begin{verbatim}
java toast
\end{verbatim}
:p.
You may also be able to use the \textbf{netrexxc} or \textbf{nrc}
command to compile and run the program with a single command (details
may vary - see the installation and user's guide document for your
implementation of NetRexx):
\begin{verbatim}
netrexxc toast -run
\end{verbatim}
:p.
Of course, NetRexx can do more than just display a character string.
Although the language has a simple syntax, and has a small number of
instruction types, it is powerful; the reference implementation of the
language allows full access to the rapidly growing collection of Java
programs known as \emph{class libraries}, and allows new class
libraries to be written in NetRexx.
:p.
The rest of this overview introduces most of the features of NetRexx.
Since the economy, power, and clarity of expression in NetRexx is best
appreciated with use, you are urged to try using the language yourself.
