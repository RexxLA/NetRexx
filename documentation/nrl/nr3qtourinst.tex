\section{Installation}
\nr{} requires an installed Java version 8 or later.

To install \nr{} download the GA package from https://www.netrexx.org/downloads.nsp.

Unzip the package to a directory of choice. As an example, we'll unzip to the 'nrx' directory in the user's home directory.

Add the bin/ directory of this 'nrx' directory to the PATH environment variable.

If the installed Java version is a JDK - i.e. the javac command is available - add the lib/NetRexxC.jar in this 'nrx' directory to the CLASSPATH environment variable.

If the installed Java version is a JRE - i.e. run-time only - add the lib/NetRexxF.jar in this 'nrx' directory to the CLASSPATH environment variable. 

Also add current directory (\textbf{.}) to the classpath.

On Linux, you could source following shell script \textbf{setnrc}:

\begin{lstlisting}
export PATH=~/nrx/bin:$PATH
export CLASSPATH=~/nrx/lib/NetRexxC.jar:.:$CLASSPATH
\end{lstlisting}

On Windows, you could use following batch file \textbf{setnrc.bat}

\begin{lstlisting}
SET PATH=%HOMEPATH%\nrx\bin;%PATH%
SET CLASSPATH=%HOMEPATH%\nrx\lib\NetRexxC.jar;.;%CLASSPATH%
\end{lstlisting}

To save such setting system-wide, please consult the appropriate documentation for your operating system.

After setting the environment, you can compile any \nr{} source file with \textbf{nrc sourcefile.nrx}. This will create a sourcefile.class, which can be executed by \textbf{java sourcefile}.
Compilation and execution can be done in one go by \textbf{nrc -run sourcefile}.

Interpretation can be started by \textbf{nrx -exec sourcefile}.

