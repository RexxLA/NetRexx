\section{Installation}
This \nr{} implementation provides a translator, compiler and interpreter, instantiated by the 
org.netrexx.process.NetRexxC Java class, and is delivered as a Java jar-file. An installed Java version 8 or later is required.

To install \nr{} download the GA package from the \nr{} website\footnote{https://www.netrexx.org/downloads.nsp}.

Unzip the package to a directory of choice. As an example, we'll unzip to the 'nrx' directory in the user's home directory.

Add the bin/ directory of this 'nrx' directory to the PATH environment variable.

If the installed Java version is a JDK - i.e. the javac command is available - add the lib/NetRexxC.jar in this 'nrx' directory to the CLASSPATH environment variable.

If the installed Java version is a JRE - i.e. run-time only - add the lib/NetRexxF.jar in this 'nrx' directory to the CLASSPATH environment variable. If you are unsure which java you have installed, use this jar.

Also add current directory (\textbf{.}) to the classpath.

On Linux with a JDK, you could source following shell script \textbf{setnrc}:

\begin{lstlisting}
export PATH=~/nrx/bin:$PATH
export CLASSPATH=~/nrx/lib/NetRexxC.jar:.:$CLASSPATH
\end{lstlisting}

On Windows with a JRE, you could use following batch file \textbf{setnrc.bat}

\begin{lstlisting}
SET PATH=%HOMEPATH%\nrx\bin;%PATH%
SET CLASSPATH=%HOMEPATH%\nrx\lib\NetRexxF.jar;.;%CLASSPATH%
\end{lstlisting}

To save such setting system-wide, please consult the appropriate documentation for your operating system.


For convenience, some shell scripts and batch files are provided in the bin/ directory.

After setting the environment, you can compile any \nr{} source file with \textbf{nrc sourcefile.nrx}. This will create a sourcefile.class, which can be executed by \textbf{java sourcefile}.
Compilation and execution can be done in one go by \textbf{nrc -run sourcefile}.

Interpretation can be started by \textbf{nrc -exec sourcefile} or \textbf{nr sourcefile}.

Under the covers, the translator translates \nr{} source code into Java source code, in memory unless the -keepasjava 
option is given. The compiler then compiles the generated Java source code either by using the javax.tools.JavaCompiler interface
 when a JDK is available, or by using the Eclipse batch
 compiler\footnote{which is called org.eclipse.jdt.internal.compiler.tool.EclipseCompiler} included in the NetRexxF.jar.

When interpreting, the \nr{} translator produces and runs the required Java bytecode and proxy classes without the need for a Java compiler.

