\section{Go instruction}\label{refgo}
\index{GO,instruction}
\index{Running methods concurrently}
\index{Instructions,GO}
\index{}
\index{,}
\index{,}
\begin{shaded}
    \begin{alltt}
        \textbf{go} \emph{expression};

        where \emph{expression} is a method call
    \end{alltt}
\end{shaded}

The \keyword{go} instruction is used to start methods concurrently. It takes a method as an argument and
starts this method in a new thread of execution.

The method can take any arguments. Any return value from the method is discarded.

\begin{lstlisting}

go counter(4)
go counter(3)
say 'counting..'

method counter(c) static
  say 'counting to 'c
  loop i = 0 to c
    say i 'from 'Thread.currentThread().getName()
  end
\end{lstlisting}

will start two threads, counting to 4 and 3 respectively.

One cannot predict the concurrency of the threads, in the example above, the exact order in which
the say statements will execute.
It is the runtime environment's scheduler which is responsible for
deciding which thread runs and for how long. This scheduling is non-deterministic.

RexxChannels (see page \pageref{refrexxchannel})provide an easy way to synchronise concurrency.
