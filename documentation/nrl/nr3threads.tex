\section{Thread Pool Support}\label{refthreads}
\index{Threads}

 Support for thread pooling is built into the \nr{} language. 


\begin{description}
\item[rtp=RexxTaskPool(size,number)]
\emph{size} is the number of parallel threads desired - default is the current number of available processors
\emph{number} is the number of the threadpool if using multiple pools - default is pool number 0

\item[rtp.start(runtask)]
runtask needs to be a class that implements the \emph{runnable} interface

\item[rtp.start(maintask,mainparm)]
\emph{maintask} is a NetRexx Java class with a standard "main" method
\emph{mainparm} is a string parm to pass to the main method at startup

\item[rtp.start("taskname",mainparm)]
\emph{taskname} is a string identifying a Java class with a standard main method
\emph{mainparm} is a string parm to pass to the main method at
startup\footnote{the start method returns the RexxTaskPool instance it
  is called on so that multiple calls can be stacked. Due to reflection use when starting "main" methods that call format cannot be interpreted - runnables interpret ok}

\item[rtp.waituntildone]
Blocks until all threads in the pool are finished

\item[rtp.waitforallpools]
Blocks until all threads in all task pools are complete
\end{description}

\textbf{Examples:}
\begin{lstlisting}[label=threadpool,caption=ThreadPool example]
RexxTaskPool(3,1).start(Test(7)).start(Test(8)).start("TestMain","9").start("enviroscan")
RexxTaskPool(9).start(Test(1)).start(Test(2)).start(TestMain("3"),"3").start(enviroscan.class)
RexxTaskPool().start(Test(4)).start(Test(5)).start(TestMain("6"),"6").waituntildone
RexxTaskPool().start(Test(4)).start(Test(5)).start(TestMain("6"),"6").waitforallpools
\end{lstlisting}