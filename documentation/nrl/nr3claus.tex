\chapter{Clauses and Instructions}\label{refclause}
  Clauses (see page \pageref{refclau})  are recognized, and can usefully be
classified, in the following order:
\begin{description}
\item[Null clauses]\label{refnullcl}
\index{Clauses,null}
\index{Null clauses,}

A clause that is empty or comprises only blanks, comments, and
continuations is a \emph{null clause} and is completely ignored by
\nr{} (except that if it includes a comment it will be traced, if
reached during execution).
\begin{shaded}
\textbf{Note: }A null clause is not an instruction, so (for example) putting an
extra semicolon after the \keyword{then} or \keyword{else} in an
\keyword{if} instruction is not equivalent to putting a dummy instruction
(as it would be in C or PL/I).
The \keyword{nop} instruction is provided for this purpose.
\end{shaded}
\item[Assignments]
\index{Instructions,assignment}
\index{Assignment,}
\index{Assignment,instruction}

Single clauses within a class and of the form
\emph{term}\textbf{=}\emph{expression}\textbf{;} are
instructions known as  \emph{assignment}s (see page \pageref{refassign}) .
An assignment gives a variable, identified by the
\emph{term}, a type or a new value.
 
In just one context, where property assignments are expected (before the
first method in a class), the "\textbf{=}" and the expression may
be omitted; in this case, the term (and hence the entire clause) will
always be a simple non-numeric symbol which names the property
\item[Method call instructions]\label{refxmeth}
\index{Method call instructions,}
\index{Instructions,method call}

A  method call instruction (see page \pageref{refmcalli})  is a clause within a
method that comprises a single term that is, or ends in, a method
invocation.
\item[Keyword instructions]\label{refkwcl}
\index{Keywords,}
\index{Keyword instructions,}
\index{Instructions,keyword}

A \emph{keyword instruction} consists of one or more clauses,
the first of which starts with a non-numeric symbol which is not the
name of a variable or property in the current class (if any) and is
immediately followed by a blank, a semicolon (which may be implied by
the end of a line), a literal string, or an operator (other than
"\textbf{=}", which would imply an assignment).
This symbol, the \emph{keyword}, identifies the instruction.
 
Keyword instructions control the external interfaces, the flow of
control, and so on.
Some  keyword instructions (see page \pageref{refkinst})  (\keyword{do}, \keyword{if},
\keyword{loop}, or \keyword{select}) can include nested instructions.
\end{description}
