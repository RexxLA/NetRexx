\chapter{Binary values and operations}\label{refbinary}
\index{Binary,arithmetic}
\index{Binary,operations}
\index{Binary,values}
 
By default, arithmetic and string operations in \nr{} are carried out
using the \nr{} string class, \textbf{R\textsc{exx}}, which offers the robust
set of operators described in \emph{Expressions and operators} (see page \pageref{refexpr}).
 
\index{Primitive types,}
\index{Types,primitive}
\nr{} implementations, however, may also provide \emph{primitive}
datatypes, as described in  \emph{Types and Classes} (see page \pageref{refprims}). These primitive types are used for compact storage
of numbers and for fast binary arithmetic, features which are built-in
to the hardware of most computers.
 
\index{Binary classes,}
To make use of binary arithmetic, a class is declared to be a
 \emph{binary class} (see page \pageref{refbincla})  by using the \keyword{binary}
keyword on the \keyword{class} instruction.
In such a class, literal strings and numeric symbols are assigned native
string or primitive types, rather than \nr{} types, where appropriate,
and native binary operations are used to implement operators where
possible, as detailed below.
Implementations may also provide a keyword on the
 \keyword{options} (see page \pageref{refoptions})  instruction that indicates that
all classes in a program are binary classes.
\footnote{
\emph{In the reference implementation, \keyword{options binary} is
used}.
}
 
\index{Binary methods,}
Alternatively, individual methods within a class may be declared to
be a  \emph{binary method} (see page \pageref{refbinme})  by using the \keyword{binary}
keyword on the \keyword{method} instruction.
 
\index{Integers, binary,}
\index{Floating-point numbers, binary,}
\index{Real numbers, binary,}
\index{Binary numbers,}
Binary classes and methods should be used with care.  Although binary
arithmetic can have a considerable performance advantage over
arithmetic that is not implemented in hardware, it can give incorrect
or unexpected results.
In particular, whole numbers (integers) are often held in fixed-sized
data areas (of 8, 16, 32, or 64 bits), and overflowing the data area
during a calculation can result in a positive number becoming negative
and vice versa.
Similarly, binary numbers that are not whole numbers (floating-point
numbers) cannot exactly represent common numbers in the decimal system
(\textbf{0.1}, \textbf{0.2}, \emph{etc.}), and hence can give unexpected
results.
\section{Operations in binary classes and methods}\label{refbindif}
 
In a binary class or method, the following (and only the following)
rules differ from the usual rules:
\begin{description}
\item[Dyadic operations in expressions]
\index{Binary operations,dyadic}
 
If the operands of a dyadic operator both have primitive numeric types
\footnote{
\emph{In the reference implementation, \textbf{boolean} is considered to
be a numeric type (having the values \textbf{0} or \textbf{1})
but \textbf{char} is not.
Characters, and strings or arrays of characters, always use the rules
defined for \nr{} strings.}
}
then binary operations are carried out.  The type of the result is
implementation defined, and is typically the type of the more precise of
the two operands, or of some minimum precision.
\footnote{
\emph{In the reference implementation, the minimum precision is 32 bits,
so an \textbf{int} is returned for results that would otherwise
be \textbf{byte} or \textbf{short}.
If both operands are \textbf{boolean}, however, and the operation is a
logical operation, then the type of the result
is \textbf{boolean}.}
}
Arithmetic operations follow the usual rules of binary arithmetic, as
defined for the underlying environment of the implementation.
 
Note that \nr{} provides both divide and integer divide operators; in
a binary class or method, the divide operator ("\textbf{/}")
converts its operands to floating-point types and returns a
floating-point result, whereas the integer divide operator
("\textbf{\%}") converts its operands to integer types and
returns an integer result.
The remainder operator must accept both integer and floating-point
types.
 
Logical operations (\emph{and}, \emph{or}, and \emph{exclusive
or}) apply to all the bits of the operands, and are not permitted
on floating-point types.
\item[Prefix operations in expressions]
\index{Binary operations,prefix}
\index{Binary operations,monadic}
 
If the operand of a prefix operator has a primitive numeric type, then
the type of the result is the type of the operand, subject to
the same minimum as dyadic operations.
Prefix plus and minus follow the rules of dyadic operators (because they
are defined as being zero plus or minus the operand) with the additional
rule that if acting on a literal number (a constant in the program) then
the result is also considered to be a literal constant.
Logical not (prefix "\textbf{\textbackslash }") does not apply to all
the bits of its operand; instead, it changes a \textbf{0}
to \textbf{1} and vice versa.
\item[Assignments]
\index{Binary classes,assignment}
\index{Binary methods,assignment}
\index{Binary literals,}
\index{Literals, binary,}
\index{Assignment,of literals}
\index{Assignment,binary}
 
In assignments where the value being assigned is the result of an
expression which comprises a string or number literal constant, the
type of the result is defined as follows:
\begin{enumerate}
\item Strings are given the native string type, even for a
single-character literal.
\footnote{
\emph{In the reference implementation, this type is \textbf{java.lang.String}}.
}
\item Numbers are given the smallest possible primitive numeric type that
will contain the literal without loss of information (or minimal loss of
information for numbers with decimal or exponential parts).
If this is smaller than the implementation-defined minimum precision
used for the result of adding the literal to \textbf{0}, then the type
of that minimum precision is used.
 
If the constant is an integer, and no primitive integer binary type has
sufficient precision to hold the number without loss of information,
then the number is treated as a literal string (that is, as though it
were enclosed in quotes).  \nr{} arithmetic would then be used if it
were involved in an arithmetic operation.
\end{enumerate}
 
These rules can apply in assignment instructions, the initial assignment
to the control variable in the \keyword{loop} instruction, or the
assignment of a default value to the argument of a method; the result
type may define the type of the variable (if new, or a method argument).
\item[Control variables in loops]
\index{Binary classes,control variables}
\index{Binary classes,LOOP instruction}
\index{Binary methods,control variables}
\index{Binary methods,LOOP instruction}
\index{Loops,in binary classes and methods}
 
In the \keyword{loop} instruction, if the control variable has a
primitive integer type, and the increment (\keyword{by} value) has a
primitive integer type, then binary arithmetic will be used for stepping
the control variable, following the rules for binary arithmetic in
expressions described above.
 
Similarly, if the control variable has a primitive integer type, and the
end (\keyword{to}) value has a primitive integer type, then binary
arithmetic will be used for the comparison that tests for loop
termination.
\item[Numeric instruction]
\index{Binary classes,NUMERIC instruction}
\index{Binary methods,NUMERIC instruction}
\index{NUMERIC,in binary classes and methods}
 
The \keyword{numeric} instruction does not affect binary operations.
It has the usual effects on operations carried out using \nr{}
arithmetic.
\end{description}
\textbf{Note: }
At all times (whether in binary classes, binary methods, or anywhere
else) implementations may use primitive types and operations, and
techniques such as late binding of types, as an optimization
providing that the results obtained are identical to those defined in
this language definition.
\section{Binary constructors}\label{refbincon}
\index{Binary constructors,}
\index{Constructors,binary}
\index{Conversion,binary constructors}
\index{Conversion,of characters}
\index{Character,encodings}
\index{Character,converting to binary}
\index{Encodings,binary}
 
\nr{} provides special constructors for implementation-defined
primitive types that allow bit-wise construction of primitives.
These \emph{binary constructors} are especially useful for
manipulating the binary encodings of individual characters.
 
The binary constructors follow the same syntax as other constructors,
with the name being that of a primitive type.  All binary constructors
take one argument, which must have a primitive type.
 
The bits of the value of the argument are extended or truncated on the
left to the same length as the bits required for the type of the
constructor (following the usual binary rules of sign extension if the
argument type is a signed numeric type), and a value with the type of
the constructor is then constructed directly from those bits and
returned.

\textbf{Example:}
 This example illustrates types from the reference implementation,
with 32-bit signed integers of type \textbf{int} and 16-bit Unicode
characters of type \textbf{char}.
\begin{alltt}
i=int 77   -- i is now the integer 77
c=char(i)  -- c is now the character 'M'
j=int(c)   -- j is now the integer 77
\end{alltt}
Note that the conversion
\begin{alltt}
j=int c
\end{alltt}
would have failed, as "\textbf{M}" is not a number.
