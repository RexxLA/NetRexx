% \section{Stream()}
\label{refrexxstream}
\index{netrexx.lang,RexxStream class}

% The RexxStream class implements the Classic Rexx inspired I/O
% functions. Streams are identified by character string identifiers and
% provide for the reading and writing of data. A \keyword{uses RexxStream} statement is
% automatically included due to the default commandline option \code{-implicituses}. All methods are \keyword{static}.

\section{charin(name [,start [,length{]}{]})}
\index{CHARIN method,}
\index{Method, built-in,CHARIN}
returns a string of up to \emph{length} characters from
the character input stream  \emph{name}. A \emph{start} value may be given to specify
an explicit read position. This read position must be positive and within
the bounds of the stream. A value of 1 for \emph{start} refers to the first character in the stream.
If \emph{length} is not specified, one character is read.

A read position, based on UTF-8 characters, is kept independently of the write position.
Charin() and linein() start at the same read position.

A charin() operation returns the character(s) read, or ''. Stream state is 'READY' when successful.

\section{charout(name [,string [,start{]}{]})}
\index{CHAROUT method,}
\index{Method, built-in,CHAROUT}
returns the count of characters remaining after attempting to
write \emph{string} to the character output stream \emph{name}. A \emph{start} value may be given
to specify an explicit write position. This write position must be positive and within the
bounds of the stream. A value of 1 for start refers to the first character in the stream.

When \emph{string} is omitted, character output stream \emph{name} is closed.

A write position, based on UTF-8 characters, is kept independently of the read position.
Charout() and lineout() start at the same write position.

The charout() operation returns 0 for success, non-zero if unsuccessful. Stream state is 'READY' when successful.

\section{chars(name)}
\index{CHARS method,}
\index{Method, built-in,CHARS}
indicates whether there are characters remaining in the character input stream
\emph{name}.

It returns 1 if at least one character can be read from \emph{name},
otherwise 0 is returned.

\section{linein(name [,line [,count {]}{]})}
\index{LINEIN method,}
\index{Method, built-in , LINEIN}
returns \emph{count} (0 or 1) lines read from the character input stream \emph{name}.
A \emph{line} number may be given to set the read position to the start of a specified line. This
line number must be positive and within the bounds of the stream. A value of 1 for line refers
to the first line in the stream. A call to linein() will return a partial line
if part of the line has already been read with charin().

A read position, based on UTF-8 characters, is kept independently of the write position.
Charin() and linein() start at the same read position.

A linein() operation returns the line read, or ''. Stream state is 'READY' when successful.


\section{lineout(name [,string, [,line{]}{]})}
\index{LINEOUT method,}
\index{Method, built-in , LINEOUT}
returns the count of lines remaining after attempting to write \emph{string} as
a line to the character output stream \emph{name}. The count will be either 0
(meaning the line was successfully written) or 1 (meaning that an error
occurred while writing the line).
A \emph{line} number may be given to set the write position to the start of a
particular line in stream \emph{name}. This \emph{line} number must be positive and
within the bounds of the stream. A value of 1 for line
refers to the first line in the stream.

When \emph{string} is omitted, character output stream \emph{name} is closed.

A write position, based on UTF-8 characters, is kept independently of the read position.
Charout() and lineout() start at the same write position.

The lineout() operation returns 0 for success, non-zero if unsuccessful. Stream state is 'READY' when successful.

\section{lines(name)}
\index{LINES method,}
\index{Method, built-in , LINES}
returns 1 if any data remains between the current read position and the end of the character input
stream \emph{name}. It returns 0 if no data remains.

\section{stream(name [,operation [,stream\_command{]]})}
\index{STREAM method,}
\index{Method, built-in,STREAM}
returns a string describing the state of, or the result of an \emph{operation}
upon the character stream \emph{name}.
\emph{Operation} is one of the following:
\begin{description}
  \item[S] State
  \item[D] Description
  \item[C] Command
\end{description}
If \emph{operation} is not specified, it defaults to \emph{State}.

Requesting the state or description of a character stream returns one of the following:
\begin{description}
  \item[READY] When a read or write operation is likely to succeed
  \item[NOTREADY] When the stream is in a condition that a read or write is not possible, for example when
  a seek operation is out of bounds
  \item[ERROR] When a stream is in error
  \item[UNKNOWN] When the state of the stream is unknown
\end{description}
Issuing \emph{operation} C(ommand) must be followed by a \emph{stream\_command} which is one of the following:
\begin{description}
  \item[OPEN [READ|WRITE|BOTH|APPEND{]}] Opens the stream in read, write mode or both (which is default), or in write append mode.
  \item[CLOSE] Closes the stream
  \item[SEEK <offset> [READ|WRITE{]}] Sets the read, write (or both) position into the stream
  \item[QUERY EXISTS] Returns the fully qualified filename of the stream if it exists, or the empty string ''
  \item[QUERY SIZE] Returns the size of the filename represented by the stream \emph{name}
  \item[QUERY DATETIME] Returns the date and time of last modification of the stream, in format 'DD-MM-YY HH:MM:SS UTC'
  \item[QUERY TIMESTAMP] Returns the date and time of last modification of the stream, in format 'YYYY-MM-DD HH:MM:SS UTC'
\end{description}

\section{Stream operations}
Streams (files) are implicitly opened, in read mode for charin/linein, in write mode for charout/lineout,
and always in \textbf{UTF-8}. UTF-8 is compatible with ASCII, operating systems mostly default to UTF-8,
and Java defaults to it since JDK18.

Reading starts at the beginning of the file, writing starts at the end of the file, unless a start position is given.

Reading and writing positions are kept independently, and are based on UTF-8 characters.

Repositioning can be costly. Current line and current char positions are kept when
possible (when starting from begin of file for instance),
 but to position a read or write file pointer to a specific line or character
number, it might be necessary to read a significant portion of the file.

When positioning write pointers, characters and lines can be
re-written, and may overwrite following characters if the line(s) or
character(s) are larger than what was there before. Note, UTF-8 characters can be represented by 1 to 4 bytes.

Stream(name, 'C', 'SEEK offset') operations are byte based, not UTF-8 character based, so it is possible
with this operation to position the file pointer not at the start of a multi-byte UTF-8 character.


\subsection{Examples}
\index{examples,Stream}
\begin{lstlisting}[label=datessexample,caption=Example of using Date()]

rc = lineout('testdata.txt', 'This is line 1')
rc = lineout('testdata.txt', 'This is line 2')
rc = lineout('testdata.txt', 'This is line 3')
rc = stream('testdata.txt', 'C', 'CLOSE')
rc = lineout('testdata.txt', 'This is line 4')

say charin('testdata.txt', 1, 5)   --  > "This "
say linein('testdata.txt')         --  > "is line 1"
say linein('testdata.txt', 4)      --  > "This is line 4"

rc = stream('testdata.txt', 'C', 'SEEK 0')

loop while lines('testdata.txt')
  say linein('testdata.txt')       -- > shows lines 1 to 4
end

say stream('testdata.txt','c','QUERY EXISTS')  -- > displays fully qualified file name
say stream('testdata.txt','c','QUERY SIZE')    -- > displays 60

rc = stream('testdata.txt', 'C', 'CLOSE')

rc  = charout('testdata.txt', 'Overwrite line 1, join line 2', 1)
rc = stream('testdata.txt', 'C', 'CLOSE')

loop while chars('testdata.txt')
  say charin('testdata.txt')'\0'   -- > shows lines 1 to 3
end

rc = stream('testdata.txt', 'C', 'SEEK 0')

loop for 20
  rc = charout('testdata.txt','€')
end
rc = charout('testdata.txt','\n')
rc = stream('testdata.txt','c','CLOSE')

-- use the chars() function to loop until EOF
loop while chars('testdata.txt') > 0
  say charin('testdata.txt')'\0'   -- > shows 20 euro signs
end

say stream('testdata.txt','c','QUERY SIZE')    -- > displays 61

-- display the last modified date of the last file
say stream('testdata.dat','c','QUERY DATETIME')  -- displays timestamp as DD-MM-YY HH:MM:SS UTC
-- in the post-2000 era
say stream('testdata.dat','c','QUERY TIMESTAMP') -- displays timestamp as YYYY-MM-DD HH:MM:SS UTC

\end{lstlisting}
