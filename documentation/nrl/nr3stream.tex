\section{The R\textsc{exx}Stream class}\label{refrexxstream}
\index{netrexx.lang,R\textsc{exx}Stream class}

The RexxStream class implements the Classic Rexx inspired I/O
functions. Streams are identified by character string identifiers and
provide for the reading and writing of data. In scripting mode (when
the class definition is generated) a \keyword{uses RexxStream} statement is
automatically included. All methods are \keyword{static}.

\begin{description}
  
\item[charin(name, [start{]}, [length{]})] returns a string read from the stream named by the first
argument.
  \index{charin,Stream}
\item[charout(name,[char{]},[start{]})] returns the count of characters remaining after attempting to
write the second argument to the stream named by the first
argument. For \emph{start}, only 1 is currently implemented (which
rewrites the file, where the default is to append to an existing file). 
  \index{charout,Stream}
\item[chars(name)] indicates whether there are characters remaining in the named
stream. Optionally, it returns a count of the characters remaining and
immediately available.
  \index{chars,Stream}
\item[linein(name,string)] reads a line from the stream named by the first argument,
unless the third argument is zero.
  \index{linein,Stream}
\item[lineout([name{]},[string{]},[line{]})] returns '1' or '0', indicating whether the
  second argument has been successfully written to the stream named by
  the first argument. A result of '1' means an unsuccessful write. The
  line argument, when 1, causes the stream to be rewritten from the
  start (line 1), where the default is to append to an existing stream. 
  \index{lineout,Stream}
\item[lines(name)] returns the number of lines remaining in the named stream.

\item[stream(name, [operation{]}, [stream\_command{]})]   \index{stream,Stream} (Operations) returns a description of the state of, or the result of an
operation upon, the stream named by the first argument. The first
argument is a stream name, the second is either
\begin{description}
\item[C(ommand)]
\item[D(escription]
\item[S(tate)]
\end{description}

When used with the S(tate) option, STREAM
returns one of the following strings:
\begin{description}
\item[ERROR]
\item[NOTREADY]
\item[READY]
\item[UNKNOWN]
\end{description}

Commands are expressions that evaluate to the following command
strings:
\begin{description}
  \item['OPEN'] opens the named stream. The default for 'OPEN' is to
    open the stream for both reading and writing data. The STREAM function itself
    returns 'READY' is succesfully opened.
  \item['CLOSE'] closes the named stream. The STREAM function itself
    returns 'READY' is succesfully closed.
  \item['QUERY EXISTS'] returns the full path specification of the
    named stream, if it exists, or a null string otherwise.
  \item['QUERY SIZE'] returns the size in bytes of a persistent
    stream.
  \item['QUERY DATETIME'] returns the date and time stamps of a
    persistent stream. This has a 2-digit year format.
  \item['QUERY TIMESTAMP'] returns the date and time stamps of a
    persistent stream in ISO-format. (This has a 4-digit year format).
  \end{description}
\end{description}

The methods \keyword{charin} and \keyword{charout} work on UTF
characters, in practice this means that what is read with
\keyword{charin}, needs to be written with \keyword{charout}.


\subsection{Examples}
\index{examples,Stream}
\begin{lstlisting}[label=datessexample,caption=Example of using Date()]
-- write two lines to the file testdata.txt
lineout('testdata.txt','the first quick brown fox')
lineout('testdata.txt','jumps over the first lazy dog')

-- write two lines to the file testdata2.txt
lineout('testdata2.txt','the second quick brown fox')
lineout('testdata2.txt','jumps over the second lazy dog')

-- close the file
stream('testdata.txt','c','CLOSE')

-- show the lines() function - loop until eof
loop i=1 while lines('testdata.txt') > 0
  say linein('testdata.txt') i
end

-- see if it exists. Returns the full path
say stream('testdata.txt','c','QUERY EXISTS')
-- query its size. Should be 56 bytes
say stream('testdata.txt','c','QUERY SIZE')

-- display lines from different files 
say linein('testdata.txt')
say linein('testdata2.txt')
say linein('testdata.txt')

-- show the charout function, which outputs UTF
loop for 15
  charout('testdata.dat','a')
end

-- read back these 15 characters
loop for 15
  say charin('testdata.dat') 'from loop 15'
end

-- close the file
stream('testdata.dat','c','CLOSE')

-- use the chars() function to loop until EOF
loop i=i while chars('testdata.dat') > 0
  say charin('testdata.dat') 'from chars' i
end

-- display the last modified date of the last file
say stream('testdata.dat','c','QUERY DATETIME')
-- in the post-2000 era
say stream('testdata.dat','c','QUERY TIMESTAMP')

\end{lstlisting}
