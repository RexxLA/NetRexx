\subsection{Binary types and conversions}\label{refobinar}
\index{Overview,binary types}
\index{Overview,conversions}
\index{Binary numbers,overview}
\index{Conversion,overview}
:p.
Most programming environments support the notion of fixed-precision
"primitive" binary types, which correspond closely to the binary
operations usually available at the hardware level in computers.
For the reference implementation, these types are:
:ul.
:li.
\emph{byte, short, int,} and \emph{long} - signed integers
that will fit in 8, 16, 32, or 64 bits respectively
:li.
\emph{float} and \emph{double} - signed floating point
numbers that will fit in 32 or 64 bits respectively.
:li.
\emph{char} - an unsigned 16-bit quantity, holding a Unicode
character
:li.
\emph{boolean} - a 1-bit logical value,
representing :m.0:em. or :m.1:em. ("false" or "true").
:eul.
:p.
Objects of these types are handled specially by the implementation
"under the covers" in order to achieve maximum efficiency; in
particular, they cannot be constructed like other objects - their
value is held directly.  This distinction rarely matters to the NetRexx
programmer: in the case of string literals an object is
constructed automatically; in the case of an :m.int:em. literal, an
object is not constructed.
:p.
Further, NetRexx automatically allows the conversion between the various
forms of character strings in implementations
:fn.
In the reference implementation, these
are :m.String:em., :m.char:em., :m.char[]:em. (an
array of characters), and the NetRexx string type, :m.Rexx:em..
:efn.
and the primitive types.  The "golden rule" that is followed by
NetRexx is that any automatic conversion which is applied must not lose
information: either it can be determined before execution that the
conversion is safe (as in :m.int:em. to :m.String:em.) or it
will be detected at execution time if the conversion fails (as
in :m.String:em. to :m.int:em.).
:p.
The automatic conversions greatly simplify the writing of programs;
the exact type of numeric and string-like method arguments rarely needs
to be a concern of the programmer.
:p.
For certain applications where early checking or performance override
other considerations, the reference implementation of NetRexx provides
options for different treatment of the primitive types:
:ol.
:li.
\texttt{options strictassign} - ensures exact type matching for all
assignments.  No conversions (including those from shorter integers to
longer ones) are applied.  This option provides stricter type-checking
than most other languages, and ensures that all types are an exact match.
:li.
\texttt{options binary} - uses implementation-dependent fixed
precision arithmetic on binary types (also, literal numbers, for
example, will be treated as binary, and local variables will be given
"native" types such as :m.int:em. or :m.String:em., where
possible).
:p.
Binary arithmetic currently gives better performance than NetRexx
decimal arithmetic, but places the burden of avoiding overflows and
loss of information on the programmer.
:eol.
:p.
The options instruction (which may list more than one option) is placed
before the first class instruction in a file; the \texttt{binary}
keyword may also be used on a \texttt{class} or \texttt{method}
instruction, to allow an individual class or method to use binary
arithmetic.
:h4.Explicit type assignment
:p.
You may explicitly assign a type to an expression or variable:
\index{Example,program}
\begin{verbatim}
i=int 3000000  -- 'i' is an 'int' with value 3000000
j=int 4000000  -- 'j' is an 'int' with value 4000000
k=int          -- 'k' is an 'int', with no initial value
say i*j        -- multiply and display the result
k=i*j          -- multiply and assign result to 'k'
\end{verbatim}
:p.
This example also illustrates an important difference between
\texttt{options nobinary} and \texttt{options binary}.  With the former
(the default) the \texttt{say} instruction would display the result
":m.1.20000000E+13:em." and a conversion overflow would be
reported when the same expression is assigned to the variable :m.k:em..
:p.
With \texttt{options binary}, binary arithmetic would be used for the
multiplications, and so no error would be detected; the \texttt{say}
would display ":m.-138625024:em." and the variable :m.k:em.
takes the incorrect result.
:h4.Binary types in practice
:p.
In practice, explicit type assignment is only occasionally needed in
NetRexx.  Those conversions that are necessary for using existing
classes (or those that use \texttt{options binary}) are generally
automatic.
For example, here is an "Applet" for use by Java-enabled
browsers:
\index{Example,applet}
\begin{verbatim}
/* A simple graphics Applet */
class Rainbow extends Applet
  method paint(g=Graphics)  -- called to repaint window
    maxx=size.width-1
    maxy=size.height-1
    loop y=0 to maxy
      col=Color.getHSBColor(y/maxy, 1, 1) -- new colour
      g.setColor(col)                     -- set it
      g.drawLine(0, y, maxx, y)           -- fill slice
    end y
\end{verbatim}
:p.
In this example, the variable :m.col:em. will have
type :m.Color:em., and the three arguments to the
method :m.getHSBColor:em. will all automatically be converted to
type :m.float:em..  As no overflows are possible in this example,
\texttt{options binary} may be added to the top of the program with no
other changes being necessary.
