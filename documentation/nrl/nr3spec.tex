\chapter{Special names and methods}\label{refspecial}
 
For convenience, \nr{} provides some \emph{special names} for naming
commonly-used concepts within terms.
These are only recognized if there is no variable of the same name
previously seen in the current scope, as described in the section on
 \emph{Terms} (see page \pageref{refterms}) .
This allows the set of special words to be expanded in the future, if
necessary, without invalidating existing variables.  Therefore, these
names are not reserved; they may be used as variable names instead, if
desired.
 
There are also two "special methods" that are used when
constructing objects.
\section{Special names}\label{specialnames}
\index{Special words,}
 
The following special names are allowed in \nr{} programs, and are
recognized independently of case.
\footnote{
\emph{Unless \keyword{options strictcase} is in effect.
}
}
With the exception of \textbf{length} and \textbf{class}, these
may only be used alone as a term or at the start of a compound term.
\begin{description}
\item[ask]\label{refswask}
\index{Special words,ask}
\index{Names,special/ask}
\index{Words,special/ask}
\index{ASK special word,}
\index{stdin, reading with ASK,}
 
Returns a string of type \textbf{R\textsc{exx}}, read as a line from the
implementation-defined default input stream (often the user's
"console").

\textbf{Example:}
\begin{lstlisting}
if ask='yes' then say 'OK'
\end{lstlisting}
 \textbf{ask} can only appear alone, or at the start of a
compound term.
\footnote{
\emph{In the reference implementation, \textbf{ask} is simply a shorthand
for \textbf{R\textsc{exx}IO.Ask()}.}
}
\item[class]\label{refswclass}
\index{Special words,class}
\index{Names, special,class}
\index{Words, special,class}
\index{CLASS,special word}
 
The object of type \textbf{Class} that describes a specific type.
This word is only recognized as the second part of a compound term,
where the evaluation of the first part of the term resulted in a
type or qualified type.

\textbf{Example:}
\begin{lstlisting}
obj=String.class
say obj.isInterface /* would say '0' */
\end{lstlisting}
\item[digits]\label{refswdigit}
\index{Special words,digits}
\index{Names,special/digits}
\index{Words,special/digits}
\index{DIGITS,special word}
 The current setting of  \keyword{numeric digits} (see page \pageref{refndigits}) ,
returned as a string of type \textbf{R\textsc{exx}}.
This will be one or more Arabic numerals, with no leading blanks, zeros,
or sign, and no trailing blanks or exponent.
 \textbf{digits} can only appear alone, or at the start of a
compound term.
\item[form]\label{refswform}
\index{Special words,form}
\index{Names,special/form}
\index{Words,special/form}
\index{FORM,special word}
 The current setting of  \keyword{numeric form} (see page \pageref{refnform}) ,
returned as a string of type \textbf{R\textsc{exx}}.
This will have either the value "\textbf{scientific}" or the
value "\textbf{engineering}".
 \textbf{form} can only appear alone, or at the start of a
compound term.
\item[length]\label{refswleng}
\index{Special words,length}
\index{Names,special/length}
\index{Words,special/length}
\index{LENGTH,special word}
 The length of an  array (see page \pageref{refarray}) , returned as an
implementation-dependent binary type or string.
This word is only recognized as the last part of a compound term,
where the evaluation of the rest of the term resulted in an array of
dimension 1.
 
\textbf{Example:}
\begin{lstlisting}
foo=char[7]
say foo.length     /* would say '7' */
\end{lstlisting}
 
Note that you can get the length of a \nr{} string with the
same syntax.
\footnote{\emph{Unless \keyword{options strictargs}} is in effect.}

In that case, however, a \textbf{length()} method is being invoked.
\item[null]\label{refswnull}
\index{Special words,null}
\index{Names,special/null}
\index{Words,special/null}
\index{NULL special word,}
\index{Empty reference, null,}
\index{References,null}
 
The \emph{empty reference}.  This is a special value that represents
"no value" and may be assigned to variables (or returned from
methods) except those whose type is both primitive and undimensioned.
It may also be be used in a comparison for equality (or inequality) with
values of suitable type, and may be given a type.
 \textbf{Examples:}
\begin{lstlisting}
blob=int[3]   -- 'blob' refers to array of 3 ints
blob=null     -- 'blob' is still of type int[],
              -- but refers to no real object
mob=Mark null -- 'mob' is type 'Mark'
\end{lstlisting}
 The \textbf{null} value may be considered to represent the state of
being uninitialized.  It can only appear as simple symbol, not as a part
of a compound term.

\item[RC]\label{refswrc}
\index{Special words,RC}
\index{Names,special/RC}
\index{Words,special/RC}
\index{RC special word,}
 
The special variable named \emph{RC} holds the returncode of the last executed command sent
to the \emph{address} environment.
When the environment was never addressed, \emph{RC} holds the value 'RC'. 
\item[source]\label{refswsourc}
\index{Special words,source}
\index{Names,special/source}
\index{Words,special/source}
\index{SOURCE special word,}
\index{Program,filename of}
\index{Class,filename of}
 
Returns a string of type \textbf{R\textsc{exx}} identifying the source of the
current class.
The string consists of the following words, with a single blank between
the words and no trailing or leading blanks:
\begin{enumerate}
\item the name of the underlying environment (\emph{e.g.}, \textbf{Java})
\item either \textbf{method} (if the term is being used within a method)
or \textbf{class} (if the term is being used within a property
assignment, before the first method in a class)
\item 
an implementation-dependent representation of the name of the
source stream for the class (\emph{e.g.}, \textbf{Fred.nrx}).
\end{enumerate}
 \textbf{source} can only appear alone, or at the start of a
compound term.
\item[sourceline]\label{refswsourl}
\index{Special words,sourceline}
\index{Names, special,sourceline}
\index{Words, special,sourceline}
\index{SOURCELINE,special word}
 
The line number of the first token of the current clause in the
\nr{} program, returned as a string of type \textbf{R\textsc{exx}}.
This will be one or more Arabic numerals, with no leading blanks, zeros,
or sign, and no trailing blanks or exponent.
 \textbf{sourceline} can only appear alone, or at the start of a
compound term.
\item[super]\label{refswsuper}
\index{Special words,super}
\index{Names,special/super}
\index{Words,special/super}
\index{SUPER,special word}
\index{References,to current object}
 
Returns a reference to the current object, with a type that is the
type of the class that the current object's class extends.
This means that a search for methods or properties
which \textbf{super} qualifies will start from the superclass rather
than in the current class.
This is used for invoking a method or property (in the superclass or one
of its superclasses) that has been overridden in the current class.
 \textbf{Example:}
\begin{lstlisting}
method printit(x)
  say 'it'          -- modification
  super.printit(x)  -- now the usual processing
\end{lstlisting}
 
If a property being referenced is in fact defined by a superclass of
the current class, then the prefix "\textbf{super.}" is perhaps
the clearest way to indicate that name refers to a property of a
superclass rather than to a local variable.
(You could also qualify it by the name of the superclass.)
 \textbf{super} can only appear alone, or at the start of a
compound term.
\item[this]\label{refswthis}
\index{Special words,this}
\index{Names,special/this}
\index{Words,special/this}
\index{THIS,special word}
\index{References,to current object}
 
Returns a reference to the current object.
When a method is invoked, for example in:
\begin{lstlisting}
word=R\textsc{exx} "hello"  -- 'word' refers to "hello"
say word.substr(3)                -- invokes substr on "hello"
\end{lstlisting}
then the method \textbf{substr} in the class \textbf{R\textsc{exx}} is
invoked, with argument \textbf{'3'}, and with the properties of the
value (object) \textbf{"hello"} available to it.
These properties may be accessed simply by name, or (more explicitly) by
prefixing the name with "\textbf{this.}".
Using "\textbf{this.}" can make a method more readable,
especially when several objects of the same type are being manipulated
in the method.
 \textbf{this} can only appear alone, or at the start of a
compound term.
\item[trace]\label{refswtrace}
\index{Special words,trace}
\index{Names,special/trace}
\index{Words,special/trace}
\index{TRACE,special word}

The current  \keyword{trace} (see page \pageref{reftrace})  setting,
returned as a \nr{} string.
This will be one of the words:
\begin{lstlisting}
off var methods all results
\end{lstlisting}

(\texttt{var} is returned when clause tracing is off but variable
tracing has then been turned on using a \keyword{trace var} instruction.)
 \textbf{trace} can only appear alone, or at the start of a
compound term.
\item[version]\label{refswvers}
\index{Special words,version}
\index{Names,special/version}
\index{Words,special/version}
\index{VERSION special word,}
 
Returns a string of type \textbf{R\textsc{exx}} identifying the version of the
\nr{} language in effect when the current class was processed.
The string consists of the following words, with a single blank between
the words and no trailing or leading blanks:
\begin{enumerate}
\item A word describing the language.  The first seven letters will be the
characters \textbf{\nr{}}, and the remainder may be used to identify
a particular implementation or language processor.
This word may not include any periods.
\item 
The language level description, which must be a number with no sign or
exponential part.
For example, "\textbf{\splice{java org.netrexx.process.NrVersion}}" is the language level of this
definition.
\item 
Three words describing the language processor release date in
the same format as the default for the R\textsc{exx} "\textbf{date()}"
function.
\footnote{
As defined in \textbf{American National Standard for Information
Technology - Programming Language REXX, X3.274-1996:}, American
National Standards Institute, New York, 1996. See also
\keyword{Date()} on page \pageref{refrexxdate}.
}
For example, "\textbf{22 May 2009}".
\end{enumerate}
 \textbf{version} can only appear alone, or at the start of a
compound term.
\end{description}
\section{Special methods}\label{refspecm}
\index{Special methods,}
\index{Methods,special}
\index{Constructors,special}
 
Constructors (methods used for constructing objects) in \nr{}
must invoke a constructor of their superclass before making any
modifications to the current object (or invoke another constructor in
the current class).
 
\index{SUPER,special method}
\index{THIS,special method}
\index{Special methods,this}
\index{Special methods,super}
This is simplified and made explicit by the provision of the special
method names \textbf{super} and \textbf{this}, which refer to
constructors of the superclass and current class respectively.  These
special methods are only recognized when used as the first, method call,
instruction in a constructor, as described in
 \emph{Methods and constructors} (see page \pageref{refmethcon}) .
Their names will be recognized independently of case.
\footnote{
\emph{Unless \keyword{options strictcase} is in effect.
}
}
 
In addition, \nr{} provides special constructor methods for the
primitive types that allow binary construction of primitives.
These are described in  \emph{Binary values and arithmetic} (see page \pageref{refbincon}).
