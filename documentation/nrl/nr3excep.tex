\chapter{Exceptions}\label{refexcep}
\index{Exceptions,}
\index{Signals,}
\index{Caught exceptions,}
 
Exceptional conditions, including errors, in \nr{} are handled by a
mechanism called \emph{Exceptions}.
When an exceptional condition occurs, a \emph{signal} takes place
which may optionally be \emph{caught} by an enclosing control
construct, as detailed below.
 
An exception can be signalled by:
\begin{enumerate}
\item the program's environment, when some processing error occurs (such
as running out of memory, or a problem discovered when reading or
writing a file)
\item a method called by a \nr{} program (if, for example, it is passed
incorrect arguments)
\item the  \texttt{signal} instruction (see page \pageref{refsignal}) .
\end{enumerate}
In all cases, the signal is handled in exactly the same way.
First, execution of the current clause ceases; no further operations
within the clause will be carried out.
\footnote{
This is the only case in which an expression will not be wholly
evaluated, for example.
}
Next, an object that represents the exception is constructed.  The type
of the exception object is implementation-dependent, as described for
the  \texttt{signal} instruction (see page \pageref{refsignal}) , and defines the
type of the exception.  The object constructed usually contains
information about the Exception (such as a descriptive string).
 
Once the object has been constructed, all active \texttt{do} groups,
\texttt{loop} loops, \texttt{if} constructs, and \texttt{select}
constructs in the active method are "unwound", starting with the
innermost, until the exception is caught by a control construct that
specifies a suitable \texttt{catch} clause (see below) for handling the
exception.
 
This unwinding takes place as follows:
\begin{enumerate}
\item No further clauses within the body of the construct will be executed
(in this respect, the signal acts like a \texttt{leave} for the
construct).
\item If a \texttt{catch} clause specifies a type to which
the exception object can be assigned (that is, it matches or is a
superclass of the type of exception object), then the
\emph{instructionlist} following that clause is executed, and the
exception is considered to be handled (no further control constructs
will be unwound).
If more than one \texttt{catch} clause specifies a suitable type, the
first is used.
\item The \emph{instructionlist} following the \texttt{finally} clause
for the construct, if any, is executed.
\item The \texttt{end} clause is executed, hence completing execution of
the construct.
(The only effect of this is that it is seen when tracing.)
\item 
If the exception was handled, then execution resumes as though the
construct completed normally.  If it was not handled, then the process
is repeated for any enclosing constructs.
\end{enumerate}
 
If the exception is not caught by any of the control constructs
enclosing the original point of the exception signal, then
the current active method is terminated, without returning any data, and
the exception is then signalled at the point where the method was
invoked (that is, in the caller).
 
The process of unwinding control constructs and terminating the method
is then repeated in each calling method until the exception is caught or
the initial program invocation method (the main method) is terminated,
in which case the program ends and the environment receives the signal
(it would usually then display diagnostic information).
\subsection{Syntax and example}
\index{CATCH,use of}
\index{FINALLY,use of}
 
The constructs that may be used to handle (catch) an exception are
\texttt{do} groups, \texttt{loop} loops, and \texttt{select} constructs.
Specifically, as shown in the syntax diagrams (\emph{q.v.}), where the
\texttt{end} clause can appear in these constructs, zero or more
\texttt{catch} clauses can be used to define exception handlers,
followed by zero or one \texttt{finally} clauses that describe
"clean-up" code for the construct.
The whole construct continues to be ended by an \texttt{end} clause.
 
The syntax of a \texttt{catch} clause is shown in the syntax diagrams.
It always specifies an \emph{exception} type, which may be
qualified.  It may optionally specify a symbol, \emph{vare}, which
is followed by an equals sign.  This indicates that when the exception
is caught then the object representing the exception will be assigned to
the variable \emph{vare}.  If new, the type of the variable will be
\emph{exception}.
 Here is an example of a program that handles some of the exceptions
signalled by methods in the \textbf{R\textsc{exx}} class; the \texttt{trace
results} instruction is included to show the flow of execution:
\index{Example,of exception handling}
\begin{alltt}
trace results
do                -- could be LOOP i=1 to 10, etc.
  say 1/arg
catch DivideException
  say 'Divide exception'
catch ex=NumberFormatException
  /* 'ex' is assigned the exception object */
  say 'Bad number for division:' ex.getMessage
finally
  say 'Done!'
end
\end{alltt}
In this example, if the argument passed to the program (and hence
placed in \textbf{arg}) is a valid number, then its inverse is
displayed.  If the argument is 0, then "\textbf{Divide
exception}" would be displayed.  If the argument were an invalid
number, the message describing the bad number would be displayed.
For any other exception (such as an \textbf{ExponentOverflowException}),
the program would end and the environment would normally report the
exception.
 
In \textbf{all} cases, the message "\textbf{Done!}" would be
displayed; this would be true even if the body of the \texttt{do}
construct executed a \texttt{return}, \texttt{leave}, or
\texttt{iterate} instruction.  Only an  \texttt{exit} (see page \pageref{refexit}) 
instruction:ea. would cause immediate termination of the construct (and
the program).
\textbf{Note: }The \texttt{finally} keyword, like \texttt{otherwise} in the
\texttt{select} construct, implies a semicolon after it, so the last
\texttt{say} instruction in the example could have appeared on the same
line as the \texttt{finally} without an intervening semicolon.
\subsection{Exceptions after catch and finally clauses}
\index{Exceptions,after CATCH clause}
\index{Exceptions,after FINALLY clause}
 
If an exception is signalled in the \emph{instructionlist} following
a \texttt{catch} or \texttt{finally} clause, then the current exception
is considered handled, the \emph{instructionlist} is terminated, and
the new exception is signalled.  It will not be caught by \texttt{catch}
clauses in the current construct.  If it occurs in the
\emph{instructionlist} following a \texttt{catch} clause, then any
\texttt{finally} instructions will be executed, as usual.
 
Similarly, executing a \texttt{return}
or \texttt{exit} instruction within either of the
\emph{instructionlists} completes the handling of any pending
signal.
\subsection{Checked exceptions}\label{refchecked}
\index{Exceptions,checked}
\index{Checked exceptions,}
 
\nr{} implementations may define certain exceptions as \emph{checked
exceptions}.  These are exceptions that the implementation considers
it useful to check; the checked exceptions that each method may signal
are recorded.  Within \texttt{do} groups, \texttt{loop} loops, and
\texttt{select} constructs, for example, it is then possible to report
if a \texttt{catch} clause tries to catch a checked exception that is
not signalled within the body of the construct.
 
Checked exceptions that are signalled within a method (by a
\texttt{signal} instruction or a method invocation) but not caught by a
\texttt{catch} clause in the method are automatically added to the
\texttt{signals} list for a method.  Implementations that support
checked exceptions are encouraged to provide options that list the
uncaught checked exceptions for methods or enforce the explicit
inclusion of some or all checked exceptions in the \texttt{signals} list
on the method instruction.
 
\emph{In the reference implementation, all exceptions are checked except
those that are subclasses of \textbf{java.lang.RuntimeException}
or \textbf{java.lang.Error}.  These latter are considered so
ubiquitous that almost all methods would signal them.}
 
\emph{Expressions assigned as the initial values of properties must not invoke
methods that may signal checked exceptions.}
 
\emph{The \texttt{strictsignal} option on the \texttt{options} instruction may
be used to enforce the inclusion of all uncaught checked exceptions in
methods' \texttt{signals} lists; this may be used to assure that
any uncaught checked exceptions are intentional.}
