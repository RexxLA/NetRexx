\chapter{Methods and Constructors}\label{refmethcon}
\index{Methods,}
\index{Methods,invocation of}
\index{Parentheses,in method calls}
\index{Comma,in method calls}
\index{,}
 
Instructions in \nr{} are grouped into \emph{methods}, which are
named routines that always belong to (are part of) a \emph{class}.
 
\index{References,to methods}
Methods are invoked by being referenced in a  term (see page \pageref{refterms}) ,
which may be part of an expression or be a clause in its own right (a
method call instruction).
In either case, the syntax used for a method invocation is:
\begin{alltt}
\emph{symbol}([\emph{expression}[,\emph{expression}]...])
\end{alltt}
\index{Arguments,of methods}
\index{Arguments,passing to methods}
 The \emph{symbol}, which must be non-numeric, is called the
\emph{name} of the method.
It is important to note that the name of the method must be followed
immediately by the "\texttt{(}", with \textbf{no} blank in
between, or the construct will not be recognized as a method call
(a \emph{blank operator} would be assumed at that point instead).
 The \emph{expression}s (separated by commas) between the
parentheses are called the \emph{arguments} to the method.
Each argument expression may include further method calls.
 The argument expressions are evaluated in turn from left to right
and the resulting values are then passed to the method (the procedure
for locating the method is described below).
The method then executes some algorithm (usually dependent on any
arguments passed, though arguments are not mandatory) and will
eventually return a value.
This value is then included in the original expression just as though
the entire method reference had been replaced by the name of a variable
whose value is that returned data.
 
For example, the \textbf{substr} method is provided for strings of
type \textbf{R\textsc{exx}} and could be used as:
\begin{alltt}
c='abcdefghijk'
a=c.substr(3,7)
/* would set A to "cdefghi" */
\end{alltt}
Here, the value of the variable \textbf{c} is a string (of
type \textbf{R\textsc{exx}}).
The \textbf{substr} (substring) method of the \textbf{R\textsc{exx}} class is
then invoked, with arguments \textbf{3} and \textbf{7}, on the value
referred to by \textbf{c}.
That is, the the properties available to (the context of)
the \textbf{substr} method are the properties constructed from the
literal string \textbf{'abcdefghijk'}.
The method returns the substring of the value, starting at the third
character and of length seven characters.
\index{Arguments,passing to methods}
 
A method may have a variable number of arguments: only those
required need be specified.
For example, \textbf{'ABCDEF'.substr(4)} would return the
string \textbf{'DEF'}, as the \textbf{substr} method will assume
that the remainder of the string is to be returned if no length is
provided.
 
Method invocations that take no arguments may omit the (empty)
parentheses in circumstances where this would not be ambiguous.
See the section on  \emph{Terms} (see page \pageref{refterms})  for details.
 
\textbf{Implementation minimum:} At least 10 argument expressions
should be allowed in a method call.
\section{Method call instructions}\label{refmcalli}
\index{Method call instructions,}
\index{Subroutines,calling}
 
When a clause in a method consists of just a term, and the final part of
the term is a method invocation, the clause is a \emph{method call
instruction}:
\begin{shaded}
\begin{alltt}
\emph{symbol}([\emph{expression}[,\emph{expression}]...]);
\end{alltt}
\end{shaded}
The method is being called as a subroutine of the current method, and
any returned value is discarded.
In this case (and in this case only), the method invoked need not return
a value (that is, the \texttt{return} instruction that ends it need not
specify an expression).
\footnote{
A method call instruction is equivalent to the \texttt{call} instruction
of other languages, except that no keyword is required.
}
 
A method call instruction that is the first instruction in a constructor
(see below) can only invoke the special constructors \textbf{this}
and \textbf{super}.
\section{Method resolution (search order)}\label{refsmeth}
\index{Resolution of methods,}
\index{Search order,for methods}
\index{Methods,resolution of}
 
Method resolution in \nr{} proceeds as follows:
\begin{itemize}
\item 
If the method invocation is the first part (stub) of a term, then:
\begin{enumerate}
\item 
The current class is searched for the method (see below for details of
searching).
\item If not found in the current class, then the superclasses of the
current class are searched, starting with the class that the current
class extends.
\item 
If still not found, then the classes listed in the \texttt{uses} phrase
of the \texttt{class} instruction are searched for the method, which in
this case must be a  static method (see page \pageref{refstatmet}) .
Each class from the list is searched for the method, and then its
superclasses are searched upwards from the class; this process is
repeated for each of the classes, in the order specified in the list.
\item 
If still not found, the method invocation must be a constructor (see
below) and so the method name, which may be qualified by a package name,
should match the name of a primitive type or a known class (type).
The specified class is then searched for a constructor that matches the
method invocation.
\end{enumerate}
\item 
If the method invocation is not the first part of the term, then the
evaluation of the parts of the term to the left of the method invocation
will have resulted in a value (or just a type), which will have a known
type (the continuation type).
Then:
\begin{enumerate}
\item 
The class that defines the continuation type is searched for the method
(see below for details of searching).
\item If not found in that class, then the superclasses of that class are
searched, starting with the class that that class extends.
\end{enumerate}
 If the search did not find a method, an error is reported.
 
If the search did find a method, that is the method which is invoked,
except in one case:
\begin{itemize}
\item If the evaluation so far has resulted in a value (an object), then
that value may have a type which is a subclass of the continuation type.
If, within that subclass, there is a method that
 exactly overrides (see page \pageref{refoverrid})  the method that was found in the
search, then the method in the subclass is invoked.
\end{itemize}

This case occurs when an object is earlier assigned to a variable of a
type which is a superclass of the type of the object.  This type
simplification hides the real type of the object from the language
processor, though it can be determined when the program is executed.
\end{itemize}
 
\index{Methods,searching for}
\index{Matching methods,}
Searching for a method in a class proceeds as follows:
\begin{enumerate}
\item 
Candidate methods in the class are selected.
To be a candidate method:
\begin{itemize}
\item 
the method must have the same name as the method invocation (independent
of the  case (see page \pageref{refcase})  of the letters of the name)
\item 
the method must have the same number of arguments as the method
invocation (or more arguments, provided that the remainder are shown as
optional in the method definition)
\item 
it must be possible to assign the result of each argument expression to
the type of the corresponding argument in the method definition (if
strict type checking is in effect, the types must match exactly).
\end{itemize}
\item 
If there are no candidate methods then the search is complete; the
method was not found.
\item If there is just one candidate method, that method is used; the
search is complete.
\item 
If there is more than one candidate method, the sum of the
 costs of the conversions (see page \pageref{refcosts})  from the type of each
argument expression to the type of the corresponding argument defined
for the method is computed for each candidate method.
\item 
The costs of those candidates (if any) whose names match the method
invocation exactly, including in case, are compared; if one has a lower
cost than all others, that method is used and the search is complete.
\item 
The costs of all the candidates are compared; if one has a lower
cost than all others, that method is used and the search is complete.
\item 
If there remain two or more candidates with the same minimum cost, the
method invocation is ambiguous, and an error is reported.
\end{enumerate}
\textbf{Note: }When a method is found in a class, superclasses of that class are
not searched for methods, even though a lower-cost method may exist in a
superclass.
\section{Method overriding}\label{refoverrid}
\index{Methods,overriding}
\index{Overriding methods,}
 
A method is said to \emph{exactly override} a method in another class
if
\begin{enumerate}
\item the method in the other class has the same name as the current method
\item the method in the other class is not \texttt{private}
\item the other class is a superclass of the current class, or is a class
that the current class implements (or is a superclass of one of those
classes).
\item the number and type of the arguments of the method in the other
class exactly match the number and type of the arguments of the current
method (where subsets are also checked, if either method has optional
arguments).
\end{enumerate}
For example, the \textbf{R\textsc{exx}} class includes a
 \textbf{substr} (see page \pageref{refsubstr})  method, which takes from one to
three strings of type \textbf{R\textsc{exx}}.  In the class:
\begin{alltt}
class mystring extends R\textsc{exx}
  method substr(n=R\textsc{exx}, length=R\textsc{exx})
    return this.reverse.substr(n, length)

  method substr(n=int, length=int)
    return this.reverse.substr(R\textsc{exx} n, R\textsc{exx} length)
\end{alltt}
the first method exactly overrides the \textbf{substr} method in
the \textbf{R\textsc{exx}} class, but the second does not, because the types of
the arguments do not match.
 
A method that exactly overrides a method is assumed to be an extension
of the overridden method, to be used in the same way.  For such a
method, the following rules apply:
\begin{itemize}
\item It must return a value of the same type as the overridden method (or
none, if the overridden method returns none).
\item It must be at least as visible as the overridden routine.
For example, if the overridden routine is \texttt{public} then it must
also be \texttt{public}.
\item If the overridden method is \texttt{static} then it must also
be \texttt{static}.
\item If the overridden method is not \texttt{static} then it must
not be \texttt{static}.
\item If the underlying implementation checks  exceptions (see page \pageref{refexcep}) ,
only those checked exceptions that are signalled by the overridden
method may be left uncaught in the current method.
\end{itemize}
\section{Constructor methods}\label{refcons}
\index{Constructors,}
\index{,}
\index{Methods,constructor}
\index{Objects,constructing}
\index{Instance, of a class,}
\index{Class,instances of}
 
As described above, methods are usually invoked in the context of an
existing value or type.
A special kind of method, called a constructor method, is used to
actually create a value of a given type (an object).
 
Constructor methods always have the same short name as the class in
which they are found, and construct and return a value of the type
defined by that class (sometimes known as an \emph{instance} of that
class).
If the class is part of a package, then the constructor call may be
qualified by the package name.
 \textbf{Example constructors:}
\begin{alltt}
File('Dan.yr.Ogof')
java.io.File('Speleogroup.letter')
R\textsc{exx}('some words')
netrexx.lang.R\textsc{exx}(1)
\end{alltt}
 
\index{Constructors,default}
There will always be at least one constructor if values can be created
for a class.  \nr{} will add a default public constructor that takes
no arguments if no constructors are provided, unless the components of
the class are all static or constant, or the class is an interface
class.
 
All constructors follow the same rules as other methods, and in
addition:
\begin{enumerate}
\item Constructor calls always include parentheses in the syntax, even
if no arguments are supplied.  This distinguishes them from a reference
to the type of the same name.
\item Constructors must call a constructor of their superclass (the class
they extend) before they carry out any initialization of their own.
This is so any initialization carried out by the superclass takes
place, and at the appropriate moment.
Only after this call is complete can they make any reference to the
special words  \textbf{this} or \textbf{super} (see page \pageref{refspecial}) .
 
Therefore, the first instruction in a constructor must be either a call
to the superclass, using the special constructor \textbf{super()}
(with optional arguments), or a call to to another constructor in the
same class, using the special constructor \textbf{this()} (with
optional arguments).
In the latter case, eventually a constructor that explicitly
calls \textbf{super()} will be invoked and the chain of local
constructor calls ends.
 
As a convenience, \nr{} will add a default call to \textbf{super()},
with no arguments, if the first instruction in a constructor is not a
call to \textbf{this()} or \textbf{super()}.
\item 
The properties of a constructed value are initialized, in the order
given in the program, after the call to \textbf{super()} (whether
implicit or explicit).
\item 
By definition, constructors create a value (object) whose type is
defined by the current class, and then return that value for use.
Therefore, the \texttt{returns} keyword on the
 \texttt{method} instruction (see page \pageref{refmethod})  that introduces the
constructor is optional (if given, the type specified must be that of
the class).
Similarly, the only possible forms of the \texttt{return} instruction
used in a constructor are either "\textbf{return this;}", which
returns the value that has just been constructed, or just
"\textbf{return;}", in which case, the "\textbf{this}" is
assumed (this form will be assumed at the end of a method, as usual, if
necessary).
\end{enumerate}
 
\index{Example,of constructors}
Here is an example of a class with two constructors, showing the use
of \textbf{this()} and \textbf{super()}, and taking advantage of
some of the assumptions:
\begin{alltt}
class MyChars extends SomeClass

  properties private
    /* the data 'in' the object */
    value=char[]

  /* construct the object from a char array */
  method MyChars(array=char[])
    /* initialize superclass */
    super()
    value=array             -- save the value

  /* construct the object from a String */
  method MyChars(s=String)
    /* convert to char[] and use the above */
    this(s.toCharArray())
\end{alltt}
 
Objects of type \textbf{MyChars} could then be created thus:
\begin{alltt}
myvar=MyChars("From a string")
\end{alltt}

or by using an argument that has type \textbf{char[]}.
