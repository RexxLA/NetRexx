\section{Do instruction}\label{refdo}
\index{Instructions,DO}
\index{Flow control,with DO construct}
\index{Group, DO,}
\index{DO group,}
\index{Simple DO group,}
\begin{shaded}
\begin{alltt}
\textbf{do} [\textbf{label} \emph{name}] [\textbf{protect} \emph{term}] [\textbf{binary]};
        \emph{instructionlist}
    [\textbf{catch} [\emph{vare} =] \emph{exception};
        \emph{instructionlist}]...
    [\textbf{finally}[;]
        \emph{instructionlist}]
\textbf{end} [\emph{name}];

where \emph{name} is a non-numeric \emph{symbol}

and \emph{instructionlist} is zero or more \emph{instruction}s
\end{alltt}
\end{shaded}
 The \keyword{do} instruction is used to group instructions together for
execution; these are executed once.
The group may optionally be given a label, and may protect an object
while the instructions in the group are executed; exceptional conditions
can be handled with \keyword{catch} and \keyword{finally}.
 
The most common use of \keyword{do} is simply for treating a number of
instructions as group.

\textbf{Example:}
\begin{lstlisting}
/* The two instructions between DO and END will both */
/* be executed if A has the value 3.                 */
if a=3 then do
  a=a+2
  say 'Smile!'
end
\end{lstlisting}
\index{Body,of group}
Here, only the first \emph{instructionlist} is used.
This forms the \emph{body} of the group.
 
The instructions in the \emph{instructionlist}s may be any assignment,
method call, or keyword instruction, including any of the more complex
constructions such as \keyword{loop}, \keyword{if}, \keyword{select}, and
the \keyword{do} instruction itself.
\subsection{Label phrase}
\index{DO instruction,LABEL}
\index{DO group,naming of}
\index{LABEL,on DO instruction}
 
If \keyword{label} is used to specify a \emph{name} for the group,
then a \keyword{leave} which specifies that name may be used to leave the
group, and the \keyword{end} that ends the group may optionally specify
the name of the group for additional checking.

\textbf{Example:}
\begin{lstlisting}
do label sticky
  x=ask
  if x='quit' then leave sticky
  say 'x was' x
end sticky
\end{lstlisting}
\subsection{Protect phrase}
\index{PROTECT,on DO instruction}
 
If \keyword{protect} is given it must be followed by a \emph{term}
that evaluates to a value that is not just a type and is not of a
primitive type; while the \keyword{do} construct is being executed, the
value (object) is protected - that is, all the instructions in the
\keyword{do} construct have exclusive access to the object.
 
Both \keyword{label} and \keyword{protect} may be specified, in any order,
if required.
\subsection{Exceptions in do groups}
 
\index{CATCH,on DO instruction}
\index{FINALLY,on DO instruction}
Exceptions that are raised by the instructions within a do group may be
caught using one or more \keyword{catch} clauses that name the
\emph{exception} that they will catch.
When an exception is caught, the exception object that holds the details
of the exception may optionally be assigned to a variable,
\emph{vare}.
 
Similarly, a \keyword{finally} clause may be used to introduce
instructions that will always be executed at the end of the group, even
if an exception is raised (whether caught or not).
 
The  \emph{Exceptions} section (see page \pageref{refexcep})  has details and
examples of \keyword{catch} and \keyword{finally}.

\subsection{Binary}
A group of one or more statements in a \code{do binary} group will
follow the semantics of binary statements in binary classes or
methods; \marginnote{\color{gray}3.04}the scope is limited to the do binary group.