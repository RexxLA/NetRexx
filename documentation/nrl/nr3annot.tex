\section{Annotation instruction}\label{refparse}
\index{Annotate,instruction}
\index{Instructions,Annotate}
\begin{shaded}
\begin{alltt}
\textbf{@}
\end{alltt}
\end{shaded}
An \emph{annotation}\footnote{The Annotation instruction is not part of
the original \nr{} language but is added due to the fact that Java
programs sometimes require the use of annotations.} starts with an \textbf{@} (commercial at sign)
and is passed through unchanged\footnote{dependent on the setting of
option -annotations, which is the default. When option -noannotations
is in effect, no annotations are passed through. In this case, no
@SuppressWarnings("unchecked") annotations are generated on methods,
which might lead to (harmless) javac warnings. } To interpret a program with an annotation is an error.

\textbf{Example:}
\begin{lstlisting}
/* standard annotations like @Override and @Deprecated are */
/* used, as are some custom ones                           */
/* (those need to be compiled first to be used)            */
options binary
@Author(name="Class Author")
class AnnotateTest
properties private unused
propz
a = ArrayList()
test = TreeMap()

  @SuppressWarnings("unchecked")
  method main(args=String[]) static
    say 'hello annotations'
    t=AnnotateTest()
    t.old()

    @Override
  method toString() returns String
    return 'Annotations'

    @Deprecated
  method old() /* a comment with an @ in it */
    say 'do no use anymore'

    @Author(name = "Jane Doe")
    @Author(name = "John Doe")
  method repeating()
    say 'repeating annotations'

    @Author( name = "Fifi the Cat", date = "2016-01-01" )
  method parameters()
    say 'parameters are possible, but all on one line'
\end{lstlisting}
