\section{The R\textsc{exx}Time class}\label{refrexxtime}
\index{netrexx.lang,R\textsc{exx}Time class}
 
The \texttt{R\textsc{exx}Time} class implements the Classic Rexx \texttt{Time()} function.
\begin{figure}[h]
   \begin{shaded}
\begin{rail}
  TIME:  'TIME' '('   option?  ')'
  ;
\end{rail}
 \end{shaded}
\end{figure}

The \texttt{time()} function can be called standalone because the
default  \code{-implicituses} commandline option causes a \texttt{uses
  RexxTime} option on the \texttt{class} statement to be included. You can use the following options to obtain specific time formats. (Only the capitalized letter is needed; all characters following it are ignored.)
\subsection{Options}
\begin{description}
\item[Civil]
  \index{Time,Civil}
returns the time in Civil format: hh:mmxx. The hours may take the values 1 through 12, and the minutes the values 00 through 59. The minutes are followed immediately by the letters am or pm. This distinguishes times in the morning (12 midnight through 11:59 a.m.—appearing as 12:00am through 11:59am) from noon and afternoon (12 noon through 11:59 p.m.—appearing as 12:00pm through 11:59pm). The hour has no leading zero. The minute field shows the current minute (rather than the nearest minute) for consistency with other TIME results.
\item[Elapsed]
    \index{Time,Elapsed}
returns sssssssss.uuuuuu, the number of seconds.microseconds since the elapsed-time clock (described later) was started or reset. The number has no leading zeros or blanks, and the setting of NUMERIC DIGITS does not affect the number. The fractional part always has six digits.
\item[Hours]
    \index{Time,Hours}
returns up to two characters giving the number of hours since midnight in the format: hh (no leading zeros or blanks, except for a result of 0).
\item[Long]
    \index{Time,Long}
returns time in the format: hh:mm:ss.uuuuuu (uuuuuu is the fraction of seconds, in microseconds). The first eight characters of the result follow the same rules as for the Normal form, and the fractional part is always six digits.
\item[Minutes]
    \index{Time,Minutes}
returns up to four characters giving the number of minutes since midnight in the format: mmmm (no leading zeros or blanks, except for a result of 0).
\item[Normal]
    \index{Time,Normal}
returns the time in the default format hh:mm:ss, as described previously. The hours can have the values 00 through 23, and minutes and seconds, 00 through 59. All these are always two digits. Any fractions of seconds are ignored (times are never rounded up). This is the default.
\item[Reset]
    \index{Time,Reset}
returns sssssssss.uuuuuu, the number of seconds.microseconds since the elapsed-time clock (described later) was started or reset and also resets the elapsed-time clock to zero. The number has no leading zeros or blanks, and the setting of NUMERIC DIGITS does not affect the number. The fractional part always has six digits.
\item[Seconds]
    \index{Time,Seconds}
returns up to five characters giving the number of seconds since midnight in the format: sssss (no leading zeros or blanks, except for a result of 0).
\end{description}
\subsection{Examples}

\begin{lstlisting}[label=timeexample,caption=Example of using Time()]
  method main(args=String[]) static
    say time()       -- 22:16:33
    say time('C')    -- 10:16pm
    say time('E')    -- 0.000000
    say time('R')    -- 0
    say time('H')    -- 22
    say time('L')    -- 22:16:33.836725
    say time('M')    -- 1336
    say time('N')    -- 22:16:33
    say time('O')    -- 22:16:33
    say time('R')    -- 0.001204
    say time('E')    -- 0.000271
    say time('S')    -- 80193
  \end{lstlisting}

