\section{Interpret instruction}\label{refinterpret}
\index{INTERPRET,instruction}
\index{Interpreting code at run-time}
\index{Instructions,INTERPRET}
\index{}
\index{,}
\index{,}
\begin{shaded}
\begin{alltt}
\textbf{interpret} \emph{expression};

\end{alltt}
\end{shaded}

The \keyword{interpret} instruction is used to process instructions at run-time.
Expression is evaluated and interpreted within the current context of the running program.

The expression must be one or more valid clauses separated by a semicolon \textbf{;}, syntactically correct, properly ended - in short,
anything that may be written within a \keyword{do} block.

When this is not the case, the keyword{interpret} instruction will signal an InterpretException exception.

The \keyword{interpret} instruction has read-write access to properties and variables, and can invoke methods
on these (when visibility and modifier attributes are suitable).

The following examples show some of the capabilities:

\begin{lstlisting}
FRED=''
data='FRED'
interpret data '= 4'
say FRED
\end{lstlisting}
will output '4'.

You can even interpret the \keyword{interpret} instruction, as in:

\begin{lstlisting}
FRED=''
interpret "data = 'FRED'; interpret  data '= 4';"
say FRED
\end{lstlisting}
which will also output '4'.

Loop constructs must be complete:

\begin{lstlisting}
interpret  'loop for 3; say "Hello there!"; end'
\end{lstlisting}
will send its greetings three times.

The \keyword{interpret} instruction is slightly more powerful in interpreted mode than compiled.

In interpreted mode, new variables can be created with the \keyword{interpret} instruction, compiled \nr{} cannot
create new variables in the calling program because of Java constraints.

The \keyword{return} instruction is the only allowed 'level changing' instruction, and only in interpreted mode:
undeterministic
\begin{lstlisting}
method aMethod
  interpret  'RETURN 1'
  -- never reached
  return 0
\end{lstlisting}
will return '1' from aMethod.
It will generate an InterpretException in compiled \nr{}.

Singaling exceptions is not supported in the \keyword{interpret} instruction, and starting threads from the instruction
is nondeterministic and unpredictable and should be avoided

