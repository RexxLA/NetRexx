\section{Interpret instruction}\label{refinterpret}
\index{INTERPRET,instruction}
\index{Interpreting code at run-time}
\index{Instructions,TRACE}
\index{}
\index{,}
\index{,}
\begin{shaded}
\begin{alltt}
\textbf{interpret} \emph{expression};

\end{alltt}
\end{shaded}

The \keyword{interpret} instruction is used to process instructions at run-time.
Expression is evaluated and interpreted within the current context of the running program.

The expression must be one or more valid clauses separated by a semicolon \textbf{;}, syntactically correct, properly ended - in short,
anything that may be written within a \keyword{do} block.

When this is not the case, the keyword{interpret} instruction will signal an
org.netrexx.process.RxQuit exception.


\begin{shaded}
\textbf{Note:}

The \keyword{interpret} instruction is currently only valid when interpreting code, i.e. with the -exec option active.\newline
Compiling the NetRexx \keyword{interpret} instruction fails with an errormessage "+++ Error: INTERPRET cannot be compiled"
\end{shaded}

The following examples show some of the capabilities:

\begin{lstlisting}
FRED=''
data='FRED'
interpret data '= 4'
say FRED
\end{lstlisting}
will output '4'.

You can even interpret the \keyword{interpret} instruction, as in

\begin{lstlisting}
FRED=''
interpret "data = 'FRED'; interpret  data '= 4';"
say FRED
\end{lstlisting}
which will also output '4'.

Loop constructs must be complete.

\begin{lstlisting}
interpret  'loop for 3; say "Hello there!"; end'
\end{lstlisting}
will send its greetings three times.

The \keyword{return} instruction is the only allowed 'level changing' instruction.

\begin{lstlisting}
method aMethod
  interpret  'RETURN 1'
  -- never reached
  return 0
\end{lstlisting}
will return '1' from aMethod.


