\chapter{Leave instruction}\label{refleave}
\index{LEAVE instruction,}
\index{Instructions,LEAVE}
\index{LEAVE instruction,use of variable on}
\index{,}
\index{Loops,termination of}
\begin{shaded}
\begin{alltt}
\textbf{leave} [\emph{name}];

where \emph{name} is a non-numeric \emph{symbol}.
\end{alltt}
\end{shaded}
 \texttt{leave} causes immediate exit from one or more \texttt{do},
\texttt{loop}, or \texttt{select} constructs.
It may only be used in the body (the first \emph{instructionlist})
of the construct.
 
Execution of the instruction list is terminated, and control is
passed to the \texttt{end} clause of the construct, just as though the
last clause in the body of the construct had just been executed or (if
a loop) the termination condition had been met normally, except that on
exit the control variable (if any) will contain the value it had when
the \texttt{leave} instruction was executed.
 
If no \emph{name} is specified, then \texttt{leave} must be
within an active loop and will terminate the innermost active loop.
 
If a \emph{name} is specified, then it must be the name of the
label (or control variable for a loop with no label), of a currently
active \texttt{do}, \texttt{loop}, or \texttt{select} construct
(which may be the innermost).  That construct (and any active constructs
inside it) is then terminated.  Control then passes to the clause
following the \texttt{end} clause that matches the
\texttt{do}, \texttt{loop}, or \texttt{select} clause identified by the
\emph{name}.
 \textbf{Example:}
\begin{alltt}
loop i=1 to 5
  say i
  if i=3 then leave
  end i
/* Would display the numbers:  1, 2, 3  */
\end{alltt}
 \textbf{Notes:}
\begin{enumerate}
\index{FINALLY,reached by LEAVE}
\item If any construct being left includes a \texttt{finally} clause, the
\emph{instructionlist} following the \texttt{finally} will be
executed before the construct is left.
\index{Active constructs,}
\index{Loops,active}
\index{Constructs,active}
\index{Names,on LEAVE instructions}
\item 
A \texttt{do}, \texttt{loop}, or \texttt{select} construct
is active if it is currently being executed.
If a method (even in the same class) is called during execution of an
active construct, then the construct becomes inactive until the method
has returned.
\texttt{leave} cannot be used to leave an inactive construct.
\item The \emph{name} symbol, if specified, must exactly match the
label (or the name of the control variable, for a loop with no label) in
the \texttt{do}, \texttt{loop}, or \texttt{select} clause in all
respects except case.
\end{enumerate}
