\section{Exception classes}\label{refnlexcep}
\index{netrexx.lang,Exceptions}
 
The classes provided for exceptions in the \textbf{netrexx.lang} package
are all subclasses of \textbf{java.lang.RuntimeException} and all have
the same content.  Each has two constructors: one taking no argument and
the other taking a string of type \textbf{java.lang.String}, which is
used for additional detail describing the exception.
 
The Exceptions are signalled as follows.
\begin{description}
\item[BadArgumentException]\label{refexpbae}
\index{BadArgumentException,}
\index{Exception,BadArgumentException}
 signalled when an argument to a method is incorrect.
\item[BadColumnException]\label{refexpbce}
\index{BadColumnException,}
\index{Exception,BadColumnException}
 signalled when a column number in a parsing template is not valid
(for example, not a number).
\item[BadNumericException]\label{refexpbne}
\index{BadNumericException,}
\index{Exception,BadNumericException}
 signalled when a \keyword{numeric digits} instruction tries to set
a value that is not a whole number, or is not positive, or is more than
nine digits.
\item[DivideException]\label{refexpdve}
\index{DivideException,}
\index{Exception,DivideException}
 signalled when an error occurs during a division.  This may be due
to an attempt to divide by zero, or when the intermediate result of an
integer divide or remainder operation is not valid.
\item[ExponentOverflowException]\label{refexpeoe}
\index{ExponentOverflowException,}
\index{Exception,ExponentOverflowException}
signalled when the exponent resulting from an operation would
require more than nine digits.
\item[InterpretException]\label{refexpie}
\index{InterpretException,}
\index{Exception,InterpretException}
signalled when an \keyword{interpret} expression cannot be parsed or when it generates a runtime exception.
\item[NoOtherwiseException]\label{refexpnoe}
\index{NoOtherwiseException,}
\index{Exception,NoOtherwiseException}
 signalled when a \keyword{select} construct does not supply an
\keyword{otherwise} clause and none of expressions on the \keyword{when}
clauses resulted in \textbf{'1'}.
\item[NotCharacterException]\label{refexpnce}
\index{NotCharacterException,}
\index{Exception,NotCharacterException}
 signalled when a conversion from a string to a single character was
attempted but the string was not exactly one character long.
\item[NotLogicException]\label{refexpnle}
\index{NotLogicException,}
\index{Exception,NotLogicException}
 signalled when a conversion from a string to a boolean was
attempted but the string was neither the string \textbf{'0'} nor the
string \textbf{'1'}.
\end{description}
 
\index{NumberFormatException,}
\index{Exception,NumberFormatException}
\index{NullPointerException,}
\index{Exception,NullPointerException}
Other exceptions, from the \textbf{java.lang} package, may also be
signalled, for example \textbf{NumberFormatException}
or \textbf{NullPointerException}.
