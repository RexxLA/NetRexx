\chapter{Appendix A - A Sample \nr{} Program}\label{refappa}
 
This appendix includes a short program, called \textbf{qtime}, which
is an example of a "real" \nr{} program.  The programs included
elsewhere in this book have been contrived to illustrate specific
points.  By contrast, \textbf{qtime} is a simple but useful tool that
genuinely improves the human factors of computer systems.  People
frequently wish to know the time of day, and this program presents this
information in a natural way.
 
The style used for this example is the same as that used throughout
the
book,
with all symbols except those describing classes being written
in lower case.  Other \nr{} programming styles are possible, of
course; \nr{} syntax is designed to permit a wide variety of styles
with a minimum of punctuation.
 
The \textbf{qtime} program is a modification of one of the first R\textsc{exx}
programs ever written (much of the program is identical).  The main
changes are:
\begin{itemize}
\item Indexed variables (brackets notation) are used instead of R\textsc{exx}
stems.
\item The \textbf{word} method from the \textbf{R\textsc{exx}} class is used
instead of the \textbf{word} R\textsc{exx} built-in function.
\item The Java \textbf{Date} class is used to determine the current
time.
\end{itemize}
 
\index{,}
\index{Programs,examples}
\index{Example,program}
\index{qtime example program,}
\textbf{qtime.nrx - Query Time}
\lstinputlisting[label=qtime,caption=qtime.nrx]{../../examples/ibm-historic/qtime.nrx}
