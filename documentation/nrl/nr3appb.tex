\chapter{JavaBean Support}\label{refappb}
 
This chapter describes the  \emph{indirect
properties} feature.
 
\index{JavaBean properties,}
\index{Properties,for JavaBeans}
The intention of this feature is to make it easier to write a certain
kind of class known as a \emph{JavaBean}.
Almost all JavaBeans will have \emph{properties}, which are data items
that a user of a JavaBean is expected to be able to customize (for
example, the text on a pushbutton).  The names and types of the
properties of a JavaBean are inferred from "\emph{design
patterns}" (in this context, conventions for naming methods) or
from PropertyDescriptor objects associated with the JavaBean.
 
The JavaBean properties do not necessarily correspond to instance
variables in the class - although very often they do.  The
JavaBean specification does not guarantee that JavaBean properties
that can be set can also be inspected, nor does it describe how
ambiguities of naming and method signatures are to be handled.
 
The \nr{}C compiler
allows a more rigorous
treatment of JavaBean properties, by allowing an optional attribute of
properties in a class that declares them to be \emph{indirect
properties}.  Indirect properties are properties of a known type
that are private to the class, but which are expected to be publicly
accessible indirectly, though certain conventional method calls.
 
Declaring properties to be indirect offers the following advantages:
\begin{itemize}
\item For many simple cases, the access methods for the properties can be
generated automatically; there is no need to explicitly code them in the
source file for the class.  This is especially helpful for Indexed
Properties (where four methods are needed, in general).
\item Where access methods are explicitly provided in the class, they can
be checked for correct form, signature and accessibility.  This detects
errors at compile time that otherwise would only be determined by
testing.
\item Similarly, attention can be drawn to the presence of methods that
may be intended to be an access method for an indirect property, but
will not be recognized as such by builders.
\end{itemize}
 The next section describes the use of indirect properties in more
detail.
\section{Indirect properties}\label{refindprop}
\index{PROPERTIES instruction,}
\index{Instructions,PROPERTIES}
\index{INDIRECT,on PROPERTIES instruction}
\index{Indirect properties,}
\index{Properties,indirect}
 
The  \keyword{properties} instruction (see page \pageref{refprop})  is used to
define the attributes of following \emph{property} variables.
The \emph{visibility} of properties may include a new alternative:
\keyword{indirect}.
Properties with this form of visibility are known as \emph{indirect
properties}.  These are properties of a known type that are private
to the class, but which are expected to be publicly accessible
indirectly, though certain conventional method calls.
 
For example, consider the simple program:
\begin{lstlisting}[label=sandwich,caption=Sandwich.nrx]
class Sandwich extends Canvas implements Serializable
  properties indirect
    slices=Color.gray
    filling=Color.red

  method Sandwich
    resize(100,30)

  method paint(g=Graphics)
    g.setColor(slices)
    g.fillRect(0, 0, size.width, size.height)
    g.setColor(filling)
    g.fillRect(12, 12, size.width-12, size.height-12)
\end{lstlisting}
This declares the \textbf{Sandwich} class as having two indirect
properties, called \textbf{slices} and \textbf{filling}, both being
of type \textbf{java.awt.Color}.
 
In the example, no access methods are provided for the properties, so
the compiler will add them.  By implementation-dependent convention, the
names are prefixed with verbs such as \textbf{get} and \textbf{set},
\emph{etc.}, and have the first character of their name uppercased to form the
method names.
Hence, in this Java-based example, the following four methods are added:
\begin{lstlisting}[label=slices,caption=Slices]
method getSlices  returns java.awt.Color
  return slices
method getFilling returns java.awt.Color
  return filling
method setSlices($1=java.awt.Color)
  slices=$1
method setFilling($2=java.awt.Color)
  filling=$2
\end{lstlisting}
(where \textbf{\$1} and \textbf{\$2} are "hidden" names used for
accessing the method arguments).
 
Note that the \keyword{indirect} attribute for a property is an
alternative to the \keyword{public}, \keyword{private}, and
\keyword{inheritable}, and \keyword{shared} attributes.
Like private properties, indirect properties can only be accessed
directly by name from within the class in which they occur; other
classes can only access them using the access methods (or other methods
that may use, or have a side-effect on, the properties).
 
Indirect properties may be \keyword{constant} (implying that only
a \texttt{get} method is generated or allowed, though the private property
may be changed by methods within the class)
or  \keyword{transient} (see page \pageref{refpropmod}) .
They may not be \keyword{static} or \keyword{volatile}.
 
In detail, the rules used for generating automatic methods for a
property whose name is \textbf{xxxx} are as follows:
\begin{enumerate}
\item A method called \textbf{getXxxx} which returns the value of the
property is generated.  The returned value will have the same type
as \textbf{xxxx}.
\item If the type of \textbf{xxxx} is \textbf{boolean} then the generated
method will be called \textbf{isXxxx} instead of \textbf{getXxxx}.
\item If the property is not \keyword{constant} then a method for setting the
property will also be generated.  This will be called \textbf{setXxxx},
and take a single argument of the same type as \textbf{xxxx}.  This
assigns the argument to the property and returns no value.
\end{enumerate}
 
If the property has an array type (for example, \textbf{char[]}),
then it must only have a single dimension.
Two further methods may then be generated, according to the rules:
\begin{enumerate}
\item A method called \textbf{getXxxx} which takes a single \textbf{int}
as an argument and which returns an item from the property array is
generated. The returned value will have the same type as \textbf{xxxx},
without the \textbf{[]}.  The integer argument is used to
index into the array.
\item As before, if the result type of the method would be \textbf{boolean}
then the name of the method will be \textbf{isXxxx} instead
of \textbf{getXxxx}.
\item If the property is not \keyword{constant} then a method for setting an
item in the property array will also be generated.
This will be called \textbf{setXxxx}, and take two arguments: the
first is an \textbf{int} that is used to select the item to be
changed, and the second is an undimensioned argument of the same type
as \textbf{xxxx}.  It assigns the second argument to the item in the
property array indexed by the first argument, and returns no value.
\end{enumerate}
 For example, for an indirect property declared thus:
\begin{alltt}
properties indirect
  fred=foo.Bar[]
\end{alltt}
the four methods generated would be:
\begin{lstlisting}[label=fred,caption=getFred/setFred]
method getFred returns foo.Bar[]; return fred
method getFred($1=int) returns foo.Bar; return fred[$1]
method setFred($2=foo.Bar[]); fred=$2
method setFred($3=int, $4=foo.Bar); fred[$3]=$4
\end{lstlisting}
 
Note that in all cases a method will only be generated if it would not
exactly
match a method explicitly coded in the current class.
\subsection{Explicit provision of access methods}
 
Often, for example when an indirect property has an on-screen
representation, it is desirable to redraw the property when the property
is changed (and in more complicated cases, there may be interactions
between properties).
These and other actions will require extra processing which will not be
carried out by automatically generated methods.  To add this processing
the access methods will have to be coded explicitly.  In the
"Sandwich" example, we only need to supply the \textbf{set}
methods, perhaps by adding the following to the example class above:
\begin{lstlisting}[label=setslices,caption=setSlices]
method setSlices(col=Color)
  slices=col      -- update the property
  this.repaint    -- redraw the component

method setFilling(col=Color)
  filling=col
  this.repaint
\end{lstlisting}
If we add these two methods, they will no longer be added
automatically (the two \textbf{get} methods will continue to be
provided automatically, however).  Further, since the names match
possible access methods for properties that are declared to be indirect,
the compiler will check the method declaration: the method signatures
and return type (if any) must be correct, for example.  Also, since the
names of access methods are case-sensitive (in a Java environment), you
will be warned if a method appears to be intended to be an access method
but the case of one or more letters is wrong.
 
Specifically, the checks carried out are as follows:
\begin{enumerate}
\item For methods whose names exactly match a potential access method for
an indirect property (that is, start with \textbf{is}, \textbf{get},
or \textbf{set}, which is then followed by the name of an indirect
property with the first character of the name uppercased):
\begin{itemize}
\item The argument list for (signature of) the method must match one of
those that could possibly be automatically generated for the property.
\item The returns type (if any) must match the expected returns type for
that method.
\item If the returns type is simply \textbf{boolean}, then the method name
must start with \textbf{is}.
Conversely, if the method name starts with \textbf{is} then the returns
type must be just \textbf{boolean}.
\item If the property is \keyword{constant} then the name of the method
cannot start with \textbf{set}.
\item A warning is given if the method is not \keyword{public} (the default).
\end{itemize}
\item For methods whose names match a potential access method, as above,
except in case:
\begin{itemize}
\item A warning is given that the method in question may be intended to
be an indirect property access method, but will not be recognized as
such by builders.
\end{itemize}
\end{enumerate}
 These checks detect a wide variety of errors at compile time, hence
speeding the development of classes that use indirect properties.
