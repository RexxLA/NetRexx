\chapter{Numbers and Arithmetic}\label{refnums}
\index{Arbitrary precision arithmetic,}
\index{Precision,arbitrary}
\index{R\textsc{exx},arithmetic}
 \nr{} arithmetic attempts to carry out the usual operations
(including addition, subtraction, multiplication, and division) in as
"natural" a way as possible.
What this really means is that the rules followed are those that are
conventionally taught in schools and colleges.
However, it was found that unfortunately the rules used vary
considerably (indeed much more than generally appreciated) from person
to person and from application to application and in ways that are not
always predictable.
The \nr{} arithmetic described here is therefore a compromise which
(although not the simplest) should provide acceptable results in most
applications.
\section{Introduction}\label{arithintro}
 
\index{Numbers,}
Numbers can be expressed in \nr{} very flexibly (leading and trailing
blanks are permitted, exponential notation may be used) and follow
conventional syntax.
Some valid numbers are:
\begin{lstlisting}
     12          /* A whole number               */
   '-76'         /* A signed whole number        */
     12.76       /* Some decimal places          */
 ' +  0.003 '    /* Blanks around the sign, etc. */
     17.         /* Equal to 17                  */
      '.5'       /* Equal to 0.5                 */
     4E+9        /* Exponential notation         */
      0.73e-7    /* Exponential notation         */
\end{lstlisting}
\index{Exponential notation,}
(Exponential notation means that the number includes a sign and a power
of ten following an "\textbf{E}" that indicates how the decimal
point will be shifted.  Thus \textbf{4E+9} above is just a short way
of writing \textbf{4000000000}, and \textbf{0.73e-7} is short
for \textbf{0.000000073}.)
\index{Operators,arithmetic}
\index{Numbers,arithmetic on}
\index{Arithmetic,operators}
\index{Integer division,}
\index{Division,integer}
\index{Remainder operator,}
 The arithmetic operators include
addition (indicated by a "\textbf{+}"),
subtraction ("\textbf{-}"),
multiplication ("\textbf{*}"),
power ("\textbf{**}"), and
division ("\textbf{/}").
There are also two further division operators:
integer divide ("\textbf{\%}") which divides and returns the integer part, and
remainder ("\textbf{//}") which divides and returns the remainder.
Prefix plus ("\textbf{+}") and
prefix minus ("\textbf{-}") operators are also provided.
\index{Rounding,}
 When two numbers are combined by an operation, \nr{} uses a set of
rules to define what the result will be (and how the result is to be
represented as a character string).
These rules are defined in the next section, but in summary:
\begin{itemize}
\item Results will be calculated with up to some maximum number of
significant digits.
That is, if a result required more than 9 digits it would normally be
rounded to 9 digits.
For instance, the division of 2 by 3 would result in 0.666666667 (it
would require an infinite number of digits for perfect accuracy).
 
You can change the default of 9 significant digits by using the
\keyword{numeric digits} instruction.  This lets you calculate using
as many digits as you need - thousands, if necessary.
\item Except for the division and power operators, trailing zeros are
preserved (this is in contrast to most electronic calculators, which
remove all trailing zeros in the decimal part of results).
So, for example:
\begin{lstlisting}
2.40 + 2  =>  4.40
2.40 - 2  =>  0.40
2.40 * 2  =>  4.80
2.40 / 2  =>  1.2
\end{lstlisting}
This preservation of trailing zeros is desirable for most
calculations (and especially financial calculations).
 If necessary, trailing zeros may be easily removed with the
 \textbf{strip} method (see page \pageref{refstrip}) , or by division by 1.
\item A zero result is always expressed as the single
digit \textbf{'0'}.
\item 
Exponential form is used for a result depending on its value and
the setting of \keyword{numeric digits} (the default is 9 digits).
If the number of places needed before the decimal point exceeds this
setting, or the absolute value of the number is less
than \textbf{0.000001}, then the number will be expressed in
exponential notation; thus
\begin{lstlisting}
1e+6 * 1e+6
\end{lstlisting}
results in "\textbf{1E+12}" instead of
"\textbf{1000000000000}", and
\begin{lstlisting}
1 / 3E+10
\end{lstlisting}
results in "\textbf{3.33333333E-11}" instead of
"\textbf{0.0000000000333333333}".
\item 
Any mixture of Arabic numerals (0-9) and  Extra digits (see page \pageref{refsyms}) 
can be used for the digits in numbers used in calculations.  The results
are expressed using Arabic numerals.
\end{itemize}
\section{Definition}\label{arithdefinition}
 This definition describes arithmetic for \nr{} strings
(type \textbf{R\textsc{exx}}).
The arithmetic operations are identical to those defined in the ANSI
standard for R\textsc{exx}.
\index{ANSI standard,arithmetic definition}
\footnote{
American National Standard for Information Technology -
Programming Language REXX, X3.274-1996, American National
Standards Institute, New York, 1996.
}
\subsection{Numbers}\label{refdefnum}
\index{Numbers,definition}
 A \emph{number} in \nr{} is a character string that includes one or
more decimal digits, with an optional decimal point.
The decimal point may be embedded in the digits, or may be prefixed or
suffixed to them.
The group of digits (and optional point) thus constructed may have
leading or trailing blanks, and an optional sign ("\textbf{+}"
or "\textbf{-}") which must come before any digits or decimal
point.
The sign may also have leading or trailing blanks.
Thus:
\index{Numeric,part of a number}
\index{Digits,in numbers}
\begin{alltt}
sign    ::=  + | -
digit   ::=  0 | 1 | 2 | 3 | 4 | 5 | 6 | 7 | 8 | 9
digits  ::=  digit [digit]...
numeric ::=  digits . [digits]
             | [.] digits
number  ::=  [blank]... [sign [blank]...]
             numeric [blank]...
\end{alltt}

\index{Extra digits,in numbers}
where if the implementation supports  extra digits (see page \pageref{refsyms}) 
these are also accepted as \emph{digit}s, providing that they
represent values in the range zero through nine.
In this case each extra digit is treated as though it were
the corresponding character in the range 0-9.
\index{Period,in numbers}
\index{. (period),in numbers}
 Note that a single period alone is not a valid number.
\subsection{Precision}\label{refndi2}
\index{Arithmetic,precision}
\index{Precision,of arithmetic}
\index{Significant digits, in arithmetic,}
\index{DIGITS,on NUMERIC instruction}
\index{NUMERIC,DIGITS}
 The maximum number of significant digits that can result from an
arithmetic operation is controlled by the \keyword{digits} keyword on the
 \keyword{numeric} instruction (see page \pageref{refnumeric}) :
\begin{alltt}
\keyword{numeric digits} [\emph{expression}];
\end{alltt}
The expression is evaluated and must result in a positive whole
number.
This defines the precision (number of significant digits) to which
arithmetic calculations will be carried out; results will be rounded to
that precision,
if necessary.
 If no expression is specified, then the default precision is used.
The default precision is 9, that is, all implementations must support
at least nine digits of precision.  An implementation-dependent maximum
(equal to or larger than 9) may apply: an attempt to exceed this will
cause execution of the instruction to terminate with an exception.
Thus if an algorithm is defined to use more than 9 digits then if
the \keyword{numeric digits} instruction succeeds then the computation
will proceed and produce identical results to any other implementation.
 Note that \keyword{numeric digits} may set values below the default of
nine.
Small values, however, should be used with care - the loss of
precision and rounding thus requested will affect all \nr{}
computations, including (for example) the computation of new values for
the control variable in loops.
 
In the remainder of this section, the notation \textbf{digits} refers
to the current setting of \keyword{numeric digits}.
This setting may also be referred to in expressions in programs by using
the \textbf{digits}  special word (see page \pageref{refspecial}) .
\subsection{Arithmetic operators}
\index{Operators,arithmetic}
\index{Numbers,arithmetic on}
\index{Arithmetic,operators}
 
\nr{} arithmetic is effected by the operators "\textbf{+}",
"\textbf{-}", "\textbf{*}", "\textbf{/}",
"\textbf{\%}", "\textbf{//}", and "\textbf{**}"
(add, subtract, multiply, divide, integer divide, remainder, and power)
which all act upon two terms, together with the prefix operators
"\textbf{+}" and "\textbf{-}" (plus and minus)
which both act on a single term.
The result of all these operations is a \nr{} string, of
type \textbf{R\textsc{exx}}.
This section describes the way in which these operations are carried
out.
 Before every arithmetic operation, the term or terms being operated
upon have any extra digits converted to the corresponding Arabic numeral
(the digits 0-9).  They then have leading zeros removed (noting the
position of any decimal point, and leaving just one zero if all the
digits in the number are zeros) and are then truncated
to \textbf{digits+1} significant digits
\footnote{
\index{Guard digit in arithmetic,}
That is, to the precision set by \keyword{numeric digits}, plus one extra
"guard" digit.
}
(if necessary) before being used in the computation.
The operation is then carried out under up to double that precision, as
described under the individual operations below.
When the operation is completed, the result is rounded if necessary to
the precision specified by the \keyword{numeric digits} instruction.
\index{Rounding,definition}
 Rounding is done in the "traditional" manner, in that the extra
(guard) digit is inspected and values of 5 through 9 are rounded up,
and values of 0 through 4 are rounded down.
\footnote{
\index{,}
Even/odd rounding would require the ability to calculate to arbitrary
precision (that is, to a precision not governed by the setting of
\keyword{numeric digits}) at any time and is therefore not the mechanism
defined for \nr{}.
}
 A conventional zero is supplied preceding a decimal point if
otherwise there would be no digit before it.  Trailing zeros are
retained for addition, subtraction, and multiplication, according to
the rules given below, except that a result of zero is always expressed
as the single character \textbf{'0'}.  For division, insignificant
trailing zeros are removed after rounding.
 
The  \textbf{format} method (see page \pageref{refformat})  is defined to allow a
number to be represented in a particular format if the standard result
provided by \nr{} does not meet requirements.
\subsection{Arithmetic operation rules - basic operators}\label{}
\index{Trailing zeros,}
\index{Arithmetic,operation rules}
 The basic operators (addition, subtraction, multiplication, and
division) operate on numbers as follows:
\begin{description}
\item[Addition and subtraction]
\index{Addition,definition}
\index{+ plus sign,addition operator}
\index{Subtraction,definition}
\index{- minus sign,subtraction operator}
\index{Prefix operators,arithmetic}

If either number is zero then the other number, rounded
to \textbf{digits} digits if necessary, is used as the result (with
sign adjustment as appropriate).
Otherwise, the two numbers are extended on the right and left as
necessary up to a total maximum of \textbf{digits+1} digits.
 
The number with smaller absolute value may therefore lose some or
all of its digits on the right.
\footnote{
In the example, the number \textbf{yy.yyyyy} would have three digits
truncated if \textbf{digits} were \textbf{5}.
}
The numbers are then added or subtracted as appropriate.  For example:
\begin{alltt}
xxxx.xxx + yy.yyyyy
\end{alltt}
becomes:
\begin{alltt}
  xxxx.xxx00
+ 00yy.yyyyy
------------
  zzzz.zzzzz
\end{alltt}
.sumadd
The result is then rounded to \textbf{digits} digits if necessary,
taking into account any extra (carry) digit on the left after an
addition, but otherwise counting from the position corresponding to the
most significant digit of the terms being added or subtracted.
Finally, any insignificant leading zeros are removed.
 The \emph{prefix operators} are evaluated using the same rules;
the operations "\textbf{+number}" and "\textbf{-number}"
are calculated as "\textbf{0+number}" and
"\textbf{0-number}", respectively.
\item[Multiplication]
\index{Multiplication,definition}
\index{* multiplication operator,}

The numbers are multiplied together ("long multiplication")
resulting in a number which may be as long as the sum of the lengths of
the two operands.  For example:
\begin{alltt}
xxx.xxx * yy.yyyyy
\end{alltt}
becomes:
\begin{alltt}
zzzzz.zzzzzzzz
\end{alltt}
and the result is then rounded to \textbf{digits} digits if
necessary, counting from the first significant digit of the result.
\item[Division]
\index{Division,definition}

For the division:
\begin{alltt}
yyy / xxxxx
\end{alltt}
the following steps are taken: first, the number
"\textbf{yyy}" is extended
with zeros on the right until it is larger than
the number "\textbf{xxxxx}" (with note being taken of the change
in the power of ten that this implies).  Thus in this example,
"\textbf{yyy}"
might become
"\textbf{yyy00}".
Traditional long division then takes place, which can be written:
\begin{alltt}
         zzzz
      .------
xxxxx | yyy00
\end{alltt}

The length of the result ("\textbf{zzzz}") is such that the
rightmost "\textbf{z}" will be at least as far right as the
rightmost digit of the (extended) "\textbf{y}" number in the
example.  During the division, the "\textbf{y}" number will be
extended further as necessary, and the "\textbf{z}" number
(which will not include any leading zeros) may increase up
to \textbf{digits+1} digits, at which point the division stops and the
result is rounded.
Following completion of the division (and rounding if necessary),
insignificant trailing zeros are removed.
\end{description}
 \textbf{Examples:}
\begin{alltt}
/* With 'numeric digits 5' */
12+7.00     ==  19.00
1.3-1.07    ==  0.23
1.3-2.07    ==  -0.77
1.20*3      ==  3.60
7*3         ==  21
0.9*0.8     ==  0.72
1/3         ==  0.33333
2/3         ==  0.66667
5/2         ==  2.5
1/10        ==  0.1
12/12       ==  1
8.0/2       ==  4
\end{alltt}
\textbf{Note: }With all the basic operators, the position of the decimal point
in the terms being operated upon is arbitrary.
The operations may be carried out as integer operations with the
exponent being calculated and applied afterwards.
Therefore the significant digits of a result are not in any way
dependent on the position of the decimal point in either of the terms
involved in the operation.
\subsection{Arithmetic operation rules - additional operators}
 The operation rules for the power ("\textbf{**}"),
integer division ("\textbf{\%}"), and remainder
("\textbf{//}") operators are as follows:
\begin{description}
\item[Power]\label{refpower}

\index{Exponentiation,definition}
\index{Power operator,definition}
The "\textbf{**}" (power) operator raises a number (on the
left of the operator) to a power (on the right of the operator).
The term on the right is rounded to \textbf{digits} digits (if
necessary), and must, after any rounding, be a whole number, which may
be positive, negative, or zero.
If negative, the absolute value of the power is used, and then the
result is inverted (divided into 1).
 
For calculating the power, the number is effectively multiplied by
itself for the number of times expressed by the power, and finally
trailing zeros are removed (as though the result were divided by one).
 In practice (see note below for the reasons), the power is
calculated by the process of left-to-right binary reduction.
For "\textbf{x**n}": "\textbf{n}" is converted to
binary, and a temporary accumulator is set to 1.
If "\textbf{n}" has the value 0 then the initial calculation is
complete.
Otherwise each bit (starting at the first non-zero bit) is inspected
from left to right.
If the current bit is 1 then the accumulator is multiplied by
"\textbf{x}".
If all bits have now been inspected then the initial calculation is
complete, otherwise the accumulator is squared by multiplication and the
next bit is inspected.
When the initial calculation is complete, the temporary result is
divided into 1 if the power was negative.
 
The multiplications and division are done under the normal
arithmetic operation rules, detailed earlier in this section, using a
precision of \textbf{digits+elength+1} digits.
Here, \textbf{elength} is the length in decimal digits of the integer
part of the whole number "\textbf{n}" (\emph{i.e.}, excluding any sign,
decimal part, decimal point, or insignificant leading zeros, as though
the operation \textbf{n\%1} had been carried out and any sign removed).
Finally, the result is rounded to \textbf{digits} digits, if
necessary, and insignificant trailing zeros are removed.
\item[Integer division]

\index{Integer division,definition}

The "\textbf{\%}" (integer divide) operator divides two numbers
and returns the integer part of the result.
The result returned is defined to be that which would result from
repeatedly subtracting the divisor from the dividend while the dividend
is larger than the divisor.  During this subtraction, the absolute
values of both the dividend and the divisor are used: the sign of the
final result is the same as that which would result if normal division
were used.
 The result returned will have no fractional part (that is, no
decimal point or zeros following it).
If the result cannot be expressed exactly within \textbf{digits}
digits, the operation is in error and will fail - that is, the
result cannot have more digits than the current setting of \keyword{numeric
digits}.
For example, \textbf{10000000000\%3} requires ten digits to express the
result exactly (\textbf{3333333333}) and would therefore fail
if \textbf{digits} were \textbf{9} or smaller.
\item[Remainder]

\index{Remainder operator,definition}
The "\textbf{//}" (remainder) operator will return the remainder
from integer division, and is defined
as being the residue of the dividend after the operation of calculating
integer division as just described.
The sign of the remainder, if non-zero, is the same as that of the
original dividend.
 This operation will fail under the same conditions as integer
division (that is, if integer division on the same two terms would
fail, the remainder cannot be calculated).
\end{description}
 \textbf{Examples:}
\begin{lstlisting}
/* Again with 'numeric digits 5' */
2**3        ==  8
2**-3       ==  0.125
1.7**8      ==  69.758
2\%3         ==  0
2.1//3      ==  2.1
10\%3        ==  3
10//3       ==  1
-10//3      ==  -1
10.2//1     ==  0.2
10//0.3     ==  0.1
3.6//1.3    ==  1.0
\end{lstlisting}
 \textbf{Notes:}
\begin{enumerate}
\item A particular algorithm for calculating powers is described, since
it is efficient (though not optimal) and considerably reduces the
number of actual multiplications performed.
It therefore gives better performance than the simpler definition of
repeated multiplication.
Since results could possibly differ from those of repeated
multiplication, the algorithm must be defined here so that different
implementations will give identical results for the same operation on
the same values.
Other algorithms for this (and other) operations may always be used, so
long as they give identical results to those described here.
\item The integer divide and remainder operators are defined so that they
may be calculated as a by-product of the standard division operation
(described above).  The division process is ended as soon as the
integer result is available; the residue of the dividend is the
remainder.
\end{enumerate}
\subsection{Numeric comparisons}\label{arithnumericcomparisons}
\index{Arithmetic,comparisons}
\index{Operators,comparative}
\index{Comparison,of numbers}
\index{Numbers,comparison of}
 Any of the  comparative operators (see page \pageref{refcomps})  may be used
for comparing numeric strings.
However, the strict comparisons (for example, "\textbf{==}" and
"\textbf{>{}>}") are not numeric comparative operators and should
not normally be used for comparing numbers, since they compare from left
to right and leading and trailing blanks (and leading zeros) are
significant for these operators.
 Numeric comparison, using the normal comparative operators, is
effected by subtracting the two numbers (calculating the difference) and
then comparing the result with \textbf{'0'} - that is, the
operation:
\begin{lstlisting}
A ? B
\end{lstlisting}
where "\textbf{?}" is any normal comparative operator, is
identical to:
\begin{lstlisting}
(A - B) ? '0'
\end{lstlisting}
It is therefore the \emph{difference} between two numbers, when
subtracted under \nr{} subtraction rules, that determines their equality.
\subsection{Exponential notation}
\index{Ten, powers of,}
\index{Pure numbers,}
 The definition of numbers  above (see page \pageref{refdefnum}) 
describes "pure" numbers, in the sense that the character strings
that describe numbers can be very long.
 \textbf{Examples:}
\begin{lstlisting}
say  10000000000 * 10000000000
/* would display: 100000000000000000000 */

say  0.00000000001 * 0.00000000001
/* would display: 0.0000000000000000000001 */
\end{lstlisting}
For both large and small numbers some form of exponential notation
is useful, both to make such long numbers more readable and to make
evaluation possible in extreme cases.  In addition, exponential notation
is used whenever the "pure" form would give misleading
information.  For example:
\begin{lstlisting}
numeric digits 5
say 54321*54321
\end{lstlisting}
would display "\textbf{2950800000}" if long form were to be
used.
This is misleading, as it appears that the result is an exact multiple
of 100000, and so \nr{} would express the result in exponential
notation, in this case "\textbf{2.9508E+9}".
\index{Numbers,definition}
\index{Mantissa of exponential numbers,}
\index{Significand of exponential numbers,}
\index{Numeric,part of a number}
\index{Powers of ten in numbers,}
\index{Exponential notation,definition}
\index{E-notation,definition}
 The definition of \emph{number} (see above) is therefore extended
by replacing the description of \textbf{numeric} by the following:
\begin{alltt}
mantissa ::=  digits . [digits]
              | [.] digits
numeric  ::=  mantissa [E sign digits]
\end{alltt}
In other words, the numeric part of a number may be followed by an
"\textbf{E}" (indicating an exponential part), a sign,
and an integer following the sign that represents a power of ten that is
to be applied.
The "\textbf{E}" may be in uppercase or lowercase.
Note that no blanks are permitted within this part of a number, but the
integer may have leading zeros.
 \textbf{Examples:}
\begin{alltt}
12E+11  =  1200000000000
12E-5   =  0.00012
 12e+4  =  120000
\end{alltt}
 All valid numbers may be used as data for arithmetic.  The results
of calculations will be returned in exponential form depending on the
setting of \keyword{numeric digits}.
If the number of places needed before the decimal point
exceeds \textbf{digits}, or if the absolute value of the result is
less than \textbf{0.000001}, then exponential form will be used.
The exponential form generated by \nr{} always has a sign following the
"\textbf{E}".
If the exponent is 0 then the exponential part is omitted - that
is, an exponential part of "\textbf{E+0}" will never be
generated.
 If the default format for a number is not satisfactory for a
particular application, then the \textbf{format} method may be used to
control its format.  Using this, numbers may be explicitly converted to
exponential form or even forced to be returned in "pure" form.
\index{Notation,scientific}
\index{Notation,engineering}
\index{Exponential notation,}
\index{E-notation,}
\index{Scientific notation,}
\index{Engineering notation,}
\index{FORM,option of NUMERIC instruction}
\index{NUMERIC,FORM}
\label{refnfo2}
 Different exponential notations may be selected with the
 \keyword{numeric form} instruction (see page \pageref{refnform}) .
This instruction allows the selection of either scientific or
engineering notation.
\emph{Scientific notation} adjusts the power of ten so there is a
single non-zero digit to the left of the decimal point.
\emph{Engineering notation} causes powers of ten to be expressed as a
multiple of three - the integer part may therefore range
from \textbf{1} through \textbf{999}.
 \textbf{Examples:}
\begin{lstlisting}
numeric form scientific
say 123.45 * 1e11
/* would display: 1.2345E+13 */

numeric form engineering
say 123.45  * 1e11
/* would display: 12.345E+12 */
\end{lstlisting}
 The default exponential notation is scientific.
\subsection{Whole numbers}\label{refwholed}
\index{Whole numbers,definition}
\index{DIGITS,effect on whole numbers}
 Within the set of numbers understood by \nr{} it is useful to
distinguish the subset defined as \emph{whole numbers}.
 
A \emph{whole number} in \nr{} is a number that has a decimal part
which is all zeros (or that has no decimal part).
\subsection{Numbers used directly by \nr{}}\label{refnumuse}
\index{Numbers,use of by \nr{}}
\index{Functions,numeric arguments of}
\index{Rounding,when numbers used}
\index{DIGITS,rounding when numbers used}
 As discussed above, the result of any arithmetic operation is
rounded (if necessary) according to the setting of \keyword{numeric digits}.
Similarly, when a number (which has not necessarily been involved in an
arithmetic operation) is used directly by \nr{} then the same rounding
is also applied, just as though the operation of adding the number
to \textbf{0} had been carried out.
After this operation, the integer part of the number must have no more
digits than the current setting of \keyword{numeric digits}.
 
In the following cases, the number used must be a whole number and
an implementation restriction on the largest number that can be used
may apply:
\begin{itemize}
\item positional patterns, including variable positional patterns,
in  parsing templates (see page \pageref{refparsing}) 
\item the power value (right hand operand) of the power operator (see page \pageref{refpower}).
\item the values of \emph{exprr} and \emph{exprf} (following the
\keyword{for} keyword) in the  \keyword{loop} instruction (see page \pageref{refloop}) 
\item the value of \emph{exprd} (following the \keyword{digits}
keyword) in the  \keyword{numeric} instruction (see page \pageref{refnumeric}) .
\end{itemize}
 \textbf{Implementation minimum:} A minimum length of 9 digits must
be supported for these uses of whole numbers by a \nr{} language
processor.
\subsection{Implementation independence}
\index{Arithmetic,implementation independence}
 The \nr{} arithmetic rules are defined in detail, so that when a
given program is run the results of all computations are sufficiently
defined that the same answer will result for all correct
implementations.  Differences due to the underlying machine
architecture will not affect computations.
 This contrasts with most other programming languages, and with
 binary arithmetic (see page \pageref{refbinary})  in \nr{}, where the
result obtained may depend on the implementation because the precision
and algorithms used by the language processor are defined by the
implementation rather than by the language.
\subsection{Exceptions and errors}
\index{Exceptions,during arithmetic}
\index{Errors during arithmetic,}
\index{Arithmetic,exceptions}
\index{Arithmetic,errors}
\index{Arithmetic,overflow}
\index{Arithmetic,underflow}
\index{Overflow, arithmetic,}
\index{Underflow, arithmetic,}
 The following exceptions and errors may be signalled during arithmetic:
\begin{itemize}
\item Divide exception
 This exception will be signalled if division by zero was attempted,
or if the integer result of an integer divide or remainder operation had
too many digits.
\item Overflow/Underflow exception
 This exception will be signalled if the exponential part of a result
(from an operation that is not an attempt to divide by zero) would
exceed the range that can be handled by the language processor, when the
result is formatted according to the current settings of \keyword{numeric
digits} and \keyword{numeric form}.
The language defines a minimum capability for the exponential part,
namely exponents whose absolute value is at least as large as the
largest number that can be expressed as an exact integer in default
precision.
Thus, since the default precision is nine, implementations must support
exponents in the range \textbf{-999999999}
through \textbf{999999999}.
\item Insufficient storage
 Storage is needed for calculations and intermediate results, and on
occasion an arithmetic operation may fail due to lack of storage.
This is considered an operating environment error as usual, rather
than an arithmetical exception.
\end{itemize}
 \emph{In the reference implementation, the exceptions and error types
used for these three cases
are \textbf{DivideException}, \textbf{ExponentOverflowException},
and \textbf{OutOfMemoryError}, respectively.}
\index{,}
\index{,}
\index{,}
