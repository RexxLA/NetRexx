\section{The R\textsc{exx}Channel class}\label{refrexxchannel}
\index{netrexx.lang,R\textsc{exx}Channel class}

The \texttt{R\textsc{exx}Channel} class implements concurrency synchronisation and communication,
mainly used by methods started with the \texttt{go} instruction. It works as a FIFO (first-in, first-out) queue,
items are read in the order they are written.

Unbuffered RexxChannels are constructed without size, a write method does not complete until the written object is read
and a read operation is blocked until something is written into the RexxChannel. Unbuffered RexxChannels are best
suited for synchronisation between concurrent methods because of the handshaking behavior.


Buffered RexxChannels are constructed with a given size. Items can be written into the RexxChannel upto the given size.
When the channel is full, the write operation blocks until a read operation consumes an item. Reading a
buffered channel is blocked when the channel is empty. Buffered RexxChannels are ideal for communication between
concurrent methods, decoupling producer and consumer.

A RexxChannel can have multiple readers and multiple writers.

RexxChannels can be closed by both readers and writers. A closed channel can be read until empty.
Reading from an empty closed channel, or writing to a closed channel generates an IOException.

The following methods are available on RexxChannels:

\begin{description}
    \item{RexxChannel()}
%    \index{RexxChannel() method,}
%    \index{Method,RexxChannel()}
    \index{Constructor,RexxChannel()}

constructor for an unbuffered RexxChannel
    \item{RexxChannel(s=int)}
%    \index{RexxChannel(s=int) method,}
%    \index{Method,RexxChannel(s=int)}
    \index{Constructor,RexxChannel(s=int)}

constructor for an buffered RexxChannel of size s
    \item{write(o=Object) signals IOException}
    \index{write(s=int) method,}
    \index{Method,write(s=int)}

put an object into a RexxChannel

    \item{read() returns Object signals IOException}
    \index{read() method,}
    \index{Method,read()}

get an Object from a RexxChannel

    \item{close()}
    \index{close() method,}
    \index{Method,close()}

close a RexxChannel
\end{description}

