\chapter{Exit instruction}\label{refexit}
\index{EXIT instruction,}
\index{Instructions,EXIT}
\index{Return code, setting on exit,}
\index{Return string, setting on exit,}
\begin{shaded}
\begin{alltt}
\textbf{exit} [\emph{expression}];
\end{alltt}
\end{shaded}
 \keyword{exit} is used to unconditionally leave a program, and
optionally return a result to the caller.
The entire program is terminated immediately.

If an \emph{expression} is given, it is evaluated and the result
of the evaluation is then passed back to the caller in an
implementation-dependent manner when the program terminates.
Typically this value is expected to be a small whole number; most
implementations will accept values in the range 0 through 250.
If no expression is given, a default result (which depends on the
implementation, and is typically zero) is passed back to the caller.

\textbf{Example:}
\begin{lstlisting}
j=3
exit j*4
/* Would exit with the value '12' */
\end{lstlisting}
\index{Running off the end of a program,}
\index{Bottom of program, reaching during execution,}
 "Running off the end" of a program is equivalent to the
instruction \textbf{return;}.  In the case where the program is simply
a stand-alone application with no \keyword{class} or \keyword{method}
instructions, this has the same effect as \textbf{exit}, in that it
terminates the whole program and returns a default result.
