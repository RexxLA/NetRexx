\chapter{Indexed strings and Arrays}\label{refinstr}
\index{Indexed references,indexed strings}
\index{Indexed strings,}
\index{Strings,indexed}
\index{Variables,indexed}
\index{Variables,subscripts}
\index{Sub-values, of strings,}
\index{Strings,sub-values of}
\index{Index strings,for sub-values}
\index{References,to indexed strings}
\index{Brackets,in indexed strings}
\index{Comma,in indexed strings}
 
Any \nr{} string (that is, a value of type \textbf{R\textsc{exx}}), has the
ability to have \emph{sub-values}, values (also of type \textbf{R\textsc{exx}})
which are associated with the original string and are indexed by an
\emph{index string} which identifies the sub-value.
Any string with such sub-values is known as an \emph{indexed string}.
 
The sub-values of a \nr{} string are accessed using \emph{indexed
references}, where the name of a variable of type \textbf{R\textsc{exx}} is
followed immediately by square brackets enclosing one or more
expressions separated by commas:
\footnote{
The notations \keyword{'['} and \keyword{']'}
indicate square brackets appearing in the \nr{} program.
}
\begin{shaded}
\begin{alltt}
\emph{symbol}'['[\emph{expression}[, \emph{expression}]...]']'
\end{alltt}
\end{shaded}
It is important to note that the \emph{symbol} that names the
variable must be followed immediately by the "\keyword{[}",
with \textbf{no} blank in between, or the construct will not be
recognized as an indexed reference.
 The \emph{expression}s (separated by commas) between the
brackets are called the \emph{indexes} to the string.
These index expressions are evaluated in turn from left to right, and
each must evaluate to a value is of type \textbf{R\textsc{exx}} or that can be
converted to type \textbf{R\textsc{exx}}.
 
The resulting index strings are taken "as-is" - that is, they
must match exactly in content, case, and length for a reference to find
a previously-set item.
They may have any length (including the null string) and value (they are
not constrained to be just those strings which are numbers, for
example).
 
If a reference does not find a sub-value, then a copy of the non-indexed
value of the variable is used.
 \textbf{Example:}
\begin{lstlisting}
surname='Unknown'         -- default value
surname['Fred']='Bloggs'
surname['Davy']='Jones'
try='Fred'
say surname[try] surname['Bert']
\end{lstlisting}
would say "\textbf{Bloggs Unknown}".
 
When multiple indexes are used, they indicate accessing a hierarchy of
strings.  A single \nr{} string has a single set of indexes and
subvalues associated with it.  The sub-values, however, are also \nr{}
strings, and so may in turn have indexes and sub-values.  When more than
one index is specified in an indexed reference, the indexes are applied
in turn from left to right to each retrieved sub-value.
 
For example, in the sequence:
\begin{lstlisting}
x='?'
x['foo', 'bar']='OK'
say x['foo', 'bar']
y=x['foo']
say y['bar']
\end{lstlisting}
both \keyword{say} instructions would display the string
"\textbf{OK}".
 Indexed strings may be used to set up "associative arrays", or
dictionaries, in which the subscript is not necessarily numeric, and
thus offer great scope for the creative programmer.
A useful application is to set up a variable in which the subscripts
are taken from the value of one or more variables, so effecting a form
of associative (content addressable) memory.
%% The  \emph{justone} program (see page \pageref{refjust1})  is an example of this
%% technique.
 \textbf{Notes:}
\begin{enumerate}
\item 
A variable of type \textbf{R\textsc{exx}} must have been assigned a value
before indexing is used on it.
This is the value that is used as the default value whenever an indexed
reference finds no sub-value.
\item 
The indexes, and hence the sub-values, of a \textbf{R\textsc{exx}} object can
be retrieved in turn using the  \keyword{over} (see page \pageref{refloopov})  keyword
of the \keyword{loop} instruction.
\item 
The  \textbf{exists} method (see page \pageref{refexists})  of the \textbf{R\textsc{exx}}
class may be used to test whether an indexed reference has an
explicitly-set value.
\item 
Assigning \textbf{null} to an indexed reference (for example, the
assignment \\ \textbf{switch[7]=null;}) drops the sub-value;
until set to a new value, any reference to the sub-value (including use
of the \textbf{exists} method) will return the same result as
when it had never been set.
\end{enumerate}
\section{Arrays}\label{refarray}
\index{Arrays,}
\index{Indexed references,arrays}
\index{Brackets,in array references}
\index{Comma,in array references}
 
In addition to indexed strings, \nr{} also includes the concept of
fixed-size \emph{arrays}, which may be used for indexing values of any
type (including strings).
 
Arrays are used with the same syntax and in the same manner as
indexed strings, but with important differences that allow for
compact implementations and access to equivalent data structures
constructed using other programming languages:
\begin{enumerate}
\item 
The indexes for arrays must be whole numbers that are zero or positive.
There will usually be an implementation restriction on the maximum value
of the index (typically 999999999 or higher).
\item 
The elements of an array are considered to be \emph{ordered}; the
first element has index \textbf{0}, the second \textbf{1}, and so on.
\item 
An array is of fixed size;
\index{Fixed size, of arrays,}
it must be constructed before use.
\item 
Variables that are assigned arrays can only be assigned arrays (of the
same dimension, see below) in the future.  That is, being an array
changes the type of a value; it becomes a
 \emph{dimensioned type} (see page \pageref{refdimtype}) .
\end{enumerate}
 
\index{Arrays,references}
\index{References,to arrays}
\index{Arrays,constructors}
\index{Constructors,array}
Array references use the \nr{} \emph{indexed reference} syntax
defined above.  The same syntax is used for constructing arrays, except
that the symbol before the left bracket describes a type (and hence may
be qualified by a package name).  The expression or expressions between
the brackets indicate the size of the array in each dimension, and must
be a positive whole number or zero:
\begin{lstlisting}
arg=String[4]       -- makes an array for four Strings
arg=java.io.File[4] -- makes an array for four Files
i=int[3]            -- makes an array for three 'int's
\end{lstlisting}

(Another way of describing this is that array constructors look
just like other object constructors, except that brackets are
used instead of parentheses.)
 
Once an array has been constructed, its elements can be referred to
using brackets and expressions, as before:
\begin{lstlisting}
i[2]=3 -- sets the '2'-indexed value of 'i'
j=i[2] -- sets 'j' to the '2'-indexed value of 'i'
\end{lstlisting}
 
Regular multiple-dimensioned arrays may be constructed and referenced by
using multiple expressions within the brackets:
\begin{lstlisting}
i=int[2,3] -- makes a 2x3 array of 'int' type objects
i[1,2]=3   -- sets the '1,2'-indexed value of 'i'
j=i[1,2]   -- sets 'j' to the '1,2'-indexed value of 'i'
\end{lstlisting}
 As with indexed strings, when multiple indexes are used, they
indicate accessing a hierarchy of arrays (the underlying model is
therefore of a hierarchy of single-dimensioned arrays).
When more than one index is specified in an indexed reference to an
array, the indexes are applied in turn from left to right to each
array.
 
As described in the section on  \emph{Types} (see page \pageref{reftypes}) , the
type of a variable that refers to an array can be set (declared) by
assignment of the type with array notation that indicates the dimension
of an array without any sizes:
\begin{lstlisting}
k=int[]     -- one-dimensional array of 'int' objects
m=float[,,] -- 3-dimensional array of 'float' objects
\end{lstlisting}

The same syntax is also used when describing an array type in the
arguments of a \keyword{method} instruction or when converting types.
For example, after:
\begin{lstlisting}
gg=char[] "Horse"
\end{lstlisting}
the variable \textbf{gg} has as its value an array of
type \textbf{char[]} containing the five
characters \textbf{H}, \textbf{o}, \textbf{r}, \textbf{s},
and \textbf{e}.
\subsection{Array initializers}\label{refarrin}
\index{Arrays,initializing}
\index{Initializing arrays,}
\index{Array initializer,in terms}
\index{Brackets,in array initializers}
\index{Square brackets,in array initializers}
 
An \emph{array initializer} is a \emph{simple term} which is
recognized if it does not immediately follow (abut) a symbol, and has
the form
\footnote{
The notations \keyword{'['} and \keyword{']'}
indicate square brackets appearing in the \nr{} program.
}
\begin{alltt}
\keyword{'['}\emph{expression}[,\emph{expression}]...\keyword{']'}
\end{alltt}
 
An array initializer therefore comprises a list of one or more
expressions, separated by commas, within brackets.  When an array
initializer is evaluated, the expressions are evaluated in turn from
left to right, and all must result in a value.
An array is then constructed, with a number of elements
equal to the number of expressions in the list, with each element
initialized by being assigned the result of the corresponding
expression.
 
The type of the array is derived by adding one dimension to the type of
the result of the first expression in the list, where the type of that
expression is determined using the same rules as are used to select the
type of a variable when it is first  assigned a value(see page \pageref{refassign}).
All the other expressions in the list must have types that could be
assigned to the chosen type without error.
 
For example, in
\begin{lstlisting}
var1=['aa', 'bb', 'cc']
var2=[char 'a', 'b', 'c']
var3=[String 'a', 'bb', 'c']
var4=[1, 2, 3, 4, 5, 6]
var5=[[1,2], [3,4]]
\end{lstlisting}
the types of the variables would
be \textbf{R\textsc{exx}[]}, \textbf{char[]}, \textbf{String[]}, \textbf{R\textsc{exx}[]},
and \textbf{R\textsc{exx}[,]} respectively.
In a binary class in the reference implementation, the types would
be \textbf{String[]}, \textbf{char[]}, \textbf{String[]}, \textbf{int[]},
and \textbf{int[,]}.
 
Array initializers are most useful for initializing properties and
variables, but like other simple terms, they may start a compound term.
 
So, for example
\begin{lstlisting}
say [1,1,1,1].length
\end{lstlisting}
would display \textbf{4}.
 Note that an array of length zero cannot be constructed with an array
initializer, as its type would be undefined.  An explicitly typed array
constructor (for example, \textbf{int[0]}) must be used.
