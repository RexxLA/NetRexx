\chapter{Notations}\label{refnotat}
\index{Syntax notation,}
\index{Syntax diagrams,notation for}
\index{Notations,syntax}
\index{Notations,in text}
\index{Diagrams, of syntax,}
 
In this part of the
book,
various notations such as changes of font are used for clarity.
Within the text, a
sans-serif bold font is used to indicate \texttt{keywords}, and an italic
font is used to indicate \emph{technical terms}.
An italic font is also used to indicate a reference to a
\emph{technical term defined elsewhere} or a \emph{word} in a
syntax diagram that names a segment of syntax.
 
Similarly, in the syntax diagrams in this
book,
words (symbols) in the
sans-serif
bold font also denote keywords or
sub-keywords, and words (such as \emph{expression}) in italics
denote a token or collection of tokens defined elsewhere.
The brackets [ and ] delimit optional (and possibly
alternative) parts of the instructions, whereas the braces \{
and \} indicate that one of a number of alternatives must be
selected.
An ellipsis (\textbf{...}) following a bracket indicates that
the bracketed part of the clause may optionally be repeated.
 
Occasionally in syntax diagrams (\emph{e.g.}, for indexed references)
brackets are "real" (that is, a bracket is required in the
syntax; it is not marking an optional part).
These brackets are enclosed in single quotes, thus:
\texttt{'['} or \texttt{']'}.
 
Note that the keywords and sub-keywords in the syntax diagrams are not
case-sensitive: the symbols "IF" "If" and "iF" would
all match the keyword shown in a syntax diagram as \texttt{if}.
 
\index{Semicolons,can be omitted}
Note also that most of the clause delimiters ("\textbf{;}") shown
can usually be omitted as they will be implied by the end of a line.
