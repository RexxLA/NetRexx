\chapter{Structured Lists Interface}\label{reflists}
\index{lists,}
\index{Structured Lists}
 
A Structured List class is used to create objects which can contain
structured or string encoded lists with associated methods for easy
access. A Structured List class is a subclass of the native NetRexx
"Rexx" class data type which implements the StructuredList interface
as described below and can be used in many cases as a normal NetRexx
data item. Assuming a class named "AStructuredList" is available to
support structured lists:

\begin{itemize}
\item A StructuredList class should have a constructor that sets the
  encoded list value
\end{itemize}
 
This creates an unexamined (unstructured) list object. The object is
an object of type Rexx with additional methods for processing lists. You
must pass a Ruleset to the list object (see the "buildlist" method
below) in order to obtain a structured version of the list object
which can be accessed with list methods. Note that due to the
immutability rule for base Rexx object values, all methods which alter
the content of a structured list return a new structured list
object. This rule also means that if a list is altered by a Java List
view, the new structured list object must be obtained from the
"currentlist" property of the Java List view. 

Structured List formats
known to \nr{} include 
\begin{itemize}
\item WORDLIST
\item DSV
\item CSV
\item TSV
\item XML
\item JSON
\item PYTHON
\end{itemize}

\section{Essential List Processing Methods}
This section lists the methods provided by the \emph{StructuredList}
class.

\begin{description}
\item[buildlist(ruleset)]

Returns a structured list form of the encoded list contained in the object's string value after examination with the provided ruleset.

\item[join(anotherStructuredList)]

Returns a structured list containing the elements of the structured list with the elements from another structured list appended to it.

\item[islist]

Returns 1 if the item is a structured list, otherwise returns 0. If a structured list is empty, index "elements" will have a 0 value. 

\item[elementcount]

Returns the count of elements in a structured list.
    
\item[getelement(index)]

Returns the element at the specified location in a structured list.

\item[getJavaList]

Returns a Java List interface view of the structured list. 

\item[index(element,start)]

Returns the numeric index of the first occurence of the given element in the list with index equal to or higher than start or 0 if not found.

\item[insertelement(index, element)]

Adds an element to a list at the specified location. Returns modified list.

\item[replaceanelement(index, element)]

Replaces an element of a list at the specified location. Returns modified list.


\item[deleteanelement(index)]

Removes an element from a list at the specified location. Returns modified list.

\item[sublist(fromIndex,toIndex)]

Creates a list which is a subset of the list containing the elements starting at the from index and ending at the to index. 
\end{description}

\section{Convenience Methods}

\begin{description}
\item[append(element)]

Adds an element to the end of a list. Returns modified list.

\item[head]

Returns the first element in a list.

\item[tail]

Returns a list containing all elements except the first in a list.

\item[count(value)]

Returns a count of how many elements have the provided value.

\item[minval]

Returns the lowest value in a list. This is a runtime error if not all list elements are numeric.

\item[maxval]

Returns the highest value in a list. This is a runtime error if not all list elements are numeric.

\item[sum]

Returns the sum of all list elements. This is a runtime error if not all list elements are numeric.

\item[reverselist]

Returns a structured list with the order of the elements reversed.

\item[flatlist]

Returns a structured list with only the list elements (all metadata is
removed). Nested sublists are also flattened. 

\end{description}


\section{NetRexx Structured List Format}
A structured list is an ordered sequence of items stored in a NetRexx
native data object (class Rexx) along with meta data describing the
list. Each element of the list can be accessed with a whole number
ranging from 1 to n where n is the number of elements in the list. The
string encoded (aka "serialized") form of the string is the base value
of the Rexx object. If an element of a list is itself a list (ie a
nested list), then elements of the sublist can also be accessed by
whole number indexes so that for example the 3rd element of the nested
list which is the second element of the first list could be found at
myList[2,3] where myList is the object containing the structured
list. Note that although a structured list is a Rexx object, changing
the list directly rather than through the StructuredList interface
methods will cause loss of metadata and unpredictable results for
further list access.

\subsection{Accessing structured lists with NetRexx facilities}
Assuming myList is a Rexx object containing a structured list, then:
\begin{description}


\item[myList["elements"]]

Provides a count of the number of elements in a structured list. This is meaningless if the item is not actually in the structured list format (ie if islist returns 0). 

\item[myList[n,m]]

A list element can be accessed by numeric index using NetRexx indexed variable syntax. If the element is also a list, sub-elements can be accessed using the multiple index syntax.
\end{description}

\subsubsection{Additional indexed values}
\begin{description}
\item["rules"]
The rules used to structure this list (the rules are also a structured list).
The following indexes must be prefixed with "\#" to use if they are whole numbers according to the NetRexx datatype BIF - this avoids conflict with list index numbers.
\item[elementname]
If an element has a name, using the name as an index will return the associated element value.

\item[element]
The element string will return the index of the first location of that element in the list.
elementvalue
If an element has a name the named value string will return the index of the first location of that value in the list.
\end{description}

\section{Structured List Ruleset Description}

A list ruleset specifies a set of delimiters and options and can be provided in one of four ways:
\begin{enumerate}
\item A null ruleset (or the string "default") signifies a default ruleset. Default rules are start/end delimiters "(" and ")", a separator " " (a blank), an escape character " (double quote), and the option "escape is quoted string mode".

\item A string such as "CSV", which is a well known list format name, selects a built-in ruleset. For example, a list in CSV format could be decoded like this: inputlist=inputstring.getlist("CSV"). Formats known to a Structured List include "WORDLIST", "DSV", "CSV", "TSV", "XML", "JSON", "PYTHON".

\item A string which provides a human readable custom set of list rules that is itself a decomposable list according to the default ruleset. Rulesets contain two sections: delimiters and options which are specified as separate sublists as in the following example: 

\begin{verbatim}
'delimiters(start("<") end(">") separator(",") meta("/") escape("\\")
 nameseparator("=") ) 
 options(separators-must-follow-sublists "adjacent separators reduce to one")'
\end{verbatim}
\item A ruleset string that is itself an encoded list according to a
  known ruleset can simply be preparsed before use like in this
  example: 
\begin{verbatim}
inputlist=inputstring.buildlist(rulesetstring.buildlist("default"))
\end{verbatim}
\end{enumerate}

\subsection{Delimiters}

Any type of delimiter can be specified but the following (along with the ruleset options) define the structure of a list. Other delimiters can be provided and will be recognized and recorded as list elements when they occur which means they are "defacto" separators for the list elements.

\begin{description}
\item[Start]

A delimiter which signals the start of a sublist. Example: start("{")

\item[End]

A delimiter which signals the end of a sublist. Example: End("}")

\item[Escape] 

An escape character used to include delimiters in the list data elements. Example: escape("||")

\item[Separator]

A delimiter used to separate list elements. Example: Separator(",")

\item[NameSeparator]

A delimiter used to associate element names with element values. Example: Namesep("=")
\end{description}

\subsection{Options}
The following options are recognized (A "0" in front of the option indicates a default value and a "1" indicates an option which overrides the default value. The additional descriptions in parenthesis are not part of the option. Options can be specified as quoted text or with dashes substituted for spaces. Options can be abbreviated as long as they are unique. Options are not case-sensitive.
\begin{description}
\item[option 1]
\begin{verbatim}
0 = separators follow sublists
1 = sublists are separators
\end{verbatim}
\item[option 2]
\begin{verbatim}
0 = adjacent separators reduce to one
1 = produce empty elements for adjacent separators
\end{verbatim}
\item[option 3]
\begin{verbatim}
0 = translate escape sequences
1 = do not translate escape sequences
\end{verbatim}
\item[option 4]
\begin{verbatim}
0 = whitespace is translated to blank (TAB,FF,LF,CR,VT)
1 = treat whitespace as text
\end{verbatim}
\item[option 5]
\begin{verbatim}
0 = escape is quoted string mode (ie "text, delimiters or double escape like this: "" more text")
1 = single character escape (ie \x)
\end{verbatim}
\item[option 6]
\begin{verbatim}
0 = attribute names are implicit (ie fun(x,y) )
1 = explicit attribute names (ie with delimiter as in fun=(x,y) or
fun:(x,y) )
\end{verbatim}
\item[option 7]
\begin{verbatim}
0 = delimiters are implicit (do not record structural delimiters)
1 = tokens are delimiters (save delimiters as separate elements)
\end{verbatim}
\item[option 8]
\begin{verbatim}
0 = keep leading and trailing whitespace
1 = strip leading and trailing whitespace
\end{verbatim}
\end{description}
