\chapter{Nop instruction}
\index{NOP instruction,}
\index{Instructions,NOP}
\index{Null instruction, NOP,}
\index{Dummy instruction, NOP,}
\begin{shaded}
\begin{alltt}
\textbf{nop};
\end{alltt}
\end{shaded}
 \keyword{nop} is a dummy instruction that has no effect.  It can be
useful as an explicit "do nothing" instruction following a
\keyword{then} or \keyword{else} clause.
 \textbf{Example:}
\begin{alltt}
select
  when a=b then nop           -- Do nothing
  when a>b then say 'A > B'
  otherwise     say 'A < B'
  end
\end{alltt}
\textbf{Note: }Putting an extra semicolon instead of the \keyword{nop} would
merely insert a null clause, which would just be ignored by \nr{}.
The second \keyword{when} clause would then immediately follow the
\keyword{then}, and hence would be reported as an error.
\keyword{nop} is a true instruction, however, and is therefore a valid
target for the \keyword{then} clause.
