\chapter{Characters and Encodings}\label{refencod}
\index{Character,encodings}
\index{Encodings, of characters,}
\index{Character,}
\index{Coded character,}
\index{Encodings,of characters}
 In the definition of a programming language it is important to
emphasize the distinction between a \emph{character} and the
\emph{coded representation}
\footnote{
These terms have the meanings as defined by the International
Organization for Standardization, in ISO 2382 :cit.Data processing
- Vocabulary:ecit..
}
(encoding) of a character.
The character "A", for example, is the first letter of the English
(Roman) alphabet, and this meaning is independent of any specific coded
representation of that character.
Different coded character sets (such as, for example, the ASCII
\footnote{
\index{ASCII,coded character set}
\index{Coded character set,ASCII}
American Standard Code for Information Interchange.
}
and EBCDIC
\footnote{
\index{EBCDIC,coded character set}
\index{Coded character set,EBCDIC}
Extended Binary Coded Decimal Interchange Code.
}
codes) use quite different encodings for this character (decimal
values 65 and 193, respectively).
 Except where stated otherwise, this
book
uses characters to convey meaning and not to imply a specific character
code (the exceptions are certain operations that specifically convert
between characters and their representations). At no time is \nr{} concerned with the glyph (actual appearance) of
a character.
\index{Character,appearance}
\index{Character,glyphs}
\index{Glyphs,}
\section{Character Sets}
\index{Character sets,}
\index{Unicode,coded character set}
\index{Coded character set,Unicode}
 Programming in the \nr{} language can be considered to involve the
use of two character sets.
The first is used for expressing the \nr{} program itself, and is the
relatively small set of characters described in the next section.
The second character set is the set of characters that can be used as
character data by a particular implementation of a \nr{} language
processor.
This character set may be limited in size (sometimes to a limit of 256
different characters, which have a convenient 8-bit representation), or
it may be much larger.  The \emph{Unicode}
\footnote{
\emph{The Unicode Standard, version 6.0}.,
The Unicode Consortium, Mountain View, 2011, ISBN 09781936213016.
}
character set, for example, allows for 1,114,112 code points, of which
currently 128,000 are defined as characters. These are represented,
depending on the serialization format, in one to four bytes.
 
Usually, most or all of the characters in the second (data) character
set are also allowed within a \nr{} program, but only within
commentary or immediate (literal) data.
 The \nr{} language explicitly defines the first character set, in
order that programs will be portable and understandable; at the same
time it avoids restrictions due to the language itself on the character
set used for data.
However, where the language itself manipulates or inspects the data (as
when carrying out arithmetic operations), there may be requirements on
the data character set (for example, numbers can only be expressed if
there are digit characters in the set).
