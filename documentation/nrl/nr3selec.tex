\chapter{Select instruction}
\index{,}
\index{Instructions,SELECT}
\index{THEN,following WHEN clause}
\index{,}
\index{,}
\index{,}
\index{Flow control,with SELECT construct}
\begin{shaded}
\begin{alltt}
\textbf{select} [\textbf{label} \emph{name}] [\textbf{protect} \emph{term}] [\textbf{case} \emph{expression}];
        \emph{whenlist}
        [\textbf{otherwise}[;] \emph{instructionlist}]
    [\textbf{catch} [\emph{vare} =] \emph{exception};
        \emph{instructionlist}]...
    [\textbf{finally}[;]
        \emph{instructionlist}]
    \textbf{end} [\emph{name}];

where \emph{name} is a non-numeric \emph{symbol}

and \emph{whenlist} is one or more \emph{whenconstruct}s

and \emph{whenconstruct} is:

    \textbf{when} \emph{expression}[, \emph{expression}]... [;] \textbf{then}[;] \emph{instruction}

and \emph{instructionlist} is zero or more \emph{instruction}s.
\end{alltt}
\end{shaded}
\index{Body,of select}
 \texttt{select} is used to conditionally execute one of several
alternatives.
The construct may optionally be given a label, and may protect an object
while the instructions in the construct are executed; exceptional
conditions can be handled with \texttt{catch} and \texttt{finally},
which follow the body of the construct.
 
Starting with the first \texttt{when} clause, each expression in
the clause is evaluated in turn from left to right, and if the
result of any evaluation is 1 (or equals the \texttt{case}
expression, see below) then the test has succeeded and the
instruction following the associated \texttt{then} (which may be
a complex instruction such as \texttt{if}, \texttt{do},
\texttt{loop}, or \texttt{select}) is executed and control will
then pass directly to the \texttt{end}.
 
If the result of all the expressions in a \texttt{when} clause
is 0, control will pass to the next \texttt{when} clause.
 
Note that once an expression evaluation in a \texttt{when}
clause has resulted in a successful test, no further expressions
in the clause are evaluated.
 
If none of the \texttt{when} expressions result in 1, then control will
pass to the instruction list (if any) following \texttt{otherwise}.
In this situation, the absence of an \texttt{otherwise} is a run-time
error.
\footnote{
\emph{In the reference implementation, a \textbf{NoOtherwiseException}
is raised.}
}
 \textbf{Notes:}
\begin{enumerate}
\item An \emph{instruction} may be any assignment, method call, or keyword
instruction, including any of the more complex constructions such as
\texttt{do}, \texttt{loop}, \texttt{if}, and the \texttt{select}
instruction itself.
A null clause is not an instruction, however, so putting an extra
semicolon after the \texttt{then} is not equivalent to putting a dummy
instruction (as it would be in C or PL/I).
The \texttt{nop} instruction is provided for this purpose.
\item The keyword \texttt{then} is treated specially, in that it need not
start a clause.
This allows the expression on the \texttt{when} clause to be terminated
by the \texttt{then}, without a "\textbf{;}" being required
- were this not so, people used to other computer languages would
be inconvenienced.
Hence the symbol \texttt{then} cannot be used as a variable name within
the expression.
\footnote{
Strictly speaking, \texttt{then} should only be recognized if not
the name of a variable.  In this special case, however, NetRexx language
processors are permitted to treat \texttt{then} as reserved in the
context of a \texttt{when} clause, to provide better performance and
more useful error reporting.
}
\end{enumerate}
\subsubsection{Label phrase}
\index{Select,label}
\index{Select,naming of}
 
\index{LABEL,on SELECT instruction}
If \texttt{label} is used to specify a \emph{name} for the select
group, then a  \texttt{leave} instruction (see page \pageref{refleave})  which
specifies that name may be used to leave the group, and the \texttt{end}
that ends the group may optionally specify the name of the group for
additional checking.
 \textbf{Example:}
\begin{alltt}
select label roman
  when a=b then say 'same'
  when a<b then say 'lo'
  otherwise
    say 'hi'
    if a=0 then leave roman
    say 'a non-0'
  end roman
\end{alltt}
In this example, if the variable \textbf{a} has the value 0
and \textbf{b} is negative then just "\textbf{hi}" is
displayed.
\subsubsection{Protect phrase}
 
\index{PROTECT,on SELECT instruction}
If \texttt{protect} is given it must be followed by a \emph{term}
that evaluates to a value that is not just a type and is not of a
primitive type;
while the \texttt{select} construct is being executed, the value
(object) is protected - that is, all the instructions in the
\texttt{select} construct have exclusive access to the object.
 
Both \texttt{label} and \texttt{protect} may be specified, in any order,
if required.
\subsubsection{Case phrase}
 
\index{CASE,on SELECT instruction}
If \texttt{case} is given it must follow any \texttt{label} or
\texttt{protect} phrase, and must be followed by an
\emph{expression}.
 
When \texttt{case} is used, the expression following it is evaluated at
the start of the \texttt{select} construct.
The result of the expression is then compared, using the strict equality
operator (:m.==:em.), to the result of evaluating the expression
or expressions in each of the \texttt{when} clauses in turn until
a match is found.  As usual, if no match is found then control
will pass to the instruction list (if any) following
\texttt{otherwise}, and in this situation the absence of an
\texttt{otherwise} is a run-time error.
 For example, in:
\begin{alltt}
select case i+1
  when 1 then say 'one'
  when 1+1 then say 'two'
  when 3, 4, 5 then say 'many'
end
\end{alltt}
then if :m.i:em. had the value 1 then the message displayed would be
":m.two:em.".
 
The third \texttt{when} clause in the example demonstrates the use of the
multiple expressions in a \texttt{when} clause in this context.
Similar to a \texttt{select} without \texttt{case}, each
expression is evaluated in turn from left to right and is then
compared to the result of the \texttt{case} expression.
As soon as one matches that result, execution of the
\texttt{when} clause stops (any further expressions are not
evaluated) and the instruction following the associated
\texttt{then} clause is executed.
 \textbf{Notes:}
\begin{enumerate}
\item When \texttt{case} is used, the result of evaluating the expression
following each \texttt{when} no longer has to be 0 or 1.  Instead, it
must be possible to compare each result to the result of the
\texttt{case} expression.
\item 
The \texttt{case} expression is evaluated only on entry to the
\texttt{select} construct; it is not re-evaluated for each \texttt{when}
clause.
\item 
An exception raised during evaluation of the \texttt{case} expression
will be caught by a suitable \texttt{catch} clause in the construct, if
one is present.
Similarly, evaluation of the \texttt{case} expression is protected by
the \texttt{protect} phrase, if one is present.
\item 
\emph{In the reference implementation, a \texttt{select case} construct will
be translated into a Java :m.switch:em. construct provided that it
meets the following criteria:}
\begin{itemize}
\item 
\emph{The type of the \texttt{case} expression
is :m.byte:em., :m.char:em., :m.int:em., or :m.short:em..}
\item 
\emph{The value of all the expressions on the \texttt{when} clauses are
primitive constants (that is, they consist of only constants of
primitive types and operators valid for them and so may be evaluated at
compile time).}
\item 
\emph{No two expressions on the \texttt{when} clauses evaluate to the same
value.}
\item 
\emph{It is not subject to tracing.}
\end{itemize}
\emph{Under these conditions the semantics of the :m.switch:em. construct
match those defined for \texttt{select}.  The example shown above would
be translated to a :m.switch:em. construct if :m.i:em. had type :m.int:em.
and \texttt{options binary} were in effect.}
\end{enumerate}
\subsubsection{Exceptions in select constructs}
 
\index{CATCH,on SELECT instruction}
\index{FINALLY,on SELECT instruction}
Exceptions that are raised by the instructions within the body of the
group, or during evaluation of the \texttt{case} expression, may be
caught using one or more \texttt{catch} clauses that name
the \emph{exception} that they will catch.
When an exception is caught, the exception object that holds the details
of the exception may optionally be assigned to a variable,
\emph{vare}.
 
Similarly, a \texttt{finally} clause may be used to introduce
instructions that will always be executed at the end of the select
group, even if an exception is raised (whether caught or not).
 
The  \emph{Exceptions} section (see page \pageref{refexcep})  has details and
examples of \texttt{catch} and \texttt{finally}.
\index{,}
