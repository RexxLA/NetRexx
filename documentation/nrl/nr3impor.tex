\chapter{Import instruction}\label{"id"}
\index{IMPORT instruction,}
\index{Instructions,IMPORT}
\begin{shaded}
\begin{alltt}
\textbf{import} \emph{name};

where \emph{name} is one or more non-numeric \emph{symbol}s separated by periods, with an optional trailing period.
\end{alltt}
\end{shaded}
 
The \texttt{import} instruction is used to simplify the use of
classes from other packages.
If a class is identified by an \texttt{import} instruction, it can then
be referred to by its short name, as given on the
 \texttt{class} instruction (see page \pageref{refclass}) , as well as by its fully
qualified name.
 
There may be zero or more \texttt{import} instructions in a program.
They must precede any \texttt{class} instruction (or any instruction
that would start the default class).
 
\index{Package,name of}
In the following description, a \emph{package name} names a package
as described under the  \texttt{package} (see page \pageref{refpackage}) 
instruction:ea..
The import \emph{name} must be one of:
\begin{itemize}
\item A qualified class name, which is a package name immediately followed
by a period which is immediately followed by a short class name -
in this case, the individual class identified is imported.
\item A package name - in this case, all the classes in the specified
package are imported.  The name may have a trailing period.
\item A partial package name (a package name with one or more parts
omitted from the right, indicated by a trailing period after the parts
that are present) - in this case, all classes in the package hierarchy
below the specified point are imported.
\end{itemize}
 \textbf{Examples:}
\begin{alltt}
import java.lang.String
import java.lang
import java.
\end{alltt}
The first example above imports a single class (which could then be
referred to simply as "\textbf{String}").
The second example imports all classes in the
"\textbf{java.lang}" package.
The third example imports all classes in all the packages whose name
starts with "\textbf{java.}".
 
\index{Imports,explicit}
When a class is imported explicitly, for example, using
\begin{alltt}
import java.awt.List
\end{alltt}
this indicates that the short name of the class (\texttt{List},
in this example) may be used to refer to the class unambiguously.
That is, using this short name will not report an ambiguous reference
warning (as it would without the \texttt{import} instruction, because
a \texttt{java.util.List} class was added in Java 1.2).
 
It follows that:
\begin{itemize}
\item Two classes imported explicitly cannot have the same short name.
\item No class in a program being compiled can have the same short name as
a class that is imported explicitly.
\end{itemize}
because in either of these situations a use of the short name would
be ambiguous.
 
Note also that an explicit import does not import the minor or dependent
classes associated with a name; they each require their own explicit
import (unless the entire package is imported).
 
\index{Imports,automatic}
\emph{In the reference implementation, the fundamental NetRexx and Java
package hierarchies are automatically imported by default, as though the
instructions:}
\begin{alltt}
import netrexx.lang.
import java.lang.
import java.io.
import java.util.
import java.net.
import java.awt.
import java.applet.
\end{alltt}
\emph{had been executed before the program begins.
In addition, classes in the current (working) directory are imported if
no \texttt{package} instruction is specified.  If a \texttt{package}
instruction is specified then all classes in that package are imported.
}
