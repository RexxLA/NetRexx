\chapter{Address instruction}\label{refparse}
\index{Address,instruction}
\index{Instructions,Address}
\begin{shaded}
\begin{alltt}
\textbf{address} \emph{[environment]} \emph{[expression]}

where \emph{environment} is a native executable or script on the path

\end{alltt}
\end{shaded}
The keyword \emph{address} temporarily or permanently changes the destination of commands. Commands are strings sent to an external environment. You can send commands by specifying clauses consisting of only an expression or by using the \keyword{Address} instruction.

To send a single command to a specified environment, code an
environment, a literal string or a single symbol, which is taken to be
a constant, followed by an expression.

The environment name is the
name of an external procedure or process that can process
commands.

The expression is evaluated to produce a character string
value, and this string is routed to the environment to be processed as
a command.

After execution of the command, environment is set back to
its original state, thus temporarily changing the destination for a
single command. The default environment is \keyword{SYSTEM}, which is the shell
on most operating systems.

After execution, the most recent return code is in the variable \keyword{RC}.