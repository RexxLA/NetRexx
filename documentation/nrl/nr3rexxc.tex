\chapter{Methods for \nr{} strings}\label{refbmeth}
\index{R\textsc{exx},class/methods of}
 This section describes the set of methods defined for the \nr{}
string class, \textbf{R\textsc{exx}}.  These are called \emph{built-in
methods}, and include character manipulation, word manipulation,
conversion, and arithmetic methods.
 
Implementations will also provide other methods for the \textbf{R\textsc{exx}}
class (for example, to implement the \nr{} operators or to provide
constructors with primitive arguments), but these are not part of the
\nr{} language.
\footnote{
\emph{Details of the methods provided in the reference implementation are
included in  Appendix C (see page \pageref{refappc}) .}
}
\section{General notes on the built-in methods:}
\begin{enumerate}
\item All methods work on a \nr{} string of type \textbf{R\textsc{exx}}; this
is referred to by the name \emph{string} in the descriptions of the
methods.  For example, if the \textbf{word} method were invoked using
the term:
\begin{alltt}
"Three word phrase".word(2)
\end{alltt}
then in the description of \textbf{word} the name
\emph{string} refers to the string "\textbf{Three word
phrase}", and the name \emph{n} refers to the string
"\textbf{2}".
\item All method arguments are of type \textbf{R\textsc{exx}} and all methods
return a string of type \textbf{R\textsc{exx}}; if a number is returned, it
will be formatted as though 0 had been added with no rounding.
\item 
The first parenthesis in a method call must immediately follow the name
of the method, with no space in between.
\item The parentheses in a method call can be omitted if no
arguments are required and the method call is part of a
 \emph{compound term} (see page \pageref{refcomterm}) .
\footnote{
Unless an implementation-provided option to disallow parenthesis
omission is in force.
}
\item A position in a string is the number of a character in the string,
where the first character is at position 1, \emph{etc.}
\item Where arguments are optional, commas may only be included between
arguments that are present (that is, trailing commas in argument lists
are not permitted).
\item A \emph{pad} argument, if specified, must be exactly one
character long.
\item If a method has a sub-option selected by the first character of a
string, that character may be in upper or lowercase.
\item Conversion between character encodings and decimal or hexadecimal
is dependent on the machine representation (encoding) of characters
and hence will return appropriately different results for Unicode,
ASCII, EBCDIC, and other implementations.
\end{enumerate}
\section{The built-in methods}\label{builtinmethods}
\begin{description}
\item[abbrev(info [,length{]})]\label{refabbrev}
\index{ABBREV method,}
\index{Method, built-in,ABBREV}
\index{Abbreviations,testing with ABBREV method}
returns 1 if \emph{info} is equal to the leading characters of
\emph{string} and \emph{info} is not less than
the minimum length, \emph{length}; 0 is returned
if either of these conditions is not met.
\emph{length} must be a non-negative whole number; the default is
the length of \emph{info}.
 \textbf{Examples:}
\begin{alltt}
'Print'.abbrev('Pri')   == 1
'PRINT'.abbrev('Pri')   == 0
'PRINT'.abbrev('PRI',4) == 0
'PRINT'.abbrev('PRY')   == 0
'PRINT'.abbrev('')      == 1
'PRINT'.abbrev('',1)    == 0
\end{alltt}
\textbf{Note: }A null string will always match if a length of 0 (or the default)
is used.
This allows a default keyword to be selected automatically if desired.
 \textbf{Example:}
\begin{alltt}
say 'Enter option:';  option=ask
select  /* keyword1 is to be the default */
  when 'keyword1'.abbrev(option) then ...
  when 'keyword2'.abbrev(option) then ...
     ...
  otherwise ...
  end
\end{alltt}
\item[abs()]\label{refabs}
\index{ABS method,}
\index{Method, built-in,ABS}
\index{Mathematical method,ABS}
\index{Absolute,value, finding using ABS method}
returns the absolute value of \emph{string}, which must be a
number.
 Any sign is removed from the number, and it is then formatted by adding
zero with a digits setting that is either nine or, if greater, the
number of digits in the mantissa of the number (excluding leading
insignificant zeros).
Scientific notation is used, if necessary.
 
\textbf{Examples:}
\begin{alltt}
'12.3'.abs              == 12.3
' -0.307'.abs           == 0.307
'123.45E+16'.abs        == 1.2345E+18
'- 1234567.7654321'.abs == 1234567.7654321
\end{alltt}

\item[b2d([n{]})]\label{refb2x}
\index{B2D method}
\index{Packing a string,with B2D}
\index{Method, built-in,B2D}
\index{Conversion,binary to decimal}
\index{Binary,conversion to decimal}
\marginnote{\color{gray}3.02} Binary to decimal.
Converts \emph{string}, a string of at least one binary
(\textbf{0} and/or \textbf{1}) digits, to an equivalent string of
decimal characters (a number), without rounding.
The returned string will use digits,
and will not include any blanks.
 If the number of binary digits in the string is not a multiple of four,
then up to three \textbf{'0'} digits will be added on the left before
conversion to make a total that is a multiple of four.
 If \emph{string} is the null string, 0 is returned. If n is not specified, \emph{string} is taken to be an unsigned number.

\textbf{Examples:}
\begin{alltt}
'01110'.b2d == 14 
'10000001'.b2d == 129 
'111110000001'.b2d == 3969 
'1111111110000001'.b2d == 65409 
'1100011011110000'.b2d == 50928 
\end{alltt}
If n is specified, string is taken as a signed number expressed in n binary characters. If the most significant (left-most) bit is zero then the number is positive; otherwise it is a negative number in twos-complement form. In both cases it is converted to a NetRexx number which may, therefore, be negative. If n is 0, 0 is always returned.

If necessary, string is padded on the left with '0' characters (note, not “signextended”), or truncated on the left, to length n characters; (that is, as though string.right(n, '0') had been executed.)

\textbf{Examples:}
\begin{alltt}
'10000001'.b2d(8) == -127 
'10000001'.b2d(16) == 129 
'1111000010000001'.b2d(16) == -3967 
'1111000010000001'.b2d(12) == 129 
'1111000010000001'.b2d(8) == -127 
'1111000010000001'.b2d(4) == 1 
'0000000000110001'.b2d(0) == 0
\end{alltt} 
\item[b2x()]\label{refb2x}
\index{B2X method}
\index{Packing a string,with B2X}
\index{Method, built-in,B2X}
\index{Conversion,binary to hexadecimal}
\index{Binary,conversion to hexadecimal}
Binary to hexadecimal.
Converts \emph{string}, a string of at least one binary
(\textbf{0} and/or \textbf{1}) digits, to an equivalent string of
hexadecimal characters.
The returned string will use uppercase Roman letters for the values A-F,
and will not include any blanks.
 If the number of binary digits in the string is not a multiple of four,
then up to three \textbf{'0'} digits will be added on the left before
conversion to make a total that is a multiple of four.
 
\textbf{Examples:}
\begin{alltt}
'11000011'.b2x  == 'C3'
'10111'.b2x     == '17'
'0101'.b2x      == '5'
'101'.b2x       == '5'
'111110000'.b2x == '1F0'
\end{alltt}
  
\item[center(length [,pad{]})]\label{refcenter}

\emph{or}
\item[centre(length [,pad{]})]\label{refcentre}
\index{CENTRE method,}
\index{CENTER method,}
\index{Method, built-in,CENTRE}
\index{Method, built-in,CENTER}
\index{Formatting,text centering}

returns a string of length \emph{length} with \emph{string}
centered in it, with \emph{pad} characters added as necessary to
make up the required length.
\emph{length} must be a non-negative whole number.
The default \emph{pad} character is blank.
If the string is longer than \emph{length}, it will be truncated at
both ends to fit.
If an odd number of characters are truncated or added, the right hand
end loses or gains one more character than the left hand end.
 
\textbf{Examples:}
\begin{alltt}
'ABC'.centre(7)          == '  ABC  '
'ABC'.center(8,'-')      == '--ABC---'
'The blue sky'.centre(8) == 'e blue s'
'The blue sky'.center(7) == 'e blue '
\end{alltt}
\textbf{Note: }This method may be called either \textbf{centre} or \textbf{center},
which avoids difficulties due to the difference between the British and
American spellings.
\item[changestr(needle, new)]\label{refchastr}
\index{CHANGESTR method,}
\index{Method, built-in,CHANGESTR}
\index{Replacing strings,using CHANGESTR}
\index{Changing strings,using CHANGESTR}

returns a copy of \emph{string} in which each occurrence of the
\emph{needle} string is replaced by the \emph{new} string.
Each unique (non-overlapping) occurrence of the \emph{needle} string
is changed, searching from left to right and starting from the first
(leftmost) position in \emph{string}.
Only the original \emph{string} is searched for the
\emph{needle}, and each character in \emph{string} can only be
included in one match of the \emph{needle}.
 
If the \emph{needle} is the null string, the result is a copy of
\emph{string}, unchanged.
 
\textbf{Examples:}
\begin{alltt}
'elephant'.changestr('e','X')    == 'XlXphant'
'elephant'.changestr('ph','X')   == 'eleXant'
'elephant'.changestr('ph','hph') == 'elehphant'
'elephant'.changestr('e','')     == 'lphant'
'elephant'.changestr('','!!')    == 'elephant'
\end{alltt}
 The  \textbf{countstr} method (see page \pageref{refcoustr})  can be used to
count the number of changes that could be made to a string in this
fashion.
\item[compare(target [,pad{]})]\label{refcompar}
\index{COMPARE method,}
\index{Method, built-in,COMPARE}
\index{Finding a mismatch using COMPARE,}
\index{Comparison,of strings/using COMPARE}

returns 0 if \emph{string} and \emph{target}
are the same.
If they are not, the returned number is positive and is the position of
the first character that is not the same in both strings.
If one string is shorter than the other, one or more \emph{pad}
characters are added on the right to make it the same length for the
comparison.
The default \emph{pad} character is a blank.
 
\textbf{Examples:}
\begin{alltt}
'abc'.compare('abc')      == 0
'abc'.compare('ak')       == 2
'ab '.compare('ab')       == 0
'ab '.compare('ab',' ')   == 0
'ab '.compare('ab','x')   == 3
'ab-- '.compare('ab','-') == 5
\end{alltt}
\item[copies(n)]\label{refcopies}
\index{COPIES method,}
\index{Method, built-in,COPIES}
\index{Copying a string using COPIES,}
\index{Repeating a string with COPIES,}
returns \emph{n} directly concatenated copies of
\emph{string}.
\emph{n} must be positive or 0; if 0, the null string is returned.
 
\textbf{Examples:}
\begin{alltt}
'abc'.copies(3) == 'abcabcabc'
'abc'.copies(0) == ''
''.copies(2)    == ''
\end{alltt}
\item[copyindexed(sub)]\label{refcopyind}
\index{COPYINDEXED method,}
\index{Method, built-in,COPYINDEXED}
\index{Indexed strings,merging}
\index{Indexed strings,copying}
\index{Copying indexed variables,}
\index{Merging indexed variables,}
copies the collection of indexed  sub-values (see page \pageref{refinstr}) 
of \emph{sub} into the collection associated with
\emph{string}, and returns the modified \emph{string}.  The
resulting collection is the union of the two collections (that is,
it contains the indexes and their values from both collections).
If a given index exists in both collections then the sub-value of
\emph{string} for that index is replaced by the sub-value from
\emph{sub}.
 
The non-indexed value of \emph{string} is not affected.
 
\textbf{Example:}
 Following the instructions:
\begin{alltt}
foo='def'
foo['a']=1
foo['b']=2
bar='ghi'
bar['b']='B'
bar['c']='C'
merged=foo.copyIndexed(bar)
\end{alltt}
then:
\begin{alltt}
merged['a'] == '1'
merged['b'] == 'B'
merged['c'] == 'C'
merged['d'] == 'def'
\end{alltt}


\index{COUNTSTR method,}
\index{Method, built-in,COUNTSTR}
\index{Counting,strings, using COUNTSTR}
\item[countstr(needle)]\label{refcoustr}
returns the count of non-overlapping occurrences of the
\emph{needle} string in \emph{string}, searching from left to
right and starting from the first (leftmost) position in
\emph{string}.
 
If the \emph{needle} is the null string, \textbf{0} is returned.
 
\textbf{Examples:}
\begin{alltt}
'elephant'.countstr('e')  == '2'
'elephant'.countstr('ph') == '1'
'elephant'.countstr('')   == '0'
\end{alltt}
 The  \textbf{changestr} method (see page \pageref{refchastr})  can be used to
change occurrences of \emph{needle} to some other string.
\item[c2d()]\label{refc2d}
\index{C2D method,}
\index{Method, built-in,C2D}
\index{Conversion,coded character to decimal}
\index{Conversion,character to decimal}
\index{Character,conversion to decimal}
\index{Coded character,conversion to decimal}

Coded character to decimal.
Converts the encoding of the character in \emph{string} (which must be
exactly one character) to its decimal representation.
The returned string will be a non-negative number that represents
the encoding of the character and will not include any sign, blanks,
insignificant leading zeros, or decimal part.
 
\textbf{Examples:}
\begin{alltt}
'M'.c2d  == '77'  -- ASCII or Unicode
'7'.c2d  == '247' -- EBCDIC
'\textbackslash{}r'.c2d == '13'  -- ASCII or Unicode
'\textbackslash{}0'.c2d == '0'
\end{alltt}
 The  \textbf{c2x} method (see page \pageref{refc2x})  can be used to
convert the encoding of a character to a hexadecimal representation.
\item[c2x()]\label{refc2x}
\index{C2X method,}
\index{Unpacking a string,with C2X}
\index{Method, built-in,C2X}
\index{Conversion,coded character to hexadecimal}
\index{Conversion,character to hexadecimal}
\index{Character,conversion to hexadecimal}
\index{Coded character,conversion to hexadecimal}

Coded character to hexadecimal.
Converts the encoding of the character in \emph{string} (which must be
exactly one character) to its hexadecimal representation (unpacks).
The returned string will use uppercase Roman letters for the values A-F,
and will not include any blanks.
Insignificant leading zeros are removed.
 
\textbf{Examples:}
\begin{alltt}
'M'.c2x  == '4D' -- ASCII or Unicode
'7'.c2x  == 'F7' -- EBCDIC
'\textbackslash{}r'.c2x == 'D'  -- ASCII or Unicode
'\textbackslash{}0'.c2x == '0'
\end{alltt}
 The  \textbf{c2d} method (see page \pageref{refc2d})  can be used to
convert the encoding of a character to a decimal number.
\item[datatype(option)]\label{refdataty}
\index{DATATYPE method,}
\index{Method, built-in,DATATYPE}
\index{Mathematical method,DATATYPE options}
\index{Types,checking with DATATYPE}
\index{Numbers,checking with DATATYPE}
\index{Whole numbers,checking with DATATYPE}
\index{Alphanumerics,checking with DATATYPE}
\index{Alphabetics,checking with DATATYPE}
\index{Letters,checking with DATATYPE}
\index{Bits,checking with DATATYPE}
\index{Binary,checking with DATATYPE}
\index{Digits,checking with DATATYPE}
\index{Lowercase,checking with DATATYPE}
\index{Mixed case,checking with DATATYPE}
\index{Whole numbers,checking with DATATYPE}
\index{Numbers,checking with DATATYPE}
\index{Symbol characters,checking with DATATYPE}
\index{Uppercase,checking with DATATYPE}
\index{Hexadecimal,checking with DATATYPE}
\index{,}
\index{,}
\index{,}

returns 1 if \emph{string} matches the description requested with
the \emph{option}, or 0 otherwise.
If \emph{string} is the null string, 0 is always returned.
 
Only the first character of \emph{option} is significant, and it may
be in either uppercase or lowercase.
The following \emph{option} characters are recognized:
\begin{description}
\item[A]\label{refdta}
(Alphanumeric); returns 1 if \emph{string} only contains
characters from the ranges "a-z", "A-Z", and "0-9".
\item[B]\label{refdtb}
(Binary); returns 1 if \emph{string} only contains the
characters "0" and/or "1".
\item[D]\label{refdtd}
(Digits); returns 1 if \emph{string} only contains
characters from the range "0-9".
\item[L]\label{refdtl}
(Lowercase); returns 1 if \emph{string} only contains
characters from the range "a-z".
\item[M]\label{refdtm}
(Mixed case); returns 1 if \emph{string} only contains
characters from the ranges "a-z" and "A-Z".
\item[N]\label{refdtn}
(Number); returns 1 if \emph{string} is a syntactically valid
\nr{} number that could be added to \textbf{'0'} without error,
\item[S]\label{refdts}
(Symbol); returns 1 if \emph{string} only contains characters
that are valid in non-numeric symbols (the alphanumeric characters and
underscore), and does not start with a digit.  Note that both uppercase
and lowercase letters are permitted.
\item[U]\label{refdtu}
(Uppercase); returns 1 if \emph{string} only contains
characters from the range "A-Z".
\item[W]\label{refdtw}
(Whole Number); returns 1 if \emph{string} is a syntactically valid
\nr{} number that can be added to \textbf{'0'} without error, and
whose decimal part after that addition, with no rounding, is zero.
\item[X]\label{refdtx}
(heXadecimal); returns 1 if \emph{string} only contains
characters from the ranges "a-f", "A-F", and "0-9".
\end{description}
 
\textbf{Examples:}
\begin{alltt}
'101'.datatype('B')    == 1
'12.3'.datatype('D')   == 0
'12.3'.datatype('N')   == 1
'12.3'.datatype('W')   == 0
'LaArca'.datatype('M') == 1
''.datatype('M')       == 0
'Llanes'.datatype('L') == 0
'3 d'.datatype('s')    == 0
'BCd3'.datatype('X')   == 1
'BCgd3'.datatype('X')  == 0
\end{alltt}
\textbf{Note: }The \textbf{datatype} method tests the meaning of the characters
in a string, independent of the encoding of those characters.  Extra
letters and Extra digits cause \textbf{datatype} to return 0 except
for the number tests ("\textbf{N}" and "\textbf{W}"),
which treat extra digits whose value is in the range 0-9 as though they
were the corresponding Arabic numeral.
\item[delstr(n [,length{]})]\label{refdelstr}
\index{DELSTR method,}
\index{Method, built-in,DELSTR}
\index{Deleting,part of a string}

returns a copy of \emph{string} with the sub-string of
\emph{string} that begins at the \emph{n}\emph{th} character, and is
of length \emph{length} characters, deleted.
If \emph{length} is not specified, or is greater than the number of
characters from \emph{n} to the end of the string, the rest of the
string is deleted (including the \emph{n}\emph{th} character).
\emph{length} must be a non-negative whole number, and \emph{n}
must be a positive whole number.  If \emph{n} is greater than the
length of \emph{string}, the string is returned unchanged.
 
\textbf{Examples:}
\begin{alltt}
'abcd'.delstr(3)    == 'ab'
'abcde'.delstr(3,2) == 'abe'
'abcde'.delstr(6)   == 'abcde'
\end{alltt}
\item[delword(n [,length])]\label{refdelword}
\index{DELWORD method,}
\index{Method, built-in,DELWORD}
\index{Deleting,words from a string}
\index{Words,deleting from a string}

returns a copy of \emph{string} with the sub-string of
\emph{string} that starts at the \emph{n}\emph{th} word, and is of
length \emph{length} blank-delimited words, deleted.
If \emph{length} is not specified, or is greater than number of
remaining words in the string, it defaults to be the remaining words
in the string (including the \emph{n}\emph{th} word).
\emph{length} must be a non-negative whole number, and \emph{n}
must be a positive whole number.  If \emph{n} is greater than the
number of words in \emph{string}, the string is returned unchanged.
The string deleted includes any blanks following the final word
involved, but none of the blanks preceding the first word involved.
 
\textbf{Examples:}
\begin{alltt}
'Now is the  time'.delword(2,2) == 'Now time'
'Now is the time '.delword(3)   == 'Now is '
'Now  time'.delword(5)          == 'Now  time'
\end{alltt}

\index{D2B method,}
\index{Method, built-in,D2B}
\index{Conversion,decimal to binary}
\index{Decimal,conversion to binary}
\index{Binary,from decimal}
\index{Binary,from decimal}
\item[d2b([n{]})]\label{refd2b}
\marginnote{\color{gray}3.02} Decimal to binary.
Returns a string of binary characters of length as needed or of length
n, which is the binary representation of the decimal number. The
returned string will use 0 and 1 characters for binary values. string
must be a whole number, and must be non-negative unless n is
specified, or an error will result. If n is not specified, the length
of the result returned is such that there are no leading 0 characters,
unless string was equal to 0 (in which case '0' is returned).

If n is specified it is the length of the final result in characters;
that is, after conversion the input string will be sign-extended to
the required length (negative numbers are converted assuming
twos-complement form). If the number is too big to fit into n
characters, it will be truncated on the left. n must be a nonnegative
whole number.

 \textbf{Examples:}
\begin{alltt}
'0'.d2b == 0 
'9'.d2b == 1001 
'19'.d2b == 10011 
'129'.d2b == 10000001 
'129'.d2b(1) == 1 
'129'.d2b(8) == 10000001 
'127'.d2b(12) == 000001111111 
'129'.d2b(16) == 0000000010000001 
'257'.d2b(8) == 00000001 
'-127'.d2b(8) == 10000001 
'-127'.d2b(16) == 1111111110000001 
'12'.d2b(0) == 
\end{alltt}

\index{D2C method,}
\index{Method, built-in,D2C}
\index{Conversion,decimal to character}
\index{Decimal,conversion to character}
\index{Coded character,from decimal}
\index{Character,from decimal}
\index{Character,from a number}
\index{Numbers,conversion to character}
\item[d2c()]\label{refd2c}
Decimal to coded character.
Converts the \emph{string} (a \nr{} \emph{number}) to a
single character, where the number is used as the encoding of the
character.
 
\emph{string} must be a non-negative whole number.
An error results if the encoding described does not produce a valid
character for the implementation (for example, if it has more
significant bits than the implementation's encoding for characters).
 
\textbf{Examples:}
\begin{alltt}
'77'.d2c  == 'M' -- ASCII or Unicode
'+77'.d2c == 'M' -- ASCII or Unicode
'247'.d2c == '7' -- EBCDIC
'0'.d2c   == '\textbackslash 0'
\end{alltt}
\item[d2x([n])]\label{refd2x}
\index{D2X method,}
\index{Method, built-in,D2X}
\index{Conversion,decimal to hexadecimal}
\index{Decimal,conversion to hexadecimal}
\index{Numbers,conversion to hexadecimal}

Decimal to hexadecimal.
Returns a string of hexadecimal characters of length as needed or of
length \emph{n}, which is the hexadecimal (unpacked) representation
of the decimal number.  The returned string will use uppercase
Roman letters for the values A-F, and will not include any blanks.
 \emph{string} must be a whole number, and must be non-negative
unless \emph{n} is specified, or an error will result.
If \emph{n} is not specified, the length of the result returned is
such that there are no leading 0 characters, unless \emph{string}
was equal to 0 (in which case \textbf{'0'} is returned).
 
If \emph{n} is specified it is the length of the final result in
characters; that is, after conversion the input string will be
sign-extended to the required length (negative numbers are converted
assuming twos-complement form).
If the number is too big to fit into \emph{n} characters, it will be
truncated on the left.
\emph{n} must be a non-negative whole number.
 
\textbf{Examples:}
\begin{alltt}
'9'.d2x       == '9'
'129'.d2x     == '81'
'129'.d2x(1)  == '1'
'129'.d2x(2)  == '81'
'127'.d2x(3)  == '07F'
'129'.d2x(4)  == '0081'
'257'.d2x(2)  == '01'
'-127'.d2x(2) == '81'
'-127'.d2x(4) == 'FF81'
'12'.d2x(0)   == ''
\end{alltt}
\item[exists(index)]\label{refexists}
\index{EXISTS method,}
\index{Method, built-in,EXISTS}
\index{Index strings,testing for}
\index{Indexed strings,testing for}
\index{Testing for indexed variables,}

returns 1 if \emph{index} names a  sub-value (see page \pageref{refinstr})  of
\emph{string} that has explicitly been assigned a value, or 0
otherwise.
 
\textbf{Example:}
 Following the instructions:
\begin{alltt}
vowel=0
vowel['a']=1
vowel['b']=1
vowel['b']=null -- drops previous assignment
\end{alltt}
then:
\begin{alltt}
vowel.exists('a') == '1'
vowel.exists('b') == '0'
vowel.exists('c') == '0'
\end{alltt}
\item[format([before [,after]])]\label{refformat}
\index{FORMAT,method}
\index{Method, built-in,FORMAT}
\index{Mathematical method,FORMAT}
\index{Formatting,numbers for display}
\index{Numbers,formatting for display}
\index{Numbers,rounding}
\index{Conversion,formatting numbers}

formats (lays out) \emph{string}, which must be a number.
 
The number, \emph{string}, is first formatted by adding zero with a
digits setting that is either nine or, if greater, the number of digits
in the mantissa of the number (excluding leading insignificant zeros).
If no arguments are given, the result is precisely that of this
operation.
 
The arguments \emph{before} and \emph{after} may be specified to
control the number of characters to be used for the integer part and
decimal part of the result respectively.  If either of these is omitted
(with no arguments specified to its right), or is \textbf{null}, the
number of characters used will be as many as are needed for that part.
 
\emph{before} must be a positive number; if it is larger than is
needed to contain the integer part, that part is padded on the left with
blanks to the requested length.
If \emph{before} is not large enough to contain the integer part
of the number (including the sign, for negative numbers), an error
results.
 
\emph{after} must be a non-negative number; if it is not the same
size as the decimal part of the number, the number will be rounded (or
extended with zeros) to fit.  Specifying 0 for \emph{after} will
cause the number to be rounded to an integer (that is, it will have no
decimal part or decimal point).
 
\textbf{Examples:}
\begin{alltt}
' - 12.73'.format         == '-12.73'
'0.000'.format            == '0'
'3'.format(4)             == '   3'
'1.73'.format(4,0)        == '   2'
'1.73'.format(4,3)        == '   1.730'
'-.76'.format(4,1)        == '  -0.8'
'3.03'.format(4)          == '   3.03'
' - 12.73'.format(null,4) == '-12.7300'
\end{alltt}
 
Further arguments may be passed to the \keyword{format} method to control
the use of exponential notation.
The full syntax of the method is then:
 
\keyword{format([before[,after[,explaces[,exdigits[,exform]]]]])}
 The first two arguments are as already described.  The other three
(\emph{explaces}, \emph{exdigits}, and \emph{exform})
control the exponent part of the result.  The default for any of the
arguments may be selected by omitting them (if there are no arguments to
be specified to their right) or by using the value \textbf{null}.
 
\emph{explaces} must be a positive number; it sets the number of
places (digits after the sign of the exponent) to be used for any
exponent part, the default being to use as many as are needed.
If \emph{explaces} is specified and is not large enough to contain
the exponent, an error results.
If \emph{explaces} is specified and the exponent will be 0,
then \emph{explaces}+2 blanks are supplied for the exponent
part of the result.
 
\emph{exdigits} sets the trigger point for use of exponential
notation.
If, after the first formatting, the number of places needed before the
decimal point exceeds \emph{exdigits}, or if the absolute value of
the result is less than \textbf{0.000001}, then exponential form will
be used, provided that \emph{exdigits} was specified.
When \emph{exdigits} is not specified, exponential notation
will never be used.
The current setting of \keyword{numeric digits} may be used for
\emph{exdigits} by specifying the special word
 \textbf{digits} (see page \pageref{refswdigit}) .
If 0 is specified for \emph{exdigits}, exponential
notation is always used unless the exponent would be 0.
 
\emph{exform} sets the form for exponential notation (if needed).
\emph{exform} may be either \textbf{'Scientific'} (the default)
or \textbf{'Engineering'}.  Only the first character of
\emph{exform} is significant and it may be in uppercase or in
lowercase.
The current setting of \keyword{numeric form} may be used by specifying
the special word  \textbf{form} (see page \pageref{refswform}) .
If engineering form is in effect, up to three digits (plus sign) may be
needed for the integer part of the result (\emph{before}).
 
\textbf{Examples:}
\begin{alltt}
'12345.73'.format(null,null,2,2) == '1.234573E+04'
'12345.73'.format(null,3,null,0) == '1.235E+4'
'1.234573'.format(null,3,null,0) == '1.235'
'123.45'.format(null,3,2,0)      == '1.235E+02'
'1234.5'.format(null,3,2,0,'e')  == '1.235E+03'
'1.2345'.format(null,3,2,0)      == '1.235    '
'12345.73'.format(null,null,3,6) == '12345.73     '
'12345e+5'.format(null,3)        == '1234500000.000'
\end{alltt}
 \textbf{Implementation minimum:} If exponents are supported in an
implementation, then they must be supported for exponents whose
absolute value is at least as large as the largest number that can be
expressed as an exact integer in default precision, \emph{i.e.}, 999999999.
Therefore, values for \emph{explaces} of up to 9 should also be
supported.
\item[insert(new [,n [,length [,pad{]]]})]\label{refinsert}
\index{INSERT method,}
\index{Method, built-in,INSERT}
\index{Inserting a string into another,}

inserts the string \emph{new}, padded or truncated to length
\emph{length}, into a copy of the target \emph{string} after the
\emph{n}\emph{th} character; the string with any inserts is returned.
\emph{length} and \emph{n} must be a non-negative whole numbers.
If \emph{n} is greater than the length of the target string,
padding is added before the \emph{new} string also.
The default value for \emph{n} is 0, which means insert before the
beginning of the string.  The default value for \emph{length} is
the length of \emph{new}.  The default \emph{pad} character is
a blank.
 
\textbf{Examples:}
\begin{alltt}
'abc'.insert('123')         == '123abc'
'abcdef'.insert(' ',3)      == 'abc def'
'abc'.insert('123',5,6)     == 'abc  123   '
'abc'.insert('123',5,6,'+') == 'abc++123+++'
'abc'.insert('123',0,5,'-') == '123--abc'
\end{alltt}

\index{LASTPOS method,}
\index{Method, built-in,LASTPOS}
\index{Finding a string in another string,}
\index{Locating,a string in another string}
\item[lastpos(needle [,start{]})]\label{reflastpos}
returns the position of the last occurrence of the string
\emph{needle} in \emph{string} (the "haystack"), searching
from right to left.
If the string \emph{needle} is not found, or is the null string,
0 is returned.
By default the search starts at the last character of
\emph{string} and scans backwards.
This may be overridden by specifying \emph{start}, the point at
which to start the backwards scan.
\emph{start} must be a positive whole number, and defaults to the
value \emph{string}\textbf{.length} if larger than that
value or if not specified (with a minimum default value of one).
 
\textbf{Examples:}
\begin{alltt}
'abc def ghi'.lastpos(' ')   == 8
'abc def ghi'.lastpos(' ',7) == 4
'abcdefghi'.lastpos(' ')     == 0
'abcdefghi'.lastpos('cd')    == 3
''.lastpos('?')              == 0
\end{alltt}

\index{LEFT method,}
\index{Method, built-in,LEFT}
\index{Formatting,text left justification}
\item[left(length [,pad{]})]\label{refleft}
returns a string of length \emph{length} containing the
left-most \emph{length} characters of \emph{string}.
The string is padded with \emph{pad} characters (or truncated) on
the right as needed.
The default \emph{pad} character is a blank.
\emph{length} must be a non-negative whole number.
This method is exactly equivalent to
\emph{string}\textbf{.substr(1}, \emph{length}
[, \emph{pad}]\textbf{)}.
 
\textbf{Examples:}
\begin{alltt}
'abc d'.left(8)     == 'abc d   '
'abc d'.left(8,'.') == 'abc d...'
'abc defg'.left(6)  == 'abc de'
\end{alltt}

\index{LENGTH,method}
\index{Method, built-in,LENGTH}
\index{Strings,length of}
\index{Data,length of}
\item[length()]\label{reflength}
returns the number of characters in \emph{string}.
 
\textbf{Examples:}
\begin{alltt}
'abcdefgh'.length == 8
''.length         == 0
\end{alltt}
\item[lower([n [,length{]}])]\label{reflower}
\index{LOWER method,}
\index{Method, built-in,LOWER}
\index{Strings,lowercasing}
\index{Lowercasing strings,}

returns a copy of \emph{string} with any uppercase characters in
the sub-string of \emph{string} that begins at the \emph{n}\emph{th}
character, and is of length \emph{length} characters, replaced by
their lowercase equivalent.
 
\emph{n} must be a positive whole number, and defaults to 1 (the
first character in \emph{string}).  If \emph{n} is greater than
the length of \emph{string}, the string is returned unchanged.
 
\emph{length} must be a non-negative whole number.
If \emph{length} is not specified, or is greater than the number of
characters from \emph{n} to the end of the string, the rest of the
string (including the \emph{n}\emph{th} character) is assumed.
 
\textbf{Examples:}
\begin{alltt}
'SumA'.lower      == 'suma'
'SumA'.lower(2)   == 'Suma'
'SuMB'.lower(1,1) == 'suMB'
'SUMB'.lower(2,2) == 'SumB'
''.lower          == ''
\end{alltt}
\item[max(number)]\label{refmax}
\index{MAX method,}
\index{Method, built-in,MAX}
\index{Mathematical method,MAX}

returns the larger of \emph{string} and \emph{number}, which
must both be numbers.  If they compare equal (that is, when subtracted,
the result is 0), then \emph{string} is selected for the result.
 
The comparison is effected using a numerical comparison with a digits
setting that is either nine or, if greater, the larger of the number of
digits in the mantissas of the two numbers (excluding leading
insignificant zeros).
 
The selected result is formatted by adding zero to the selected number
with a digits setting that is either nine or, if greater, the number of
digits in the mantissa of the number (excluding leading insignificant
zeros).
Scientific notation is used, if necessary.
 
\textbf{Examples:}
\begin{alltt}
0.max(1)          ==1
'-1'.max(1)       ==1
'+1'.max(-1)      ==1
'1.0'.max(1.00)   =='1.0'
'1.00'.max(1.0)   =='1.00'
'123456700000'.max(1234567E+5)   == '123456700000'
'1234567E+5'.max('123456700000') == '1.234567E+11'
\end{alltt}
\item[min(number)]\label{refmin}
\index{MIN method,}
\index{Method, built-in,MIN}
\index{Mathematical method,MIN}

returns the smaller of \emph{string} and \emph{number}, which
must both be numbers.  If they compare equal (that is, when subtracted,
the result is 0), then \emph{string} is selected for the result.
 
The comparison is effected using a numerical comparison with a digits
setting that is either nine or, if greater, the larger of the number of
digits in the mantissas of the two numbers (excluding leading
insignificant zeros).
 
The selected result is formatted by adding zero to the selected number
with a digits setting that is either nine or, if greater, the number of
digits in the mantissa of the number (excluding leading insignificant
zeros).
Scientific notation is used, if necessary.
 
\textbf{Examples:}
\begin{alltt}
0.min(1)          ==0
'-1'.min(1)       =='-1'
'+1'.min(-1)      =='-1'
'1.0'.min(1.00)   =='1.0'
'1.00'.min(1.0)   =='1.00'
'123456700000'.min(1234567E+5)   == '123456700000'
'1234567E+5'.min('123456700000') == '1.234567E+11'
\end{alltt}

\index{OVERLAY method,}
\index{Method, built-in,OVERLAY}
\index{Overlaying a string onto another,}
\item[overlay(new [,n [,length [,pad{]]]})]\label{refoverlay}
overlays the string \emph{new}, padded or truncated to length
\emph{length}, onto a copy of the target \emph{string} starting
at the \emph{n}\emph{th} character; the string with any overlays is
returned.  Overlays may extend beyond the end of the original
\emph{string}.
If \emph{length} is specified it must be a non-negative whole
number.
If \emph{n} is greater than the length of
the target string, padding is added before the \emph{new} string
also.
The default \emph{pad} character is a blank, and the default value
for \emph{n} is 1.
\emph{n} must be greater than 0.
The default value for \emph{length} is the length of \emph{new}.
 
\textbf{Examples:}
\begin{alltt}
'abcdef'.overlay(' ',3)      == 'ab def'
'abcdef'.overlay('.',3,2)    == 'ab. ef'
'abcd'.overlay('qq')         == 'qqcd'
'abcd'.overlay('qq',4)       == 'abcqq'
'abc'.overlay('123',5,6,'+') == 'abc+123+++'
\end{alltt}

\index{POS position method,}
\index{Method, built-in,POS}
\index{Finding a string in another string,}
\index{Locating,a string in another string}
\index{Searching a string for a word or phrase,}
\item[pos(needle [,start{]})]\label{refpos}
returns the position of the string \emph{needle}, in
\emph{string} (the "haystack"), searching from left to right.
If the string \emph{needle} is not found, or is the null string,
0 is returned.
By default the search starts at the first character of
\emph{string} (that is, \emph{start} has the value 1).
This may be overridden by specifying \emph{start} (which must be a
positive whole number), the point at which to start the search; if
\emph{start} is greater than the length of \emph{string} then 0
is returned.
 \textbf{Examples:}
\begin{alltt}
'Saturday'.pos('day')    == 6
'abc def ghi'.pos('x')   == 0
'abc def ghi'.pos(' ')   == 4
'abc def ghi'.pos(' ',5) == 8
\end{alltt}

\index{REVERSE method,}
\index{Method, built-in,REVERSE}
\item[reverse()]\label{refreverse}
returns a copy of \emph{string}, swapped end for end.
 
\textbf{Examples:}
\begin{alltt}
'ABc.'.reverse        == '.cBA'
'XYZ '.reverse        == ' ZYX'
'Tranquility'.reverse == 'ytiliuqnarT'
\end{alltt}

\item[right(length [,pad{]})]\label{refright}
\index{RIGHT method,}
\index{Method, built-in,RIGHT}
\index{Formatting,text right justification}
\index{Leading zeros,adding with the RIGHT method}
\index{Zeros,adding on the left}
\index{Zeros,padding}

returns a string of length \emph{length} containing the
right-most \emph{length} characters of \emph{string} -
that is, padded with \emph{pad} characters (or truncated) on the
left as needed.  The default \emph{pad} character is a blank.
\emph{length} must be a non-negative whole number.
 
\textbf{Examples:}
\begin{alltt}
'abc  d'.right(8)  == '  abc  d'
'abc def'.right(5) == 'c def'
'12'.right(5,'0')  == '00012'
\end{alltt}

\index{SEQUENCE method,}
\index{Method, built-in,SEQUENCE}
\index{Collating sequence, using SEQUENCE,}
\item[sequence(final)]\label{refsequen}
 returns a string of all characters, in ascending order of encoding,
between and including the character in \emph{string} and the
character in \emph{final}.
\emph{string} and \emph{final} must be single characters;
if \emph{string} is greater than \emph{final}, an error is
reported.
 
\textbf{Examples:}
\begin{alltt}
'a'.sequence('f')           == 'abcdef'
'\\0'.sequence('\\x03')       == '\\x00\\x01\\x02\\x03'
'\\ufffe'.sequence('\\uffff') == '\\ufffe\\uffff'
\end{alltt}

\index{SIGN method,}
\index{Method, built-in,SIGN}
\index{Mathematical method,SIGN}
\item[sign()]\label{refsign}
returns a number that indicates the sign of \emph{string}, which
must be a number.
\emph{string} is first formatted, just as though the operation
"\textbf{string+0}" had been carried out with sufficient digits
to avoid rounding.
If the number then starts with \textbf{'-'} then \textbf{'-1'} is
returned; if it is \textbf{'0'} then \textbf{'0'} is returned; and
otherwise \textbf{'1'} is returned.
 
\textbf{Examples:}
\begin{alltt}
'12.3'.sign    ==  1
'0.0'.sign     ==  0
' -0.307'.sign == -1
\end{alltt}

\index{SOUNDEX method,}
\index{Method, built-in,SOUNDEX}
\index{Normalizing a string by its sound,SOUNDEX}
\item[soundex()]\label{refsoundex}
\marginnote{\color{gray}3.08}
 returns the normalized soundex value of the string. This implementation
is for the English language.  

\textbf{Examples:}
\begin{alltt}
'EULER'.soundex()  == 'E460'
\end{alltt}

\index{SPACE method,}
\index{Method, built-in,SPACE}
\index{Formatting,text spacing}
\index{Blank,removal with SPACE method}
\item[space([n [,pad{]]})]\label{refspace}
returns a copy of \emph{string} with the blank-delimited words in
\emph{string} formatted with \emph{n} (and only \emph{n})
\emph{pad} characters between each word.
\emph{n} must be a non-negative whole number.
If \emph{n} is 0, all blanks are removed.
Leading and trailing blanks are always removed.
The default for \emph{n} is 1, and the default \emph{pad}
character is a blank.
 
\textbf{Examples:}
\begin{alltt}
'abc  def  '.space        == 'abc def'
'  abc def '.space(3)     == 'abc   def'
'abc  def  '.space(1)     == 'abc def'
'abc  def  '.space(0)     == 'abcdef'
'abc  def  '.space(2,'+') == 'abc++def'
\end{alltt}

\index{STRIP method,}
\index{Method, built-in,STRIP}
\index{Leading blanks,removal with STRIP method}
\index{Blank,removal with STRIP method}
\index{Zeros,removal with STRIP method}
\index{Leading zeros,removal with STRIP method}
\index{Character,removal with STRIP method}
\index{Trailing blanks,removal with STRIP method}
\item[strip([option [,char{]]}])]\label{refstrip}
returns a copy of \emph{string} with Leading, Trailing, or Both
leading and trailing characters removed, when the first character of
\emph{option} is L, T, or B respectively (these may be given in
either uppercase or lowercase).  The default is B.
The second argument, \emph{char}, specifies the character to be
removed, with the default being a blank.
If given, \emph{char} must be exactly one character long.
 
\textbf{Examples:}
\begin{alltt}
'  ab c  '.strip        == 'ab c'
'  ab c  '.strip('L')   == 'ab c  '
'  ab c  '.strip('t')   == '  ab c'
'12.70000'.strip('t',0) == '12.7'
'0012.700'.strip('b',0) == '12.7'
\end{alltt}

\index{SUBSTR method,}
\index{Method, built-in,SUBSTR}
\index{Extracting,a sub-string}
\index{Sub-string, extracting,}
\item[substr(n [,length [,pad{]]})]\label{refsubstr}
returns the sub-string of \emph{string} that begins at the
\emph{n}\emph{th} character, and is of length \emph{length}, padded
with \emph{pad} characters if necessary.
\emph{n} must be a positive whole number, and \emph{length} must
be a non-negative whole number.
If \emph{n} is greater than \emph{string}\textbf{.length},
then only pad characters can be returned.
 If \emph{length} is omitted it defaults to be the rest of the
string (or 0 if \emph{n} is greater than the length of the string).
The default \emph{pad} character is a blank.
 
\textbf{Examples:}
\begin{alltt}
'abc'.substr(2)       == 'bc'
'abc'.substr(2,4)     == 'bc  '
'abc'.substr(5,4)     == '    '
'abc'.substr(2,6,'.') == 'bc....'
'abc'.substr(5,6,'.') == '......'
\end{alltt}
\textbf{Note: }In some situations the positional (numeric) patterns of parsing
templates are more convenient for selecting sub-strings, especially if
more than one sub-string is to be extracted from a string.

\index{SUBWORD method,}
\index{Method, built-in,SUBWORD}
\index{Extracting,words from a string}
\index{Words,extracting from a string}
\item[subword(n [,length{]})]\label{refsubword}
returns the sub-string of \emph{string} that starts at the
\emph{n}\emph{th} word, and is up to \emph{length} blank-delimited
words long.
\emph{n} must be a positive whole number; if greater than the number
of words in the string then the null string is returned.
\emph{length} must be a non-negative whole number.
If \emph{length} is omitted it defaults to be the remaining words
in the string.
The returned string will never have leading or trailing blanks, but
will include all blanks between the selected words.
 
\textbf{Examples:}
\begin{alltt}
'Now is the  time'.subword(2,2) == 'is the'
'Now is the  time'.subword(3)   == 'the  time'
'Now is the  time'.subword(5)   == ''
\end{alltt}

\index{TRANSLATE method,}
\index{Method, built-in,TRANSLATE}
\index{Translation,with TRANSLATE method}
\index{Re-ordering characters,with TRANSLATE method}
\index{Moving characters, with TRANSLATE method,}
\index{Strings,moving with TRANSLATE method}
\index{,}
\index{Replacing strings,using TRANSLATE}
\index{Changing strings,using TRANSLATE}
\item[translate(tableo, tablei [,pad{]})]\label{reftrans}
returns a copy of \emph{string} with each character in
\emph{string} either unchanged or translated to another character.
 
The \textbf{translate} method acts by searching the input translate
table, \emph{tablei}, for each character in \emph{string}.
If the character is found in \emph{tablei} (the first, leftmost,
occurrence being used if there are duplicates) then the corresponding
character in the same position in the output translate table,
\emph{tableo}, is used in the result string; otherwise the original
character found in \emph{string} is used.
The result string is always the same length as \emph{string}.
 
The translate tables may be of any length, including the null string.
The output table, \emph{tableo}, is padded with \emph{pad} or
truncated on the right as necessary to be the same length as
\emph{tablei}.
The default \emph{pad} is a blank.
 
\textbf{Examples:}
\begin{alltt}
'abbc'.translate('\&','b')           == 'a\&\&c'
'abcdef'.translate('12','ec')       == 'ab2d1f'
'abcdef'.translate('12','abcd','.') == '12..ef'
'4123'.translate('abcd','1234')     == 'dabc'
'4123'.translate('hods','1234')     == 'shod'
\end{alltt}
\textbf{Note: }The last two examples show how the \textbf{translate} method
may be used to move around the characters in a string.
In these examples, any 4-character string could be specified as the
first argument and its last character would be moved to the beginning of
the string.
Similarly, the term:
\begin{alltt}
'gh.ef.abcd'.translate(19970827,'abcdefgh')
\end{alltt}
(which returns "\textbf{27.08.1997}") shows how a string (in
this case perhaps a date) might be re-formatted and merged with other
characters using the \textbf{translate} method.

\index{TRUNC method,}
\index{Method, built-in,TRUNC}
\index{Truncating numbers,}
\index{Numbers,truncating}
\index{Formatting,numbers with TRUNC}
\item[trunc([n{]})]\label{reftrunc}
returns the integer part of \emph{string}, which must be a
number, with \emph{n} decimal places (digits after the decimal
point).
\emph{n} must be a non-negative whole number, and defaults to zero.
 
The number \emph{string} is formatted by adding zero with a digits
setting that is either nine or, if greater, the number of digits in the
mantissa of the number (excluding leading insignificant zeros).
It is then truncated to \emph{n} decimal places (or trailing zeros
are added if needed to make up the specified length).
If \emph{n} is 0 (the default) then an integer with no decimal
point is returned.
The result will never be in exponential form.
 
\textbf{Examples:}
\begin{alltt}
'12.3'.trunc         == 12
'127.09782'.trunc(3) == 127.097
'127.1'.trunc(3)     == 127.100
'127'.trunc(2)       == 127.00
'0'.trunc(2)         == 0.00
\end{alltt}

\index{UPPER method,}
\index{Method, built-in,UPPER}
\index{Strings,uppercasing}
\index{Uppercasing strings,}
\item[upper([n [,length{]]})]\label{refupper}
returns a copy of \emph{string} with any lowercase characters in
the sub-string of \emph{string} that begins at the \emph{n}\emph{th}
character, and is of length \emph{length} characters, replaced by
their uppercase equivalent.
 
\emph{n} must be a positive whole number, and defaults to 1 (the
first character in \emph{string}).  If \emph{n} is greater than
the length of \emph{string}, the string is returned unchanged.
 
\emph{length} must be a non-negative whole number.
If \emph{length} is not specified, or is greater than the number of
characters from \emph{n} to the end of the string, the rest of the
string (including the \emph{n}\emph{th} character) is assumed.
 
\textbf{Examples:}
\begin{alltt}
'Fou-Baa'.upper        == 'FOU-BAA'
'Mad Sheep'.upper      == 'MAD SHEEP'
'Mad sheep'.upper(5)   == 'Mad SHEEP'
'Mad sheep'.upper(5,1) == 'Mad Sheep'
'Mad sheep'.upper(5,4) == 'Mad SHEEp'
'tinganon'.upper(1,1)  == 'Tinganon'
''.upper               == ''
\end{alltt}

\index{VERIFY method,}
\index{Method, built-in,VERIFY}
\index{Strings,verifying contents of}
\item[verify(reference [,option [,start{]]})]\label{refverify}
verifies that \emph{string} is composed only of characters
from \emph{reference}, by returning the position of the first
character in \emph{string} that is not also in
\emph{reference}.  If all the characters were found in
\emph{reference}, 0 is returned.
 The \emph{option} may be either \textbf{'Nomatch'} (the
default) or \textbf{'Match'}.  Only the first character of
\emph{option} is significant and it may be in uppercase or in
lowercase.
If \textbf{'Match'} is specified, the position of the first character
in \emph{string} that \textbf{is} in \emph{reference} is
returned, or 0 is returned if none of the characters were found.
 The default for \emph{start} is 1 (that is, the search starts at
the first character of \emph{string}).
This can be overridden by giving a different \emph{start} point,
which must be positive.
 If \emph{string} is the null string, the method returns 0,
regardless of the value of the \emph{option}.
Similarly if \emph{start} is greater than
\emph{string}\textbf{.length}, 0 is returned.
 If \emph{reference} is the null string, then the returned value
is the same as the value used for \emph{start},
unless \textbf{'Match'} is specified as the \emph{option}, in
which case 0 is returned.
 
\textbf{Examples:}
\begin{alltt}
'123'.verify('1234567890')          == 0
'1Z3'.verify('1234567890')          == 2
'AB4T'.verify('1234567890','M')     == 3
'1P3Q4'.verify('1234567890','N',3)  == 4
'ABCDE'.verify('','n',3)            == 3
'AB3CD5'.verify('1234567890','m',4) == 6
\end{alltt}

\index{WORD method,}
\index{Method, built-in,WORD}
\index{Words,extracting from a string}
\item[word(n)]\label{refword}
returns the \emph{n}\emph{th} blank-delimited word in
\emph{string}.
\emph{n} must be positive.
If there are fewer than \emph{n} words in \emph{string}, the
null string is returned.
This method is exactly equivalent to
\emph{string}\textbf{.subword(}\emph{n},\textbf{1)}.
 
\textbf{Examples:}
\begin{alltt}
'Now is the time'.word(3) == 'the'
'Now is the time'.word(5) == ''
\end{alltt}

\index{WORDINDEX method,}
\index{Method, built-in,WORDINDEX}
\index{Words,locating in a string}
\item[wordindex(n)]\label{refwordind}
returns the character position of the \emph{n}\emph{th}
blank-delimited word in \emph{string}.
\emph{n} must be positive.
If there are fewer than \emph{n} words in the string, 0 is returned.
 
\textbf{Examples:}
\begin{alltt}
'Now is the time'.wordindex(3) == 8
'Now is the time'.wordindex(6) == 0
\end{alltt}

\index{WORDLENGTH method,}
\index{Method, built-in,WORDLENGTH}
\index{Words,finding length of}
\item[wordlength(n)]\label{refwordlen}
returns the length of the \emph{n}\emph{th} blank-delimited word in
\emph{string}.
\emph{n} must be positive.
If there are fewer than \emph{n} words in the string, 0 is returned.
 
\textbf{Examples:}
\begin{alltt}
'Now is the time'.wordlength(2)    == 2
'Now comes the time'.wordlength(2) == 5
'Now is the time'.wordlength(6)    == 0
\end{alltt}

\index{WORDPOS method,}
\index{Method, built-in,WORDPOS}
\index{Searching a string for a word or phrase,}
\index{Locating,a word or phrase in a string}
\index{Words,finding in a string}
\item[wordpos(phrase [,start{]})]\label{refwordpos}
searches \emph{string} for the first occurrence of the sequence
of blank-delimited words \emph{phrase}, and returns the word number
of the first word of \emph{phrase} in \emph{string}.  Multiple
blanks between words in either \emph{phrase} or \emph{string}
are treated as a single blank for the comparison, but otherwise the
words must match exactly.  Similarly, leading or trailing blanks on
either string are ignored.
If \emph{phrase} is not found, or contains no words, 0 is returned.
 By default the search starts at the first word in \emph{string}.
This may be overridden by specifying \emph{start} (which must be
positive), the word at which to start the search.
 
\textbf{Examples:}
\begin{alltt}
'now is the time'.wordpos('the')       == 3
'now is the time'.wordpos('The')       == 0
'now is the time'.wordpos('is the')    == 2
'now is the time'.wordpos('is    the') == 2
'now is the time'.wordpos('is  time')  == 0
'To be or not to be'.wordpos('be')     == 2
'To be or not to be'.wordpos('be',3)   == 6
\end{alltt}

\index{WORDS method,}
\index{Method, built-in,WORDS}
\index{Words,counting, using WORDS}
\index{Counting,words, using WORDS}
\item[words()]\label{refwords}
returns the number of blank-delimited words in \emph{string}.
 
\textbf{Examples:}
\begin{alltt}
'Now is the time'.words == 4
' '.words               == 0
''.words                == 0
\end{alltt}

\index{X2B method,}
\index{Unpacking a string,with X2B}
\index{Method, built-in,X2B}
\index{Conversion,hexadecimal to binary}
\index{Hexadecimal,conversion to binary}
\item[x2b()]\label{refx2b}
Hexadecimal to binary.
Converts \emph{string} (a string of at least one hexadecimal
characters) to an equivalent string of binary digits.
Hexadecimal characters may be any decimal digit character (0-9) or any
of the first six alphabetic characters (a-f), in either lowercase or
uppercase.
 \emph{string} may be of any length; each hexadecimal character
with be converted to a string of four binary digits.
The returned string will have a length that is a multiple of four, and
will not include any blanks.
 
\textbf{Examples:}
\begin{alltt}
'C3'.x2b  == '11000011'
'7'.x2b   == '0111'
'1C1'.x2b == '000111000001'
\end{alltt}

\index{X2C method,}
\index{Packing a string,with X2C}
\index{Method, built-in,X2C}
\index{Conversion,hexadecimal to character}
\index{Hexadecimal,conversion to character}
\index{Coded character,from hexadecimal}
\index{Character,from hexadecimal}
\index{Character,from a number}
\index{Numbers,conversion to character}
\item[x2c()]\label{refx2c}
Hexadecimal to coded character.
Converts the \emph{string} (a string of hexadecimal characters) to
a single character (packs).
Hexadecimal characters may be any decimal digit character (0-9) or any
of the first six alphabetic characters (a-f), in either lowercase or
uppercase.
 
\emph{string} must contain at least one hexadecimal character;
insignificant leading zeros are removed, and the string is then padded
with leading zeros if necessary to make a sufficient number of
hexadecimal digits to describe a character encoding for the
implementation.
 
An error results if the encoding described does not produce a valid
character for the implementation (for example, if it has more
significant bits than the implementation's encoding for characters).
 \textbf{Examples:}
\begin{alltt}
'004D'.x2c == 'M' -- ASCII or Unicode
'4d'.x2c   == 'M' -- ASCII or Unicode
'A2'.x2c   == 's' -- EBCDIC
'0'.x2c    == '\textbackslash 0'
\end{alltt}
 The  \textbf{d2c} method (see page \pageref{refd2c})  can be used to
convert a \nr{} number to the encoding of a character.

\index{X2D method,}
\index{Method, built-in,X2D}
\index{Conversion,hexadecimal to decimal}
\index{Hexadecimal,conversion to decimal}
\item[x2d([n{]})]\label{refx2d}
Hexadecimal to decimal.
Converts the \emph{string} (a string of hexadecimal characters) to
a decimal number, without rounding.
If \emph{string} is the null string, 0 is returned.
 
If \emph{n} is not specified, \emph{string} is taken to
be an unsigned number.
 
\textbf{Examples:}
\begin{alltt}
'0E'.x2d    == 14
'81'.x2d    == 129
'F81'.x2d   == 3969
'FF81'.x2d  == 65409
'c6f0'.x2d  == 50928
\end{alltt}
 
If \emph{n} is specified, \emph{string} is taken as a signed
number expressed in \emph{n} hexadecimal characters.
If the most significant (left-most) bit is zero then the number is
positive; otherwise it is a negative number in twos-complement form.
In both cases it is converted to a \nr{} number which may,
therefore, be negative.
If \emph{n} is 0, 0 is always returned.
 
If necessary, \emph{string} is padded on the left
with \textbf{'0'} characters (note, not "sign-extended"), or
truncated on the left, to length \emph{n} characters; (that is, as
though \emph{string}\textbf{.right(}\emph{n}, \textbf{'0')}
had been executed.)
 
\textbf{Examples:}
\begin{alltt}
'81'.x2d(2)   == -127
'81'.x2d(4)   == 129
'F081'.x2d(4) == -3967
'F081'.x2d(3) == 129
'F081'.x2d(2) == -127
'F081'.x2d(1) == 1
'0031'.x2d(0) == 0
\end{alltt}
 The  \textbf{c2d} method (see page \pageref{refc2d})  can be used to convert
a character to a decimal representation of its encoding.
\end{description}

