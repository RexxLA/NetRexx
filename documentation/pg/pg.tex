\documentclass[10pt]{book}
\usepackage[FINAL]{../boilerplate/rexx} 
\usepackage{hyperref}
\usepackage{graphics}
\usepackage{fontspec} 
%\setmainfont[Mapping=tex-text]{Garamond Premier Pro}
\setmainfont[Mapping=tex-text]{Times New Roman}
\usepackage{tabularx}
\usepackage{booktabs}
\usepackage{makeidx}
\usepackage[all]{xy}
\usepackage{lingmacros}
\usepackage{color}
\usepackage{xcolor}
%\usepackage[usenames]{color}
\usepackage{listings}
\usepackage{caption}
\DeclareCaptionFont{white}{\color{white}}
\DeclareCaptionFormat{listing}{\colorbox{gray}{\parbox{\textwidth}{#1#2#3}}}
\captionsetup[lstlisting]{format=listing,labelfont=white,textfont=white}
\usepackage{listings}
\lstdefinelanguage{NetRexx}
{morekeywords={abstract,adapter,binary,case,catch,class,constant,dependent,deprecated,digits,do,else,end,engineering,extends,final,finally,for,forever,if,implements,indirect,import,indirect,inheritable,interface,iterate,label,leave,loop,method,native,nop,numeric,options,otherwise,over,package,parent,parse,private,properties,protect,public,return,returns,rexx,say,scientific,set,digits,form,select,shared,signal,signals,sourceline,static,super,then,this,until,used,upper,volatile,when,where,while},
sensitive=false,
extendedchars=false,
morecomment=[s]={/*}{*/},
morecomment=[l]{--},
morecomment=[s]{/**}{*/},
morestring=[b]",
morestring=[d]",
morestring=[b]',
morestring=[d]'}

\lstset{language=NetRexx,
  captionpos=t,
  tabsize=3,
  alsolanguage=Rexx,
  keywordstyle=\color{blue},
  commentstyle=\color{cyan},
  stringstyle=\color{red},
  numbers=left,
  numberstyle=\tiny,
  numbersep=5pt,
  breaklines=true,
  showstringspaces=false,
  index=[1][keywords],
  columns=flexible,
  basicstyle=\fontsize{8}{8}\fontspec{Source Code Pro},emph={label}}

\lstset{language=NetRexx,
  captionpos=t,
  tabsize=3,
%  frame=lines,
  keywordstyle=\color{blue},
  commentstyle=\color{green},
  stringstyle=\color{red},
  numbers=left,
  numberstyle=\tiny,
  numbersep=5pt,
  breaklines=true,
  showstringspaces=false,
  basicstyle=\small,emph={label}}
\usepackage{../boilerplate/rail}
\hyphenation{Net-Rexx Net-Rexx-A Net-Rexx-C Net-Rexx-R}
\makeindex
\begin{document}        
\DeclareGraphicsExtensions{.jpg,.png}
\setcounter{tocdepth}{1} 
\title{NetRexx Programming Guide}
\author{RexxLA}
\date{Version 3.01 of \today}
\maketitle
\pagenumbering{Roman}
\pagestyle{plain}
\frontmatter
\pagenumbering{Roman}
\pagestyle{plain}
\section*{Temporary Disclaimer}
This is not official in any way yet
\newpage
\tableofcontents
\newpage
\pagenumbering{arabic}
\frontmatter
\large
\chapter*{\fontspec{IBM Plex Serif}\LARGE The \nr{} Programming Series}
This book is part of a library, the \emph{\nr{} Programming Series}, documenting the \nr{} programming language and its use and applications. This section lists the other publications in this series, and their roles. These books can be ordered in convenient hardcopy and electronic formats from the Rexx Language Association.
\newline
\newline
\begin{tabularx}{\textwidth}{>{\bfseries}lX}
\toprule
%% Quick Start Guide & This guide is meant for an audience that has done some programming and wants to start quickly. It starts with a quick tour of the language, and a section on installing the \nr{} translator and how to run it. It also contains help for troubleshooting if anything in the installation does not work as designed, and states current limits and restrictions of the open source reference implementation.
%% \\\midrule
Programming Guide & The Programming Guide is the one manual that at the same time teaches programming, shows lots of examples as they occur in the real world, and explains about the internals of the translator and how to interface with it.
\\\midrule
Language Reference & Referred to as the NRL, this is meant as the formal definition for the language, documenting its syntax and semantics, and prescribing minimal functionality for language implementers.
\\\midrule
Pipelines Guide \& Reference & The Data Flow oriented companion to \nr{}, with its CMS Pipelines compatible syntax, is documented in this manual. It discusses running Pipes for \nr{} in the command shell and the Workspace, and has ample examples of defining your own stages in \nr{}.
\\\bottomrule
\end{tabularx}
%%% Local Variables: 
%%% mode: latex
%%% TeX-master: t
%%% End: 

\chapter{Typographical conventions}
In general, the following conventions have  been observed in the NetRexx publications:
\begin{itemize}
\item Body text is in this font
\item Examples of language statements are in a \textbf{bold} type
\item Variables or strings as mentioned in source code, or things that appear on the console, are in a \texttt{typewriter} type
\item Items that are introduced, or emphasized, are in an \emph{italic} type
\item Included program fragments are listed in this fashion:
\begin{lstlisting}[label=example,caption=Example Listing]
-- salute the reader
say 'lectorem salutat'
\end{lstlisting}
\item Syntax diagrams take the form of so-called \emph{Railroad Diagrams} to convey structure, mandatory and optional items
\begin{rail}
AggregateExpression : ("AVG" |"MAX" |"MIN" |"SUM")
 (
   (
    'DISTINCT' ?  StateFieldPathExpression
   ) | 'COUNT'
   (
    'DISTINCT' ?  IdentificationVariable
                  | StateFieldPathExpression
                  | SingleValuedAssociationPathExpression
   )
 )
   ;
\end{rail}
%%% Local Variables: 
%%% mode: latex
%%% TeX-master: t
%%% End: 
\end{itemize}
\chapter{Introduction}
This document is the \emph{Quick Start Guide} for the reference implementation of
\nr{}. \nr{} is a \emph{human-oriented} programming language which makes
writing and using Java\footnote{Java is a trademark of Oracle, Inc.}
classes quicker and easier than writing in Java. It is part of the Rexx
language family, under the governance of the Rexx Language
Association.\footnote{\url{http.www.rexxla.org}} \nr{} has been
developed and was made available as a free download by IBM since 1995
and is free and open source since June 8, 2011.

In this Quick Start Guide, you’ll find information on
\begin{enumerate} 
\item How easy it is to write for the JVM: A Quick Tour of \nr{}
\item Installing \nr{} 
\item Using the \nr{} translator as a compiler, interpreter, or
  syntax checker 
\item Troubleshooting when things do not work as expected
\item Current restrictions.
\end{enumerate} 
The \nr{} documentation and software are distributed
by The Rexx Language Association under the \textsc{ICU} license. For
the terms of this license, see the included \textsc{LICENSE} file in
this package.

For details of the \nr{} language, and the latest news, downloads,
etc., please see the \nr{} documentation included with the package
or available at: \url{http://www.netrexx.org}.

\begin{shaded}\noindent
The highest Java version that is supported in this version, 3.09, is
\emph{Java 8}. Higher versions are not yet supported due to changes in
Java, including incompatibilities introduced with the Java module system.
\end{shaded}\indent

\mainmatter
\chapter{Meet the Rexx Family}
\section{Once upon a Virtual Machine}
On the 22nd of March 1979, to be precise, Mike Cowlishaw of IBM had a
vision of an easier to use command processor for VM, and wrote down a
specification over the following days.
VM\textsuperscript{\texttrademark} (now called z/VM) is the original Virtual Machine operating system,  stemming from an
era in which time sharing was acknowledged to be the wave of the
future and when systems as CTSS (on the IBM 704) and TSS (on the IBM
360 Family of computers) were early timesharing systems, that offered
the user an illusion of having a large machine for their exclusive
use, but fell short of virtualising the entire hardware. The CP/CMS
system changed this; CP virtualised the hardware completely and CMS was the OS
running on CP. CMS knew a succession of command interpreters, called
EXEC, EXEC2 and Rexx\textsuperscript{\texttrademark}  (originally REX - until IBM Legal interfered) -
the EXEC roots are the explanation why some people refer to an NetRexx
program as an ``exec''. As a prime example of a \emph{backronym}, Rexx
stands for ``Restructured Extended Executor''. It
can be defended that Rexx came to be as a reaction on EXEC2, but it
must be noted that both command interpreters shipped around the same
time. From 1988 on Rexx was available on MVS/TSO and other systems,
like DOS, Amiga and various Unix systems. Rexx was branded the
official SAA procedures language and was implemented on all IBM's
Operating Systems; most people got to know Rexx on OS/2. In the late
eighties the Object-Oriented successor of Rexx, Object Rexx, was
designed by Simon Nash and his colleagues in the IBM Winchester
laboratory. Rexx was thereafter known as Classic Rexx. Several open
source versions of Classic Rexx were made over the years, of which
Regina is a good example.

\section{Once upon another Virtual Machine}
In 1995 Mike Cowlishaw ported Java\textsuperscript{\texttrademark} to OS/2\textsuperscript{\texttrademark} and soon after started with
an experiment to run Rexx on the JVM\textsuperscript{\texttrademark}. With Rexx generally considered
the first of the general purpose scripting languages, NetRexx\textsuperscript{\texttrademark}  is the
first alternative language for the JVM. The 0.50 release, from April
1996, contained the NetRexx runtime classes and a translator written
in Rexx but tokenized and turned into an OS/2 executable. The 1.00
release came available in January 1997 and contained a translator
bootstrapped to NetRexx. The Rexx string type that can also handle
unlimited precision numerics is called Rexx in Java and NetRexx.
Where Classic Rexx was positioned as a system \emph{glue} language and
application macro language, NetRexx is seen as the one language that
does it all, delivering system level programs or large applications.

Release 2.00 became available in August 2000 and was a major upgrade,
in which interpreted execution was added. Until that release, NetRexx
only knew \emph{ahead of time} compilation (AOT).

Mike Cowlishaw left IBM in March 2010. IBM announced the transfer of NetRexx source code to the Rexx Language Association (RexxLA) on June 8, 2011, 14 years after the v1.0 release.

On June 8th, 2011, IBM released the NetRexx source code to RexxLA
under the ICU open source license. RexxLA shortly after released this
as NetRexx 3.00 and has followed with updates.
\section{Features of NetRexx}
\begin{description}
\item[Ease of use]
The NetRexx language is easy to read and write because many instructions are meaningful English words. Unlike some lower level programming languages that use abbreviations, NetRexx instructions are common words, such as \textbf{say}, \textbf{ask}, \textbf{if...then...else}, \textbf{do...end}, and \textbf{exit}.
\item[Free format] There are few rules about NetRexx format. You need not start an instruction in a particular column, you can also skip spaces in a line or skip entire lines, you can have an instruction span many lines or have multiple instructions on one line, variables do not need to be pre-defined, and you can type instructions in upper, lower, or mixed case.
\item[Convenient built-in functions] NetRexx supplies built-in functions
  that perform various processing, searching, and comparison
  operations for both text and numbers. Other built-in functions
  provide formatting capabilities and arithmetic calculations.
\item[Easy to debug]
When a NetRexx exec contains an error, messages with meaningful explanations are displayed on the screen. In addition, the \textbf{trace} instruction provides a powerful debugging tool.
\item[Interpreted]
The NetRexx language is an interpreted language. When a NetRexx exec
runs, the language processor directly interprets each language
statement, or translates the program in JVM bytecode.
\item[Extensive parsing capabilities]
NetRexx includes extensive parsing capabilities for character
manipulation. This parsing capability allows you to set up a pattern
to separate characters, numbers, and mixed input.
\item[Seamless use of JVM Class Libraries]
NetRexx can use any class, and class library for the JVM (written in
Java or other JVM languages) in a seamless manner, that is, without
the need for extra declarations or definitions in the source code.
\end{description}

\chapter{Learning to program}
\section{Console Based Programs}
One way that a computer can communicate with a user is to ask
questions and then compute results based on the answers typed in. In
other words, the user has a conversation with the computer. You can
easily write a list of  NetRexx instructions that will conduct a conversation. We call such a list of instructions a program.
The following listing shows a sample NetRexx program. The sample program asks the user to give his name, and then responds to him by name. For instance, if the user types in the name Joe, the reply Hello Joe is displayed. Or else, if the user does not type anything in, the reply Hello stranger is displayed.
First, we shall discuss how it works; then you can try it out for
yourself.
\begin{lstlisting}[label=hello,caption=Hello Stranger]
/* A conversation */
say "Hello! What's your name?"
who=ask
if who = '' then say "Hello stranger"
else say "Hello" who
\end{lstlisting}
Briefly, the various pieces of the sample program are:
\begin{description}
\item[\texttt{/* ... */}] A comment explaining what the program is
  about. Where Rexx programs on several platforms must start with a comment, this is not
  a hard requirement for NetRexx anymore. Still, it is a good idea to start
  every program with a comment that explains what it does.
\item [\texttt{say}] An instruction to display Hello! What' s your name? on the screen.
\item [\texttt{ask}] An instruction to read the response entered from the keyboard and put it into the computer's memory.
\item [\texttt{who}] The name given to the place in memory where the user's response is put.
\item [\texttt{if}] An instruction that asks a question. 
\item [\texttt{who = ''}] A test to determine if who is empty.
\item [\texttt{then}] A direction to execute the instruction that follows, if the tested condition is true.
\item [\texttt{say}] An instruction to display Hello stranger on the screen.
\item [\texttt{else}] An alternative direction to execute the
  instruction that follows, if the tested condition is not true. Note
  that in NetRexx, else needs to be on a separate line.
\item [\texttt{say}] An instruction to display Hello, followed by whatever is in who on the screen.
\end{description}
The text of your program should be stored on a disk that you have
access to with the help of an \emph{editor} program. On Windows,
notepad or (notepad++), jEdit, X2 or SlickEdit are suitable
candidates. On Unix based systems, including MacOSX, vim or emacs are
plausible editors. If you are on z/VM or z/OS, XEDIT or ISPF/PDF are a
given. More about editing NetRexx code in chapter \ref{editors},
\emph{Editor Support}, on page \pageref{editors}. 

When the text of the program is stored in a file, let's say we called
it \texttt{hello.nrx}, and you installed NetRexx as indicated in the
\emph{NetRexx Quick Beginning Guide}, we can run it with
\begin{verbatim}
    nrc -exec hello
\end{verbatim}
and this will yield the result:
\begin{verbatim}
NetRexx portable processor, version NetRexx after3.01, build 1-20120406-1326
Copyright (c) RexxLA, 2011.  All rights reserved.
Parts Copyright (c) IBM Corporation, 1995,2008.
Program hello.nrx
===== Exec: hello =====
Hello! What’s your name?
\end{verbatim}
If you do not want to see the version and copyright message every
time, which would be understandable, then start the program with:
\begin{verbatim}
nrc -exec -nologo hello
\end{verbatim}
This is what happened when Fred tried it.
\begin{verbatim}
Program hello.nrx
===== Exec: hello =====
Hello! What’s your name?
Fred
Hello Fred
\end{verbatim}
The \textbf{ask} instruction paused, waiting for a reply. Fred typed
Fred on the command line and, when he pressed the ENTER key, the
\textbf{ask} instruction put the word Fred into the place in the
computer's memory called ``who''. The \textbf{if} instruction asked,
is ``who'' equal to nothing:
\begin{verbatim}
who = '' 
\end{verbatim}
meaning, is the value of ``who''  (in this case, Fred) equal to
nothing:
\begin{verbatim}
"Fred = ''
\end{verbatim}
This was not true; so, the instruction after \texttt{then} was not executed; but the instruction
after \texttt{else}, was.

But when Mike tried it, this happened:
\begin{verbatim}
Program hello.nrx
===== Exec: hello =====
Hello! What’s your name?

Hello stranger
Processing of 'hello.nrx' complete
\end{verbatim}
Mike did not understand that he had to type in his name. Perhaps the
program should have made it clearer to him. Anyhow, he just pressed
ENTER. The \textbf{ask} instruction put '' (nothing) into the place in the
computer's memory called ``who''. The \textbf{if} instruction asked, is:
\begin{verbatim}
who = ''
\end{verbatim}
meaning, is the value of ``who'' equal to nothing:
\begin{verbatim}
'' = ''
\end{verbatim}
 
In this case, it was true. So, the instruction after \textbf{then} was executed; but the instruction after \textbf{else} was not.

\chapter{NetRexx as a Scripting Language}
You can use NetRexx as a simple scripting language without having
knowledge of, using any of the features that is needed in a Java
program that runs on the JVM. 

Scripts can be written very fast. There is
no overhead, such as defining a class, constructors and methods, and the programs contain only
the necessary instructions.

The scripting feature can be used for test purposes. It is an easy and convenient way of entering some statements and testing them.
The scripting feature can also be used for the start sequence of a
NetRexx application.

Scripts can be interpreted or compiled - there is no rule that a
script needs to be interpreted. In both cases, interpreted or
compiled, the NetRexx translator adds the necessary overhead to enable
the JVM to execute the resulting program.

\chapter{NetRexx as an Interpreted Language}

\chapter{NetRexx as a Compiled Language}

\chapter{Calling non-JVM programs}

\chapter{Using NetRexx classes from Java}
If you are a Java programmer, using a NetRexx class from Java is just as easy as using a Java class from NetRexx. NetRexx compiles to Java classes that can be used by Java programs.
You should import the netrexx.lang package to be able to use the short
class name for the Rexx (NetRexx string and numerics) class.

A NetRexx method without a returns keyword can return nothing, which is the void type in Java, or a Rexx string. 
NetRexx is case independent; Java is case dependent. NetRexx generates the Java code with the case used in the class and method instructions. For example, if you named your class Spider in the NetRexx source file, the resulting Java class file is Spider.class.
The public class name in your source program must match the NetRexx
source file name. For example, if your source file is SPIDER.NRX, and
your class is Spider, NetRexx generates a warning and changes the
class name to SPIDER to match the file name. A Java program using the
class name Spider would not find the generated class, because its name
is SPIDER.class - if the compile succeeded, which is not guaranteed in
case of casing mismatches.
If you have problems, compile your NetRexx program with the options
-keepasjava -format. You then can look at the generated java file for the correct spelling style and method parameters.

\chapter{Classes}
\section{Classes}
\section{Properties}
\section{Methods}
\section{Inheritance}
\section{Overriding Methods}
\section{Overriding Properties}
\chapter{Incorporating Class Libraries}
\section{The Collection Classes}
\section{CLASSPATH}
\chapter{Input and Output}
\section{The File Class}
\section{Streams}
\section{Line mode I/O}
\subsection{Line mode I/O using BufferedReader and PrintWriter}
\subsection{Line mode I/O using BufferedReader and BufferedWriter}
\section{Byte Oriented I/O}
\section{Data Oriented I/O}
\section{Object Oriented I/O using Serialization}
\chapter{Algorithms in NetRexx}
\section{Fibonacci}
\lstinputlisting[label=fibonacci,caption=Fibonacci]{../../../examples/rosettacode/RCFibonacciSeq.nrx}
\chapter{Using Parse}
\chapter{Using Trace}

\chapter{User Interfaces}
\section{AWT}
\section{Swing}
\subsection{Creating NetRexx Swing interfaces with NetBeans}
\section{Web Frameworks}
\subsection{JSF}
\chapter{Network Programming}
\section{Using Uniform Resource Locators (URL)}
\section{TCP/IP Socket I/O}
\section{RMI: Remote Method Interface}

\chapter{Database Connectivity with JDBC}

\chapter{Threads}

\chapter{Component Based Programming: Beans}

\chapter{Interfacing to Open Object Rexx}
\section{BSF4ooRexx}

\chapter{Editor support}\label{editors}

\backmatter
\listoffigures
\listoftables
\printindex
\end{document} 
