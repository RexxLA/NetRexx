\chapter{JPMS, The Java Platform Module System}
Java9+ introduced the Java Platform Module System (JPMS) per JSR 376.
While Java 9 still loads external classes from files and jar-files, all
run-time packages now are bundled in modules.
\nr{} 3.xx was not capable to load classes from the JPMS.

\nr{} 4 is now supporting the JPMS to find and load its needed run-time
packages.\footnote{You'll find, in the \nr{} source code, updates in RxClasser, where method importclasses() is extended to look
for packages and classes in JPMS' jrt:/ file system.

Method \emph{packmodfind()} walks the jrt:/ directory tree at initialisation and
registers all found packages. Method modfind() registers classes when imported.
Method loadclass() loads the class image from the JPMS.}

From a \nr{} source code perspective, nothing changes. NetRexx is agnostic
about modules. It processes packages and classes whether found in directories,
jar-files or - now - modules.
All existing source code should run unmodified, with the exception that possibly
classes could need to be called explicitly when short-named classes now
exhibit ambiguous classes if the short-named class is found in more than one
module/package.
E.g.
\begin{verbatim}
/modules/java.base/java/util/spi/ToolProvider.class
/modules/java.compiler/javax/tools/ToolProvider.class
\end{verbatim}

\nr{} 4 depends on JSR 203 (NIO.2) and thus requires a minimum JDK level of Java 7,
whereas \nr{} 3 runs on Java 6.
\nr{} 4 compiles and runs on Java 7/8 (without JPMS) and on Java 9+ (with JPMS).
\section{CLASSPATH}
Most implementations of Java use an environment variable called
CLASSPATH to indicate a search path for Java classes. The Java Virtual
Machine and the \nr{} translator rely on the CLASSPATH value to find directories, zip files, and jar files which may contain Java classes. 
The procedure for setting the CLASSPATH environment variable depends on your operating system (and there may be more than one way).
\begin{itemize}
\item For Linux and Unix (BASH, Korn, or Bourne shell), use:
\begin{verbatim}
        CLASSPATH=<newdir>:\$CLASSPATH 
        export CLASSPATH
\end{verbatim}

\item Changes for re-boot or opening of a new window should be placed in your /etc/profile, .login, or .profile file, as appropriate. 
\item For Linux and Unix (C shell), use:
\begin{verbatim}
        setenv CLASSPATH <newdir>:\$CLASSPATH 
\end{verbatim}
Changes for re-boot or opening of a new window should be placed in
your .cshrc file. If you are unsure of how to do this, check the
documentation you have for installing the Java toolkit.
\item For Windows operating systems, it is best to set the system wide
  environment, which is accessible using the Control Panel (a search
  for ``environment'' offsets the many attempts to relocate the exact
  dialog in successive Windows Control Panel versions somewhat).
\end{itemize}
The \emph{Quick Start Guide} has more information about CLASSPATH.

\section{Adding modules to a compile run}
Extra modules are provided to the java runtime with the --module-path argument, --module-path describes a list of directories containing module jar files.

Because, by default, NetRexx programs compile in the unnamed module, any referenced jar file must be included in the \texttt{-{}-add-modules} option at run-time.
So when \texttt{-{}-module\-path PATH1:PATH2} is needed at compile-time, \texttt{-{}-module-path PATH1:PATH2 -{}-add-module JAR1,JAR2} is needed at run-time.

Java also supports the \texttt{-{}-upgrade-module-path PATH1:PATH2} option, which 'adds' all modules in the given paths to the unnamed module.

Setting the options in the \keyword{JDK\_JAVA\_OPTIONS} environment variable will allow NetRexx to find classes in the given modules.

By convenience, NetRexx supports both --module-path and --upgrade-module-path:

\begin{verbatim}
$ export JDK_JAVA_OPTIONS='--upgrade-module-path PATH1:PATH2'
$ nrc -run code-needing-classes-from-modules-outside-jrt
\end{verbatim}
or (for Windows)
\begin{verbatim}
> set JDK_JAVA_OPTIONS=--upgrade-module-path "PATH1;PATH2"
> nrc -run code-needing-classes-from-modules-outside-jrt
\end{verbatim}
