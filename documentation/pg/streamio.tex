\chapter{Stream I/O}\label{refstreamio}

\marginnote{\color{gray}4.03} Stream I/O was one of the casualties of
an IBM Source Code Freeze around the \Rexx{} 4.00 era. This meant that
TSO (MVS - OS390 - z/OS) \Rexx{} did not receive the code, and had to settle for EXECIO\index{EXECIO}, originally a VM utility program, which was adapted to handle \Rexx{} stems. Regina and Open Object Rexx (ooRexx) do have Stream I/O, while several dialects have idiosyncratic I/O implementations, some based on the C standard library. For a long time, MVS/TSO had a separately installable stream I/O library; not many IBM mainframe sites did install it, and consequently it could not be counted upon to be available. These functions are, however, available on z/OS USS (Unix Systems Services).

Like the rest of \Rexx{}, stream I/O was designed to make life easier on the software developer, and without any doubt, it does. For this reason, \nr{} now contains a \keyword{RexxStream} class, which is inspired on the Classic \Rexx version. As mentioned on page \pageref{refscripting}, this class is automatically available when using \nr{} in \emph{scripting mode}, where the expectation is it will be used most. On first sight, these functions look line Classic \Rexx{} in the sense that they are not predicated on an object; in fact they are static methods of the \keyword{RexxStream} class.\footnote{For this reason they are listed in the list of Rexx string functions in the \nr{} Language Reference, but explained in their own chapter of that manual.}

Data can be written to, and read from, files, without going through \emph{open} and \emph{close} routines; also, in the same manner information on file metadata can be obtained.

\section{Lines() and Linein()}
As an example, it is now possible to write a very short script that reads lines from a file for inspecting or transforming their content, in a few lines.

\lstinputlisting[label=streadlines,caption=]{./readlines.nrx}

\begin{shaded}
\bash[stdout]
nrc -verbose0 -exec readlines.nrx
\END
\end{shaded}

This script loops while the function \keyword{lines('testdata.txt'} returns a number greater than 0, and prints the line number in front of the line. (Notice how the loop statement always increases the loop variable \emph{i} when the loop is executed). These functions are Unicode compatible.

\section{Chars() and CharIn()}
For \keyword{char()} and \keyword{charin()} the same rules apply, but they work on one (Unicode) character at a time. Because of this, they are not as fast, as they cannot optimally use a buffered input mechanism.

  




