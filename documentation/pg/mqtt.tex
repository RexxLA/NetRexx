\chapter{MQTT}\label{mqtt}
 
 
\section{Pub/Sub with MQ Telemetry }
 
Publish/subscribe (pub/sub) is a model that lends itself very well to a number of one publisher, many subscriber type of applications; the tools to enter this technology have never been as available as they are now. Also, MQTT is a small protocol that needs to be taken seriously: Facebook has recently become one of the largest users.
 
Designed as a low-overhead on-the-wire protocol for brokers in the Internet-of-things age, MQTT is an exciting new development in the Messaging and Queueing realm. It is a good choice for any broker functionality, as the minimal message overhead is 2 bytes, but the maximum messages size, in one of the more popular open source brokers is a good 250MB, which give you a message size that is a lot higher than anything possible in the early years of MQ Series back in the nineties. It is now possible to do development with an entry level, entirely open source suite, and scale up to commercial, clustered and highly available implementations when needed, since the protocol has is supported by the base IBM WebSphere MQ product and is an added deliverable in WSMQ 7.5, after being available as an installable add-on for several years.
 
Here I will show how extremely straightforward it is to create a pub/sub application using this technology. These examples use NetRexx, the Eclipse PAHO Java client library and the open source Mosquitto broker; all these components are completely free and open source. I have installed Mosquitto on my MacBook using the brew system(fn), which makes it as much trouble as “sudo brew install mosquitto”.  NetRexx is an excellent language for these examples, as it is compact and avoids the C-inspired ceremony of Java language syntax; if your project requires Java, you can just save the generated Java source (using the new –keepasjava option).
 
Mosquitto(fn) is written by Roger Light as an open source equivalent of IBM’s rsmb (real small message broker) example application, which is free but lacks source code. It is a small broker application that nevertheless runs production sized workloads. As MQTT, as opposed to the MQI or JMS API’s you use when developing a messaging application, is an on-the-wire protocol (commercial messaging systems tend to have their own, unpublished, on-the-wire protocols), we need an API to use it. This API consists of a set of calls that do the formatting of the messages to the requirements of the on-the-wire protocol for you. The messages themselves are just byte-arrays, which gives you the ultimate freedom in designing their content. It is not unusual for connected devices to encode their information in a few bits; on the other hand, there is no reason not to use extreme verbosity in messages; as long as you send the .getBytes that your String yields, MQTT will send it. When encoding information in a compact way, the protocol design will really pay off, because the protocol overhead, in comparison with http and other chatty protocols, is very low. A limited set of quality of service options (qos) will indicate if you want send and pray, acknowledged delivery or acknowledged one-time-only delivery.
 
The API library that was chosen for these examples is that from the Eclipse PAHO project. This project, which is in its early stages, has C, Javascript and Java client libraries available. I chose the Java client because the JVM environment is where most of the organizations that I work for will use it. The PAHO Java client library is donated by IBM and written by Dave Locke; it is in active development. If you want to see how the protocol moves in packets over the network, I can recommend Wireshark, which does a good job of recognizing them (if you run on the standard port 1883) and showing you the message types (like ACK) and their bytes.
 
After having put the NetRexx(.jar) and paho client jars on your classpath, you are good to go. The first example here is the publisher – this is not a fragment, but the complete code. For production code we might add some more checks, as enterprise environments always are prone to suddenly run low on disk space and suffer missing authorizations, but it works as it stands. Do note that you do not have to define a message topic in advance – just think of one any use it, at least if you are in your own environment. With Mosquitto, there wasn’t anything to define in advance, and the running Publisher (happily lifted from the Java example) in NetRexx was actually the first time I talked to Mosquitto on my MacBook.
 
\lstinputlisting[label=mqlistener,caption=MQTT Publish Sample]{../../examples/enterprise/mqtt/Publish.nrx}
 
Topics can have a hierarchical organization; this structure is put in by composing trees of topics, which are strings separated by ‘/’. In this way, it is easy to compose a /news/economics/today topic string that gives some structure to the publication. The classification is entirely up to the designer.
 
Messaging in its original form is an asynchronous technology, and for this reason the API offers a callback option, where the callback receives the results of your publish action in an asynchronous way. The broker assigns a message id which you receive back.
 
 
The second source fragment (and again, it is no fragment but the entire application program) shows the subscriber.
 
\lstinputlisting[label=mqlistener,caption=MQTT Subscribe Sample]{../../examples/enterprise/mqtt/Subscribe.nrx}
  
Security is outside of the scope of this introduction which shows you the sourcecode of a simple pub/sub application, but in Mosquitto the traffic can be secured using SSL certificates and userid/password combinations; also, the access to topics can be limited. In terms of availability, the Mosquitto configuration file offers an opportunity to send all messages for a defined set of topics to another connected broker, which might be in a different part of the world, or your home, to enable a redundant setup. While the broker does not offer the queue – transmission queue - channel setup with retrying channels that MQ does, the client API has some facilities to locally save the messages and retry if the communication was lost. Also, the last-will-and-testament facility is something that traditional MQ does not have.
 
 