\chapter{Building the \nr{} translator}
It is easy to build the \nr{} translator from source. Prerequisites
are:
\begin{enumerate}
\item A Java Virtual Machine
\item A Git client
\end{enumerate}

\nr{} can be built on all platforms it runs on. NetRexx has been
bootstrapped since 1996 and subsequently has been used to compile
itself. Every checkout of the source code contains the 'bootstrap'
compiler, which is normally the previous release version. Only the official
release branches contain the same release of the compiler - to prove
that it still can compile itself on release. Theoretically, it is
possible to break things by introducing changes that preclude the compiler to compile itself - it is our job that
these changes are not released to a wider audience, but rolled back in
time.
\section{Repository}
The \nr{} source code repository is hosted at the SourceForge Git repository. To get the code
on your system, you should register at the NetRexx project at SourceForce and clone the repository
using Git. For this version management package there are many
graphical user interfaces, but what is shown here, is the command line
version. Choose a suitable place as working directory - you can later
move it around as you please.
\begin{alltt}
git clone https://git.code.sf.net/p/netrexx/code netrexx-code
\end{alltt}

\begin{shaded}\noindent
\textbf{Note:} This will checkout
the whole repository to your local system; including previous
versions, experimental branches and personal sandboxes of other
developers. 
\end{shaded}\noindent
The master branch contains the most current version of the source
code, including the documentation, examples and test cases. 
% It looks
% like:
% \bash[stdout]
% tree ../..
% \END
\section{The buildfile}
The official buildfile is called \keyword{build.xml} and the "Another Neat Tool"
\keyword{ant} utility is used for building \nr{} from source. This
\keyword{build.xml} contains tasks to build a number of targets, as listed below:
\begin{alltt}
ant -p
    Buildfile: ./netrexx-code/build.xml

    Main targets:

    apidocs           create API documentation
    clean             delete all built files
    clean.jar         delete built jars
    clean.javadocs    delete built javadocs
    clean.process     delete built translator files
    clean.runtime     delete built runtime files
    clean.tests       delete test files
    compile           compile all (except tests)
    compile.process   compile translator
    compile.runtime   compile runtime
    compile.tests     compile tests
    default           build and test distribution
    init              Set build number and document version level
    jars              create jars
    package           build distribution package
    post.jar.prepare  post jar build - define new NetRexx compiler
    setecj            set compiler to ecj
    setjavac          set compiler to javac
    showprops         Displays default property settings
    tests             compile and run tests
    withecj           build and test distribution with ecj
    withjavac         build and test distribution with javac
    withjavadocs      build distribution and javadocs with test
    Default target: default
\end{alltt}
To build the translator, make sure that the top level directory that is cloned from git is the
current directory, and issue the command:
\begin{alltt}
    ant
\end{alltt}
If the short-hand ant script is not available in your build environment, use the \nr{} supplied
ant-launcher.jar located in the  \keyword{./ant} directory
\begin{alltt}
    java -jar ant/ant-launcher.jar
\end{alltt}

This will build the \keyword{default} target, which runs the following tasks in sequence :
setecj, prepare, compile.runtime, compile.process, compile, init, jars, post.jar.prepare,
compile.tests, -checkRunTestsRequired, run.tests, tests and withecj.

The compiler/translator is built from source and creates a \keyword{./build}
directory in the current directory. The NetRexxC.jar, NetRexxF.jar and NetRexxR.jar file are created
in the \keyword{./build/lib} directory by the archiving process which is started by the \keyword{jars}
ant-task. These new jar files can be used immediately, by having them (NetRexxC.jar will suffice) on the classpath.

The \nr{} source files are located in the src directory.

The \keyword{./src/netrexx/lang} directory contains the core of \nr{}, \keyword{./src/org/netrexx/process} contains
the translator, compiler and interpreter. The \keyword{./src/org/netrexx/jsr223} directory contains the framework which
provides \nr{} as a JSR223 scripting language, and \keyword{./src/org/netrexx/njppipes} holds the sourcecode for the
\nr{} Pipelines compiler and stages.

The \keyword{default} build process produces the following jar files:

\begin{tabularx}{\textwidth}{>{\bfseries}lX}
\toprule
NetRexxC.jar & This jar includes the core \nr{} classes and the \nr{} translator, compiler and interpreter.
Include this jar in your classpath if you have a Java Development Kit (JDK) installed in your build system and you
want to compile (or interpret) \nr{} programs.

The jar also contains the \nr{} implementation of CMS Pipelines, nrws, the \nr{} workspace, and the
NetRexx JSR223 Scripting Engine.
\\\midrule
NetRexxF.jar & This jar file contains the same as NetRexxC.jar with addition to a slightly modified Eclipse
Java compiler. Use this jar file if you only have a Java runtime (i.e. no javac).
\\\midrule
NetRexxR.jar & This jar file contains the core of \nr{}. Ship this jar file with any compiled \nr{} program where
you expect NetRexxC.jar or NetRexxF.jar to be absent.
\\\bottomrule
\end{tabularx}

For a virgin start, issue
\begin{alltt}
    ant clean
\end{alltt}
to remove all previously built files.

There is no target defined to build the documentation, which is built manually using the TextTools/build.rexx
program (see https://github.com/RexxLA/TextTools).

The documentation is however handled by the package target, where all generated pdfs from
\keyword{documentation/nrl, documentation/pg, documentation/ug, documentation/njpipes} are archived in the \nr{}
distribution zip file.

\section{Testing}
Testing is included in a normal build. When testing, the newly built NetRexxC.jar file is used in the classpath.

All \nr{} files located in directories \keyword{src/org/netrexx/diag} and \keyword{test} are run to test all instructions and features.
Any obvious, and non-obvious, possible code error in the \nr{} core and translator source is very likely to be detected by the tests.

\section{Preparing a new release}
When preparing to release a new version, whether major or minor, update file \keyword{org/netrexx/process/NrVersion.nrx}.

Its private properties \keyword{version, mod and procdate} are referenced during the generation of documentation and
other target files.

The following
\begin{lstlisting}
class NrVersion
  properties private
    version   = '4.06'
    procdate  = '03 Mar 2024'
    copyright = 'Copyright (c) RexxLA, 2011,2024.   All rights reserved.\nParts Copyright (c) IBM Corporation, 1995,2008.'
    mod       = 'GA'
\end{lstlisting}

builds NetRexx-4.06-GA.
\section{Package a new release}
As a final verification, copy the newly built NetRexxC.jar and NetRexxF.jar from the \keyword{./build/lib} directory
to the \keyword{./lib} directory, and build the \nr{} translator using the new jar files by issuing
\keyword{ant clean default}.

Next, build the documentation from the \keyword{./documentation} directories.

Finally, create the release package by issuing \keyword{ant package}.

The \nr{} release package is delivered as a zip file, containing the following:
\begin{enumerate}
    \item The NetRexxC.jar, NetRexxF.jar and NetRexxR.jar files
    \item The documentation pdfs from \keyword{./documentation/nrl, ./documentation/pg, ./documentation/ug,
        ./documentation/njpipes}.
    \item The \keyword{tools} directory with a number of utilities
    \item A large number of examples in the \keyword{examples} directory
    \item The keyword{bin} directory with scripts to launch the translator
    \item The readme file and release notes
\end{enumerate}


Normally only beta and General Available (GA) builds are published on \url{https://netrexx.org}.




%package, which currently is
%being integrated into the \keyword{test} directory. This
%directory contains, in addition to the traditional ``diag'' tests that
%have been modified to run under jUnit, some of the tests for the newer
%functionality. These tests are accessible using a \keyword{make}
%process that uses \keyword{makefile} as its build build file. The
%command
%\begin{alltt}
%make test
%%\end{alltt}
%will compile and run the tests; jUnit will report on progress and
%results.
%% \bash[stdout]
% cd ..\..\test
% make test
% \END
