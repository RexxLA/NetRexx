\chapter{Building the \nr{} translator}
It is easy to build the \nr{} translator from source. Prerequisites
are:
\begin{enumerate}
\item A Java Virtual Machine
\item A Git client
\end{enumerate}

\nr{} can be built on all platforms that it runs on. NetRexx has been
bootstrapped since 1996 and subsequently has been used to compile
itself. Every checkout of the source code contains the 'bootstrap'
compiler, which is normally the previous release version. Only the official
release branches contain the same release of the compiler - to prove
that it still can compile itself on release. Theoretically, it is
possible to break things by introducing changes that preclude the compiler to compile itself - it is our job that
these changes are not released to a wider audience, but rolled back in
time.
\section{Repository}
The \nr{} source code repository is hosted at the SourceForge Git repository. To get the code
on your system, you should register at the NetRexx project at SourceForce and clone the repository
using Git. For this version management package there are many
graphical user interfaces, but what is shown here, is the command line
version. Choose a suitable place as working directory - you can later
move it around as you please.
\begin{alltt}
git clone git://git.code.sf.net/p/netrexx/code netrexx-code
\end{alltt}

\begin{shaded}\noindent
\textbf{Note:} This will checkout
the whole repository to your local system; including previous
versions, experimental branches and personal sandboxes of other
developers. 
\end{shaded}\noindent
The master branch contains the most current version of the source
code, including the documentation, examples and test cases. 
% It looks
% like:
% \bash[stdout]
% tree ../..
% \END
\section{The buildfile}
The official buildfile is called \keyword{build.xml} and the
\keyword{ant} utility is used for building \nr{} from source. This
file contains a number of tasks. To build
the translator, make sure that the top level directory that is cloned from git is the
current directory, and issue the command:
\begin{alltt}
java -jar ant/ant-launcher.jar compile
\end{alltt}
followed by
\begin{alltt}
java -jar ant/ant-launcher.jar jars
\end{alltt}
This will build the compiler from source and create a \keyword{build}
directory in the current directory. In \keyword{build/lib} the
NetRexxC and NetRexxR jars are put by the archiving process that is
started by the \keyword{jar} task. These new jars can be used
immediately, by having them (NetRexxC will suffice) on the classpath.
\section{Testing}
Currently, there are two locations that contain the tests. The first
is the \keyword{org.netrexx.process.diag} package, which currently is
being integrated into the \keyword{trunk/test} directory. This
directory contains, in addition to the traditional ``diag'' tests that
have been modified to run under jUnit, some of the tests for the newer
functionality. These tests are accessible using a \keyword{make}
process that uses \keyword{makefile} as its build build file. The
command
\begin{alltt}
make test
\end{alltt}
will compile and run the tests; jUnit will report on progress and
results.
% \bash[stdout]
% cd ..\..\test
% make test
% \END
