\index{RxModel}
\chapter{Source Code Formatting}

\section{RxModel}

Rexx is mostly a free-form programming language allowing each programmer to write code in their own unique style. Over time, large programs with many contributing authors may become difficult to read for new users. RxModel is a NetRexx source code formatter that merges each style into one form. Visually, it lays out nesting and control flow in a plain and simple manner.

The default built-in model is Model Zero and does nothing. Model One produces clean, indented source without comments. Model Two produces clean, indented and compacted source with merged comments. Model Three is a clever extension of Model One inserting commented braces for code-folding editors.

Before starting, make a backup of your original source files. Your code must compile cleanly before using RxModel. To use RxModel you only need to pass -model[0-3] as an argument to the NetRexx Compiler. RxModel does have problems with some clauses that can be resolved by editing the original NetRexx file and running RxModel again. When the command finishes, you are left with the untouched NetRexx file and a Model file with ".mod" as its extension. After review, you can replace the NetRexx file with the new Model file. You must make sure the Model file stills compiles. If you used Model Two, please ensure all comments have been preserved and apply in context as the author intended.     


For NetRexx developers, a powerful example is running RxModel against NetRexx's own code base.
 
git clone https://git.code.sf.net/p/netrexx/code netrexx-code

cd ./netrexx-code

ant clean

edit ./src/org/netrexx/process/RxFlag.nrx

Change line 57 from "int     0" to "int     1" and save.  

ant

Rename the original cloned ./lib/NetRexxC.jar

Replace it with the freshly built one in ./build/lib/NetRexxC.jar

ant clean

ant

Now, for each "*.nrx" inside the ./build/classes sub-directories you have a "*.mod" file.

A developer might want to keep the newly created NetRexxC.jar or NetRexxF.jar that defaults to Model One without passing in -model1 for everyday personal use. A Model One file will be generated each time the compiler is used.

\section{Beyond RxModel}

Under ./tools/java2xml is a NetRexx front-end to the sourceforge project XES that converts Java to XML. It is older but it is currently a better option than the buggy one provided by JavaParser. Under ./examples/javaparser, NrxWriter.nrx is an almost perfect port of the DefaultPrettyPrinterVisitor included with the current JavaParser project. It could become the next generation of java2nrx or modified for other special projects. NrxJava.nrx is a very simple use case of the DefaultPrettyPrinterVisitor in JavaParser to test NrxWriter. Also, included is NrxYaml.nrx for those who like working with YAML files.