%%%%%% .* ------------------------------------------------------------------
% .* \nr{} now-moved to programmer's Guide                                              mfc
% .* Copyright (c) IBM Corporation 1996, 2000.  All Rights Reserved.
% .* ------------------------------------------------------------------

\section{Programmatic use of the \nr{}C translator}
\nr{}C can be used in a program, to compile \nr{} programs from files,
or to compile from strings in memory. 

\section{Compiling from memory strings}
Programs may also be compiled from memory strings by passing an array
of strings containing programs to the translator using these methods:

\begin{lstlisting}[label=frommemory,caption=From Memory]
method main(arg=Rexx, programarray=String[], log=PrintWriter null) static returns int
method main2(arg=String[], programarray=String[], log=PrintWriter null) static returns int
\end{lstlisting}

Any programs passed as strings must be named in the arg parameter before any programs contained in files are named.
For convenience when compiling a single program, the program can be
passed directly to the compiler as a String with this method:

\begin{lstlisting}[label=string,caption=With String argument]
method main(arg=Rexx, programstring=String, logfile=PrintWriter null) constant returns int
\end{lstlisting}

Here is an example of compiling a \nr{} program from a string in
memory:

\begin{lstlisting}[label=memexample,caption=Example of compiling from String]
import org.netrexx.process.NetRexxC
program = "say 'hello there via NetRexxC'"
NetRexxC.main("myprogram",program)
\end{lstlisting}
This will generate a myprogram.class file on disk.

\section{Compile, load and go}
There is also a \marginnote{\color{gray}3.08} possibility to generate a program dynamically, compile it from a string, and
execute the resulting class(es) without writing intermediate class
files to disk. In this case, even the classfile is executed from
memory with a special classloader, the RxMapClassLoader. The compile,
load and go method calls the main method of the generated class,
without arguments. This method is used to dynamically generate and
execute NetRexx Pipelines. An example:

\begin{lstlisting}[label=memloadgoexample,caption=Example of compile-load-go]
nrxsrc  = "say 'hello dynamic world'"
clsName = 'helloDyn'
org.netrexx.process.NetRexxC.clgMain(clsName,nrxsrc)
\end{lstlisting}

\section{Using the generated classfiles array}
There is an overloaded main method in NetRexxC that does the compile
in-memory and fills a specified collection with the compiled classfile
images. This is in fact what clgMain() does. The generated classfile,
including their classloader, are then usable in your own code.

\begin{lstlisting}[label=memclasslistexample,caption=Example of using the classes collection]
classList = ArrayList()
    rc = main(arg' -nologo -verbose0',[String programstring],null,classList)
    mapLoader = RxMapClassLoader classList.get(0)
    do
      c = mapLoader.findClass(arg)
      paramTypes = Class[1]
      paramTypes[0] = String[].class
      methodName = String "main"
      m = c.getMethod(methodName, paramTypes)
      params = Object[1]
      params[0] = String[0]
      m.invoke(null, params)
    catch e=Exception
      say 'Reflection exception encountered:'
      say e.getMessage()
    end
\end{lstlisting}


\section{JSR199}
NetRexx uses the jsr-199 way of invoking the compiler, and a .java
file is not written to disk by default. \marginnote{\color{gray}3.04} The following program
illustrates how a complete program can be compiled and executed
without anything being written to disk.

\begin{lstlisting}[label=jsr199,caption=JSR199 example]
import org.netrexx.

pname="jsr199hello"                                --        program name
nrp=' say "hello"\n say arg \n say "program complete" '             --        NetRexx program code

classlist=ArrayList()        --    this requests a class loader and class files returned in memory
NetRexxC.main(pname "-verbose0", nrp, null, classlist) --        ask NetRexxC to compile from string nrp
loader=ClassLoader classlist.get(0)    --        find class loader build by NetRexx translator
pclass=loader.loadClass(pname)        --        load our class file into the jvm
pclass.getMethod("main", [Class -
String[0].getClass()]).invoke(null,[Object [String 'argument
string']])        --        locate main, call it with reflection = all done!
\end{lstlisting}
