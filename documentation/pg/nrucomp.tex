%%%%%% .* ------------------------------------------------------------------
% .* \nr{} User's Guide                                              mfc
% .* Copyright (c) IBM Corporation 1996, 2000.  All Rights Reserved.
% .* ------------------------------------------------------------------

\index{compiling, \nr{} programs}
\index{using the translator, as a Compiler}

\chapter{Using the translator}
\index{using the translator}
This section of the document tells you how to use the
translator package.

The \nr{} translator may be used as a compiler or as an interpreter
(or it can do both in a single run, so parsing and syntax checking are
only carried out once).  It can also be used as simply a syntax checker.

When used as a compiler, the intermediate Java source code may be
retained, if desired.  Automatic formatting, and the inclusion of comments
from the \nr{} source code are also options.

\section{Using the translator as a compiler}
The installation instructions for the \nr{} translator describe how to
use the package to compile and run a simple \nr{} program
(\emph{hello.nrx}).  When using the translator in this way (as a
compiler), the translator parses and checks the \nr{} source code, and
if no errors were found then generates Java source code.  This Java
code is then compiled into bytecodes
(\emph{.class} files) using a Java compiler, in a process called AOT
compilation. By default,
the \emph{javac} compiler in the Java toolkit is used.

This section explains more of the options available to you when using
the translator as a compiler.
% .*
\section{The translator command}
\index{command, for compiling}

\index{\nr{}C, class}
The translator is invoked by running a Java program (class) which is
called 
\begin{lstlisting}
org.netrexx.process.NetRexxC
\end{lstlisting}  
(\code{\nr{}C}, for short). This can be run by using the Java interpreter, for example,
by the command:
\begin{lstlisting}
java org.netrexx.process.NetRexxC
\end{lstlisting}
\index{scripts, \nr{}C}
\index{\nr{}C, scripts}
\index{scripts, nrc}
\index{nrc scripts}
or by using a system-specific script (such as \emph{\nr{}C.cmd}.
or \emph{nrc.bat}).  In either case, the compiler invocation is followed
by one or more file specifications (these are the names of the files
containing the \nr{} source code for the programs to be compiled).

\index{file specifications}
File specifications may include a path; if no path is given then
\nr{}C will look in the current (working) directory for the file.
\nr{}C will add the extension \emph{.nrx} to input program names (file
specifications) if no extension was given.

So, for example, to compile \emph{hello.nrx} in the current directory,
you could use any of:
\begin{lstlisting}
java org.netrexx.process.NetRexxC hello
java org.netrexx.process.NetRexxC hello.nrx
NetRexxC hello.nrx
nrc hello
\end{lstlisting}
(the first two should always work, the last two require that the
system-specific script be available).  The resulting \emph{.class} file
is placed in the current directory, and the \emph{.crossref}
(cross-reference) file is placed in the same directory as the source
file (if there are any variables and the compilation has no errors).

Here is an example of compiling two programs, one of which is in the
directory \emph{d:\textbackslash myprograms}:
\begin{lstlisting}
nrc hello d:\myprograms\test2.nrx
\end{lstlisting}

In this case, again, the \emph{.class} file for each program is placed
in the current directory.

Note that when more than one program is specified, they are all compiled
within the same class context.  That is, they can see the
classes, properties, and methods of the other programs being compiled,
much as though they were all in one file.
\footnote{The programs do, however, maintain their independence (that is, they may
have different \textbf{options}, \textbf{import}, and \textbf{package}
instructions).}
This allows mutually interdependent programs and classes to be compiled
in a single operation.
Note that if you use the \textbf{package} instruction you should also
read the more detailed \emph{Compiling multiple
programs} section.% \ref{multiple} on page \pageref{multiple}.

\index{completion codes, from translator}
\index{return codes, from translator}
On completion, the \nr{}C class will exit with one of three return
values: 0 if the compilation of all programs was successful, 1 if there
were one or more Warnings, but no errors, and 2 if there were one or
more Errors. The result can be forced to 0 for warnings only with the
\emph{-warnexit0} option.

\index{option words}
\index{flags}
As well as file names, you can also specify various option words, which
are distinguished by the word being prefixed with \emph{-}.  These
flagged words (or flags) may be any of the option words allowed
on the \nr{} \textbf{options} instruction (see the \nr{} language documentation, and the below paragraph).  These options words can be freely mixed with file
specifications.  To see a full list of options, execute the \nr{}C with the --help option
command without specifying any files. As this command states, all options may have prefix 'no' added for the inverse effect.

\subsection{Options}
\index{compiling,options}

Here are some examples:
\begin{lstlisting}
java org.netrexx.process.NetRexxC hello -keep -strictargs
java org.netrexx.process.NetRexxC -keep hello wordclock
java org.netrexx.process.NetRexxC hello wordclock -nocompile
nrc hello
nrc hello.nrx
nrc -run hello
nrc -run Spectrum -keep
nrc hello -binary -verbose1
nrc hello -noconsole -savelog -format -keep
\end{lstlisting}

Option words may be specified in lowercase, mixed case, or uppercase.
File specifications are platform-dependent and may be case sensitive,
though \nr{}C will always prefer an exact case match over a mismatch.

\textbf{Note:} The \emph{-run} option is implemented by a script (such
as \emph{nrc.bat} or \emph{\nr{}C.cmd}), not by the translator; some
scripts (such as the \emph{.bat} scripts) may require that
the \emph{-run} be the first word of the command arguments, and/or be in
lowercase.  They may also require that only the name of the file be
given if the \emph{-run} option is used.  Check the commentary at the
beginning of the script for details.

\section{Compiling multiple programs and using packages}
\index{compiling,multiple programs}

When you specify more than one program for \nr{}C to compile, they are
all compiled within the same class context: that is, they can see
the classes, properties, and methods of the other programs being
compiled, much as though they were all in one file.

This allows mutually interdependent programs and classes to be compiled
in a single operation.  For example, consider the following two programs
(assumed to be in your current directory, as the files \emph{X.nrx}
and \emph{Y.nrx}):
\begin{lstlisting}[label=dependencies,caption=Dependencies]
/* X.nrx */
class X
  why=Y null

/* Y.nrx */
class Y
  exe=X null
\end{lstlisting}
Each contains a reference to the other, so neither can be compiled in
isolation.  However, if you compile them together, using the command:
\begin{lstlisting}
nrc X Y
\end{lstlisting}
 the cross-references will be resolved correctly.

The total elapsed time will be significantly less, too, as the classes
on the CLASSPATH need to be located only once, and the class files used
by the \nr{}C compiler or the programs themselves will also only be
loaded (and JIT-compiled) once.

\index{projects, compiling}
\index{packages, compiling}
\index{compiling,packages}
This example works as you would expect for programs that are not in
packages.  There is a restriction, though, if the classes you are
compiling \emph{are} in packages (that is, they include a
\textbf{package} instruction).  \nr{}C uses either the \emph{javac}
compiler or the Eclipse batch compiler \emph{ecj} to generate the \emph{.class} files, and for mutually-dependent
files like these; both require the source files to be in the
Java \emph{CLASSPATH}, in the sub-directory described by the \textbf{package}
instruction.


So, for example, if your project is based on the tree:

\texttt{D:\textbackslash myproject}

 if the two programs above specified a package, thus:
\begin{lstlisting}[label=packagedep,caption=Package Dependencies]
/* X.nrx */
package foo.bar
class X
  why=Y null

/* Y.nrx */
package foo.bar
class Y
  exe=X null
\end{lstlisting}


\begin{enumerate}
\item
You should put these source files in the directory:
\emph{D:\textbackslash myproject\textbackslash foo\textbackslash bar}
\item
The directory \emph{D:\textbackslash myproject} should appear in your CLASSPATH
setting (if you don't do this, \emph{javac} will complain that it cannot
find one or other of the classes).
\item
You should then make the current directory be \emph{D:\textbackslash
myproject\textbackslash foo\textbackslash bar}
and then compile the programs using the command \emph{nrc X Y},
as above.
\end{enumerate}

With this procedure, you should end up with the \emph{.class} files in
the same directory as the \emph{.nrx} (source) files, and therefore also
on the CLASSPATH and immediately usable by other packages.  In general,
this arrangement is recommended whenever you are writing programs that
reside in packages.

\textbf{Notes:}
\begin{enumerate}
\item
When \emph{javac} is used to generate the \emph{.class} files, no
new \emph{.class} files will be created if any of the programs being
compiled together had errors - this avoids accidentally generating
mixtures of new and old \emph{.class} files that cannot work with each
other.
\item
If a class is abstract or is an adapter class then it should be placed
in the list before any classes that extend it (as otherwise any
automatically generated methods will not be visible to the subclasses).
\end{enumerate}

% \nr{}C can be used in a program, to compile \nr{} programs from files,
% or to compile from strings in memory. 

% \section{Compiling from memory strings}
% Programs may also be compiled from memory strings by passing an array
% of strings containing programs to the translator using these methods:

% \begin{lstlisting}[label=frommemory,caption=From Memory]
% method main(arg=Rexx, programarray=String[], log=PrintWriter null) static returns int
% method main2(arg=String[], programarray=String[], log=PrintWriter null) static returns int
% \end{lstlisting}

% Any programs passed as strings must be named in the arg parameter before any programs contained in files are named.
% For convenience when compiling a single program, the program can be
% passed directly to the compiler as a String with this method:

% \begin{lstlisting}[label=string,caption=With String argument]
% method main(arg=Rexx, programstring=String, logfile=PrintWriter null) constant returns int
% \end{lstlisting}

% Here is an example of compiling a \nr{} program from a string in
% memory:

% \begin{lstlisting}[label=memexample,caption=Example of compiling from String]
% import org.netrexx.process.NetRexxC
% program = "say 'hello there via NetRexxC'"
% NetRexxC.main("myprogram",program)
% \end{lstlisting}
% This will generate a myprogram.class file on disk.

% \section{Compile, load and go}
% There is also a \marginnote{\color{gray}3.08} possibility to generate a program dynamically, compile it from a string, and
% execute the resulting class(es) without writing intermediate class
% files to disk. In this case, even the classfile is executed from
% memory with a special classloader, the RxMapClassLoader. The compile,
% load and go method calls the main method of the generated class,
% without arguments. This method is used to dynamically generate and
% execute NetRexx Pipelines. An example:

% \begin{lstlisting}[label=memloadgoexample,caption=Example of compile-load-go]
% nrxsrc  = "say 'hello dynamic world'"
% clsName = 'helloDyn'
% org.netrexx.process.NetRexxC.clgMain(clsName,nrxsrc)
% \end{lstlisting}

% \section{Using the generated classfiles array}
% There is an overloaded main method in NetRexxC that does the compile
% in-memory and fills a specified collection with the compiled classfile
% images. This is in fact what clgMain() does. The generated classfile,
% including their classloader, are then usable in your own code.

% \begin{lstlisting}[label=memclasslistexample,caption=Example of using the classes collection]
% classList = ArrayList()
%     rc = main(arg' -nologo -verbose0',[String programstring],null,classList)
%     mapLoader = RxMapClassLoader classList.get(0)
%     do
%       c = mapLoader.findClass(arg)
%       paramTypes = Class[1]
%       paramTypes[0] = String[].class
%       methodName = String "main"
%       m = c.getMethod(methodName, paramTypes)
%       params = Object[1]
%       params[0] = String[0]
%       m.invoke(null, params)
%     catch e=Exception
%       say 'Reflection exception encountered:'
%       say e.getMessage()
%     end
% \end{lstlisting}


% \section{JSR199}
% NetRexx uses the jsr-199 way of invoking the compiler, and a .java
% file is not written to disk by default. \marginnote{\color{gray}3.04} The following program
% illustrates how a complete program can be compiled and executed
% without anything being written to disk.

% \begin{lstlisting}[label=jsr199,caption=JSR199 example]
% import org.netrexx.

% pname="jsr199hello"                                --        program name
% nrp=' say "hello"\n say arg \n say "program complete" '             --        NetRexx program code

% classlist=ArrayList()        --    this requests a class loader and class files returned in memory
% NetRexxC.main(pname "-verbose0", nrp, null, classlist) --        ask NetRexxC to compile from string nrp
% loader=ClassLoader classlist.get(0)    --        find class loader build by NetRexx translator
% pclass=loader.loadClass(pname)        --        load our class file into the jvm
% pclass.getMethod("main", [Class -
% String[0].getClass()]).invoke(null,[Object [String 'argument
% string']])        --        locate main, call it with reflection = all done!
% \end{lstlisting}

% Programmatic use of the translator - the compiler(NetRexxC) and the
% interpreter (NetRexxA) - used to be in this chapter, but the options
% to do this have multiplied over the years, and are now beyond the scope
% of this Quick Start Guide. They are documented in the \emph{Programming Guide}.

\index{build systems}

\chapter{Using build systems - ANT}
\index{ant}

From the command line, different build systems can be used to build an
entire project in one go. This chapter explains how to use ANT, one of
the early Java cross-platform build tools. With \emph{ant}, the
specification for the build needs to be provided in an \emph{.xml} file; the
default is \emph{build.xml}. \nr{} itself is built using \emph{ant};
its build.xml can be checked out in the git repository. Two scenarios
for building with \emph{ant} are mentioned in the
following sections. Unlike \emph{make}, ant does not work with command
lines, but with specialized Java tasks, to make this build system
platform independent. A special \nr{} ant task (written in \nr{}) is packaged in the
NetRexxC.jar and NetRexxF.jar files, this needs to be specified in the
build file; the small ant-netrexx.jar file also can be used.

The official Apache package for \emph{ant} has the original \nr{}
optional task written in the Java language; this can be used, but is
not up to date with the RexxLA version.

Note that when building \nr{} from source, there are two bootstrapping
situations: \nr{} is written in itself, and is built using the
optional \nr{} \emph{ant} task, written in \nr{}, using \emph{ant}.

\section{In-source, no packages}
In this scenario, the build is in-source, this means the program
source files and the class files are interspersed in the same
directory; this is often the case with small projects that only have a
few source files and no package structure. This situation enables a
very small buildfile, with only two 'build goals' in it: prepare compile and clean, identified by \texttt{<target>} XML tags. In this case, the
'compile' goal is the default, as indicated on the \texttt{<project>}
tag, \texttt{default=} attribute. We also need to include a
\texttt{<taskdef>} tag for ant to find the \nr{} task.

Also, we assume that the environment settings for the current user are
in effect, notably the one for \texttt{CLASSPATH}. Larger projects
will probably package their own libraries, and possibly need to
specify build- and runtime classpaths; these are not needed here.

\lstinputlisting[label=in-source-build,caption=In-source
build]{../../../../examples/ant-task/in-source-build/build.xml}

This build process will be run when the user enters the \texttt{ant} command,
and the result is a number of class files - if there are no errors. In
case of errors, no class files are produced. On subsequent runs, only
the classes of which the source files are newer than the class files,
will be compiled - this makes for an efficient build process.

\section{With package structure}
For a slightly larger project, which has its own package structure, we
can use a slightly more complicated build file, that will serve a lot
of projects of this kind. In this scenario, the source files are in a
\emph{src} directory, and the class files will be compiled to a file
system directory structure based on the package names. As an example,
if the file hello.nrx is in a src subdirectory of the project, and its
package name is org.rexxla.examples, the hello.class file will be in a
subdirectory <project>/war/WEB-INF/classes/org/rexxla/examples/.

For universal usability, e.g. in a JEE webserver as Tomcat, Jetty or
JBoss, we use the WAR file structure, as is the standard for these
application servers..

Next to the environment, we define two properties for the \nr{}
optional \emph{ant} task: we tell it to generate Java source files ('keepasjava'), and
to replace Java source that is already there without asking.

\lstinputlisting[label=in-source-build,caption=With-packages
build]{../../../../examples/ant-task/with-packages/build.xml}

In an analogous way, we compile the sources there might be in .java
files in a larger project with the \emph{javac} task. 

In the \emph{libs} target we create the output directories as
indicated in the standard. The compile task then translates the .nrx
source files to the \emph{respective} files in the target
directories, by using a compile and a copy task. This enables us to
have the same package structure in the source and target directories,
which then are ready to be compressed - and packaged - into a .war
file, which is a standard \emph{web archive}, with the \texttt{ant war} command.


The \emph{clean} task deletes the whole directory tree that starts
with \texttt{war}, which is a very efficient way to clean out all
built objects (except the compressed war file itself).
