
\chapter{Translator inner workings}

The translator source is part of the package
\keyword{org.netrexx.process}, located in the \keyword{./src/org/netrexx/process} directory of the git repository.\\
The runtime support, including the \keyword{Rexx} type, is in the package \keyword{netrexx.lang} in \keyword{./src/netrexx/lang}.

This chapter documents the inner workings of the
translator. Its purpose is to assist with debugging
serious problems or ease the introduction to the toolset for
programmers who want to help the open source effort forwards.

To delve deeper into the inner workings of the translator, you are encouraged to read the \nr{} source files,
a great resource to discover the language's potential.\\ Also take a look at Kermit Kiser's
\keyword{./documentation/pg/NetRexxCLogic.odg}
Open Document Graphics file, and the \nr{} UML diagrams at \keyword{./examples/uml}.
Finally, running the translator with -diag and -verbose5 arguments is very informative.


\section{Translating, compiling and interpreting}
The translator (see \keyword{RxTranslator.nrx}) passes four times through the \nr{} source files.
\begin{itemize}
    \item pass 0 : tokenises the source and parses prolog (options, package and import statements) and
 class instructions
    \item pass 1 : processes resolution of properties and methods
    \item pass 2 : parses method bodies and generates Java code
    \item pass 3 : code interpretation
    \item pass 4 : code compilation
\end{itemize}
Pass 3 and 4 are mutually exclusive, the presence of the \keyword{-exec} argument triggers interpretation
as pass 3 (see \keyword{RxInterpreter.nrx}),
otherwise the generated Java source code is compiled into Java class file(s) as pass 4.


Generally, the translator source files can be categorised in two sections.
\keyword{Nr*.nrx} files are implementations
of the \keyword{RxClauseParser} interface, which all define the following methods
\begin{enumerate}
    \item scan(), called three times (pass 0, 1 and 2), lexical syntax scan of the \nr{} clause
    \item getAssigns(), returns names of variables which received new values, mostly null
    \item generate(), generates Java source code from the \nr{} clause during pass 4
    \item interpret(), interprets the \nr{} clause, called by \keyword{RxInterpreter.nrx} on pass 3
\end{enumerate}

The remaining \keyword{Rx*.nrx} files are supporting input/output streaming, parsing and other translator functionality.


The following tables list all translator source files with their main function.

\begin{table}\caption{Clause parser source files}
\begin{tabularx}{\textwidth}{>{\bfseries}lX}
    \toprule
    RxClauseParser.nrx  &The interface used to group classes that parse and process individual clauses
    \\\midrule
    NrAddress.nrx   &An object that represents a parsed Address clause
    \\\midrule
    NrAnnotate.nrx  &An object that represents a parsed Annotation clause
    \\\midrule
    NrAssign.nrx    &An object that represents a parsed assignment instruction
    \\\midrule
    NrBlock.nrx     &An object that represents a generalised block start clause, extended by NrDo, NrLoop, and NrSelect
    \\\midrule
    NrCatch.nrx     &An object that represents a parsed Catch clause
    \\\midrule
    RxClass.nrx     &An object that represents a parsed Class clause
    \\\midrule
    NrDo.nrx        &An object that represents a parsed Do clause
    \\\midrule
    NrElse.nrx      &An object that represents a parsed Else clause
    \\\midrule
    NrEnd.nrx       &An object that represents a parsed End clause
    \\\midrule
    NrExit.nrx      &An object that represents a parsed Exit clause
    \\\midrule
    NrFinally.nrx   &An object that represents a parsed Finally clause
    \\\midrule
    NrIf.nrx        &An object that represents a parsed If clause
    \\\midrule
    NrImport.nrx    &An object that represents a parsed Import instruction
    \\\midrule
    NrInterpret.nrx &An object that represents a parsed Interpret clause
    \\\midrule
    NrIterate.nrx   &An object that represents a parsed Iterate clause
    \\\midrule
    NrLeave.nrx     &An object that represents a parsed Leave clause
    \\\midrule
    NrLevel.nrx     &An object that represents a traversable stack
    \\\midrule
    NrLoop.nrx      &An object that represents a parsed Loop instruction
    \\\midrule
    NrMethodcall.nrx    &An object that represents a method call instruction
    \\\midrule
    RxMethod.nrx    &An object that represents a Method instruction, extended by NrMethod
    \\\midrule
    NrMethod.nrx    &An object that represents a Method instruction
    \\\midrule
    NrNop.nrx       &An object that represents a parsed Nop clause
    \\\midrule
    NrNumeric.nrx   &An object that represents a parsed Numeric instruction
    \\\midrule
    NrOptions.nrx   &An object that represents a parsed Options instruction
    \\\midrule
    NrOtherwise.nrx &An object that represents a parsed Otherwise clause
    \\\midrule
    NrPackage.nrx   &An object that represents a parsed package instruction
    \\\midrule
    NrParse.nrx     &An object that represents a parsed Parse instruction
    \\\midrule
    NrProperties.nrx    &An object that represents a parsed properties instruction
    \\\midrule
    NrReturn.nrx    &An object that represents a parsed Return instruction
    \\\midrule
    NrSay.nrx       &An object that represents a parsed Say clause
    \\\midrule
    NrSelect.nrx    &An object that represents a parsed Select clause
    \\\midrule
    NrSignal.nrx    &An object that represents a parsed Signal instruction
    \\\midrule
    NrThen.nrx      &An object that represents a parsed Then instruction
    \\\midrule
    NrTrace.nrx     &An object that represents a parsed Trace instruction
    \\\midrule
    NrWhen.nrx      &An object that represents a parsed When clause
    \\\bottomrule
\end{tabularx}
\end{table}


\begin{table}\caption{Class related translator source files}
\begin{tabularx}{\textwidth}{>{\bfseries}lX}
    \toprule
    RxClasser.nrx       &The generalised class processor. The RxClasser object handles
    everything to do with classes: finding them, testing for them, and so on.
    \\\midrule
    RxClassImage.nrx    &An object that processes class files (or file images)
    \\\midrule
    RxClassInfo.nrx     &An object that describes what we know about a class
    \\\midrule
    RxClassPool.nrx     &The pool of RxClassInfo objects
    \\\midrule
    RxPersistClass.nrx  &An object that persists class files
    \\\midrule
    RxMapClassLoader.nrx &A class loader that loads classes from a map
    \\\midrule
    RxByteArrayJavaClass.nrx &Represents a Java class as a byte array
    \\\midrule
    RxType.nrx          &Describes the class and dimensions of an item (object or primitive).
    \\\midrule
    RxField.nrx         &An object that represents a known field, this can refer to a property or a method
    \\\bottomrule
\end{tabularx}
\end{table}


\begin{table}\caption{Translation and parsing source files}
\begin{tabularx}{\textwidth}{>{\bfseries}lX}
    \toprule
    NetRexxC.nrx        &A command shell to translate and compile or interpret one or more \nr{} programs
    \\\midrule
    RxTranslator.nrx    &An instance of a Venta translator, the 'boss'. Compilation occurs here unless -exec
    \\\midrule
    RxFlag.nrx          &An object that describes the per-program or per-compile flags
    \\\midrule
    RxParser.nrx        &The program parser for \nr{}, called three times, once for each pass. Its provider
    is RxClauser, which supplies tokenised clauses on demand
    \\\midrule
    RxClauser.nrx       &The lexical processor for \nr{} syntax. Given a streamer object which supplies lines of source code,
    the clauser object will parse the input lines and supply logical \nr{} clauses,
    expressed as a RxClause object (primarily an array of RxTokens)
    \\\midrule
    RxClause.nrx        &An object that represents a tokenised clause
    \\\midrule
    RxToken.nrx         &An object that represents a token in the input source file
    \\\midrule
    RxExprParser.nrx    &Parses an expression and constructs the corresponding RxCode object
    \\\midrule
    RxTermParser.nrx    &Parses a term and constructs the corresponding RxCode object
    \\\midrule
    RxCursor.nrx        &An object that represents the current context of execution or parsing. This may be
    either during initial parsing or per-thread while executing.
    \\\midrule
    RxCode.nrx          &An object that represents a sequence of code. Depending on the kind of the code,
    this may correspond to program, class, or method-level data, for example Java sourcecode, bytecodes,
    constants, etc.
    \\\midrule
    RxExpr.nrx          &An object that represents a parsed expression
    \\\midrule
    RxArray.nrx         &An object that represents a parsed array reference
    \\\midrule
    RxTracer.nrx        &Manages code generation for tracing
    \\\midrule
    RxModel.nrx         &Generates a model of a \nr{} program from RxClauser clauses, cleans code written
    in different styles, and indents for good code folding in your favorite editor
    \\\bottomrule
\end{tabularx}
\end{table}


\begin{table}\caption{Interpretation related source files}
\begin{tabularx}{\textwidth}{>{\bfseries}lX}
    \toprule
    NetRexxA.nrx        &\nr{} external API to translate and interpret one or more \nr{} programs
    \\\midrule
    NetRexxInterpreter.nrx &An early version of NetRexxA.nrx
    \\\midrule
    NetRexxI.nrx        &\nr{} internal API to support the compiled \keyword{interpret} instruction
    \\\midrule
    RxInterpreter.nrx   &The interpreter object, it takes requests (equivalent to execution requests in java.lang.reflect) and executes them, either using the reflection API or using direct interpretation for local classes
    \\\midrule
    RxInterpreterHelper.nrx   &The interpreter helper object, allows RxInterpreter to run as nodecimal
    \\\midrule
    RxSignal.nrx        &An object that represents an active Exception during execution
    \\\midrule
    RxSignalPend.nrx    &An object that wraps an invocation target Exception so we can tunnel it
    upwards without checking
    \\\midrule
    RxScriptEngine.nrx  &A JSR223 javax.script implementation supporting the compiled \keyword{interpret} instruction
    \\\midrule
    RxScriptEngineFactory.nrx  &The factory object which exposes metadata describing RxScriptEngine
    \\\midrule
    RxProxy.nrx         &The Proxy object, provides a proxy class as a byte array
    \\\midrule
    RxProxyLoader.nrx   &The ProxyLoader factory, used for loading proxies of local classes
    \\\bottomrule
\end{tabularx}
\end{table}


\begin{table}\caption{Other miscelaneous source files}
\begin{tabularx}{\textwidth}{>{\bfseries}lX}
    \toprule
    NrVersion.nrx       &Implements all metadata associated with a  \nr{} translator version
    \\\midrule
    RxProcessor.nrx     &Processor-level constants and repository of the master change list
    \\\midrule
    RuntimeConstants.nrx    &The Java RuntimeConstants
    \\\midrule
    NetRexxC.properties &The NetRexxC messages master file, each line is a single message, indexed by key
    \\\midrule
    NrAnsi.nrx          &Implements support for ANSI control sequences
    \\\midrule
    RxMessage.nrx       &Displays an error or warning message, identified by a key in NetRexxC.properties
    \\\midrule
    RxError.nrx         &Processes a compile-time error, constructs an RxMessage
    \\\midrule
    RxWarn.nrx          &Display an warning message during parsing
    \\\midrule
    RxQuit.nrx          &Processes a Quit (terminal error)
    \\\midrule
    RxBabel.nrx         &The interface that defines the methods that implement language-specific utility routines
    \\\midrule
    NrBabel.nrx         &Implements language-specific utility routines
    \\\midrule
    RxSource.nrx        &The interface that defines a valid program source
    \\\midrule
    RxFileReader.nrx    &Makes a file into an RxSource
    \\\midrule
    RxStreamer.nrx      &Manages the output streams for a program
    \\\midrule
    RxChunk.nrx         &An object that represents an output (Java) code chunk
    \\\midrule
    RxException.nrx     &An object that represents an exception reference
    \\\midrule
    RxProgram.nrx       &An object that describes per-program data
    \\\midrule
    RxMessageOutput.nrx &The interface to RxProgram
    \\\midrule
    RxForwardingJavaFileManager.nrx &Forwards calls to Java FileManager
    \\\midrule
    RxInMemoryJavaFileObject.nrx    &Provides Java source code as a CharSequence
    \\\midrule
    RxConverter.nrx     &A maker object that costs and effects conversions
    \\\midrule
    RxConvert.nrx       &An object that describes the cost of a conversion and the procedure
    to be used to effect the conversion.
    \\\midrule
    RxVariable.nrx      &An object that represents a variable local to a class, can be an
    argument to a method, a class property or local variable
    \\\midrule
    RxVarpool.nrx       &An object that manages and maintains RxVariables, exists on
    the class and method level
    \\\bottomrule
\end{tabularx}
\end{table}

