\chapter{Introduction}
The Programming Guide is the book that has the broadest scope of the publications in the \emph{NetRexx Programming Series}. Where the \emph{Language Reference} and the \emph{Quick Beginnings} need to be limited to a formal description and definition of the NetRexx language for the former, and a Quick Tour and Installation instructions for the latter, this book has no such limitations. It teaches programming, discusses computer language history and comparative linguistics, and shows many examples on how to make NetRexx work with diverse techologies as TCP/IP, Relational Database Management Systems, Messaging and Queuing (MQ\textsuperscript{\texttrademark}) systems, J2EE Containers as JBOSS\textsuperscript{\texttrademark} and IBM WebSphere Application Server\textsuperscript{\texttrademark}, discusses various rich- and thin client Graphical User Interface Options, and discusses ways to use NetRexx on various operating platforms. For many people, the best way to learn is from examples instead of from specifications. For this reason this book is rich in example code, all of which is part of the NetRexx distribution, and tested and maintained. This has had its effect on the volume of this book, which means that unlike the other publications in the series, it is probably not a good idea to print it out in its entirety; its size will relegate it to being used electronically.
\section*{Terminology}
The \emph{NetRexx Language Reference (NRL)} is the source of the definitive truth about the language. In this \emph{Programming Guide}, terminology is sometimes used more loosely than required for the more formal approach of the NRL. For example, there is a fine line distinguishing \emph{statement}, \emph{instruction} and \emph{clause}, where the latter is a more Rexx-like concept that is not often mentioned in relation to other languages (if they are not COBOL or SQL). While we try not to be confusing, \emph{clause} and \emph{statement} will be interchangibly used, as are \emph{instruction} and \emph{keyword instruction}.
\section*{Acknowledgements}
As this book is a compendium of decades of Rexx and NetRexx knowledge, it stands upon the shoulders of many of its predecessors, many of which are not available in print anymore in their original form, or will never be upgraded or actualized; we are indebted to many anonymous\footnote{because they are unacknowledged in the original publications} authors of IBM product documentation, and many others that we do know, and will thank in the following. If anyone knows of a name not mentioned here that should be, please be in touch. 

A big IOU goes out to Alan Sampson, who singlehandedly contributed more than one hundred NetRexx programming examples. The Redbook authors (Peter Heuchert, Frederik Haesbrouck, Norio Furukawa, Ueli Wahli, Kris Buelens, Bengt Heijnesson, Dave Jones and Salvador Torres) have provided some important documents that have shown, in an early stage, how almost everything on the JVM is better and easier done in NetRexx. Kermit Kiser also provided examples and did maintenance on the translator. Bill Finlason provided the Eclipse instructions. If anyone feels their copyright is violated, please do let us know, so we can take out offending passages or paraphrase them beyond recognition. As the usage of all material in this publication is quoted for educational use, and consists of short fragments, a fair use clause will apply in most jurisdictions.

