\chapter{Interfacing to Scripting Languages}
\nr{} contains standardized Java Scripting
support\marginpar{\color{gray}3.03}, and the NetRexxC.jar file is a self-contained JSR223 scripting engine. This facility opens up a number of possibilities to interface in a standardized manner with several scripting languages and other infrastructure, and offers an easy way for including interpreted \nr{} code in JVM applications. JSR223 is a standard for interacting with scripting languages that consists of:
\begin{enumerate}
\item A mechanism to find out for which scripting languages support is
  available
\item A way to choose one of them
\item An eval() call to dynamically specify and execute a program
\item A \emph{bindings} mechanism to bind variable names to values, to exchange objects with scripts
\item Optionally, a way to execute methods, functions or routines from
  larger programs
\item Optionally, a way to keep already compiled scripts around for repeated execution (with associated higher performance)
\end{enumerate}

The JSR223 specification\footnote{\url{http://www.jcp.org/en/jsr/detail?id=223}}
details the calls that are available in the \keyword{javax.scripting}
package. To use the JSR223 interface, Java 6 or higher is
required. The JAR file specification defines a service as a well-known
set of interfaces and (usually) abstract classes. A service provider
is a specific implementation of such a service. For scripting, the
service consists of \keyword{javax.script.ScriptEngineFactory}. All
classes that implement this interface are service providers. Service
providers identify themselves by placing a so-called
provider-configuration file in META-INF/services. Its filename
corresponds to the fully qualified name of the service class, which is
\keyword{javax.script.ScriptEngineFactory}. Each line of this file contains the
fully qualified name of a service provider. The factory
class of the \nr{} connector is \keyword{org.netrexx.jsr223.NetRexxScriptEngineFactory}. So the file \keyword{META-INF/services/javax.script.ScriptEngineFactory} contains one line with exactly this class name.
\section{Which JSR223 engines are on my system?}
The number of JSR223 engines available varies per JVM
implementation. The following code can be used to list these.
\lstinputlisting[label=enumeratejsr223,caption=Enumerate the JSR223 Engines on a JVM]{../../examples/jsr223/ListScriptingEngines.nrx} 

For example, the Java 8 SE version by Oracle on macOS delivers out of
the box:
\bash[stdout]
java ListScriptingEngines
\END

As one can see, the name of the engine, the language and its release are standard features for this query. The NetRexxC.jar file on the classpath adds the \nr{} implementation.
There can be any number of additional jar archives on the classpath to deliver engines for different JSR223 implementations for different languages.
\section{Selecting an engine}
When developing a program one is probably interested in using a specific implementation, and it is possible to request the loading of a specific JSR223 engine by name.
\begin{lstlisting}[label=choosingjsr223,caption=Choosing an engine]
import javax.script.

manager = ScriptEngineManager()
nrEngine = manager.getEngineByName("NetRexx")
\end{lstlisting}
The language engine can be selected by its short name, so there is no need to specify the longer name or its version.
\section{Evaluating a script}
This example shows how to do a simple thing that illustrates the value of being able to do this from other environments: calculating some number with \emph{numeric precision} set to some value that other languages cannot handle.
\begin{lstlisting}[label=evaljsr223,caption=Evaluating a script]
/* simple script invocation */
nrEngine.eval('numeric digits 17; say 111111111 * 111111111')
\end{lstlisting}
The output from this script would be:
\begin{alltt}
12345678987654321
\end{alltt}
\section{Bindings}
Bindings are name-value pairs whose keys are strings - they can be of
\Rexx{} type. Their behavior is defined through the
\keyword{javax.script.Bindings} interface. As for
\keyword{ScriptContext}, a basic implementation is provided called
\keyword{SimpleBindings}. Although bindings belong to script contexts,
\keyword{ScriptEngine} provides \keyword{createBindings()}, which
returns an uninitialized binding. Another method,
\keyword{getBindings()}, exists to return the bindings of a certain
scope. There are at least two scopes,
\keyword{ScriptContext.GLOBAL\_SCOPE} and
\keyword{ScriptContext.ENGINE\_SCOPE}. They represent key-value pairs
that are either visible to all instances of a script engine that have
been created by the same \keyword{ScriptengineManager}, or visible
only during the lifetime of a certain script engine instance. The
following program illustrates the use of bindings to store a value,
42, into the binding called \keyword{answer} and then using its
retrieved value in the evaluation of the statement \keyword{'say ``the
  answer is'' answer '}. The next action uses the handle \keyword{one}
for a value of 1, and uses its retrieved value to add it to the value
previously contained in the binding \keyword{answer}.
\lstinputlisting[label=bindingsjsr223,caption=Object Bindings]{../../examples/jsr223/ScriptDemo2.nrx} 
Note that in line two, the invocation is shortened a bit by getting rid of the intermediate \keyword{manager} object for instantiation of the language interface. Also note that in line 10, we chose, for illustration purposes, to store the \keyword{one} object into the bindings structure using a different name, \keyword{onemore}. This shows that the string used as identifier for the object is just a handle to it, and nothing more.
This would yield:
\bash[stdout]
java ScriptDemo2
\END
The different possibilities and language combinations will be
discussed in the paragraphs below.

\subsection{Obtaining a returncode}
The variable binding used for the return code from the NetRexx program
is called \code{returnobject}. This program illustrates its use: 
\lstinputlisting[label=rcjsr223,caption=Obtaining a returncode]{../../examples/jsr223/ScriptDemo3.nrx} 
\bash[stdout]
java ScriptDemo3
\END

 \section{Interpreted execution of \nr{} scripts from jrunscript}
 Another way of calling any NetRexx program, for interpretation, is to use the standard jrunscript executable that is included in Java 1.6 and beyond. For example, in the examples/rosettacode directory, one could specify:
\begin{verbatim}
jrunscript -l netrexx -cp $CLASSPATH -f RCSortingHeapsort.nrx
\end{verbatim}
The -l option instructs the jrunscript handler to choose NetRexx as
its standard scripting language. For NetRexx to be eligible as a
scripting language, NetRexxC.jar must be on the jrunscript classpath,
which is a separate classpath from the standard one. In this setup,
even \nr{} programs with a filename that is not valid as a classname,
can be executed as an interpreted script.

% \section{Using JavaScript from \nr{} programs}
% JavaScript support is built in from Java 1.6 onwards, and using it
% does not require placing another library on the classpath.
% Using JavaScript from \nr{} can have benefits, for example when using types native
% to JavaScript, like the JSON data interchange format.
% \lstinputlisting[label=rcjsr223,caption=Run JavaScript]{../../examples/jsr223/JSONDemo.nrx}

% which yields the following result:
% \bash[stdout]
% java JSONDemo
% \END

\section{Using AppleScript on macOS}
On macOS you can run an AppleScript using \nr{}. 
\lstinputlisting[label=rcjsr223,caption=Run AppleScript]{../../examples/jsr223/RunAppleScript.nrx}
 The AppleScript interpreter expects end-of-line characters at the end
 of every line, so make sure to include them in your script. The above
 script shows it is fairly straightforward to put a dialog box with a
 question on the screen. The example shows how to give an argument
 (ARGV) to a method, and how to put the method name in the bindings
 object in order to return the result upon evaluation. 
 
\section{Execution of \nr{} scripts from ANT tasks}
The jsr223 engine enables us to execute NetRexx scripts from the
ant\footnote{\url{http://ant.apache.org}} building tool using the
\code{<script>} tag. This was already possible using the BSF library,
where \nr{} was one of the originally supported languages,
but has become more straightforward with jsr223 scripting.
\lstinputlisting[label=anr223,caption=Run a \nr{} script in Ant]{../../examples/jsr223/antscript.xml}

Note that properties can be set in other parts of the ant xml file and
used in the ant script.
This script yields the following output:
\bash[stdout]
java -cp $CLASSPATH:../../ant/ant-launcher.jar:../../ant/ant.jar org.apache.tools.ant.launch.Launcher -f antscript.xml
\END

The task may use the BSF scripting manager or the JSR 223 manager that
is included in JDK6 and higher. This is controlled by the manager
attribute. The JSR 223 scripting manager is indicated by "javax", as
shown on line 7.

All items (tasks, targets, etc) of the running project are accessible from the script, using either their name or id attributes (as long as their names are considered valid Java identifiers, that is). This is controlled by the "setbeans" attribute of the task. The name "project" is a pre-defined reference to the Project, which can be used instead of the project name. The name "self" is a pre-defined reference to the actual <script>-Task instance.
From these objects you have access to the Ant Java API.

A classpath for execution of the script can be set using the
\code{classpath} attribute. A script contained in a separate file can
be executed using the \code{src} attribute.
\section{Integration of NetRexx scripting in applications}
Several applications offer a facility to script functionality using
the javax.scripting interface, akin to the way applications use the
RexxSAA interface for this purpose. 
 
\section{Interfacing between ooRexx and \nr{} using BSF4ooRexx}
BSF is a system for language interaction that originated in a research project at IBM, and predates JSR223 (and certainly its implementation in Java 6) for a number of years. BSF 2.x has its own interface, while modern BSF versions are an implementation of the JSR223 interfaces. BSF4ooRexx enables a bidirectional interface between ooRexx and Java, and enables one to use the large class library support for Java in ooRexx programs, but likewise the execution of ooRexx code from Java (including \nr{}) programs. BSF4ooRexx contains some special support for JVM programs written in \nr{}.

\section{General jsr-223 Implementation Notes}
This section describes some notes pertaining to specific jsr223 for
\nr{} design and implementation decisions.

\begin{itemize}
\item All engine scope bindings are passed to the script as variables - note that binding names containing periods have the periods changed to underscores to be legal variable names.
\item The NetRexx script engine is reused unless the script returned via an "exit" statement and the bindings are persistent which means that scripts will see the bindings (Objects) created by previous scripts
\item Arguments are passed both as the normal \code{arg} string and as the array binding \code{javax.script.argv} i.e. script variable \code{javax\_script\_argv}.
\item Scripts are executed via the NetRexxA API for interpreting a program from a string so they are not written to files.
\item The current version of the engine has no other optimization and only support for bare minimum JSR223 features (No compilable, invokeable, preparse or caching or user profiles or console, etc.).
\item When running as an Ant Script task, properties whose names
  contain periods are not passed to the bindings and must be accessed
  via project.getProperty('some.name') The workaround is to define a local Ant property as a global first and the scriptengine will overlay the global value with the local value in the bindings map
\item When running as an Ant Script task, properties can be set via
  project.setProperty('some.name', 'some value')
\item Script parms can be passed in an "arg" binding. Parse flags can be passed with a "netrexxflags" binding or in Ant with the usual "ant.netrexx.verbose", etc properties.
\item Ant scripts can use the nested classpath facility - It is automatically added to the classpath that NetRexx scans. Likewise any path segments from a thread context URLclassloader are added.
\item The engine will run programs (ie that have a main class) as well as scripts but bindings cannot then be auto added to the program namespace so
programs have to load bindings like this: \code{NetRexxScriptEngine.getObject("objectname")}
\end{itemize}
