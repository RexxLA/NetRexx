% .* ------------------------------------------------------------------
% .* \nr{} User's Guide                                              mfc
% .* Copyright (c) IBM Corporation 1996, 2000.  All Rights Reserved.
% .* ------------------------------------------------------------------
\chapter{Using \nr{} for Web applets}
\index{Web applets, writing}
\index{applets for the Web, writing}
Java Web applets are a deprecated application model, depending on web
browser plugins, and will be removed from the JDK. This
chapter will be removed when \nr{} support for Java versions that
includes web applets ends. Note that, for some time now, no mainstream web browser
supports Java applets.
Web applets can be written one of two styles:
\begin{itemize}
\index{binary arithmetic, used for Web applets}
\item Lean and mean, where binary arithmetic is used, and only core
Java classes (such as \emph{java.lang.String}) are used.  This is
recommended for World Wide Web pages, which may be accessed by people
using a slow internet connection.
Several examples using this style are included in the \nr{} package
\index{NervousTexxt example}
\index{ArchText example}
(eg., \emph{NervousTexxt.nrx} or \emph{ArchText.nrx}).
\item Full-function, where decimal arithmetic is used, and advantage is
taken of the full power of the \nr{} runtime (Rexx) class.
This is appropriate for intranets, where most users will have fast
connections to servers.
An example using this style is included in the \nr{} package
\index{WordClock example}
(\emph{WordClock.nrx}).
\end{itemize}
If you write applets which use the \nr{} runtime (or any other Java
classes that might not be on the client browser), the rest of this
section may help in setting up your Web server.

\index{HTTP server setup}
\index{Web server setup}
\index{runtime/web server setup}
A good way of setting up an HTTP (Web) server for this is to keep all
your applets in one subdirectory.  You can then make the \nr{} runtime
classes (that is, the classes in the package known to the Java Virtual
Machine as \emph{netrexx.lang}) available to all the applets by
unzipping \nr{}R.jar into a subdirectory \emph{netrexx/lang} below
your applets directory.
\newline
For example, if the root of your server data tree is
\begin{verbatim}
D:\mydata
\end{verbatim}
 you might put your applets into
\begin{verbatim}
D:\mydata\applets
\end{verbatim}
and then the \nr{} classes (unzipped from \nr{}R.jar) should be in
the directory
\begin{verbatim}
D:\mydata\applets\netrexx\lang
\end{verbatim}

The same principle is applied if you have any other non-core Java
packages that you want to make available to your applets: the classes in
a package called \emph{iris.sort.quicksorts} would go in a subdirectory
below \emph{applets} called \emph{iris/sort/quicksorts}, for example.

Note that since Java 1.1 or later it is possible to use the
classes direct from the \nr{}R.jar file. Please see the Java documentation for details.

