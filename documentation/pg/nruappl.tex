\chapter{User Interfaces}
\section{AWT}

\section{Web Applets using AWT}
\index{Web applets, writing}
\index{applets for the Web, writing}

Web applets can be written one of two styles:
\begin{itemize}
\index{binary arithmetic, used for Web applets}
\item Lean and mean, where binary arithmetic is used, and only core
Java classes (such as \emph{java.lang.String}) are used.  This is
recommended for optimizing webpages which may be accessed by people
using a slow internet connection.
Several examples using this style are included in the NetRexx package
\index{NervousTexxt example}
\index{ArchText example}
like the two listed below.
\lstinputlisting[label=nervtxt,caption=Nervous
Texxt]{../../../examples/ibm-historic/NervousTexxt.nrx}

\lstinputlisting[label=archtxt,caption=ArchText]{../../../examples/ibm-historic/ArchText.nrx}
\item Full-function, where decimal arithmetic is used, and advantage is
taken of the full power of the NetRexx runtime \emph{Rexx} class.

An example using this style is the below
\index{WordClock example}
\emph{WordClock.nrx}.
\lstinputlisting[label=wordclock,caption=WordClock]{../../../examples/ibm-historic/WordClock.nrx}
\end{itemize}
If you write applets which use the NetRexx runtime (or any other Java
classes that might not be on the client browser), the rest of this
section may help in setting up your Web server.

\index{HTTP server setup}
\index{Web server setup}
\index{runtime/web server setup}
A good way of setting up an HTTP (Web) server for this is to keep all
your applets in one subdirectory.  You can then make the NetRexx runtime
classes (that is, the classes in the package known to the Java Virtual
Machine as \emph{netrexx.lang}) available to all the applets by
unzipping NetRexxR.jar into a subdirectory \emph{netrexx/lang} below
your applets directory.
\newline
For example, if the root of your server data tree is
\begin{verbatim}
D:\mydata
\end{verbatim}
 you might put your applets into
\begin{verbatim}
D:\mydata\applets
\end{verbatim}
and then the NetRexx classes (unzipped from NetRexxR.jar) should be in
the directory
\begin{verbatim}
D:\mydata\applets\netrexx\lang
\end{verbatim}

The same principle is applied if you have any other non-core Java
packages that you want to make available to your applets: the classes in
a package called \emph{iris.sort.quicksorts} would go in a subdirectory
below \emph{applets} called \emph{iris/sort/quicksorts}, for example.

Note that since Java 1.1 or later it is possible to use the
classes direct from the NetRexxR.jar file.

\section{Swing}
Swing is the most commonly used name for the second attempt from the
SUN engineers to provide a graphical user interface library for the
JVM. With AWT also acknowledged by SUN to be a quick attempt that was
made just before release of the first Java package, it became clear
that it was rather taxing on system resources without compensation by
a pretty look. A case in point is the event mechanism, that
indiscriminately sends around mouse and keyboard events even when
nobody is listening to them. The architecture for Swing prescribes
registering for events before they are produced, and tries to have the
drawing done by the Java graphics engine instead of leaning heavily on
the operating system's native GUI functionality. The user interface
widgets that are produced by Java are called 'light' and their looks
can be changed by applying different skins, called \emph{'look-and-feel'} (LAF)
libraries.

In the first months of its existence Swing gathered quite a bad reputation because it made
the Java 1.2 releases that contained it very slow in starting up
programs that used the library. Consequently, much was invested in
performance studies by SUN engineers and these problems were
solved. One of the things that came out is that dividing the libraries
in a great many classes, done for performance reasons, worked
counterproductive. All these problems were solved over the years, and
developments in hardware and multithreading took care of the rest, and
nowadays Swing is a valid way of producing a rich client user
interface.

For esthetical reasons, it is best to research a bit in the third
party look-and-feel libraries that can be obtained. Swing can be made
to look beautiful, but it takes some care and the defaults are not helping.
\subsection{Creating NetRexx Swing interfaces with NetBeans}
\section{Web Frameworks}
\subsection{JSF}
