\chapter{Date and Time Arithmetic}\label{refdatetimearith}
\index{RexxDate}
\index{RexxTime}
\nr{} inherited \marginnote{\color{gray}4.02} the Classic Rexx Date and Time classes \keyword{RexxDate} and \keyword{RexxTime} in order to make it easier for \Rexx{} users to do Date and Time arithmetic in a familiar fashion. The implementation does not use Java Date logic (which changed over the years and became, from the \Rexx{} users point of view, vastly more complex). The results are equal to those of the mainstream Classic Rexx implementations.

Here are some examples how to use the \nr{} built-in functions to solve usual date calculation and conversion problems:

\lstinputlisting[label=dttodaytomorrow,caption=]{./todaytomorrow.nrx}

\begin{shaded}
\bash[stdout]
nrc -verbose0 -exec todaytomorrow.nrx
\END
\end{shaded}

\lstinputlisting[label=dtdateconv,caption=]{./dateconv.nrx}

\begin{shaded}
\bash[stdout]
nrc -verbose0 -exec dateconv.nrx
\END
\end{shaded}

\lstinputlisting[label=dtdayofyear,caption=]{./dayofyear.nrx}

\begin{shaded}
\bash[stdout]
nrc -verbose0 -exec dayofyear.nrx
\END
\end{shaded}

\lstinputlisting[label=dtdaydiff,caption=]{./datediff.nrx}

\begin{shaded}
\bash[stdout]
nrc -verbose0 -exec datediff.nrx
\END
\end{shaded}

\lstinputlisting[label=dtweekday,caption=]{./weekday.nrx}

\begin{shaded}
\bash[stdout]
nrc -verbose0 -exec weekday.nrx
\END
\end{shaded}

\section{Epoch}
The start date of the \Rexx{} \keyword(Date) function (01/01/0001) is different from the Posix (unix-linux) epoch (01/01/1970). With this algorithm Posix epoch based dates can be used with \nr{}.

\lstinputlisting[label=dtunixepoch,caption=]{./unixepoch.nrx}

\begin{shaded}
\bash[stdout]
nrc -verbose0 -exec unixepoch.nrx
\END
\end{shaded}

(The built-in RexxStream stream function has this already built in:

\lstinputlisting[label=dtfiledate,caption=]{./filedate.nrx}

\begin{shaded}
\bash[stdout]
nrc -verbose0 -exec filedate.nrx
\END
\end{shaded}

see page \pageref{refstreamio} for more examples of \nr{} Stream I/O.)