\chapter{Using a Docker image or the Native Compilers for JVM releases after 9}\label{native compiler}
While the portable version of the \nr{} cannot find classes that are in Java modules, macOS and Linux users on X86\_64 do not need to install a backlevel JVM if their system already contains a newer version. One alternative is using a Docker image to compile in a container. Another is to use the \emph{native} (i.e. compiled to native machine instructions for the instruction set architecture - operating system combination) versions of the translator.
\section{Which to choose}
Apart from having different JVM levels installed and having the ability to change between these with a script, the suggestion is to use a Docker container when you already are using Docker, this is the easiest option. This images also enable quick testing of different versions of NetRexx itself. When Docker is new to you, and you are running modern versions of macOS or Linux on a 64bit Intel machine, then the native executable might be a good alternative. Sometimes these have slightly better compile time performance. In both approaches, the final product is a .class file - except when running \nr{} Pipelines, where no executable is produced.
\section{Native executables}
Native executables are produced using the Graal JVM and enable executing JVM-8 level code with a JVM that is included in the executable module. The resulting class files are usable in recent versions of Java, for example JVM versions 9 to 12. They are an experimental feature. The \nr{} translator uses JVM features that require a fallback-JVM for execution; when one installs the Graal JVM for a specific machine architecture, it can be used to produce native executables from \nr{} source that do not require a JVM at all on the target machine. The natively executable versions of the translator are delivered in separate distributions, one for Linux and one for macOS. In each distribution, these executables and jars are to be found:
\begin{enumerate}
\item nrc
\item pipe
\item pipc
\item nrws
\item NetRexxC.jar
\end{enumerate}
These should be in a directory on the executable PATH environment variable, or executed with ./ (for example, ./nrc). Care should be taken to disable alias statements that point to the .class versions of these executables.
\subsection{Classpath considerations}
The executable version needs to be able to find NetRexx.jar in a subdirectory build/lib relative to itself. When unzipping the distribution, this has been taken care of. On the \emph{nrc} command, the classpath can be specified. In some cases, it seems necessary to have a copy of NetRexxC.jar in the current directory.\footnote{under investigation; this might be a Graal bug/feature.}
\section{Docker Image}
Docker is a container technology. It is available for Mac, Windows and Linux. Information can be found at http://docker.com.
The Community Edition can be used free of charge.
In short, a container is combined from several images and runs a Linux distribution that is mapped to calls for the host OS.
Starting with 3.07, docker images will be available for NetRexx. This has several advantages:
\begin{enumerate}
\item New releases can be tested without impacting the current installation of NetRexx
\item Easy testing on multiple JDK / JRE versions (OPen JDK, IBM J9, etc)
\item No JDK or JRE is required on your desktop / laptop / work machine to develop and run NetRexx programs
\item No installation of NetRexx and its required path and classpath 
\item An image is delivered with a Java version that is tested with the NetRexx release - so it is known to work
\item The eclipse batch compiler will not be required
\item NetRexx, its batchfiles and its classpath will be setup already to avoid all installation problems going forward
\item When your app needs native calls, only one version needs to be produced and maintained
\item It will insulate against incompatible JVM changes where NetRexx development needs to catch up
\item It will be the start of a distribution mechanism for NetRexx applications cq. libraries
\end{enumerate}
Two ways of using the image are foreseen:
A shell within the image
Working with a bind mounted directory in the shell of your local machine
Work with a shell within the image
As producing data within the image generally is not recommended this also involves a bind-mounted directory, but you will work inside of the shell in the docker container and you can use all the tools provided in the image. 
A suitable command line would be
\begin{verbatim}
docker run --rm -it -v "\$PWD":/nrx -w /nrx rvjansen/netrexx:latest zsh
\end{verbatim}
If you want to keep changes in the container (for example, when you added tools or configuration that are useful and need to go into a new image, based on this image), do not use the --rm switch. The docker documentation explains how to commit this container and tag its new image. The rvjansen/netrexx:3.07 will be downloaded once from the docker hub, when it is not on the local machine yet. It will know it has been downloaded the next time you start this image. Also, it will detect when the image has been updated.
The -it switcher are needed for an interactive terminal session. the -v switch bind mounts the current directory into a directory /nrx in the image.

Compile or exec from a shell on your host machine
The term 'host machine' is used here to indicate the fact that the docker image can be seen as running a guest OS\footnote{while in reality it is a separated session on the same Linux kernel}. 
A suitable command line would be: (assuming you want to compile a class called RSAnrx in the local directory)
\begin{verbatim}
docker run --rm -v "\$PWD":/nrx -w /nrx rvjansen/netrexx:latest nrc RSAnrx
\end{verbatim}
Here, --rm will make sure the container is not kept, the -v tells docker to bind mount the current directory to a directory /nrx within the container, and -w sets this as the working directory. The rvjansen/netrexx:3.07 will be downloaded once from the docker hub, when it is not on the local machine yet. It will know it has been downloaded the next time you start this image. Of course, in most shells is it possible to alias this command, or start a batchfile, c.q. a shell script containing this.

