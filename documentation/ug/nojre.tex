\chapter{Running on a JRE-only environment\label{nosdk}}
 
\section{Eclipse Batch Compiler}
NetRexx (since the 3.01 package) can be used on a JRE-only environment; it does not need an SDK (JDK) when the included ecj (Eclipse Compiler for Java) is available on the classpath. This compiler is a part of the Eclipse JDT Core, which is the Java infrastructure of the Java IDE. This is an incremental Java compiler. It is based on technology evolved from the VisualAge for Java compiler and maintained by IBM and the Eclipse Foundation. In particular, it allows one to run and debug code which still contains unresolved errors. Future releases of NetRexx might be exploring more of the features of this compiler, like the extensive error reporting and  Currently, the 4.2 level of the core compiler jar is delivered with NetRexx. There are other standalone Java compilers, but after extensive research we have chosen to include this one. Using the \emph{–nocompile} and \emph{–keepasjava} options it is always possible to substitute your own compilers as subsequent stages in the build process.
 
\section{The nrx.compiler property}
The NetRexx language processor is a translator package that either interprets or executes NetRexx language source, and (by default) compiles the generated Java language source code with the SDK-included \emph{javac} compiler, or rather, the Java compiler class sun.tools.javac.Main class that is delivered (in most implementations) in the tools.jar file, that is also called by the javac executable. A new property is introduced to make the language processor choose the ecj compiler\footnote{the -D option is used on the java command line to specify a system property to the java VM}:
 
\begin{verbatim}
-Dnrx.compiler=ecj
\end{verbatim}
 
This directs the NetRexxC processor to use the ecj compiler to do the java compile step instead of javac. For retroactive continuity, this property can also be set to javac - which is still the default when the property is not specified. The \texttt{NetRexxC} command script can, on systems that do not have a javac compiler installed, be changed to state
 
\begin{verbatim}
java -Dnrx.compiler=ecj org.netrexx.process.NetRexxC $*
\end{verbatim}
 
In this case all compiles started with the \texttt{nrc} command will use the Eclipse compiler. Only in case of Java compiler errors, when the compiler output will be shown, will the difference be apparent. Installer support is planned to include this property automatically when during NetRexx installation the javac compiler jar is not detected. When compiling using the \texttt{-time} option, the right compiler name will be indicated.
 
\section{Interpreting}
For completeness, it is confirmed here that interpretative execution also works on a JRE-only system, and does not require a Java compiler.
 