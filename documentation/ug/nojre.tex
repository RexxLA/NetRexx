\chapter{Running on a JRE-only environment\label{nosdk}}
 
\section{Eclipse Batch Compiler}
\nr{} can be used on a JRE-only environment; it \marginnote{\color{gray}3.01} does not need an SDK (JDK) when the included ecj (Eclipse Compiler for Java) jar file is available on the classpath. This compiler is a part of the Eclipse JDT Core, which is the Java infrastructure of the Java IDE. This is an incremental Java compiler. It is based on technology evolved from the VisualAge for Java compiler and maintained by IBM and the Eclipse Foundation. In particular, it allows one to run and debug code which still contains unresolved errors. Future releases of \nr{} might be exploring more of the features of this compiler, like the extensive error reporting. Currently, the \ecjjarname{} level of the core compiler jar is delivered with \nr{}. There are other standalone Java compilers, but after extensive research we have chosen to include this one. Using the \emph{–nocompile} and \emph{–keepasjava} options it is always possible to substitute your own compilers as subsequent stages in the build process.
 
\section{The -ecj and -javac translator options}
The \nr{} language processor is a translator package that either interprets or executes \nr{} language source, and (by default) compiles the generated Java language source code with the SDK-included \emph{javac} compiler, or rather, the Java compiler class sun.tools.javac.Main class that is delivered (in most implementations) in the tools.jar file, which is also called by the javac executable. An option  \marginnote{\color{gray}3.04}is introduced to make the language processor choose the ecj compiler.
 
\begin{verbatim}
nrc -ecj sourcefile.nrx
\end{verbatim}
 
This directs the \nr{}C processor to use the ecj compiler to do the java compile step instead of javac. This option can also be set to javac - which is still the default when the option is not specified. The \texttt{\nr{}C} command script can, on systems that do not have a javac compiler installed, be changed to state
 
\begin{verbatim}
java org.netrexx.process.NetRexxC -ecj $*
\end{verbatim}
 
In this case all compiles started with the \texttt{nrc} command will use the Eclipse compiler. Only in case of Java compiler errors, when the compiler output will be shown, will the difference be apparent. Installer support is planned to include this property automatically when during \nr{} installation the javac compiler jar is not detected. When compiling using the \texttt{-time} option, the right compiler name will be indicated.
\section{The netrexx\_java environment variable}\index{netrexx\_java (environment variable}
The \nr{}C compile scripts pass the environment variable \texttt{netrexx\_java} to the Java VM at start. The compiler selection can be placed in the environment (in a slightly adapted and more historic form) and no change to the \nr{}C script is required. In Windows for example:
\begin{verbatim}
set netrexx_java=-Dnrx.compile=ecj
\end{verbatim}
A scan will be performed for a suitable compiler when the preferred one is not found. \marginnote{\color{gray}3.04}

\section{Interpreting}
For completeness, it is confirmed here that interpretative execution also works on a JRE-only system, and does not require a Java compiler. The \nr{} translator produces the required bytecode and proxy classes without any need for a Java compiler.
 