.* ------------------------------------------------------------------
.* NetRexx User's Guide                                              mfc
.* Copyright (c) IBM Corporation 1996, 2000.  All Rights Reserved.
.* ------------------------------------------------------------------
:h2 id=useapplet.Using NetRexx for Web applets
.pi /Web applets, writing
.pi /applets for the Web, writing
:p.
Web applets can be written one of two styles:
:ul.
:li.
.pi /binary arithmetic, used for Web applets
:q.Lean and mean:eq., where binary arithmetic is used, and only core
Java classes (such as :m.java.lang.String:em.) are used.  This is
recommended for World Wide Web pages, which may be accessed by people
using a slow dial-up connection.
Several examples using this style are included in the NetRexx package
.pi /NervousTexxt example
.pi /ArchText example
(&eg., :m.NervousTexxt.nrx:em. or :m.ArchText.nrx:em.).
:li.
:q.Full-function:eq., where decimal arithmetic is used, and advantage is
taken of the full power of the NetRexx runtime (Rexx) class.
This is appropriate for intranets, where most users will have fast
connections to servers.
An example using this style is included in the NetRexx package
.pi /WordClock example
(:m.WordClock.nrx:em.).
:eul.
:p.
If you write applets which use the NetRexx runtime (or any other Java
classes that might not be on the client browser), the rest of this
section may help in setting up your Web server.
:p.
.pi /HTTP server setup
.pi /Web server setup
.pi /runtime/web server setup
A good way of setting up an HTTP (Web) server for this is to keep all
your applets in one subdirectory.  You can then make the NetRexx runtime
classes (that is, the classes in the package known to the Java Virtual
Machine as :m.netrexx.lang:em.) available to all the applets by
unzipping NetRexxR.jar into a subdirectory :m.netrexx/lang:em. below
your applets directory.
:p.
For example, if the root of your server data tree is
:xmp.
D:/mydata
:exmp.
:pc.then you might put your applets into
:xmp.
D:/mydata/applets
:exmp.
:pc.and then the NetRexx classes (unzipped from NetRexxR.jar) should be in
the directory
:xmp.
D:/mydata/applets/netrexx/lang
:exmp.
:p.
The same principle is applied if you have any other non-core Java
packages that you want to make available to your applets: the classes in
a package called :m.iris.sort.quicksorts:em. would go in a subdirectory
below :m.applets:em. called :m.iris/sort/quicksorts:em., for example.
:p.
Note that with Java 1.1 or later it should be possible to use the
classes direct from the NetRexxR.jar file providing that the browser
being used is at a Java 1.1 level.  This may also depend on your server
being set up correctly.  Please see the Java documentation for details.
.*
