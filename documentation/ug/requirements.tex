\chapter{Requirements}
\nr{} \splice{java org.netrexx.process.NrVersion} runs on a wide variety of
hardware and operating systems; all releases are tested on (non-exhaustive):
\begin{enumerate}
\item Windows Desktop and Server editions, with OpenJDK, Oracle and
  IBM JVMs, GRaalVM
\item Linux, with OpenJDK, Oracle and IBM JVMs, GraalVM, including z/Linux
\item macOS with OpenJDK,  Oracle JVM, GraalVM and Amazon VM (Corretto)
\item Android on ARM hardware with Dalvik virtual machine
\item z/OS, z/Linux with IBM JVM.
\item The Raspberry Pi, using Raspbian Linux and its included JDK, or OpenJDK
\end{enumerate}
\nr{} runs equally well on 32- or 64-bit JVMs. As the translator is
a command line tool, no graphics configuration is required, and
headless operation is supported. Care has been taken to keep the \nr{} runtime small.

The class file format, however, of the current release distribution,
is\splice{java VersionTest}.; for older formats, you
can build \nr{} yourself or request assistance from the development
team\footnote{see the NetRexx project at SourceForge.net} for a special build.
\begin{shaded}\noindent
Since \nr{} 3, \nr{} requires only a
JRE\footnote{Java Runtime Environment} for program development, where previously a
Java JDK\footnote{Software Development Kit} was required. For serious development
purposes a JDK is recommended, as the tools found therein might
assist the development process.
\end{shaded}\indent

\begin{shaded}\noindent
Starting with \nr{} 4, the Java Platform Module System is supported.
\end{shaded}\indent

