\chapter{Requirements}
\nr{} \nrversion{} runs on a wide variety of
hardware and operating systems; all releases are tested on (non-exhaustive):
\begin{enumerate}
\item Windows Desktop and Server editions, with Oracle and IBM JVMs
\item Linux, with Oracle and IBM JVMs, including z/Linux
\item MacOSX with OpenJDK and Apple JVM
\item Android on ARM hardware with Dalvik virtual machine
\item z/OS OMVS
\item eComstation 2.x (OS/2) with eComstation Java 1.6
\item The Raspberry Pi, using Raspbian Linux and Oracle Embedded
  Edition ARM JDK-8
\end{enumerate}
\nr{} runs equally well on 32- or 64-bit JVMs. As the translator is
a command line tool, no graphics configuration is required, and
headless operation is supported. Care is taken to keep the \nr{} runtime small, and to keep
compatibility with earlier(post-beta) Java releases, older operating systems and
limited devices environments. 

The class file format, however, of the current release distribution, is\splice{cd ../../test;java VersionTest}.; for older formats, you
can build \nr{} yourself or request assistance from the development
team\footnote{developers@netrexx.kenai.com; you will
  need to be member of the Kenai \nr{} project} for a special build.
\begin{shaded}\noindent
Since release 3.01, \nr{} requires only a
JRE\footnote{Java Runtime Environment} for program development, where previously a
Java SDK\footnote{Software Development Kit} (earlier name: JDK) was required. For serious development
purposes a Java SDK is recommended, as the tools found therein might
assist the development process.
\end{shaded}\indent