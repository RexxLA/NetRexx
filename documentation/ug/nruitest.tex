.* ------------------------------------------------------------------
.* NetRexx User's Guide                                              mfc
.* Copyright (c) IBM Corporation 1996, 2000.  All Rights Reserved.
.* ------------------------------------------------------------------
:h2 id=testing.Testing the NetRexx installation
.pi /testing, NetRexx installation
.pi /installation/testing of
:p.
After installing NetRexx, it is recommended that you test that it is
working correctly.  If there are any problems, check the
:a id=problems.:cit.Installation Problems:ecit. section:ea..
:p.
To test your installation, make the directory to which you copied the
executables the current directory, then (being very careful to get the
case of letters correct):
:ol.
:li.
Enter the command
\end{verbatim}
java COM.ibm.netrexx.process.NetRexxC hello
:exmp.
:pc.This should run the NetRexx compiler, which first translates the
NetRexx program :m.hello.nrx:em. to the Java program :m.hello.java:em..
It then invokes the default Java compiler (:m.javac:em.), to compile the
file :m.hello.java:em. to make the binary class file :m.hello.class:em..
The intermediate :m..java:em. file is then deleted, unless an error
occurred or you asked for it to be kept.
:fn.
For example, by specifying the :m.-keep:em. or :m.-nocompile:em. flags.
:efn.
:p.
If you get errors from Java and you're running Java 1.2 or later,
first re-check the final two steps in the :a id=inj12.:cit.Installing
for Java 1.2+:ecit. section:ea. before trying the
:a id=problems.:cit.Installation Problems:ecit. section:ea..
.* - - -
:li.
Enter the command
\end{verbatim}
java hello
:exmp.
:pc.This runs (interprets the bytecodes in) the :m.hello.class:em. file,
which should display a simple greeting.  On some systems, you may
first have to add the directory that contains the :m.hello.class:em.
file to the CLASSPATH setting so Java can find it.
.* - - - -
.cp 3
:li.
With the sample scripts provided (:m.NetRexxC.cmd:em., :m.NetRexxC.bat:em.,
or :m.NetRexxC.sh:em.,), or the equivalent in the scripting language of
your choice, the steps above can be combined into a simple single
command such as:
\end{verbatim}
NetRexxC.sh -run hello
:exmp.
:p.
This package also includes a trivial :m.nrc.cmd:em., and
matching :m.nrc.bat:em. and :m.nrc:em. scripts, which simply pass on
their arguments to NetRexxC; :q.:m.nrc:em.:eq. is just a shorter name
that saves keystrokes, so for the last example you could type:
\end{verbatim}
nrc -run hello
:exmp.
:pc.Note that scripts may be case-sensitive, and unless running the OS/2
Rexx script, you will probably have to spell the name of the program
exactly as it appears in the filename.  Also, to use :m.-run:em., you
may need to omit the :m..nrx:em. extension.
:p.
You could also edit the appropriate :m.nrc.cmd:em., :m.nrc.bat:em.,
or :m.nrc:em. script and add your favourite :q.default:eq. NetRexxC
options there.
For example, you might want to add the :m.-prompt:em. flag (described
later) to save reloading the translator before every compilation.
If you do change a script, keep a backup copy so that if you install
a new version of the NetRexx package you won't overwrite your changes.
:eol.
