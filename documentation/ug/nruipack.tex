% .* ------------------------------------------------------------------
% .* NetRexx User's Guide                                              mfc
% .* Copyright (c) IBM Corporation 1996, 2000.  All Rights Reserved.
% .* ------------------------------------------------------------------
\chapter{Unpacking the NetRexx package}
\index{unpacking}
\index{zip files, unpacking}
\index{NetRexx package}
\index{package/NetRexx}
:p.
The NetRexx package is shipped as a collection of files compressed into
the file :m.NetRexx.zip:em..
:p.
You probably know how to handle :m..zip:em. files, but a word of
caution: the packages contain directory structures, and files with
:q.long names:eq. (that is, not of 8.3 maximum length names) which are
case-sensitive.  Many utilities, including some Windows versions of
:kw.unzip:ekw., can lose case information, truncate names, or fail to
restore directories.
.* - - - -
:h4.Unpacking the NetRexx.zip file
:p.
\index{Info-ZIP utility}
\index{PKZIP utility}
\index{jar command, used for unzipping}
The most common utilities for :q.unzipping:eq. are :kw.Info-ZIP:ekw.,
:kw.WinZip:ekw., and :kw.PKZIP:ekw..  An :kw.unzip:ekw. command is also
included in most Linux distributions.
You can also use the :kw.jar:ekw. command which comes with all Java
development kits.
:p.
Choose where you want the NetRexx directory tree to reside, and unpack
the zip file in the directory which will be the parent of the NetRexx
tree.
Here are some tips:
:ul.
:li.
Ensure that you are unzipping to a disk that supports long file names
(for example, an HPFS disk or equivalent on OS/2 or Windows).
:li.
:kw.Info-ZIP::ekw. use version 5.12 (August 1994) or later.  The syntax for
unzipping NetRexx.zip is simply
\end{verbatim}
unzip NetRexx
:exmp.
:pc.which should create the files and directory structure directly.
.* Please see later in this document for complete installation
.* instructions.
:li.
:kw.WinZip::ekw. all versions support long file names.
:li.
:kw.PKZIP::ekw. use a version that supports long file names.  The syntax
for unzipping NetRexx.zip is
\end{verbatim}
pkunzip -d NetRexx
:exmp.
:pc.which should create the files and directory structure directly.  The
:q.:m.-d:em.:eq. flag indicates that directory structure should be
preserved.
:li.
:kw.Linux unzip::ekw. use the syntax: :m.unzip -a NetRexx:em..  The
:q.:m.-a:em.:eq. flag will automatically convert text files to Unix
format.
:li.
:kw.jar::ekw. The syntax for unzipping NetRexx.zip is
\end{verbatim}
jar xf NetRexx.zip
:exmp.
:pc.which should create the files and directory structure directly.
The :q.:m.x:em.:eq. indicates that the contents should be extracted, and
the :q.:m.f:em.:eq. indicates that the zip file name is specified.  Note
that the extension (:m..zip:em.) is required.
:eul.
.cp 8
:p.
After unpacking, the following directories should have been created:
:fn.
On Unix and Linux systems, the directory separator will be :q./:eq.
instead of :q.\:eq..
:efn.
:dl termhi=4 tsize=50mm.
:dt.NetRexx
:dd.:p.Root of the tree, which should contain the
file :m.read.me.first:em., which contains quick installation
instructions
:dt.NetRexx\browse
:dd.:p.
The directory which contains documentation and sample programs and
applets.
To view these, point your web browser at :m.NetRexx\browse\netrexx.html:em..
You can also go straight to this User's Guide by
browsing :m.NetRexx\browse\nrusers.html:em..
:dt.NetRexx\lib
\index{class files,translator}
\index{translator,class files}
:dd.:p.
Contains the NetRexx compiler/interpreter classes (in :m.NetRexxC.jar:em.).
:dt.NetRexx\runlib
:dd.:p.
\index{runtime,class files}
\index{class files,runtime}
Contains the NetRexx runtime classes (in :m.NetRexxR.jar:em.).  These are
included in the NetRexxC.jar, so are not normally needed.
:dt.NetRexx\netrexx\lang
:dd.:p.
Contains the NetRexx runtime class files for access by a browser while
running the applet samples.
:dt.NetRexx\bin
:dd.:p.
Contains sample scripts making it easier to use the
compiler/interpreter.   The simple test case :m.hello.nrx:em. is also
included.
:edl.
