% .* ------------------------------------------------------------------
% .* \nr{} User's Guide                                              mfc
% .* Copyright (c) IBM Corporation 1996, 2000.  All Rights Reserved.
% .* ------------------------------------------------------------------
\chapter{Using the prompt option}
\index{compiling,interactive}
\index{option,prompt}
\index{flag,prompt}
\index{prompt option}
\index{interactive translation}
The \textbf{prompt} option may be be used for interactive invocation of
the translator. This requests that the processor not be ended after a
file (or set of files) has been processed.  Instead, you will be
prompted to enter a new request.  This can either repeat the process
(perhaps if you have altered the source in the meantime), specify a new
set of files, or alter the processing options.
\newline
On the second and subsequent runs, the processor will re-use class
information loaded on the first run.  Also, the classes of the processor
itself (and the \emph{javac} compiler, if used) will not need to be
verified and JIT-compiled again.  These savings allow extremely fast
processing, as much as fifty times faster than the first run for small
programs.
\newline
When you specify \emph{-prompt} on a \nr{}C command, the \nr{}
program (or programs) will initially be processed as usual, according to
the other flags specified.  Once processing is complete, you will be
prompted thus:
\begin{verbatim}
Enter new files and additional options, '=' to repeat, 'exit' to end:
\end{verbatim}.
\newline
At this point, you may enter:
\begin{itemize}
\item One or more file names (with or without additional flags): the previous
process, modified by any new flags, is repeated using the source file
or files specified.  Files named previously are not included in the
process (unless they are named again in the new list of names).
\item Additional flags (without any new files): the previous process, modified
by the new flags, is repeated, on the same files as before.
Note that flags are accumulated; that is, flags are not reset to
defaults between prompts.
\item The character \emph{=} this simply repeats the previous process,
on the same file or files (which may have had their contents changed
since the last process) and using the same flags.  This is especially
useful when you simply wish to re-compile (or re-interpret, see below)
the same file or files after editing.
\index{interactive translation,repeating}
\item
\index{interactive translation,exiting}
The word \emph{exit}, which causes \nr{}C to cease execution
without any more prompts.
\item
Nothing (just press Enter or the equivalent) -- usage hints, including
the full list of possible options, etc., are displayed and you are then
prompted again.
\end{itemize}

