\chapter{Installation}
This chapter of the document tells you how to unpack, install, and test the NetRexx translator package. This will install documentation, samples, and executables. It will first state some generic steps that are sufficient for most users. The appendices contain very specific instructions for a range of platforms that NetRexx is used on. 
Note that to run any of the samples, or use the NetRexx translator, you must have already installed the Java runtime (and toolkit, if you want to compile NetRexx programs using the default compiler). 
The NetRexx samples and translator, as of version 3.01, will run on 
Java version 1.5 or later\footnote{For earlier versions of Java, NetRexx 2.05 is available from the NetRexx.org website.}. To do anything more than run NetRexx programs with the runtime package, a Java software development kit is required. You can test whether Java is installed, and its version, by trying the following command at a command prompt:
\begin{verbatim} 
    java –version
\end{verbatim}
which should display a response similar to this:
\begin{verbatim} 
java version "1.6.0_26"
Java(TM) SE Runtime Environment (build 1.6.0_26-b03-383-11A511)
Java HotSpot(TM) 64-Bit Server VM (build 20.1-b02-383, mixed mode)
\end{verbatim}
For more information on Java installation:
\begin{enumerate} 
\item For some operating environments
\item For other operating systems, see the Oracle Java web page\footnote{at \url{http://www.javasoft.com}} – or other suppliers of Java toolkits.
\end{enumerate}
\section{Unpacking the NetRexx package}
The NetRexx package is shipped as a collection of files compressed into the file NetRexx<version>.zip. 
Most modern operating environments can uncompress a .zip package by doubleclicking.
\subsection{Unpacking the NetRexx.zip file} 
An unzip command is included in most Linux distributions, and Mac OSX. You can also use the jar command which comes with all Java development kits. 
Choose where you want the NetRexx directory tree to reside, and unpack the zip file in the directory which will be the parent of the NetRexx tree. Here are some tips: 
The syntax for unzipping NetRexx.zip is simply
\begin{verbatim}
        unzip NetRexx
\end{verbatim}
which should create the files and directory structure directly.
\index{unpacking}
\index{zip files, unpacking}
\index{NetRexx package}
\index{package/NetRexx}
\begin{itemize} 
\item WinZip: all versions may be used
\item Linux unzip: use the syntax: unzip –a NetRexx. The “–a” flag will automatically convert text files to Unix format if necessary
\item jar: The syntax for unzipping NetRexx.zip is
\begin{verbatim} 
jar xf NetRexx.zip 
\end{verbatim}
\index{jar command, used for unzipping}
\end{itemize}
which should create the files and directory structure directly. The “x” indicates that the contents should be extracted, and the “f” indicates that the zip file name is specified. Note that the extension (.zip) is required. 

After unpacking, the following directories1 should have been created:
-TODO-
\section{Installing the NetRexx Translator}
The NetRexx package includes the NetRexx translator – a Java application which can be used for compiling, interpreting, or syntax-checking NetRexx programs. The procedure for installation is briefly as follows\footnote{For Windows operating system, forward slashes are backslashes.} (full details are given later):
\begin{enumerate}
\item Make the translator visible to the Java Virtual Machine (JVM) - either:
\begin{itemize} 
\item Add the full path and filename of the NetRexx/lib/NetRexxC.jar to the CLASSPATH environment variable for your operating system. 
Note: if you have a NetRexxC.zip in your CLASSPATH from an earlier version of NetRexx, remove it (NetRexxC.jar replaces NetRexxC.zip).
\item Or (deprecated): Copy the file NetRexx/lib/NetRexxC.jar to the jre/lib/ext directory in the Java installation tree. The JVM will automatically find it there and make it available\footnote{ This has serious drawbacks, however:
This breaks NetRexx applications running in custom class loader environments such as jEdit and NetRexxScript, as well as some JSP containers.
As soon as the Java version is updates, NetRexx applications may mysteriously – due to the now obsolete path - fail.
Running multiple versions of Java and NetRexx for testing purposes will become very hard when this way of installing is chosen.}.
\end{itemize}
\item Copy all the files in the NetRexx/bin directory to a directory in your PATH (perhaps the /bin directory in the Java installation tree). This is not essential, but makes shorthand scripts and a test case available. 
\item Make the file /lib/tools.jar (which contains the javac compiler) in the Java tree visible to the JVM. You can do this either by adding its path and filename to the CLASSPATH environment variable, or by moving it to the jre/lib/ext directory in the Java tree. This file sometime goes under different names, that will be mentioned in the platform-specific appendices.
\item Test the installation by making the /bin directory the current directory and issuing the following two commands exactly as written: 
\begin{verbatim}
       java org.netrexx.process.NetRexxC hello
       java hello
\end{verbatim}
The first of these should translate the test program and then invoke the javac compiler to generate the class file (hello.class) for the program. The second should run the program and
display a simple greeting.
\end{enumerate}
If you have any problems or errors in the above process, please read the detailed instructions and problem-solving tips that follow. 

\section{Installing just the NetRexx Runtime}
\index{runtime,installation}
\index{installation,runtime only}
\index{NetRexxR runtime classes}
If you only want to run NetRexx programs and do not wish to compile or
interpret them, or if you would like to use the NetRexx string (Rexx)
classes from other languages, you can install just the NetRexx runtime
classes.
\newline
To do this, follow the appropriate instructions for installing the
compiler, but use the NetRexxR.jar instead of NetRexxC.jar.
The NetRexxR.jar file can be found in the \emph{NetRexx/runlib} directory.
\newline
You do not need to use or copy the executables in the \emph{NetRexx/bin}
directory.
\newline
The NetRexx class files can then be referred to from Java or NetRexx
programs by importing the package \emph{netrexx.lang}.  For
example, a string might be of class \emph{netrexx.lang.Rexx}.
\newline
For information on the \emph{netrexx.lang.Rexx} class and other classes
in the runtime, see the \emph{NetRexx Language Reference} document.

\textbf{note} If you have already installed the NetRexx translator
(NetRexxC.jar) then you do not need to install NetRexxR.jar; the latter
contains only the NetRexx runtime classes, and these are already
included in NetRexxC.jar.

\section{Setting the CLASSPATH}\label{install_classpath}
Most implementations of Java use an environment variable called CLASSPATH to indicate a search path for Java classes. The Java Virtual Machine and the NetRexx translator rely on the CLASSPATH value to find directories, zip files, and jar files which may contain Java classes. 
The procedure for setting the CLASSPATH environment variable depends on your operating system (and there may be more than one way). Please refer to Appendix 1 for your specific platform.
\begin{itemize}
\item For Linux and Unix (BASH, Korn, or Bourne shell), use:
\begin{verbatim}
        CLASSPATH=<newdir>:\$CLASSPATH 
        export CLASSPATH
\end{verbatim}

\item Changes for re-boot or opening of a new window should be placed in your /etc/profile, .login, or .profile file, as appropriate. 
\item For Linux and Unix (C shell), use:
\begin{verbatim}
        setenv CLASSPATH <newdir>:\$CLASSPATH 
\end{verbatim}
Changes for re-boot or opening of a new window should be placed in
your .cshrc file. If you are unsure of how to do this, check the
documentation you have for installing the Java toolkit.
\item For Windows operating systems, it is best to set the system wide
  environment, which is accessible using the Control Panel (a search
  for ``environment'' offsets the many attempts to relocate the exact
  dialog in successive Windows Control Panel versions somewhat).
\item In Windows \emph{Powershell}, limitations set by the
  administrator can determine which kind of scripting (using
  Powershell, not \nr) can be undertaken. It might be difficult to
  modify the enviroment, and a different from scripting under the
  \texttt{cmd.exe} processor is that the environment is local to an
  execution unit of a line. When changing the environment is allowed,
  and a Powershell script is used to start the \nr translator, this is
  how it can be done:
 
\end{itemize}

\section{Testing the NetRexx Installation}\label{testing}
After installing NetRexx, it is recommended that you test that it is working correctly. If there are any problems, check the Installation Problems section. 
To test your installation, make the directory to which you copied the executables the current directory, then (being very careful to get the case of letters correct):
\begin{enumerate}
\item Enter the command
 \begin{verbatim}
        java org.netrexx.process.NetRexxC hello
\end{verbatim}
This should run the NetRexx compiler, which first translates the NetRexx program hello.nrx to the Java program hello.java. It then invokes the default Java compiler (javac), to compile the file hello.java to make the binary class file hello.class. The intermediate .java file is then deleted, unless an error occurred or you asked for it to be kept.4 
\item Enter the command
 \begin{verbatim} 
java hello 
\end{verbatim}
This runs (interprets the bytecodes in) the hello.class file, which should display a simple greeting. On some systems, you may first have to add the directory that contains the hello.class file to the CLASSPATH setting so Java can find it. 
\item With the sample scripts provided (NetRexxC.cmd, NetRexxC.bat, or NetRexxC.sh,), or the equivalent in the scripting language of your choice, the steps above can be combined into a simple single command such as:
  \begin{verbatim}
 NetRexxC.sh –run hello
\end{verbatim}
This package also includes a trivial nrc, and matching nrc.cmd and nrc.bat scripts, which simply pass on their arguments to NetRexxC; “nrc” is just a shorter name that saves keystrokes, so for the last example you could type: 
  \begin{verbatim}
        nrc –run hello
\end{verbatim}
Note that scripts may be case-sensitive, and you will probably have to spell the name of the program exactly as it appears in the filename. Also, to use –run, you may need to omit the .nrx extension. 
You could also edit the appropriate nrc.cmd, nrc.bat, or nrc script and add your favourite “default” NetRexxC options there. For example, you might want to add the –prompt flag (described later) to save reloading the translator before every compilation. If you do change a script, keep a backup copy so that if you install a new version of the NetRexx package you won’t overwrite your changes. 
\end{enumerate}