\chapter{Installing and running on the Raspberry Pi}
\subsection{Running \nr{} in 10 minutes on the Raspberry Linux/ARM system}
\index{Raspberry Pi}
\index{installation,Raspberri Pi}
This install guide is different, in the sense that it describes the entire setup of the Raspberry Pi system, including \nr{}.

\subsubsection{Linux on ARM}
The Raspberry Pi is an inexpensive computer, containing an ARM
architecture CPU on a board the size of a credit card, which sells for
\$35. It boots from an SD card, the kind that is used for digital cameras. In a few small steps you can be up and running with \nr{}. Recent Raspbian distributions already contain Java. 
\begin{itemize}
\item Use an SD card of suitable size (and known brand)\footnote{Not all cards work; the large brands do. SanDisk Ultra SDHC cards are verified to work.}, at least 2GB but 8 or 16 is advisable
\item Download the raspbian image from \url{http://www.raspberrypi.org/downloads}
\item Hook up an SD Card writer (the one in your digital camera probably also works) to the USB port of your computer
\item While taking good care not to overwrite your harddisk, use \emph{dd} or, on Windows, \emph{Win32DiskManager} to write the image to the SD card. This takes a minute. Good instructions are at \url{http://elinux.org/RPi_Easy_SD_Card_Setup}
\item Now unpack the Raspberry Pi, connect the hdmi to a tv or via an hdmi-monitor cable to a monitor, connect a keyboard (mouse can be attached later, if at all), and connect the mini-usb adapter to the power socket. I used a spare plug from an old phone. It boots and gives a lot of Unix messages. The first boot is not very quick. Connect an ethernet cable to your router\footnote{The entire installation can be done without connection a monitor if so desired. You can find the Raspberry on your network by using \texttt{nmap}, or looking at your router interface. Be sure to re-enable ssh when running raspi-config.}.
\item You land in the raspi-config system. Resize the partitions to fill your SD card. Change the password for the pi user, set the default locale, and enable ssh. You can worry with the other options later.
\item Note the IP address that the system received from DHCP
\item Login from another system, for example using Putty (for Windows) or use ssh pi@your.ip.add.ress (these are the numbers of an IP4 address)
\item Use scp or ftp (binary mode) to transmit NetRexxC.jar or
  NetRexxF.jar to the system, or install the whole \nr{} package. There is an unzip command available
\item Set path and classpath as indicated earlier, and run \nr{}. You have the option to develop and compile on the Raspberry, or just upload class files to it.
\end{itemize}
