\chapter{Installing and running on the Raspberry Pi}
\subsection{Running NetRexx in 10 minutes on the Raspberry Linux/ARM system}
\index{Raspberry Pi}
\index{installation,Raspberri Pi}
This install guide is different, in the sense that it describes the entire setup of the Raspberry Pi system, including NetRexx.

\subsubsection{Linux on ARM}
The Raspberry Pi is an inexpensive computer, containing an ARM architecture CPU on a board the size of a credit card. It boots from an SD card, the kind you have in your digital camera. In a few small steps you can be up and running with Java and NetRexx. 
\begin{itemize}
\item For speed, resist the urge to unpack the Raspberry, but do unpack an (brand) SD card of suitable size, at least 2GB but 8 or 16 is advisable
\item Download the raspbian image from \url{http://http://www.raspberrypi.org/downloads} 
\item Hook up the SD Card writer to the USB port of your computer
\item While taking good care not to overwrite your harddisk, use \emph{dd} or, on Windows, \emph{Win32DiskManager} to write the image to the SD card. This takes a minute. Good instructions are at \url{http://elinux.org/RPi_Easy_SD_Card_Setup}
\item Now unpack the Raspberry Pi, connect the hdmi to a tv or via an hdmi-monitor cable to a monitor, connect a keyboard (mouse can be attached later, if at all), and connect the mini-usb adapter to the power socket. I used a spare plug from an old phone. It boots and gives a lot of Unix messages. The first boot is not very quick. Connect an ethernet cable to your router.
\item You land in the raspi-config system. Resize the partitions to fill your SD card. Change the password for the pi user, and enable ssh. You can worry with the other options later.
\item Note the IP address that the system received from DHCP
\item Login from another system, for example using Putty (for Windows) or use ssh pi@your.ip.add.ress (these are the numbers of an IP4 address)
\item When logged on, issue \texttt{sudo su -}. You are root now; be careful.
\item Issue \texttt{aptitude install openjdk-6-jdk}. Enter through the prompts.
\item Use scp or ftp (binary mode) to transmit the NetRexxC.jar to the system, or install the whole NetRexx package. There is an unzip command available
\item Set path and classpath as indicated earlier. And run NetRexx. It is possible to develop on the Raspberry, or just upload class files to it. It has a 700Mhz CPU, it is good to adapt your expectations of its performance to this fact.
\end{itemize}