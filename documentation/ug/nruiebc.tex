% .* ------------------------------------------------------------------
% .* NetRexx User's Guide                                              mfc
% .* Copyright (c) IBM Corporation 1996, 2000.  All Rights Reserved.
% .* ------------------------------------------------------------------
\chapter{Installing on an EBCDIC system}


\index{EBCDIC installations}
\index{installation,EBCDIC systems}


The NetRexx binaries are identical for all operating systems; the same
NetRexxC.jar runs everywhere \footnote{Many thanks to Mark Cathcart
  and John Kearney for contributing the details to the original version of this section.}.
However, during installation it is important to ensure that binary files
are treated as binary files, whereas text files (such as the
accompanying HTML and sample files) are translated to the local code
page as required.

The simplest way to do this is to first install the package on a
workstation, following the instructions above, then copy or FTP the
files you need to the EBCDIC machine.  Specifically:
\begin{itemize}
\item The NetRexxC.jar file should be copied as-is, that is, use
FTP or other file transfer with the BINARY option.  The CLASSPATH should
be set to include this NetRexxC.jar file.
\item Other files (documentation, etc.) should be copied as Text (that is,
they will be translated from ASCII to EBCDIC).
\end{itemize}

In general, files with extension \emph{.au}, \emph{.class}, \emph{.gif}, \emph{.jar},
or \emph{.zip} are binary files; all others are text files.
Setting the classpath might look like this:
\begin{verbatim}
export CLASSPATH=\$CLASSPATH:/u/j390/j1.1.8/lib/NetRexxC.jar
\end{verbatim}
