% .* ------------------------------------------------------------------
% .* \nr{} User's Guide                                              mfc
% .* Copyright (c) IBM Corporation 1996, 2000.  All Rights Reserved.
% .* ------------------------------------------------------------------
\chapter{Installing on an IBM Mainframe}
\subsection{EBCDIC Systems: z/OS}
\index{EBCDIC installations}
\index{installation,EBCDIC systems}

\subsubsection{Prerequisites for z/OS}
To use \nr{} on z/OS you must have access to an OMVS
prompt (z/OS Unix Systems Services\footnote{IBM Manuals SA22-7801-12 ``Unix System
  Services User's Guide''
  and  SA22-7802-12 ``Unix System Services Reference''}  shell for
3270 terminals), or have access using ssh or telnet; Java must be
installed.

Access to the OMVS command can be regulated through a security profile, so your userid
must be in the right RACF, ACF2 or TOP SECRET class. You will need a home
directory specified in this OMVS class, and this directory needs to be
mounted, preferably as a permanent mount.

If this is arranged and working, you need to verify if there is a Java
runtime available. Test this with the command 
\begin{lstlisting}
java -version
\end{lstlisting}
Java \minimalJVMversion{}, or more recent, is needed for \nr{}\splice{java org.netrexx.process.NrVersion}.

\subsubsection{Uploading the \nr{} translator jar}

The \nr{} binaries are identical for all operating systems; the same
NetRexxC.jar runs everywhere\footnote{Thanks to Mark Cathcart
  and John Kearney for contributing the details to the original version of this section.}.
However, during installation it is important to ensure that binary files
are treated as binary files, whereas text files (such as the
accompanying HTML and sample files) need to be translated to the local code
page as required. 

The simplest way to do this is to first install the package on a
workstation, following the instructions above, then copy or FTP the
files you need to the mainframe.  The files need to be placed in an
HFS to be used by OMVS; FTP and sftp can directly place the files in
an HFS or ZFS
home directory, while IND\$FILE can place them into a traditional data
set.

Specifically:
\begin{itemize}
\item The NetRexxC.jar file should be copied as-is, that is, use
FTP or other file transfer with the BINARY option.  Note that
\code{sftp} defaults to binary, while \code{scp} to z/OS translates
ASCII to EBCDIC and is not usable for this purpose. The CLASSPATH should
be set to include this NetRexxC.jar file. When using IND\$FILE as a
file transfer mechanism to a traditional MVS data set, make sure it is
allocated as a load library with \texttt{lrecl 0} and a large
blocksize. A variable length record also works, for example, a dataset
defined as \texttt{dsorg=ps, recfm=vb, lrecl=1250, blksize=12500}
works without a problem.
\item Other files (documentation, etc.) should be copied as Text (that is,
they will be translated from ASCII to EBCDIC). This can be done by specifying type TEXT on the ftp
command, or use the ASCII CRLF option on the IND\$FILE command.
\end{itemize}

In general, files with extension \emph{.au}, \emph{.class}, \emph{.gif}, \emph{.jar},
or \emph{.zip} are binary files; all others are text files. You may
opt to leave the additional files on a workstation, the mainframe
really only needs the .jar file, NetRexxC.jar (or NetRexxR.jar if you
are only planning to run already compiled classfiles).
Setting the classpath might look like this for a Java 1.6 installation
on a recent z/OS:
\begin{lstlisting}
JAVA_HOME=/opt/ibm/java-s390x-60
export JAVA_HOME
CLASSPATH=$CLASSPATH:$JAVA_HOME/lib/tools.jar
CLASSPATH=$CLASSPATH:$JAVA_HOME/jre/lib/s390x/default/jclSC160/vm.jar
CLASSPATH=$CLASSPATH:/u/[your userid]/lib/NetRexxC.jar
export CLASSPATH
\end{lstlisting}
For a Java 1.6.1 installation, the following settings were
encountered:
\begin{lstlisting}
JAVA_HOME=/usr/lpp/java/J6.0.1
export JAVA_HOME
CLASSPATH=$CLASSPATH:$JAVA_HOME/lib/tools.jar
CLASSPATH=$CLASSPATH:$JAVA_HOME/lib/s390/default/jclSC160/vm.jar
CLASSPATH=$CLASSPATH:/u/[your userid]/lib/NetRexxC.jar
export CLASSPATH
\end{lstlisting}
For a 64 bits Java 1.7.0 installation, these settings work:
\begin{lstlisting}
JAVA_HOME=/usr/lpp/java/J7.0_64
export JAVA_HOME
CLASSPATH=$CLASSPATH:$JAVA_HOME/lib/tools.jar
CLASSPATH=$CLASSPATH:$JAVA_HOME/lib/s390x/default/jclSC170/vm.jar
CLASSPATH=$CLASSPATH:/u/[your userid]/lib/NetRexxC.jar
export CLASSPATH
\end{lstlisting}

Note that you are free to put the NetRexxC.jar archive in any
location, as long as the classpath correctly refers to it. The vm.jar
has to be on the classpath because otherwise Object.class will not be
found by the \nr{}C translator.

The \texttt{OCOPY} command can be used under TSO to copy the uploaded
NetRexxC.jar to a path in an HFS dataset:
\begin{lstlisting}
/* rexx */                                             
"free fi(pathname)"                                    
"free fi(sysut1)"                                      
"alloc fi(pathname) path('/u/[your userid]/lib/NetRexxC.jar')"
"alloc fi(sysut1) dsn('netrexx.new')"                  
"ocopy indd(sysut1) outdd(pathname) binary"            
\end{lstlisting}
This works when the NetRexxC.jar file already exists, if that is not the
case, just issue \texttt{touch NetRexxC.jar} in that directory, the
copy command will overwrite that empty file.

Be sure to add the \texttt{-Xquickstart} option to the java command in
the nrc binary file in your path, or add it as an alias.
\begin{lstlisting}
java -Xquickstart org.netrexx.process.NetRexxC $*
\end{lstlisting}
because this will shorten the startup time required to a more or less
acceptable time.

When this is done, we can run some tests with it and see that
everything works. Edit a program source file with \texttt{oedit},
which works just like the ISPF/PDF editor and compile or interpret it
like we do on other versions of Unix. \nr{} programs can access HFS (and
ZFS) files the same way it does on Windows and Unix, and also network
programming with TCP/IP works in the same way from OMVS.

For a description how \nr{} can be used in a traditional MVS
workload environment, with batch JCL and using VSAM and sequential
data sets and PDS directories, you are referred to the \emph{\nr{}
  Programming Guide)}.
\subsubsection{A note on character sets}
z/OS USS is an EBCDIC Unix version, do note that the \texttt{-utf8}
option does only work when your source file actually is encoded in
utf8.
\subsection{Linux on Z}
Installing on Linux on Z (sometimes referred to as \emph{z/Linux}) is straightforward. Make sure the NetRexxC.jar
is copied untranslated to the z/Linux file system using ftp, scp or
some other file transfer technology, and take into
account that the IBM JVM has Object.class in the vm.jar archive. At
the moment, if not installed already, Java for z/Linux is a free download from the IBM website. With
Linux on Z versions that have a VNC server installed and available, Java
Graphical User Interfaces (GUI) can be used without installing X client software. 
