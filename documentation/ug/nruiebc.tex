% .* ------------------------------------------------------------------
% .* \nr{} User's Guide                                              mfc
% .* Copyright (c) IBM Corporation 1996, 2000.  All Rights Reserved.
% .* ------------------------------------------------------------------
\chapter{Installing on an IBM Mainframe}
\subsection{EBCDIC Systems: z/OS, z/VM}
\index{EBCDIC installations}
\index{installation,EBCDIC systems}

\subsubsection{Prerequisites for z/OS}
Realistically, to use \nr{} on z/OS you must have access to an OMVS
prompt (z/OS Unix Systems Services\footnote{IBM Manuals SA22-7801-12 ``Unix System
  Services User's Guide''
  and  SA22-7802-12 ``Unix System Services Reference''}  shell for 3270 terminals), have
access using ssh or telnet, and Java must be installed. While this
access used to be scarce, more and more installations have this as a
standard. Of course, if you are systems programming staff you can
arrange for most of this yourself, and if you are not, you need to
befriend the staff that can.

Access to the OMVS command is regulated through a security profile, so your userid
must be in the right RACF, ACF2 or TOP SECRET class. You will need a home
directory specified in this OMVS class, and this directory needs to be
mounted, preferably as a permanent mount.

If this is arranged and working, you need to verify if there is a Java
runtime available. Test this with the command 
\begin{verbatim}
java -version
\end{verbatim}
Any version of Java will do, although newer is better. Generally,
versions from 1.4 to 1.6 are found on mainframes nowadays. If the
command is not found, don't despair; it might be installed but it may
be not found on the \$PATH variable. This can be arranged for in the
\texttt{.profile} or \texttt{.bash\_profile} file in your home direcory. This variable works
just as in other versions of Unix\footnote{z/OS is officially a
  version of Unix, this in addition to everything that it already was}, see
page \pageref{install_classpath}. If Java is not installed, it is time
someone did it; there are SMP/E and non-SMP/E installers (using a
shell script) available -
the latter comes in handy for a quick install.

\subsubsection{Uploading the \nr translator jar}

The \nr{} binaries are identical for all operating systems; the same
NetRexxC.jar runs everywhere\footnote{Many thanks to Mark Cathcart
  and John Kearney for contributing the details to the original version of this section.}.
However, during installation it is important to ensure that binary files
are treated as binary files, whereas text files (such as the
accompanying HTML and sample files) need to be translated to the local code
page as required. 

The simplest way to do this is to first install the package on a
workstation, following the instructions above, then copy or FTP the
files you need to the mainframe.  The files need to be placed in an
HFS to be used by OMVS; FTP can directly places the files in an HFS
home directory, while IND\$FILE can place them into a traditional data
set.

Specifically:
\begin{itemize}
\item The NetRexxC.jar file should be copied as-is, that is, use
FTP or other file transfer with the BINARY option.  The CLASSPATH should
be set to include this NetRexxC.jar file. When using IND\$FILE as a
file transfer mechanism to a traditional MVS data set, make sure it is
allocated as a load library with \texttt{lrecl 0} and a large blocksize.
\item Other files (documentation, etc.) should be copied as Text (that is,
they will be translated from ASCII to EBCDIC). This can be done by specifying type TEXT on the ftp
command, or use the ASCII CRFL option on the IND\$FILE command.
\end{itemize}

In general, files with extension \emph{.au}, \emph{.class}, \emph{.gif}, \emph{.jar},
or \emph{.zip} are binary files; all others are text files. You may
opt to leave the additional files on a workstation, the mainframe
really only needs the .jar file, NetRexxC.jar (or NetRexxR.jar if you
are only planning to run already compiled classfiles).
Setting the classpath might look like this on a recent z/OS:
\begin{verbatim}
JAVA_HOME=/opt/ibm/java-s390x-60
export JAVA_HOME
CLASSPATH=$CLASSPATH:$JAVA_HOME/lib/tools.jar
CLASSPATH=$CLASSPATH:$JAVA_HOME/jre/lib/s390x/default/jclSC160/vm.jar
CLASSPATH=$CLASSPATH:/u/[your userid]/lib/NetRexxC.jar
export CLASSPATH
\end{verbatim}
Note that you are free to put the NetRexxC.jar archive in any
location, as long as the classpath correctly refers to it. The vm.jar
has to be on the classpath because otherwise Object.class will not be
found by the \nr{}C translator.

When this is done, we can run some tests with it and see that
everything works. Edit a program source file with \texttt{oedit},
which works just like the ISPF/PDF editor and compile or interpret it
like we do on other versions of Unix. \nr programs can access HFS (and
ZFS) files the same way it does on Windows and Unix, and also network
programming with TCP/IP works in the same way from OMVS.

For a description how \nr can be used in a traditional MVS
workload environment, with batch JCL and using VSAM and sequentials
data set and PDS directories, you are referred to the \emph{\nr{}
  Programming Guide)}.
\subsection{z/Linux}
Installing on z/Linux is straightforward. Make sure the NetRexxC.jar
is copied untranslated to the z/Linux file system using ftp, scp or
some other file transfer technology, and take into
account that the IBM JVM has Object.class in the vm.jar archive. At
the moment, if not installed already, Java for z/Linux is a free download from the IBM website. With
z/Linux versions that have a VNC server installed and available, Java
Graphical User Interfaces (GUI) can be used without installing X client software. 