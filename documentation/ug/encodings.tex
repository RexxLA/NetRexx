\chapter{Unicode}
The JVM works with Unicode as a string representation; for this reason the display of characters in alphabets other than the latin alphabet does not pose a problem. To work with Unicode and internationalization in a straightforward way, a combination of factors must be present. The operating system, your editor, shell and character set support must be compatible with Unicode. A set fonts very seldom contains glyphs\footnote{This is a typographical term for character form} for all Unicode code points (values). Be certain to save the program file as the right type; some editors can save as ASCII, UTF-8 and UTF-16. Some editors seem to support Unicode but have made mistakes in the implementation. The NetRexx translator has a \texttt{-utf8} option that makes it accept this encoding in the source. This option is not necessary for the use of Unicode in \emph{variables} - this always works, it being the native encoding of the JVM. The option is rather meant to enable specification of NetRexx syntax elements in Unicode. This makes it possible to use Class names, Method names and variable names composed of Unicode characters.

Some things to think of when using the \texttt{-utf8} option
\begin{itemize}
\item It is not the default.
\item The option \texttt{-utf8} can be specified in the program source, but the value of this option on the compiler command line must be equal to the value of the program option. Here the rule that the last specified value for an option is applicable, does not count
\item When method names are specified in Unicode, they need to be \emph{symbols} and not escaped Unicode characters
\item When you use Unicode in a Class name, be sure that the program file name matches that.
\item A filename in Unicode might still spell trouble when using it in conjunction with version management software, sharing it using email or other usages that are not limited to one file system and encoding method.
\end{itemize}

