There are a number of options for the translator, some of which can be specified on the translator command line, and others also in the program source on the \textbf{option} statement. In the following table, c stands for \emph{commandline only}, s stands for \emph{source} and b stands for \emph{both}.
\begin{longtable}[l]{|l|p{10cm}|l|}
\caption{ Options } \\
\hline
\rowcolor[gray]{0.8} \bfseries Option & \bfseries Meaning & \bfseries Place   \
\endfirsthead
\multicolumn{3}{r}%
{{\tablename\ \thetable{} -- \emph{continued from previous page}}} \\
\endhead
\hline \multicolumn{3}{r}{\emph{Continued on next page}}
\endfoot

\endlastfoot
\rowcolor[gray]{0.8} \bfseries \huge   & \normalsize  &  \\
\hline
-arg words & interpret; remaining words are arguments & c \\
\hline
-binary &  classes are binary classes & b \\
\hline
 -classpath  & specify a classpath & c \\
\hline
 -compile  & compile (default; -nocompile implies -keep) & c \\
\hline
 -comments     & copy comments across to generated .java &b \\
\hline
 -compact      & display error messages in compact form &b \\
\hline
 -console   & display messages on console (default) &c \\
\hline
 -crossref     & generate cross-reference listing &b \\
\hline
 -decimal      & allow implicit decimal arithmetic &b \\
\hline
 -diag         & show diagnostic messages &b \\
\hline
 -exec        & interpret with no argument words &c \\
\hline
-explicit     & local variables must be explicitly declared &b \\
\hline
-format       & format output file (pretty-print) &b \\
\hline
-java         & generate Java source code for this program &b \\
\hline
 -keep         & keep any completed .java file (as xxx.java.keep) &c \\
\hline
-keepasjava   & keep any completed .java file (as xxx.java) &c \\
\hline
 -logo         & display logo (banner) after starting &b \\
\hline
-prompt       & prompt for new request after processing &c \\
\hline
-savelog      & save messages in NetRexxC.log &c \\
\hline
 -replace      & replace .java file even if it exists &b \\
\hline
 -sourcedir    & force output files to source directory &b \\
\hline
  -strictargs   & empty argument lists must be specified as () &b \\
\hline
  -strictassign & assignment must be cost-free &b \\
\hline
  -strictcase   & names must match in case &b \\
\hline
  -strictimport & all imports must be explicit &b \\
\hline
  -strictmethods & superclass methods are not compared to local methods for best match &b \\
\hline
  -strictprops  & even local properties must be qualified &b \\
\hline
  -strictsignal & signals list must be explicit &b \\
\hline
  -symbols      & include symbols table in generated .class files &b \\
\hline
-time         & display timings &c \\
\hline
 -trace[n]     & trace stream [1 or 2], or 0 for NOTRACE &b \\
\hline
 -utf8         & source file is in UTF8 encoding &b \\
\hline
 -verbose[n]   & verbosity of progress reports [0-5] &b \\
\hline
 -warnexit0    & exit with a zero returncode on warnings &c \\
\hline
\end{longtable}

\subsubsection{Options valid for the options statement and on the commandline}
These are the options that can be used on the \textbf{options} statement:
\begin{description}
\index{option, binary}
\index{flag, binary}
\index{binary option}
\item[-binary]
All classes in this program will be binary classes. In binary classes, literals are assigned binary (primitive) or native string types, rather than \nr{} types, and native binary operations are used to implement operators where appropriate, as described in “Binary values and operations”. In classes that are not binary, terms in expressions are converted to the \nr{} string type, Rexx, before use by operators.

\index{option,comments}
\index{flag,comments}
\index{comments option}
\item[-comments]
Comments from the \nr{} source program will be passed through to the Java output file (which may be saved with a .java.keep or .java extension by using the -keep and -keepasjava command options, respectively).

\index{option,compact}
\index{flag,compact}
\index{compact option}
\item[-compact]
Requests that warnings and error messages be displayed in compact form. This format is more easily parsed than the default format, and is intended for use by editing environments.
Each error message is presented as a single line, prefixed with the error token identification enclosed in square brackets. The error token identification comprises three words, with one blank separating the words. The words are: the source file specification, the line number of the error token, the column in which it starts, and its length. For example (all on one line):
\begin{verbatim}
  [D:\test\test.nrx 3 8 5] Error: The external name
  'class' is a Java reserved word, so would not be
  usable from Java programs
\end{verbatim}
Any blanks in the file specification are replaced by a null ('\textbackslash 0') character. Additional words could be added to the error token identification later.

\index{option,crossref}
\index{flag,crossref}
\index{crossref option}
\item[-crossref]
Requests that cross-reference listings of variables be prepared, by class.
\index{option,decimal}
\index{flag,decimal}
\index{decimal option}
\item[-decimal]
Decimal arithmetic may be used in the program. If nodecimal is specified, the language processor will report operations that use (or, like normal string comparison, might use) decimal arithmetic as an error. This option is intended for performance-critical programs where the overhead of inadvertent use of decimal arithmetic is unacceptable.
\index{option,diag}
\index{flag,diag}
\index{diag option}
\item[-diag]
Requests that diagnostic information (for experimental use only) be displayed. The diag option word may also have side-effects.
\index{option,explicit}
\index{flag,explicit}
\index{explicit option}
\item[-explicit]
Requires that all local variables must be explicitly declared (by assigning them a type but no value) before assigning any value to them. This option is intended to permit the enforcement of “house styles” (but note that the \nr{} compiler always checks for variables which are referenced before their first assignment, and warns of variables which are set but not used).
\index{option,format}
\index{flag,format}
\index{format option}
\item[-format]
Requests that the translator output file (Java source code) be formatted for improved readability. Note that if this option is in effect, line numbers from the input file will not be preserved (so run-time errors and exception trace-backs may show incorrect line numbers).
\index{option,java}
\index{flag,java}
\index{java option}
\item[-java]
Requests that Java source code be produced by the translator. If nojava is specified, no Java source code will be produced; this can be used to save a little time when checking of a program is required without any compilation or Java code resulting.
\index{option,logo}
\index{flag,logo}
\index{logo option}
\item[-logo]
Requests that the language processor display an introductory logotype sequence (name and version of the compiler or interpreter, etc.).
\index{option,sourcedir}
\index{flag,sourcedir}
\index{sourcedir option}
\item[-sourcedir]
Requests that all .class files be placed in the same directory as the source file from which they are compiled. Other output files are already placed in that directory. Note that using this option will prevent the -run command option from working unless the source directory is the current directory.
\index{option,strictargs}
\index{flag,strictargs}
\index{strictargs option}
\item[-strictargs]
Requires that method invocations always specify parentheses, even when no arguments are supplied. Also, if strictargs is in effect, method arguments are checked for usage – a warning is given if no reference to the argument is made in the method.
\index{option,strictassign}
\index{flag,strictassign}
\index{strictassign option}
\item[-strictassign]
Requires that only exact type matches be allowed in assignments (this is stronger than Java requirements). This also applies to the matching of arguments in method calls.
\index{option,strictcase}
\index{flag,strictcase}
\index{strictcase option}
\item[-strictcase]
Requires that local and external name comparisons for variables, properties, methods, classes, and special words match in case (that is, names must be identical to match).
\index{option,strictimport}
\index{flag,strictimport}
\index{strictimport option}
\item[-strictimport]
Requires that all imported packages and classes be imported explicitly using import instructions. That is, if in effect, there will be no automatic imports, except those related to the package instruction.
\index{option,strictmethods}
\index{flag,strictmethods}
\index{strictmethods option}
\item[-strictmethods]
Superclass methods are not compared to local methods for best match.
\index{option,strictprops}
\index{flag,strictprops}
\index{strictprops option}
\item[-strictprops]
Requires that all properties, including those local to the current class, be qualified in references. That is, if in effect, local properties cannot appear as simple names but must be qualified by this. (or equivalent) or the class name (for static properties).
\index{option,strictsignal}
\index{flag,strictsignal}
\index{strictsignal option}
\item[-strictsignal]
Requires that all checked exceptions signalled within a method but not caught by a catch clause be listed in the signals phrase of the method instruction.
\index{option,symbols}
\index{flag,symbols}
\index{symbols option}
\item[-symbols]
Symbol table information (names of local variables, etc.) will be included in any generated .class file. This option is provided to aid the production of classes that are easy to analyse with tools that can understand the symbol table information. The use of this option increases the size of .class files.
\index{option,trace, traceX}
\index{flag,trace, traceX}
\index{trace, traceX option}
\item[-trace, -traceX]
If given as \textbf{-trace}, \textbf{-trace1}, or \textbf{-trace2}, then trace instructions are accepted. The trace output is directed according to the option word: \textbf{-trace1} requests that trace output is written to the standard output stream, \textbf{-trace} or \textbf{-trace2} imply that the output should be written to the standard error stream (the default).
\index{option,utf8}
\index{flag,utf8}
\index{utf8 option}
\item[-utf8]
If given, clauses following the options instruction are expected to be encoded using UTF-8, so all Unicode characters may be used in the source of the program.
In UTF-8 encoding, Unicode characters less than '\textbackslash u0080' are represented using one byte (whose most-significant bit is 0), characters in the range '\textbackslash u0080' through '\textbackslash u07FF' are encoded as two bytes, in the sequence of bits:
\begin{verbatim}
  110xxxxx 10xxxxxx
\end{verbatim}
where the eleven digits shown as x are the least significant eleven bits of the character, and characters in the range '\textbackslash u0800' through '\textbackslash uFFFF' are encoded as three bytes, in the sequence of bits:
\begin{verbatim}
  1110xxxx 10xxxxxx 10xxxxxx
\end{verbatim}
where the sixteen digits shown as x are the sixteen bits of the character.
If noutf8 is given, following clauses are assumed to comprise only Unicode characters in the range '\textbackslash x00' through '\textbackslash xFF', with the more significant byte of the encoding of each character being 0.
Note: this option only has an effect as a compiler option, and applies to all programs being compiled. If present on an options instruction, it is checked and must match the compiler option (this allows processing with or without utf8 to be enforced).
\index{option,verbose, verboseX}
\index{flag,verbose, verboseX}
\index{verbose, verboseX option}
\item[-verbose, -verboseX]
Sets the “noisiness” of the language processor. The digit X may be any of the digits 0 through 5; if omitted, a value of 3 is used. The options \textbf{-noverbose} and \textbf{verbose0} both suppress all messages except errors and warnings
\end{description}

\subsubsection{Options valid on the commandline}
The translator also implements some additional option words, which
control compilation features.  These cannot be used on the
\textbf{options} instruction\footnote{Although at the moment, there will be no indication of this}, and are:
\begin{description}
\index{option,arg words}
\index{flag,arg words}
\index{arg words option}
\item[-arg]
The \textbf{-arg} \emph{words} option is used when interpreting
programs, it indicates that after the \textbf{-arg} statement,
commandline arguments for ther interpreted program follow

\index{option,classpath}
\index{flag,classpath}
\index{classpath option}
\item[-classpath]
The -classpath option allows dynamic specification of the classpath
used by the \nr{}C compiler without having to depend on the
CLASSPATH environment variable. (since: \nr{} 3.02)
.
\index{option,exec}
\index{flag,exec}
\index{exec option}
\item[-exec]
The \textbf{-exec} \emph{words} option is used when interpreting programs. With this option, no commandline arguments are possible.
\index{option,keep}
\index{flag,keep}
\index{keep option}
\item[-keep]
keep the intermediate \emph{.java} file for each program.  It is kept in
the same directory as the \nr{} source file as \emph{xxx.java.keep},
where \emph{xxx} is the source file name.  The file will also be kept
automatically if the \emph{javac} compilation fails for any reason.
\index{option,keepasjava}
\index{flag,keepasjava}
\index{keepasjava option}
\item[-keepasjava]
keep the intermediate \emph{.java} file for each program.  It is kept in
the same directory as the \nr{} source file as \emph{xxx.java},
where \emph{xxx} is the source file name.  Implies -replace. Note: use this option carefully in mixed-source projects where you might have .java source files around.
\item[-nocompile]
\index{option, nocompile}
\index{flag, nocompile}
\index{nocompile option}
do not compile (just translate).  Use this option when you want to use a
different Java compiler.  The \emph{.java} file for each program is kept
in the same directory as the \nr{} source file, as the
file \emph{xxx.java.keep} (where \emph{xxx} is the source file name).
\item[-noconsole]
\index{option, noconsole}
\index{flag, noconsole}
\index{noconsole option}
do not display compiler messages on the console (command display
screen).  This is usually used with the \emph{savelog} option.
\item[-savelog]
\index{option, savelog}
\index{flag, savelog}
\index{savelog option}
write compiler messages to the file \emph{\nr{}C.log}, in the current
directory.
This is often used with the \emph{noconsole} option.
\item[-time]
\index{option, time}
\index{flag, time}
\index{time option}
display translation, \emph{javac} or \emph{ecj} compile, and total times (for the sum
of all programs processed).
\item[-run]
\index{option, run}
\index{flag, run}
\index{run option}
run the resulting Java class as a stand-alone application, provided that
the compilation had no errors.
\index{option,warnexit0}
\index{flag,warnexit0}
\index{warnexit0 option}
\item[-warnexit0]
Exit the translator with returncode 0 even if warnings are issued. Useful with build tools that would otherwise exit a build.
\end{description}
