.* ------------------------------------------------------------------
.* NetRexx User's Guide                                              mfc
.* Copyright (c) IBM Corporation 1996, 2000.  All Rights Reserved.
.* ------------------------------------------------------------------
:h2 id=iexec.Installing the NetRexx translator
.pi /installation/of translator
.pi /installation/quick
.pi /NetRexx package/quick installation
.pi /package/NetRexx
:p.
The NetRexx package includes the NetRexx translator &dash. a Java
application which can be used for compiling, interpreting, or
syntax-checking NetRexx programs.  The procedure for installation is
briefly as follows (full details are given later):
:ol.
:li.
Make the translator visible to the Java Virtual Machine (JVM):
:ul.
:li.
If you are running Java :b.1.2:eb. or later, copy the
file :m.NetRexx\lib\NetRexxC.jar:em. to the :m.jre\lib\ext:em. directory
in the Java installation tree.
The JVM will automatically find it there and make it available.
:li.
If you are using an earlier Java version (:b.1.1.2:eb. through
:b.1.1.8:eb.) instead add the full path and filename of
the :m.NetRexx\lib\NetRexxC.jar:em. to the CLASSPATH environment
variable for your operating system.
:eul.
:pc.
Note: if you have a NetRexxC.zip in your CLASSPATH from an earlier
version of Rexx, remove it (NetRexxC.jar replaces NetRexxC.zip).
:li.
Copy all the files in the :m.NetRexx\bin:em. directory to a directory in
your PATH (perhaps the :m.\bin:em. directory in the Java installation
tree).  This is not essential, but makes shorthand scripts and a test
case available.
:li.
If you are running Java :b.1.2:eb. or later, make the
file :m.\lib\tools.jar:em. (which contains the :m.javac:em. compiler) in
the Java tree visible to the JVM.  You can do this either by adding its
path and filename to the CLASSPATH environment variable, or by moving it
to the :m.jre\lib\ext:em. directory in the Java tree.
:li.
Test the installation by making the :m.\bin:em. directory the current
directory and issuing the following two commands exactly as written:
:xmp.
java COM.ibm.netrexx.process.NetRexxC hello
java hello
:exmp.
:pc.
The first of these should translate the test program and then invoke
the :m.javac:em. compiler to generate the class file
(:m.hello.class:em.) for the program.  The second should run the program
and display a simple greeting.
:eol.
:p.
If you have any problems or errors in the above process, please read the
detailed instructions and problem-solving tips that follow.
