.* ------------------------------------------------------------------
.* NetRexx User's Guide                                              mfc
.* Copyright (c) IBM Corporation 1996, 2000.  All Rights Reserved.
.* ------------------------------------------------------------------
:h2 id=inj12.Installing for Java 1.2+
.pi /Java/installation
.pi /installation/Java 1.2
:p.
This section gives a detailed procedure for installing NetRexx using a
Java 1.2, or later, development kit.  The steps are as follows:
:ol.
:li.
:b.Locate the Java home directory.:eb.
The name of the Java home directory will vary depending on the operating
system you are using.  Some possibilities are:
:xmp.
\java13\jre
c:\java1.2\jre
/opt/IBMJava2-13/jre
:exmp.
:pc.It will contain directories such as :q.:m.bin:em.:eq. and
:q.:m.lib:em.:eq..
:note. if your Java home directory is on a CD-ROM, or on an unwritable
shared disk, you'll need to leave the NetRexx files elsewhere and set up
CLASSPATH and PATH environment settings to refer to them.
Consult your Java toolkit and operating system documentation for details
on how to do this.
.* - - - - -
:li.
:b.Copy or move the NetRexxC.jar file to the Java :m.lib\ext:em.
directory.:eb.
For Java to be able to find the NetRexx classes, it's simplest to just
put them (that is, the NetRexxC.jar file which contains them) in the
Java extension directory.  This is the :m.\lib\ext:em. directory below
the Java home directory, which should already exist (if it does not,
then you probably have the wrong :m.lib:em. directory &dash. its parent
should be called :m.jre:em.).
The JVM will then automatically find it there when it is needed.
:p.
Alternatively, you could add the NetRexxC.jar file to the CLASSPATH, as
described in the :a id=inj11.:cit.Installing for Java 1.1:ecit. section:ea..
:p.
Note: if you have a NetRexxC.zip in your CLASSPATH from an earlier
version of Rexx, remove it (NetRexxC.jar replaces NetRexxC.zip).
.* - - - - -
.cp 3
:li.
:b.Copy the executables.:eb.
Copy all the files in the :m.NetRexx\bin:em. directory to a directory
which is in your search PATH (perhaps the :m.\bin:em. directory below
the Java home directory).  This will allow them to be invoked simply by
typing their name at a command prompt.
:p.
The files in the :m.bin:em. directory should be:
:xmp.
hello.nrx    -- a simple NetRexx program for testing
NetRexxC.cmd -- the NetRexx compiler command in Rexx
NetRexxC.bat -- similar NetRexx compiler batch script (Windows .bat)
NetRexxC.sh  -- similar NetRexx compiler shell script for Linux and Unix
nrc.cmd      -- shorter name for NetRexxC.cmd
nrc.bat      -- shorter name for NetRexxC.bat
nrc          -- shorter name for NetRexxC.sh
:exmp.
:p.
The :m..cmd:em. files are simple Rexx scripts for making it easier to
use the translator.  You don't have to use these, but they save some
typing.  They should require little modification to run under the Rexx
interpreter for your platform; for details of Rexx interpreters,
see: :m.
http://www2.hursley.ibm.com/rexx/
:em.
:pc.
Under Windows, the :m..bat:em. files should serve the same
purpose &dash. they are not as flexible as the Rexx :m..cmd:em. files,
but will save typing.
:p.
Similarly, under Linux or other Unix systems, the NetRexxC.sh and nrc
script simplify use of the translator.  You may need to indicate these
are executable, using (for example) the commands: :m.chmod 751
NetRexxC.sh:em. and :m.chmod 751 nrc:em. and (unless you used
the :m.unzip -a:em. command to unpack them) you may need to run dos2unix
on both of them (this converts CRLF to LF).
:p.
Alternatively, the :m.NetRexx\bin:em. directory could be added to the
PATH environment setting.
.* - - - -
.cp 3
:li.
:b.Add the Java tools.jar file to the CLASSPATH.:eb.
If you would like NetRexx to use the :m.javac:em. compiler for
compilations (which is the default) then you will need to add the Sun
tools collection to the classpath.  This is usually found in a file
called :m.tools.jar:em. in the Java :m.lib:em. directory.  For example:
:xmp.
set classpath=%CLASSPATH%;c:\java1.2\lib\tools.jar
:exmp.
:pc.(with the path changed as appropriate) would add the tools.jar to the
existing classpath in an OS/2 or Windows system.
:p.
The procedure for setting the CLASSPATH variable depends on your
operating system (and there may be more than one way), as
described in the :a id=inscp.:cit.Setting the CLASSPATH:ecit. section:ea..
:p.
Alternatively, the tools.jar file could be moved to the Java :m.lib\ext:em.
directory.
.* - - - - -
:li.
:b.Add the current directory to the CLASSPATH.:eb.
If it is not already there, add a reference to the current directory to
the CLASSPATH, for example:
:xmp.
set classpath=.;%CLASSPATH%
:exmp.
:p.
This is needed if you cannot compile and run test programs from within a
directory other than in the Java tree.
.* - - - - -
:eol.
:p.
NetRexx installation is now complete.  Now would be a good time to check
that it works, as described in the :a id=testing.:cit.Testing the NetRexx
Installation:ecit. section:ea..
