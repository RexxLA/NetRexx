\chapter{Troubleshooting\label{troubleshooting}}
\begin{description} 
\item[Can't find class org.netrexx.process.NetRexxC]
  probably means that the NetRexxC.jar file has not been specified in
  your CLASSPATH setting, or is misspelled, or is in the wrong case,
  or (for Java 1.2 or later) is not in the Java
  \textbackslash lib\textbackslash ext directory. Note that in the latter case there are two lib directories in the Java tree; the correct one is in the Java Runtime Environment directory (jre).
The Setting the CLASSPATH section contains information on setting the CLASSPATH.
\item[+++ Error: The class 'java.lang.Object'] cannot be found. You are running with an IBM JVM or JRE. The java.lang.Object class is packaged in the file vm.jar, which needs to be on your CLASSPATH
\item[Can't find class hello] may mean that the directory with the hello.class file is not in your CLASSPATH (you may need to add a ?.;’ to the CLASSPATH, signifying the current directory), or either the filename or name of the class (in the source) is spelled wrong (the java command is [very] case-sensitive). Note that the name of the class must not include the .class extension.
\item[Exception ... NoClassDefFoundError: sun/tools/javac/Main] This indicates that you are running Java 1.2 or later but did not add the Java tools to your CLASSPATH (hence Java could not find the javac compiler). See the Installing for Java 1.2+ section for more details, and an alternative action.
Alternatively, you may be trying to use NetRexx under Visual J++, which needs a different procedure. You can check whether javac is available and working by issuing the javac command at a command prompt; it should respond with usage instructions.
\item[Error opening the file 'hello.java'][C:\textbackslash Program Files(86) \textbackslash javajdk1 7.0.05 jrebinhello.java (Access is denied)] - your userid needs write authorization on the current directory. Please copy the source file to a writeable directory and try again.
\item[Extra blanks] You have an extra blank or two in the CLASSPATH. Blanks should only occur in the middle of directory names (and even then, you probably need some double quotes around the SET command or the CLASSPATH segment with the blank). The JVM is sensitive about this.
\item[Permission Denied] You are trying the NetRexxC.sh or nrc scripts under Linux or other Unix system, and are getting a Permission denied message. This probably means that you have not marked the scripts as being executable. To do this, use the chmod command, for example: chmod 751 NetRexxC.sh.
\item[No such file]  You are trying the NetRexxC.sh or nrc scripts under Linux or other Unix system, and are getting a No such file or syntax error message from bash. This probably means that you did not use the unzip -a command to unpack the NetRexx package, so CRLF sequences in the scripts were not converted to LF.
\item You didn't install on a file system that supports long file names (for example, on OS/2 or Windows you should use an HPFS or FAT32 disk or equivalent). Like most Java applications, NetRexx uses long file names.
\item[Downlevel zip] You have a down-level unzip utility, or changed the name of the NetRexxC.jar file so that it does not match the spelling in the classpath. For example, check that the name of the file NetRexxC.jar’ is exactly that, with just three capital letters.
\item You have only the Java runtime installed, and not the toolkit. If the toolkit is installed, you should have a program called javac on your computer. You can check whether javac is available and working by issuing the javac command at a command prompt; it should respond with usage information.
\item[out of environment space] when trying to set CLASSPATH under Win9x-DOS can be remedied by adding /e:4000 to the ?Cmd line’ entry for the MS-DOS prompt properties (try command /? for more information).
\item[java.lang.OutOfMemoryError] when running the compiler probably means that the maximum heap size is not sufficient. The initial size depends on your Java virtual machine; you can change it to (say) 24 MegaBytes by setting the environment variable:
\begin{verbatim}
  SET NETREXX\_JAVA=-Xmx24M
\end{verbatim}
The NetRexxC.cmd and .bat files add the value of this environment variable to the options passed to java.exe. If you're not using these, modify your java command or script appropriately.
\item[Down-level Java]  You have a down-level version of Java installed. NetRexxC will run only on Java version 1.1.2 (and later versions). You can check the version of Java you have installed using the command 
\begin{verbatim}
java -version’
\end{verbatim}
\item[applet viewer needed] Some of the samples must be viewed using the Java toolkit applet-viewer or a Java-enabled browser. Please see the hypertext pages describing these for detailed instructions. In general, if you see a message from Java saying:
\begin{verbatim}
  void main(String argv[]) is not defined
\end{verbatim}
this means that the class cannot be run using just the ?java’ command; it must be run from another Java program, probably as an applet.
\end{description} 