.* ------------------------------------------------------------------
.* NetRexx User's Guide                                              mfc
.* Copyright (c) IBM Corporation 1996, 2000.  All Rights Reserved.
.* ------------------------------------------------------------------
:h2 id=inunix.Installation on a Linux or Unix system
.pi /Linux/installations
.pi /installation/Linux systems
.pi /Unix/installations
.pi /installation/Unix systems
:p.
The NetRexx binaries are identical for all operating systems; the same
NetRexxC.jar runs everywhere, and the same installation process is used
as on other systems.
Some changes may be needed to text files, however (especially to the
shell scripts), and there are alternatives to the :q.standard:eq.
installation process.  Here are some tips:
.*
:ul.
:li.
It is strongly recommended that you use the :m.unzip:em. command with
the :m.-a:em. flag, if available.  This will automatically convert text
files to Unix text file format, while leaving binaries (such
as :m..jar:em., :m..class:em., or :m..gif:em. files) unchanged.
:li.
If you cannot use the :m.unzip -a:em. command, you may need to take
special action to use text files, such as the documentation or shell
scripts.
In the NetRexx package text files use a two-byte line end sequence
(CRLF) whereas some Unix programs (including bash, the shell
interpreter) only accept the one-byte (LF) line end sequence.  Some Unix
file systems convert the files automatically, but if you are getting
a :kw.No such file:ekw. or :kw.syntax error:ekw. message from bash you
probably need to use the :m.dos2unix:em. command, to convert CRLF to LF.
For example: :m.dos2unix NetRexxC.sh:em..
:li.
File access control information is not preserved in the package.  You
may therefore get a a :kw.Permission denied:ekw. message when you try
and run the scripts, indicating that the files are not marked as
executable.
To mark them as executable, use the :m.chmod:em. command, for
example: :m.chmod 751 NetRexxC.sh:em..
.* -----
:li.
Instead of moving the files to specific locations, as suggested in the
general installation instructions, you can instead link them
symbolically.  For example, something like:
:xmp.
ln -s /usr/local/NetRexx/bin/* /usr/local/bin/.
:exmp.
:pc.would link the shell scripts directory into a different :m.bin:em.
directory.
:eul.
