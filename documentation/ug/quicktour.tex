\chapter{A Quick Tour of \nr{}}
This chapter summarizes the main features of \nr{}, and is intended
to help you start using it quickly. It is assumed that you have some
knowledge of programming in a language such as Rexx, C, BASIC, or
Java, but extensive experience with programming is not needed.

This is not a complete tutorial, though – think of it more as a
\emph{taster}; it covers the main points of the language and shows some
examples you can try or modify. For full details of the language,
consult the next chapters in this book and the \nr{} Programming Guide.

\section{\nr{} programs}
The structure of a \nr{} program is extremely simple. This sample
program, “toast”, is complete, documented, and executable as it
stands:
\lstinputlisting[label=cheers,caption=Toast]{../../examples/quicktour/toast.nrx}
This program consists of two lines: the first is an optional comment that describes the purpose of the program, and the second is a \textbf{say} instruction. \textbf{say} simply displays the result of the expression following it – in this case just a literal string (you can use either single or double quotes around strings, as you prefer).
To run this program using the reference implementation of \nr{},
create a file called toast.nrx and copy or paste the two lines above
into it. You can then use the \nr{}C Java program to compile it:
\begin{lstlisting}
    java org.netrexx.process.NetRexxC toast
\end{lstlisting}
(this should create a file called toast.class), and then use
the \texttt{java} command to run it:
\begin{lstlisting}
    java toast
\end{lstlisting}
You may also be able to use the netrexxc or nrc command to compile and
run the program with a single command (details may vary – see the
installation and user’s guide document for your implementation of
\nr{}):
\begin{lstlisting}
    netrexxc toast -run
\end{lstlisting}
Of course, \nr{} can do more than just display a character string. Although the language has a simple syntax, and has a small number of instruction types, it is powerful; the reference implementation of the language allows full access to the rapidly growing collection of Java programs known as class libraries, and allows new class libraries to be written in \nr{}.
The rest of this overview introduces most of the features of \nr{}. Since the economy, power, and clarity of expression in \nr{} is best appreciated with use, you are urged to try using the language yourself.
\section{Expressions and variables}
Like \textbf{say} in the “toast” example, many instructions in \nr{} include expressions that will be evaluated. \nr{} provides arithmetic operators (including integer division, remainder, and power operators), several concatenation operators, comparison operators, and logical operators. These can be used in any combination within a \nr{} expression (provided, of course, that the data values are valid for those operations).

All the operators act upon strings of characters (known as \emph{\nr{}
strings}), which may be of any length (typically limited only by the
amount of storage available). Quotes (either single or double) are
used to indicate literal strings, and are optional if the literal
string is just a number. For example, the expressions:
\begin{lstlisting}
    '2' + '3'
    '2' + 3
     2 + 3
\end{lstlisting}
would all result in '5'.

The results of expressions are often assigned to \emph{variables}, using a
conventional assignment syntax:
\begin{lstlisting}[label=assignment,caption=Assignment]
    var1=5            /* sets var1 to '5'    */
    var2=(var1+2)*10  /* sets var2 to '70'   */
\end{lstlisting}
You can write the names of variables (and keywords) in whatever mixture of uppercase and lowercase that you prefer; the language is not case-sensitive.
This next sample program, “greet”, shows expressions used in various
ways:
\lstinputlisting[label=greet,caption=Greet]{../../examples/quicktour/greet.nrx}
After displaying a prompt, the program reads a line of text from the
user (“ask” is a keyword provided by \nr{}) and assigns it to the
variable answer. This is then tested to see if any characters were
entered, and different actions are taken accordingly; for example, if
the user typed “\texttt{Fred}” in response to the prompt, then the program
would display:
\begin{lstlisting}
Hello Fred!
\end{lstlisting}
As you see, the expression on the last \textbf{say} (display) instruction
concatenated the string “Hello” to the value of variable answer with a
blank in between them (the blank is here a valid operator, meaning
“concatenate with blank”). The string “!” is then directly
concatenated to the result built up so far. These unobtrusive
operators (the \emph{blank operator} and abuttal) for concatenation are very
natural and easy to use, and make building text strings simple and
clear.

The layout of instructions is very flexible. In the “greet” example,
for instance, the \textbf{if} instruction could be laid out in a number of
ways, according to personal preference. Line breaks can be added at
either side of the \textbf{then} (or following the \textbf{else}).

In general, instructions are ended by the end of a line. To continue a
instruction to a following line, you can use a hyphen (minus sign)
just as in English:
\begin{lstlisting}[label=continue,caption=Continuation]
    say 'Here we have an expression that is quite long,' -
        'so it is split over two lines'
\end{lstlisting}
This acts as though the two lines were all on one line, with the hyphen and any blanks around it being replaced by a single blank. The net result is two strings concatenated together (with a blank in between) and then displayed.
When desired, multiple instructions can be placed on one line with the
aid of the semicolon separator:
\begin{lstlisting}[label=multiple,caption=Multiple Instructions]
    if answer='Yes' then do; say 'OK!'; exit; end
\end{lstlisting}
(many people find multiple instructions on one line hard to read, but sometimes it is convenient).
\section{Control instructions}
\nr{} provides a selection of \emph{control} instructions, whose form was
chosen for readability and similarity to natural languages. The
control instructions include \textbf{if... then... else} (as in the “greet”
example) for simple conditional processing:
\lstinputlisting[label=dttodaytomorrow,caption=]{./ifthenelse.nrx}

\begin{shaded}
\bash[stdout]
nrc -verbose0 -exec ifthenelse
\END
\end{shaded}

%% \begin{lstlisting}[label=Conditional,caption=Conditional]
%%     if ask='Yes' then say "You answered Yes"
%%                  else say "You didn't answer Yes"
%% \end{lstlisting}
\textbf{select... when... otherwise... end} for selecting from a number of
alternatives:
\lstinputlisting[label=dttodaytomorrow,caption=]{./test_case.nrx}

\begin{shaded}
\bash[stdout]
nrc -verbose0 -exec test_case
\END
\end{shaded}

%% \begin{lstlisting}[label=selectwhenotherwise,caption=select - when - otherwise]
%%     select
%%       when a>0 then say 'greater than zero'
%%       when a<0 then say 'less than zero'
%%       otherwise say 'zero'
%%       end
%%     select case i+1
%%       when 1 then say 'one'
%%       when 1+1 then say 'two'
%%       when 3, 4, 5 then say 'many'
%%       end
%% \end{lstlisting}
\textbf{do... end} for grouping:
\lstinputlisting[label=dttodaytomorrow,caption=]{./do_end.nrx}

\begin{shaded}
\bash[stdout]
nrc -verbose0 -exec do_end
\END
\end{shaded}

%% \begin{lstlisting}[label=doend,caption=do - end]
%%     if a>3 then do
%%       say 'A is greater than 3; it will be set to zero'
%%       a=0
%%     end
%% \end{lstlisting}
and \textbf{loop... end} for repetition:
\lstinputlisting[label=dttodaytomorrow,caption=]{./loop_end.nrx}

\begin{shaded}
\bash[stdout]
nrc -verbose0 -exec loop_end
\END
\end{shaded}

%% \begin{lstlisting}[label=loopend,caption=loop - end]
%%     /* repeat 10 times; I changes from 1 to 10 */
%%     loop i=1 to 10
%%     say i 
%%     end i
%% \end{lstlisting}
The \textbf{loop} instruction can be used to step a variable
\textbf{to} some limit, \textbf{by} some increment, \textbf{for} a
specified number of iterations, and \textbf{while} or \textbf{until}
some condition is satisfied. \textbf{loop forever} is also provided,
and \textbf{loop over} can be used to work through a collection of
variables.

Loop execution may be modified by \textbf{leave} and \textbf{iterate} instructions that significantly reduce the complexity of many programs.
The \textbf{select}, \textbf{do}, and \textbf{loop} constructs also have the ability to “catch”
exceptions (see \ref{exceptions} on page \pageref{exceptions}.) that occur in the body of the construct. All
three, too, can specify a \textbf{finally} instruction which introduces
instructions which are to be executed when control leaves the
construct, regardless of how the construct is ended.


\section{\nr{} arithmetic}
Character strings in \nr{} are commonly used for arithmetic
(assuming, of course, that they represent numbers). The string
representation of numbers can include integers, decimal notation,
and exponential notation; they are all treated the same way. Here are
a few:
\begin{lstlisting}
    '1234'
    '12.03'
    '-12'
    '120e+7'
\end{lstlisting}
The arithmetic operations in \nr{} are designed for people rather than machines, so are decimal rather than binary, do not overflow at certain values, and follow the rules that people use for arithmetic. The operations are completely defined by the ANSI X3.274 standard for Rexx, so correct implementations always give the same results.
An unusual feature of \nr{} arithmetic is the \textbf{numeric} instruction:
this may be used to select the \emph{arbitrary precision} of
calculations. You may calculate to whatever precision that you wish
(for financial calculations, perhaps), limited only by available
memory. For example:
\begin{lstlisting}[label=Digits,caption=Digits]
    numeric digits 50
    say 1/7
\end{lstlisting}
which would display
\begin{lstlisting}
    0.14285714285714285714285714285714285714285714285714
\end{lstlisting}
The numeric precision can be set for an entire program, or be adjusted at will within the program. The \textbf{numeric} instruction can also be used to select the notation (\emph{scientific} or \emph{engineering}) used for numbers in exponential format.
\nr{} also provides simple access to the native binary arithmetic of
computers. Using binary arithmetic offers many opportunities for
errors, but is useful when performance is paramount. You select binary
arithmetic by adding the instruction:
\begin{lstlisting}
    options binary
\end{lstlisting}
at the top of a \nr{} program. The language processor will then use
binary arithmetic (see page \pageref{binarith}) instead of \nr{}
decimal arithmetic for calculations, if it can, throughout the
program\footnote{The binary keyword can also be used for a Class or a
Method statement, and for a do..end block.}.
\section{Doing things with strings}
A character string is the fundamental datatype of \nr{}, and so, as
you might expect, \nr{} provides many useful routines for
manipulating strings. These are based on the functions of Rexx, but
use a syntax that is more like Java or other similar languages:
\begin{lstlisting}[label=strings,caption=Strings]
    phrase='Now is the time for a party'
    say phrase.word(7).pos('r')
\end{lstlisting}
The second line here can be read from left to right as:
\begin{quote}take the variable phrase, find the seventh word, and then find the position of
the first “r” in that word.\end{quote}
This would display “3” in this case, because “r” is the third character in “party”.

(In Rexx, the second line above would have been written using nested
function calls:
\begin{lstlisting}[label=nested,caption=Rexx: Nested]
    say pos('r', word(phrase, 7))
\end{lstlisting}
which is not as easy to read; you have to follow the nesting and then
backtrack from right to left to work out exactly what’s going on.)

In the \nr{} syntax, at each point in the sequence of operations
some routine is acting on the result of what has gone before. These
routines are called \emph{methods}, to make the distinction from functions
(which act in isolation). \nr{} provides (as methods\footnote{and,
since the 4.04 release, also available in the Classic Rexx notation.}) most of the
functions that were evolved for Rexx, including:
\begin{itemize}
\item \texttt{changestr} (change all occurrences of a substring to another)
\item \texttt{copies} (make multiple copies of a string)
\item \texttt{lastpos} (find rightmost occurrence)
\item \texttt{left} and \texttt{right} (return leftmost/rightmost character(s))
\item \texttt{pos} and \texttt{wordpos} (find the position of string or a word in a string)
\item \texttt{reverse} (swap end-to-end)
\item \texttt{space} (pad between words with fixed spacing)
\item \texttt{strip} (remove leading and/or trailing white space)
\item \texttt{verify} (check the contents of a string for selected characters)
\item \texttt{word}, \texttt{wordindex}, \texttt{wordlength}, and \texttt{words} (work with words).
\end{itemize}
These and the others like them, and the parsing described in the next section, make it especially easy to process text with \nr{}.
\section{Parsing strings}
The previous section described some of the string-handling facilities
available; \nr{} also provides string parsing, which is an easy way
of breaking up strings of characters using simple pattern matching.

A \textbf{parse} instruction first specifies the string to be parsed. This can be any term, but is often taken simply from a variable. The term is followed by a \emph{template} which describes how the string is to be split up, and where the pieces are to be put.
\subsection{Parsing into words}
The simplest form of parsing template consists of a list of variable
names. The string being parsed is split up into words (sequences of
characters separated by blanks), and each word from the string is
assigned (copied) to the next variable in turn, from left to
right. The final variable is treated specially in that it will be
assigned a copy of whatever is left of the original string and may
therefore contain several words. For example, in:
\begin{lstlisting}[label=parsingstrings,caption=Parsing Strings]
parse 'This is a sentence.' v1 v2 v3
\end{lstlisting}
the variable v1 would be assigned the value “This”, v2 would be assigned the value
“is”, and v3 would be assigned the value “a sentence.”.
\subsection{Literal patterns}
A literal string may be used in a template as a pattern to split up
the string. For example
\begin{lstlisting}[label=parse,caption=Parse]
    parse 'To be, or not to be?'   w1 ',' w2 w3 w4
\end{lstlisting}
would cause the string to be scanned for the comma, and then split at that point; each section is then treated in just the same way as the whole string was in the previous example.

Thus, w1 would be set to “To be”, w2 and w3 would be assigned the values “or” and “not”, and w4 would be assigned the remainder: “to be?”. Note that the pattern itself is not assigned to any variable.
The pattern may be specified as a variable, by putting the variable
name in parentheses. The following instructions:
\begin{lstlisting}[label=comma,caption=Parse with comma]
    comma=','
    parse 'To be, or not to be?'   w1 (comma) w2 w3 w4
\end{lstlisting}
therefore have the same effect as the previous example.
\subsection{Positional patterns}
The third kind of parsing mechanism is the numeric positional
pattern. This allows strings to be parsed using column positions.

This works in the same way as the string pattern except that it
specifies a column number. So:
\begin{lstlisting}[label=positional,caption=Positional Parsing]
    parse 'Flying pigs have wings' x1 5 x2
\end{lstlisting}                        
would split the string at the fifth column, so x1 would be ”Flyi” 
and x2 would start at column 5 and so be ”ng pigs have wings”.
                                        
\section{Indexed strings}
\nr{} provides indexed strings, adapted from the compound variables of Rexx. Indexed strings form a powerful “associative lookup”, or \emph{dictionary}, mechanism which can be used with a convenient and simple syntax.

\nr{} string variables can be referred to simply by name, or also by
their name qualified by another string (the \emph{index}). When an index is
used, a value associated with that index is either set:
\begin{lstlisting}[label=index,caption=Index]
    fred=0         -- initial value
    fred[3]='abc'  -- indexed value
\end{lstlisting}
or retrieved:
\begin{lstlisting}[label=retrieving,caption=Retrieving]
    say fred[3]    -- would say "abc"
\end{lstlisting}
in the latter case, the simple (initial) value of the variable is
returned if the index has not been used to set a value. For example,
the program:
\begin{lstlisting}[label=woof,caption=Woof]
    bark='woof'
    bark['pup']='yap'
    bark['bulldog']='grrrrr'
    say bark['pup'] bark['terrier'] bark['bulldog']
\end{lstlisting}
would display
\begin{lstlisting}
    yap woof grrrrr
\end{lstlisting}
Note that it is not necessary to use a number as the index; any
expression may be used inside the brackets; the resulting string is
used as the index. Multiple dimensions may be used, if required:
\begin{lstlisting}[label=dimensions,caption=Multiple Dimensions]
    bark='woof'
    bark['spaniel', 'brown']='ruff'
    bark['bulldog']='grrrrr'
    animal='dog'
    say bark['spaniel', 'brown'] bark['terrier'] bark['bull'animal]
\end{lstlisting}
which would display
\begin{lstlisting}
    ruff woof grrrrr
\end{lstlisting}
Here’s a more complex example using indexed strings, a test program
with a function (called a \emph{static method} in \nr{}) that removes all
duplicate words from a string of words:
\begin{lstlisting}[label=refjust1,caption=justonetest.nrx]
    /* justonetest.nrx -- test the justone function.      */
    say justone('to be or not to be')  /* simple testcase */
    exit
    /* This removes duplicate words from a string, and    */
    /* shows the use of a variable (HADWORD) which is     */
    /* indexed by arbitrary data (words).                 */
    method justone(wordlist) static
      hadword=0         /* show all possible words as new */
      outlist=''            /* initialize the output list */
      loop while wordlist\=''  /* loop while we have data */
        /* split WORDLIST into first word and residue     */
        parse wordlist word wordlist
        if hadword[word] then iterate /* loop if had word */
        hadword[word]=1 /* remember we have had this word */
        outlist=outlist word   /* add word to output list */
        end
      return outlist /* finally return the result */
\end{lstlisting}
Running this program would display just the four words “to”, “be”, “or”, and “not”.
\section{Arrays}
\nr{} also supports fixed-size \emph{arrays}. These are an ordered set of
items, indexed by integers. To use an array, you first have to
construct it; an individual item may then be selected by an index
whose value must be in the range \texttt{0} through \texttt{n–1}, where n is the number
of items in the array:
\begin{lstlisting}[label=arrays,caption=Arrays]
    array=String[3]        -- make an array of three Strings
    array[0]='String one'  -- set each array item
    array[1]='Another string'
    array[2]='foobar'
    loop i=0 to 2          -- display the items
      say array[i]
      end
\end{lstlisting}
This example also shows \nr{} \emph{line comments}; the sequence “--” (outside of literal strings or “/*” comments) indicates that the remainder of the line is not part of the program and is commentary.

\nr{} makes it easy to initialize arrays: a term which is a list of
one or more expressions, enclosed in brackets, defines an array. Each
expression initializes an element of the array. For example:
\begin{lstlisting}[label=initializingelements,caption=Initializing elements]
words=['Ogof', 'Ffynnon', 'Ddu']
\end{lstlisting}
would set words to refer to an array of three elements, each referring to a string. So, for
example, the instruction:
\begin{lstlisting}[label=addresselement,caption=Address Array Element]
say words[1]
\end{lstlisting}
would then display 
\begin{lstlisting}
Ffynnon
\end{lstlisting}

\section{Things that aren’t strings}
In all the examples so far, the data being manipulated (numbers, words, and so on) were expressed as a string of characters. Many things, however, can be expressed more easily in some other way, so \nr{} allows variables to refer to other collections of data, which are known as \emph{objects}.

Objects are defined by a name that lets \nr{} determine the data and methods that are associated with the object. This name identifies the type of the object, and is usually called the \emph{class} of the object.

For example, an object of class Oblong might represent an oblong to be manipulated and displayed. The oblong could be defined by two values: its width and its height. These values are called the \emph{properties} of the Oblong class.

Most methods associated with an object perform operations on the object; for example a size method might be provided to change the size of an Oblong object. Other methods are used to construct objects (just as for arrays, an object must be constructed before it can be used). In \nr{} and Java, these \emph{constructor} methods always have the same name as the class of object that they build (“Oblong”, in this case).

Here’s how an Oblong class might be written in \nr{} (by convention,
this would be written in a file called \texttt{Oblong.nrx}; implementations
often expect the name of the file to match the name of the class
inside it):
\lstinputlisting[label=oblong,caption=Oblong]{../../examples/quicktour/Oblong.nrx}
To summarize:
\begin{enumerate}
\item A class is started by the \textbf{class} instruction, which names the class.
\item The \textbf{class} instruction is followed by a list of the properties of the object. These can be assigned initial values, if required.
\item The properties are followed by the methods of the object. Each
method is introduced by a \textbf{method} instruction which names the method
and describes the arguments that must be supplied to the method. The
body of the method is ended by the next method instruction (or by the
end of the file).
\end{enumerate}
The \texttt{Oblong.nrx} file is compiled just like any other \nr{} program,
and should create a \emph{class file} called \texttt{Oblong.class}. Here’s a program
to try out the Oblong class:
\lstinputlisting[label=tryoblong,caption=Try Oblong]{../../examples/quicktour/tryOblong.nrx}
When tryOblong.nrx is compiled, you’ll notice (if your compiler makes a cross-reference listing available) that the variables \texttt{first} and \texttt{second} have type \texttt{Oblong}. These variables refer to Oblongs, just as the variables in earlier examples referred to \nr{} strings.

Once a variable has been assigned a type, it can only refer to objects of that type. This helps avoid errors where a variable refers to an object that it wasn’t meant to.
\subsection{Programs are classes, too}
It’s worth pointing out, here, that all the example programs in this overview are in fact classes (you may have noticed that compiling them with the reference implementation creates \texttt{xxx.class} files, where \texttt{xxx} is the name of the source file). The environment underlying the implementation will allow a class to run as a stand-alone \emph{application} if it has a static method called \texttt{main} which takes an array of strings as its argument.

If necessary (that is, if there is no class instruction) \nr{} automatically adds the necessary class and method instructions for a stand-alone application, and also an instruction to convert the array of strings (each of which holds one word from the command string) to a single \nr{} string.

The automatic additions can also be included explicitly; the “toast”
example could therefore have been written:
\begin{lstlisting}[label=toast,caption=New Toast]
    /* This wishes you the best of health. */
    class toast
      method main(argwords=String[]) static
        arg=Rexx(argwords)
        say 'Cheers!'
\end{lstlisting}
though in this program the argument string, \texttt{arg}, is not used,
which will be indicated in a compile warning, for this (in this case,) superflous assignment.
\section{Extending classes}
It is common, when dealing with objects, to take an existing class and
extend it. One way to do this is to modify the source code of the
original class – but this is not always available, and with many different people modifying a class, classes could rapidly get overcomplicated.

Languages that deal with objects, like \nr{}, therefore allow new
classes of objects to be set up which are derived from existing
classes. For example, if you wanted a different kind of Oblong in
which the Oblong had a new property that would be used when printing
the Oblong as a rectangle, you might define it thus:
\lstinputlisting[label=charoblong,caption=charOblong]{../../examples/quicktour/charOblong.nrx}
% \begin{lstlisting}[label=charoblong,caption=charOblong.nrx]
%     /* charOblong.nrx -- an oblong class with character */
%     class charOblong extends Oblong
%       printchar              -- the character for display
%       /* Constructor to make a new oblong with character */
%       method charOblong(newwidth, newheight, newprintchar)
%         super(newwidth, newheight)  -- make an oblong
%         printchar=newprintchar         -- and set the character
%       /* 'Print' the oblong */
%       method print
%         loop for super.height
%           say printchar.copies(super.width)
%           end
% \end{lstlisting}
There are several things worth noting about this example:
\begin{enumerate}
\item The “\texttt{extends Oblong}” on the class instruction means that this class is an extension of the Oblong class. The properties and methods of the Oblong class are \emph{inherited} by this class (that is, appear as though they were part of this class).
Another common way of saying this is that “\texttt{charOblong}” is a \emph{subclass} of “\texttt{Oblong}” (and “\texttt{Oblong}” is the \emph{superclass} of “\texttt{charOblong}”).
\item This class adds the \texttt{printchar} property to the properties already defined for Oblong.
\item The constructor for this class takes a width and height (just like Oblong) and adds a third argument to specify a print character. It first invokes the constructor of its superclass (Oblong) to build an Oblong, and finally sets the printchar for the new object.
\item The new charOblong object also prints differently, as a rectangle of characters, according to its dimension. The \texttt{print} method (as it has the same name and arguments – none – as that of the superclass) replaces (overrides) the \texttt{print'} method of Oblong.
\item The other methods of Oblong are not overridden, and therefore
  can be used on charOblong objects.
\end{enumerate}
The \texttt{charOblong.nrx} file is compiled just like \texttt{Oblong.nrx} was, and
should create a file called \texttt{charOblong.class}.

Here’s a program to try it out
\lstinputlisting[label=trycharoblong,caption=trycharOblong]{../../examples/quicktour/trycharOblong.nrx}
% \begin{lstlisting}[label=trycharoblong,caption=tryCharOblong.nrx]
%     /* trycharOblong.nrx -- try the charOblong class */
%     first=charOblong(5,3,'#')  -- make an oblong
%     first.print                -- show it
%     first.relsize(1,1).print   -- enlarge and print again
%     second=charOblong(1,2,'*') -- make another oblong
%     second.print               -- and print it
% \end{lstlisting}
This should create the two charOblong objects, and print them out in a simple “character graphics” form. Note the use of the method \texttt{relsize} from Oblong to resize the charOblong object.
\subsection{Optional arguments}
All methods in \nr{} may have optional arguments (omitted from the
right) if desired. For an argument to be optional, you must supply a
default value. For example, if the charOblong constructor was to have
a default value for printchar, its method instruction could have been
written
\begin{lstlisting}[label=default,caption=Default value X]
method charOblong(newwidth, newheight, newprintchar='X')
\end{lstlisting}
which indicates that if no third argument is supplied then \texttt{'X'} should be used. A program
creating a charOblong could then simply write:
\begin{lstlisting}[label=default,caption=Default value]
first=charOblong(5,3) -- make an oblong
\end{lstlisting}
which would have exactly the same effect as if \texttt{'X'} were specified as
the third argument.


\section{Tracing}
\nr{} tracing is defined as part of the language. The flow of
execution of programs may be traced, and this trace can be viewed as
it occurs (or captured in a file). The trace can show each clause as
it is executed, and optionally show the results of expressions,
etc. For example, the \textbf{trace results} in the program
“\texttt{trace1.nrx}”:
\lstinputlisting[label=dttodaytomorrow,caption=]{./trace1.nrx}

would result in:

\begin{shaded}
\bash[stderr]
nrc -verbose0 -exec trace1
\END
\end{shaded}

%% \begin{lstlisting}[label=trace,caption=Trace]
%%     trace results
%%     number=1/7
%%     parse number before '.' after
%%     say after'.'before
%% \end{lstlisting}

%% \begin{lstlisting}
%%        ––– trace1.nrx
%%      2 *=* number=1/7
%%        >v> number "0.142857143"
%%      3 *=* parse number before '.' after
%%        >v> before "0"
%%        >v> after "142857143"
%%      4 *=* say after'.'before
%%        >>> "142857143.0"
%%     142857143.0
%% \end{lstlisting}
where the line marked with “\texttt{–––}” indicates the context of the trace, lines marked with “\texttt{*=*}” are the instructions in the program, lines with “\texttt{>v>}” show results assigned to local variables, and lines with “\texttt{>>>}” show results of unnamed expressions.

Further, \textbf{trace methods} lets you trace the use of all methods in a
class, along with the values of the arguments passed to each
method. Here’s the result of adding \texttt{trace methods} to the Oblong class
shown earlier and then running \texttt{tryOblong}:
“\texttt{trace1.nrx}”:
\lstinputlisting[label=dttodaytomorrow,caption=]{./tryOblong.nrx}

would result in:

\begin{shaded}
\bash[stderr]
nrc -verbose0 -exec tryOblong
\END
\end{shaded}

%% \begin{lstlisting}
%%         ––– Oblong.nrx
%%       8 *=*     method Oblong(newwidth, newheight)
%%         >a> newwidth "5"
%%         >a> newheight "3"
%%      26 *=*     method print
%%     Oblong 5 x 3
%%      20 *=*     method relsize(relwidth, relheight)–
%%  21 *–*
%%     >a> relwidth "1"
%%     >a> relheight "1"
%%  26 *=*     method print
%% Oblong 6 x 4
%% returns Oblong
%%      10 *=*     method Oblong(newwidth, newheight)
%%         >a> newwidth "1"
%%         >a> newheight "2"
%%      26 *=*     method print
%%     Oblong 1 x 2
%% \end{lstlisting}
where lines with “>a>” show the names and values of the arguments.

It is often useful to be able to find out when (and where) a variable’s value is changed. The \textbf{trace var} instruction does just that; it adds names to or removes names from a list of monitored variables. If the name of a variable in the current class or method is in the list, then \textbf{trace results} is turned on for any assignment, \textbf{loop}, or \textbf{parse} instruction that assigns a new value to the named variable.

Variable names to be added to the list are specified by listing them after the \textbf{var} keyword. Any name may be optionally prefixed by a – sign., which indicates that the variable is to be removed from the list.

For example, the program “\texttt{trace2.nrx}”:
\lstinputlisting[label=dttodaytomorrow,caption=]{./trace2.nrx}

would result in:

\begin{shaded}
\bash[stderr]
nrc -verbose0 -exec trace2
\END
\end{shaded}

%% \begin{lstlisting}[label=trace2,caption=trace2.nrx]
%%     trace var a b -- now variables a and b will be traced
%%     a=3
%%     b=4
%%     c=5
%%     trace var –b c -- now variables a and c will be traced
%%     a=a+1
%%     b=b+1
%%     c=c+1
%%     say a b c
%% \end{lstlisting}
%% would result in:
%% \begin{lstlisting}
%%         --- trace2.nrx
%%   3 *=* a=3
%%     >v> a "3"
%%   4 *=* b=4
%%     >v> b "4"
%%   8 *=* a=a+1
%%     >v> a "4"
%%  10 *=* c=c+1
%%     >v> c "6"
%% 4 5 6
%% \end{lstlisting}

\section{Binary types and conversions}\label{binarith}

Most programming environments support the notion of fixed-precision
“primitive” binary types, which correspond closely to the binary
operations usually available at the hardware level in computers. For
the reference implementation, these types are:
\begin{itemize}
\item \emph{byte}, \emph{short}, \emph{int}, and \emph{long} – signed integers that will fit in 8, 16, 32, or 64 bits respectively
\item \emph{float} and \emph{double} – signed floating point numbers that will fit in 32 or 64 bits respectively.
\item \emph{char} – an unsigned 16-bit quantity, holding a Unicode character
\item \emph{boolean} – a 1-bit logical value, representing 0 or 1
    (“false” or “true”).
\end{itemize}
Objects of these types are handled specially by the implementation “under the covers” in order to achieve maximum efficiency; in particular, they cannot be constructed like other objects – their value is held directly. This distinction rarely matters to the \nr{} programmer: in the case of string literals an object is constructed automatically; in the case of an \texttt{int} literal, an object is not constructed.

Further, \nr{} automatically allows the conversion between the
various forms of character strings in implementations\footnote{In the
  reference implementation, these are String, char, char[] (an array
  of characters), and the \nr{} string type, Rexx.} and the
primitive types. The “golden rule” that is followed by \nr{} is that
any automatic conversion which is applied must not lose information:
either it can be determined before execution that the conversion is
safe (as in \texttt{int} to \texttt{String}) or it will be detected at
execution time if the conversion fails (as in \texttt{String} to
\texttt {int}).

The automatic conversions greatly simplify the writing of programs; the exact type of numeric and string-like method arguments rarely needs to be a concern of the programmer.
For certain applications where early checking or performance override
other considerations, the reference implementation of \nr{}
provides options for different treatment of the primitive types:
\begin{enumerate}
\item \textbf{options strictassign} – ensures exact type matching for all assignments. No conversions (including those from shorter integers to longer ones) are applied. This option provides stricter type-checking than most other languages, and ensures that all types are an exact match.
\item \textbf{options binary} – uses implementation-dependent fixed precision
arithmetic on binary types (also, literal numbers, for example, will
be treated as binary, and local variables will be given “native”
types such as \texttt{int} or \texttt{String}, where possible).
\end{enumerate}
Binary arithmetic currently gives better performance than \nr{}
decimal arithmetic, but places the burden of avoiding overflows and
loss of information on the programmer.

The options instruction (which may list more than one option) is placed before the first class instruction in a file; the \textbf{binary} keyword may also be used on a \textbf{class} or \textbf{method} instruction, to allow an individual class or method to use binary arithmetic.
\subsection{Explicit type assignment}
You may explicitly assign a type to an expression or variable:
\begin{lstlisting}[label=assigningtype,caption=Assigning Type]
i=int 3000000  -- 'i' is an 'int' with value 3000000
j=int 4000000  -- 'j' is an 'int' with value 4000000
k=int -- 'k' is an 'int', with no initial value
say i*j -- multiply and display the result
k=i*j -- multiply and assign result to 'k'
say k
\end{lstlisting}
This example also illustrates an important difference between
\textbf{options nobinary} and \textbf{options binary}. With the former
(the default) the \textbf{say} instruction would display the result
“\texttt{1.20000000E+13}” and a conversion overflow would be reported
when the same expression is assigned to the variable k.

With \textbf{options binary}, binary arithmetic would be used for the
multiplications, and so no error would be detected; the say would
display “–138625024” (note the minus sign) and the variable \texttt{k} takes the incorrect result.
\subsection{Binary types in practice}
In practice, explicit type assignment is only occasionally needed in
\nr{}. Those conversions that are necessary for using existing
classes (or those that use \textbf{options binary}) are generally
automatic. For example, here is an Applet for use by Java-enabled
browsers:
\begin{lstlisting}[label=asimpleapplet,caption=A Simple Applet]
    /* A simple graphics Applet */
    class Rainbow extends Applet
      method paint(g=Graphics)  -- called to repaint window
        maxx=size.width-1
        maxy=size.height-1
        loop y=0 to maxy
          col=Color.getHSBColor(y/maxy, 1, 1) -- new colour
          g.setColor(col)                     -- set it
          g.drawLine(0, y, maxx, y)           -- fill slice
       end y
\end{lstlisting}
In this example, the variable \texttt{col} will have type \texttt{Color}, and the three
arguments to the method \texttt{getHSBColor} will all automatically be
converted to type float. As no overflows are possible in this example,
\textbf{options binary} may be added to the top of the program with no other
changes being necessary.

\section{Exception and error handling}\label{exceptions}
\nr{} does not have a \textbf{goto} instruction, but a \textbf{signal} instruction is provided for abnormal transfer of control, such as when something unusual occurs. Using \textbf{signal} raises an \emph{exception}; all control instructions are then “unwound” until the exception is caught by a control instruction that specifies a suitable catch instruction for handling the exception.

Exceptions are also raised when various errors occur, such as
attempting to divide a number by zero. For example:
\begin{lstlisting}[label=exception,caption=Exception]
    say 'Please enter a number:'
    number=ask
    do
      say 'The reciprocal of' number 'is:' 1/number
    catch DivideException
      say 'Sorry, could not divide "'number'" into 1'
      say 'Please try again.'
    end
\end{lstlisting}
Here, the \textbf{catch} instruction will catch any exception that is raised when the division is attempted (conversion error, divide by zero, \emph{etc.}), and any instructions that follow it are then executed. If no exception is raised, the \textbf{catch} instruction (and any instructions that follow it) are ignored.

Any of the control instructions that end with \textbf{end} (\textbf{do}, \textbf{loop}, or \textbf{select}) may be modified with one or more \textbf{catch} instructions to handle exceptions.

\section{Summary and Information Sources}
The \nr{} language, as you will have seen, allows the writing of programs for the Java environment with a minimum of overhead and “boilerplate syntax”; using \nr{} for writing Java classes could increase your productivity by 30\% or more.
Further, by simplifying the variety of numeric and string types of
Java down to a single class that follows the rules of Rexx strings,
programming is greatly simplified. Where necessary, however, full
access to all Java types and classes is available.

Other examples are available, including both stand-alone applications and samples of applets for Java-enabled browsers (for example, an applet that plays an audio clip, and another that displays the time in English). You can find these from the \nr{} web pages, at
    \url{http://www.netrexx.org}.
Also at that location, you’ll find the \nr{} language specification
and other information, and downloadable packages containing the
\nr{} software and documentation. There is a large selection of
\nr{} examples available at \url{http://www.rosettacode.org}.
The software should run on any platform that has a Java Virtual
Machine (JVM) available.
