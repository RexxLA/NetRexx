\chapter{Introduction}
This document is the \emph{Quick Start Guide} for the reference implementation of
NetRexx. NetRexx is a \emph{human-oriented} programming language which makes
writing and using Java\footnote{Java is a trademark of Oracle, Inc.}
classes quicker and easier than writing in Java. It is part of the Rexx
language family, under the governance of the Rexx Language
Association.\footnote{\url{http.www.rexxla.org}} NetRexx has been
developed and was made available as a free download by IBM since 1995
and is free and open source since June 8, 2011.

In this Quick Start Guide, you’ll find information on
\begin{enumerate} 
\item How easy it is to write for the JVM: A Quick Tour of NetRexx
\item Installing NetRexx 
\item Using the NetRexx translator as a compiler, interpreter, or
  syntax checker 
\item Troubleshooting when things do not work as expected
\item Current restrictions.
\end{enumerate} 
The NetRexx documentation and software are distributed
by The Rexx Language Association under the \textsc{ICU} license. For
the terms of this license, see the included \textsc{LICENSE} file in
this package.

For details of the NetRexx language, and the latest news, downloads,
etc., please see the NetRexx documentation included with the package
or available at: \url{http://www.netrexx.org}.

\chapter{Requirements}
Since release 3.01 (August 2012), NetRexx requires only a
JRE\footnote{Java Runtime Environment} for program development, where previously a
Java SDK\footnote{Software Development Kit} (earlier name: JDK) was required. For serious development
purposes a Java SDK is recommended, as the tools found therein might
assist the development process. NetRexx runs on a wide variety of
hardware and operating systems; all releases are tested on (non-exhaustive):
\begin{enumerate}
\item Windows Desktop and Server editions, with Oracle and IBM JVMs
\item Linux, with Oracle and IBM JVMs, including z/Linux
\item MacOSX with OpenJDK and Apple JVM
\item Android on ARM hardware with Dalvik virtual machine
\item z/OS OMVS
\item eComstation 2.x (OS/2) with eComstation Java 1.6
\item The Raspberry Pi, using Raspbian Linux and Oracle Embedded
  Edition ARM JDK-8
\end{enumerate}
NetRexx runs equally well on 32- or 64-bit JVMs. As the translator is
a command line tool, no graphics configuration is required, and
headless operation is supported. Care is taken to keep the NetRexx runtime small, and to keep
compatibility with earlier(post-beta) Java releases, older operating systems and
limited devices environments. The class file format, however, of
current release distributions is 1.6; for older formats, you
can build NetRexx yourself or request assistance from the development
team (\nolinebreak[4]developers@netrexx.kenai.com)\footnote{You will
  need to be member of the Kenai NetRexx project} for a special build.

