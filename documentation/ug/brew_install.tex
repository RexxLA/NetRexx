\section{Installing using Homebrew}

Homebrew is a popular package manager for macOS and Linux. The website
is at \url{brew.sh}. It can be installed using a simple curl script,
and for the management of packages no sudo access is
required\footnote{It is even disallowed to used sudo with brew}.

\nr{} can be installed using \code{brew}. The \rexx{}LA repository is
used for the \nr{} recipes.

\begin{lstlisting}
  brew tap rexxla/rexxla
\end{lstlisting}

Using the above \code{tap} command connects Homebrew with the
\rexx{}LA repository. The following recipes become available:

\begin{lstlisting}
  netrexx
  netrexx-openjdk
\end{lstlisting}

The latter installs \nr{} including a current version of the
OpenJDK. This is a good solution for the situation where there is no
JVM on the machine, and you want the JVM being managed by the Homebrew
package manager.

The former installs just the current version of \nr{} where it is
assumed you have a working version of Java on your machine.

