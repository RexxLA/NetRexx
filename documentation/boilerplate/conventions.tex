\chapter{Typographical conventions}
In general, the following conventions have  been observed in the NetRexx publications:
\begin{itemize}
\item Body text is in this font
\item Examples of language statements are in a \textbf{bold} type
\item Variables or strings as mentioned in source code, or things that appear on the console, are in a \texttt{typewriter} type
\item Items that are introduced, or emphasized, are in an \emph{italic} type
\item Included program fragments are listed in this fashion:
\begin{lstlisting}[label=example,caption=Example Listing]
-- salute the reader
say 'lectorem salutat'
\end{lstlisting}
\item Syntax diagrams take the form of so-called \emph{Railroad Diagrams} to convey structure, mandatory and optional items
\begin{rail}
AggregateExpression : ("AVG" |"MAX" |"MIN" |"SUM")
 (
   (
    'DISTINCT' ?  StateFieldPathExpression
   ) | 'COUNT'
   (
    'DISTINCT' ?  IdentificationVariable
                  | StateFieldPathExpression
                  | SingleValuedAssociationPathExpression
   )
 )
   ;
\end{rail}
%%% Local Variables: 
%%% mode: latex
%%% TeX-master: t
%%% End: 
\end{itemize}