\documentclass[10pt]{book}
\usepackage[FINAL]{../../../boilerplate/rexx} 
\usepackage{hyperref}
\usepackage{graphics}
\usepackage{geometry}
\usepackage{setspace}
\usepackage{etoolbox}
\usepackage{fontspec}
\usepackage{tocloft}
\usepackage{titlesec}
\setmainfont[Mapping=tex-text]{IBM Plex Serif}
\setmonofont[Mapping=tex-text,Scale=0.82]{IBM Plex Mono}
\usepackage{tabularx}
\usepackage{booktabs}
\usepackage{makeidx}
\usepackage{color}
\usepackage{xcolor}
\usepackage{listings}

%fancyhdr for page headers, footers, numbers
\usepackage{fancyhdr}

\pagestyle{fancy}
\fancyhf{} % Clear default header/footer settings

\fancyheadoffset{0pt}
\addtolength{\headsep}{15pt}
\setlength{\headheight}{22.27003pt}

% Define the headers
\fancyhead[RO]{\normalfont \nouppercase{\rightmark}}  % Chapter title on right pages, aligned right
\fancyhead[LE]{\normalfont \nouppercase{\leftmark}}   % Section title on left pages, aligned right

% Ensure chapter and section marks update correctly
%% \renewcommand{\chaptermark}[1]{\markright{#1}\normalfont \thechapter.\ #1}
%% \renewcommand{\sectionmark}[1]{\markleft{#1}\normalfont \thesection.\ #1}

\fancyfoot[C]{\thepage}

% avoid headers on empty pages
\usepackage{emptypage}

\usepackage{caption}
\usepackage{longtable}
\usepackage{colortbl}
\usepackage{framed}
\usepackage{fancyvrb}
\definecolor{shadecolor}{rgb}{0.9,0.9,0.9}
\definecolor{nrblue}{RGB}{38,139,210}
\definecolor{nrgreen}{RGB}{65,133,153}
\definecolor{nrcyan}{RGB}{42,161,152 }
\definecolor{nrorange}{RGB}{203 ,75,22}
\definecolor{nrgrey}{RGB}{101,123,131}
\usepackage{alltt}
%% \DeclareCaptionFont{white}{\color{white}}
%% \DeclareCaptionFormat{listing}{\colorbox{gray}{\parbox{\textwidth}{#1#2#3}}}
%% \captionsetup[lstlisting]{format=listing,labelfont=white,textfont=white}
\usepackage[official]{eurosym}
%% \makeatletter
%% \@empty\z@\@empty
%% \lst@CCPutMacro\lst@ProcessOther {"2D}{\lst@ttfamily{-{}}{-{}}} 
%% \makeatother
%%\renewcommand*\familydefault{\sfdefault}
\lstdefinelanguage{NetRexx}
{morekeywords={abstract,adapter,binary,case,catch,class,constant,dependent,deprecated,digits,do,else,end,engineering,extends,final,finally,for,forever,if,implements,indirect,import,indirect,inheritable,interface,iterate,label,leave,loop,method,native,nop,numeric,options,otherwise,over,package,parent,parse,private,properties,protect,public,return,returns,rexx,say,scientific,set,digits,form,select,shared,signal,signals,sourceline,static,super,then,this,until,used,upper,volatile,when,where,while},
sensitive=false,
extendedchars=false,
morecomment=[s]={/*}{*/},
morecomment=[l]{--},
morecomment=[s]{/**}{*/},
morestring=[b]",
morestring=[d]",
morestring=[b]',
morestring=[d]'}

\lstset{language=NetRexx,
  captionpos=t,
  tabsize=3,
  alsolanguage=Rexx,
  keywordstyle=\color{blue},
  commentstyle=\color{cyan},
  stringstyle=\color{red},
  numbers=left,
  numberstyle=\tiny,
  numbersep=5pt,
  breaklines=true,
  showstringspaces=false,
  index=[1][keywords],
  columns=flexible,
  basicstyle=\fontsize{8}{8}\fontspec{Source Code Pro},emph={label}}

%% \input{../../../boilerplate/assemblerformat}
%% \input{../../../boilerplate/jclformat}
\usepackage{pst-barcode,pstricks-add}
\usepackage{bashful}
\usepackage{metalogo}
\usepackage{marginnote}
\usepackage{pdfpages}
\usepackage{svg}
\usepackage{float}
\makeindex
\DeclareGraphicsExtensions{.jpg,.png}
\setlength{\parskip}{8pt}
\setlength{\parindent}{0pt}
\usepackage{enumitem}
\usepackage{underscore}
\usepackage{babel}
\usepackage{graphicx}
\usepackage{array}
% \usepackage{tabularx}
\usepackage{multirow}
\usepackage{threeparttable}
\usepackage{longtable}
\usepackage{booktabs}
\usepackage{makeidx}
\usepackage{framed}
\usepackage{alltt}
\usepackage{barracuda}
\usepackage{longtable}
\usepackage{graphicx}
\usepackage{wrapfig}
\usepackage{multicol}
\usepackage[CJK]{ucharclasses}
\usepackage[comma]{natbib}           % CSLI Pubs favored bibliography package.
%% \bibliographystyle{cslipubs-natbib}  % CSLI Pubs bibliography format.
\bibliographystyle{plainnat}  % CSLI Pubs bibliography format.


%% This books starts at chapter 0  
%% \setcounter{chapter}{-1}
%% the font used for tables
\AtBeginEnvironment{longtable}{\fontspec{IBM Plex Mono}\small}

%% \setlength{\labelsep}{2em} % Distance between label and text
\setlength{\labelwidth}{75pt} % Width of the label box  

\setlist[description]{font=\normalfont}

\usepackage{sectsty}
%% \partfont{\color{nrgreen}}
%% \chapterfont{\fontspec{Avenir Light}}
%% \sectionfont{\fontspec{Avenir Light}}
%% \subsectionfont{\fontspec{Avenir Light}}
%% \subsubsectionfont{\fontspec{Avenir Light}}

\usepackage[Sonny]{fncychap}
\makeatletter
\renewcommand{\@makechapterhead}[1]{%
  \thispagestyle{plain}%
  \vspace*{50pt}%
  {\parindent \z@
   \raggedleft
   \normalfont\fontspec{IBM Plex Serif}\LARGE \@chapapp{} \thechapter
   \par\nobreak
   \vskip 10pt
   \hrule height 0.4pt width \textwidth \relax
   \vskip 10pt
   {\normalfont\fontspec{IBM Plex Serif}\Huge #1\par}%
   \vskip 40pt
  }}
\makeatother

\titleformat{\section}
  {\normalfont\fontspec{IBM Plex Serif}\Large\bfseries}
  {\thesection}{1em}{}
  [\vspace{0.5ex}\titlerule]

\titlespacing*{\section}{0pt}{2.5ex plus 1ex minus .2ex}{1.5ex plus .2ex}


\raggedbottom

\geometry{left=1in, top=1in, bottom=1in, right=1in}
\setstretch{1.2}
\titlespacing*{\section}{0pt}{*1.5}{*0.5}
\titlespacing*{\subsection}{0pt}{*1.2}{*0.4}

\setlength{\parskip}{0.5em}
\raggedbottom

\makeatletter
\patchcmd{\@startsection}{\@afterindenttrue}{\@afterindentfalse}{}{}
\makeatother

%% heading number left overhang
\makeatletter
\def\@seccntformat#1{\protect\makebox[0pt][r]{\csname
the#1\endcsname\quad}}
\makeatother

%% use the heading fonts in the TOC
%% \renewcommand{\cftsecfont}{\sffamily\bfseries} % Matches section font
%% \renewcommand{\cftsubsecfont}{\sffamily} % Example for subsections

\renewcommand{\cftpartfont}{\sffamily}
\renewcommand{\cftchapfont}{\sffamily} % Matches section font
\renewcommand{\cftchappagefont}{\sffamily} % Matches section font
\renewcommand{\cftsubsecfont}{\sffamily} % Example for subsections
\renewcommand{\cftsubsubsecfont}{\sffamily} % Example for subsections
\renewcommand{\cftsecfont}{\sffamily} % Matches section font

\titleformat{name=\chapter,numberless}[display]
  %{\fontspec{IBM Plex Serif Light}\filcenter} % formatting applied to the
  % heading text
    {\filcenter} % formatting applied to the heading text
  {}
  {0pt}
  %% {\titlerule \vspace{1ex} \fontspec{IBM Plex Serif Light} } 
 {\titlerule \vspace{1ex} }  % code before the heading text

\titlespacing*{name=\chapter,numberless}
  {0pt}{-1cm}{1cm}  % left, before, after

  \renewcommand{\contentsname}{\fontspec{IBM Plex Serif Light}Contents}
  \renewcommand{\indexname}{\fontspec{IBM Plex Serif Light}Index}
  \renewcommand{\listtablename}{\fontspec{IBM Plex Serif Light}List of tables}
  \renewcommand{\listfigurename}{\fontspec{IBM Plex Serif Light}List of figures}
  
  \sectionfont{%
    \hrule                % place a horizontal line
    \vspace{1ex}%         % add some vertical space
    \fontspec{IBM Plex Serif Light}       % set section title size & style
  }

% defines for consistent use
\newcommand{\nr}{Net\textsc{Rexx}}
\newcommand{\Rexx}{R\textsc{exx}}
\newcommand{\nrpackagename}{\splice{java GetPackageName}}
\newcommand{\minimalJVMversion}{8}
\newcommand{\maximalJVMversion}{20}
\newcommand{\keyword}[1]{\texttt{#1}}
\newcommand{\code}[1]{\texttt{#1}}
\newcommand{\thisyear}{\splice{java TexYear}}
\newcommand{\ecjjarname}{ecj-I20201218-1800-NRX.jar}
\newcommand{\msd}[1]{\msdhelper#1\relax}
\newcommand{\msdhelper}[1]
  {\ifx\relax#1\else
    \ifx-#1--{}\else#1\fi
    \expandafter\msdhelper\fi}
  \newcommand{\doublehyphen}{\mbox{\msd{``-~-''}}}
  \newcommand{\doublehyphenunquoted}{\mbox{\msd{-~-}}}
\newcommand{\simularelease}{12.00, date of release  9 aug 1985}
