\documentclass[10pt, openany]{book}
\usepackage[FINAL]{../boilerplate/rexx} 
\usepackage{hyperref}
\usepackage{graphics}
\usepackage{fontspec}
\setmainfont[Mapping=tex-text]{TeX Gyre Pagella}
%% \setmainfont[Mapping=tex-text]{Georgia Pro}
\setmonofont[Mapping=tex-text,Scale=0.82]{IBM Plex Mono}
\usepackage{tabularx}
\usepackage{booktabs}
\usepackage{makeidx}
\usepackage[all]{xy}
%\usepackage{lingmacros}
\usepackage{color}
\usepackage{xcolor}
\usepackage{listings}
\usepackage{caption}
\usepackage{longtable}
\usepackage{colortbl}
\usepackage{framed}
\usepackage{fancyvrb}
\definecolor{shadecolor}{rgb}{0.9,0.9,0.9}
\definecolor{nrblue}{RGB}{38,139,210}
\definecolor{nrgreen}{RGB}{65,133,153}
\definecolor{nrcyan}{RGB}{42,161,152 }
\definecolor{nrorange}{RGB}{203 ,75,22}
\definecolor{nrgrey}{RGB}{101,123,131}
\usepackage{alltt}
\DeclareCaptionFont{white}{\color{white}}
\DeclareCaptionFormat{listing}{\colorbox{gray}{\parbox{\textwidth}{#1#2#3}}}
\captionsetup[lstlisting]{format=listing,labelfont=white,textfont=white}
\usepackage{listings}
\usepackage[official]{eurosym}
\makeatletter
\lst@CCPutMacro\lst@ProcessOther {"2D}{\lst@ttfamily{-{}}{-{}}}
\@empty\z@\@empty
\makeatother
\lstdefinelanguage{NetRexx}
{morekeywords={abstract,adapter,binary,case,catch,class,constant,dependent,deprecated,digits,do,else,end,engineering,extends,final,finally,for,forever,if,implements,indirect,import,indirect,inheritable,interface,iterate,label,leave,loop,method,native,nop,numeric,options,otherwise,over,package,parent,parse,private,properties,protect,public,return,returns,rexx,say,scientific,set,digits,form,select,shared,signal,signals,sourceline,static,super,then,this,until,used,upper,volatile,when,where,while},
sensitive=false,
extendedchars=false,
morecomment=[s]={/*}{*/},
morecomment=[l]{--},
morecomment=[s]{/**}{*/},
morestring=[b]",
morestring=[d]",
morestring=[b]',
morestring=[d]'}

\lstset{language=NetRexx,
  captionpos=t,
  tabsize=3,
  alsolanguage=Rexx,
  keywordstyle=\color{blue},
  commentstyle=\color{cyan},
  stringstyle=\color{red},
  numbers=left,
  numberstyle=\tiny,
  numbersep=5pt,
  breaklines=true,
  showstringspaces=false,
  index=[1][keywords],
  columns=flexible,
  basicstyle=\fontsize{8}{8}\fontspec{Source Code Pro},emph={label}}

\usepackage{../boilerplate/rail}
\usepackage{pst-barcode,pstricks-add}
\usepackage{bashful}
\usepackage{metalogo}
\usepackage{marginnote}
\usepackage{pdfpages}
\usepackage{float}
\hyphenation{Net-Rexx Net-Rexx-A Net-Rexx-C Net-Rexx-R Mac-OSX mac-OS
  infra-structure im-ple-men-ta-tion-de-pen-dent}
\makeindex
\DeclareGraphicsExtensions{.jpg,.png}
\setlength{\parskip}{4pt}
\setlength{\parindent}{0pt}
\usepackage{enumitem}
% defines for consistent use
\newcommand{\nr}{Net\textsc{Rexx}}
\newcommand{\Rexx}{R\textsc{exx}}
\newcommand{\nrpackagename}{\splice{java GetPackageName}}
\newcommand{\minimalJVMversion}{8}
\newcommand{\maximalJVMversion}{19ea}
\newcommand{\keyword}[1]{\texttt{#1}}
\newcommand{\code}[1]{\texttt{#1}}
\newcommand{\thisyear}{\splice{java TexYear}}
\newcommand{\ecjjarname}{ecj-I20201218-1800-NRX.jar}
\newcommand{\msd}[1]{\msdhelper#1\relax}
\newcommand{\msdhelper}[1]
  {\ifx\relax#1\else
    \ifx-#1--{}\else#1\fi
    \expandafter\msdhelper\fi}
  \newcommand{\doublehyphen}{\mbox{\msd{``-~-''}}}
  \newcommand{\doublehyphenunquoted}{\mbox{\msd{-~-}}}
%%% Local Variables: 
%%% mode: latex
%%% TeX-master: t
%%% End: 
