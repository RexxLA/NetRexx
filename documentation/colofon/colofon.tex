 \documentclass[10pt]{book}
\usepackage[FINAL]{../boilerplate/rexx} 
\usepackage{hyperref}
\usepackage{graphics}
\usepackage{fontspec} 
\setmainfont[Mapping=tex-text]{Garamond Premier Pro}
\setmonofont[Mapping=tex-text,Scale=0.85]{Consolas Bold}
\usepackage{tabularx}
\usepackage{booktabs}
\usepackage{makeidx}
\usepackage[all]{xy}
%\usepackage{lingmacros}
\usepackage{color}
\usepackage{xcolor}
\usepackage{listings}
\usepackage{caption}
\usepackage{longtable}
\usepackage{colortbl}
\usepackage{framed}
\usepackage{fancyvrb}
\definecolor{shadecolor}{rgb}{0.9,0.9,0.9}
\usepackage{alltt}
\DeclareCaptionFont{white}{\color{white}}
\DeclareCaptionFormat{listing}{\colorbox{gray}{\parbox{\textwidth}{#1#2#3}}}
\captionsetup[lstlisting]{format=listing,labelfont=white,textfont=white}
\usepackage{listings}
\makeatletter
\lst@CCPutMacro\lst@ProcessOther {"2D}{\lst@ttfamily{-{}}{-{}}}
\@empty\z@\@empty
\makeatother
\lstdefinelanguage{NetRexx}
{morekeywords={abstract,adapter,binary,case,catch,class,constant,dependent,deprecated,digits,do,else,end,engineering,extends,final,finally,for,forever,implements,indirect,import,indirect,inheritable,interface,iterate,label,leave,loop,method,native,nop,numeric,options,otherwise,over,package,parent,parse,private,properties,protect,public,return,returns,rexx,say,scientific,set,digits,form,select,shared,signal,signals,sourceline,static,super,then,this,until,used,upper,volatile,where,while},
sensitive=false,
morecomment=[l]={--},
morecomment=[s]={/**}{*/},
morestring=[b]",
morestring=[d]",
morestring=[b]',
morestring=[d]'}

\lstset{language=NetRexx,
  alsolanguage=Java,
  basicstyle=\small,
  stringstyle=\ttfamily}

\usepackage{../boilerplate/rail}
\usepackage{pst-barcode,pstricks-add}
\hyphenation{Net-Rexx Net-Rexx-A Net-Rexx-C Net-Rexx-R Mac-OSX infra-structure}
\makeindex
\DeclareGraphicsExtensions{.jpg,.png}
\newcommand{\nr}{NetRexx}
\newcommand{\nrversion}{3.03} 
%%% Local Variables: 
%%% mode: latex
%%% TeX-master: t
%%% End: 

\begin{document}    
\renewcommand{\isbn}{978-90-819090-2-0}    
\setcounter{tocdepth}{1} 
\title{Colofon}
\author{Rene Jansen}
\date{Version 1 of \today}
\maketitle
\pagenumbering{Roman}
\pagestyle{plain}
\frontmatter
\pagenumbering{Roman}
\pagestyle{plain}
\input{../boilerplate/bookmeta}
\tableofcontents
\newpage
\pagenumbering{arabic}
\frontmatter
\large
\chapter{The NetRexx Programming Series}
This book is part of a library, the \emph{NetRexx Programming Series}, documenting the NetRexx programming language and its use and applications. This section lists the other publications in this series, and their roles. These books can be ordered in convenient hardcopy and electronic formats from the Rexx Language Association.
\newline
\newline
\newline
\begin{tabularx}{\textwidth}{>{\bfseries}lX}
\toprule
Quick Start Guide & This guide is meant for an audience that has done some programming and wants to start quickly. It starts with a quick tour of the language, and a section on installing the NetRexx translator and how to run it. It also contains help for troubleshooting if anything in the installation does not work as designed, and states current limits and restrictions of the open source reference implementation.
\\\midrule
Programming Guide & The Programming Guide is the one manual that at the same time teaches programming, shows lots of examples as they occur in the real world, and explains about the internals of the translator and how to interface with it.
\\\midrule
Language Reference & Referred to as the NRL, this is the formal definition for the language, documenting its syntax and semantics, and prescribing minimal functionality for language implementors. It is the definitive answer to any question on the language, and as such, is subject to approval of the NetRexx Architecture Review Board on any release of the language (including its NRL).
\\\midrule
NJPipes Reference & The Data Flow oriented companion to NetRexx, with its CMS Pipes compatible syntax, is documented in this manual. It discusses installing and running Pipes for NetRexx, and has ample examples of defining your own stages in NetRexx.
\\\bottomrule
\end{tabularx}
%%% Local Variables: 
%%% mode: latex
%%% TeX-master: t
%%% End: 

\chapter{Typographical conventions}
In general, the following conventions have  been observed in the NetRexx publications:
\begin{itemize}
\item Body text is in this font
\item Examples of language statements are in a \textbf{bold} type
\item Variables or strings as mentioned in source code, or things that appear on the console, are in a \texttt{typewriter} type
\item Items that are introduced, or emphasized, are in an \emph{italic} type
\item Included program fragments are listed in this fashion:
\begin{lstlisting}[label=example,caption=Example Listing]
-- salute the reader
say 'lectorem salutat'
\end{lstlisting}
\item Syntax diagrams take the form of so-called \emph{Railroad Diagrams} to convey structure, mandatory and optional items
\begin{rail}
AggregateExpression : ("AVG" |"MAX" |"MIN" |"SUM")
 (
   (
    'DISTINCT' ?  StateFieldPathExpression
   ) | 'COUNT'
   (
    'DISTINCT' ?  IdentificationVariable
                  | StateFieldPathExpression
                  | SingleValuedAssociationPathExpression
   )
 )
   ;
\end{rail}
%%% Local Variables: 
%%% mode: latex
%%% TeX-master: t
%%% End: 
\end{itemize}
\chapter{Introduction}
The Programming Guide is the book that has the broadest scope of the publications in the \emph{NetRexx Programming Series}. Where the \emph{Language Reference} and the \emph{Quick Beginnings} need to be limited to a formal description and definition of the NetRexx language for the former, and a Quick Tour and Installation instructions for the latter, this book has no such limitations. It teaches programming, discusses computer language history and comparative linguistics, and shows many examples on how to make NetRexx work with diverse techologies as TCP/IP, Relational Database Management Systems, Messaging and Queuing (MQ\textsuperscript{\texttrademark}) systems, J2EE Containers as JBOSS\textsuperscript{\texttrademark} and IBM WebSphere Application Server\textsuperscript{\texttrademark}, discusses various rich- and thin client Graphical User Interface Options, and discusses ways to use NetRexx on various operating platforms. For many people, the best way to learn is from examples instead of from specifications. For this reason this book is rich in example code, all of which is part of the NetRexx distribution, and tested and maintained. This has had its effect on the volume of this book, which means that unlike the other publications in the series, it is probably not a good idea to print it out in its entirety; its size will relegate it to being used electronically.
\section*{Acknowledgements}
As this book is a compendium of decades of Rexx and NetRexx knowledge, it stands upon the shoulders of many of its predecessors, many of which are not available in print anymore in their original form, or will never be upgraded or actualized; we are indebted to many anonymous (because unacknowledged in the original publications) authors of IBM product documentation, and many others that we do know, and will thank in the following. If anyone knows of a name not mentioned here that should be, please be in touch. 

A big IOU goes out to Alan Sampson, who singlehandedly contributed more then one hundred NetRexx programming examples. The Redbook authors (Peter Heuchert, Frederik Haesbrouck, Norio Furukawa, Ueli Wahli) have provided an important document that has shown, in an early stage, how almost everything on the JVM is easier done in NetRexx. Kermit Kiser also provided examples and did maintenance on the translator. If anyone feels their copyright is violated, please do let us know, so we can take out offending passages or modify them beyond recognition. As the usage of all material in this publication is quoted for educational use, and consists of short fragments, a fair use clause will apply in most jurisdictions.


\mainmatter
\chapter{Structure}
\section{Prerequisites}
There arevery few prerequisites for running the book production
toolchain, at least in comparison to other toolchains. All executables
but one are readily available in compiled form on the net. A few
remarks.
\begin{enumerate}
\item The toolchain uses the \textbf{xelatex} version of Latex, because of
its easy integration of platform fonts and its facility to directly
produce PDF files. This program is available in the \emph{livetex}
distribution, to be found at \url{http://}.
\item The make utility (in the GNU version) must be available for the
  platform. All modern platforms have it included, there is a binary
  for Windows downloadable from \url{http://}.
\item For the utility that produces railroad diagrams, there are
  platform versions checked into the \nr source repository
\item to view the generated
\end{enumerate}

\section{The Makefile}
The directory ../netrexx/netrexxc/trunk/documentation/ contains a lot
of files. There is one subdirectory that contains the main files for
this colofon, it is name \textbf{colofon}. So in
../netrexx/netrexxc/trunk/documentation/colofon there are files we are
concentrating on here. The file to look into to find your starting
point is the file \textbf{makefile}. In the makefile the rules for the
make utility are found. Executing the command
\begin{verbatim}
make -B
\end{verbatim}
causes the book to be rebuilt from its sources. Sources are defined as
files ending in .tex, but there are also prerequisites in the form of
files that are pregenerated from *.tex files in previous runs. The
standard makefile re-runs the commands until it has determined that
all prerequisites have been met - this is at least its intention.

The main sourcefile for this book is called colofon.tex . It imports
different files from a ../boilerplate directory, which contains
materials shared by all books in the series. This is seen in the
import and include tags. It includes the file
\textbf{introduction.tex} to show how files are included, but the text
for these chapters the input is in the files colofon.txt itself. How
and why to split input text files is up to the author. The User's
Guide (Quick Start Guide) for example has a lot if imports/includes,
because the original material contained a lot of separate files. The
Programmer's Guide has less files, but uses the \emph{listings}
package a lot to import \nr source files from the examples section. 
\backmatter
\listoffigures
\listoftables
\lstlistoflistings
\printindex
\clearpage
\psset{unit=1in}
\begin{pspicture}(3.5,1in)
  \psbarcode{\isbn}{includetext guardwhitespace}{isbn}
\end{pspicture}
\end{document} 
