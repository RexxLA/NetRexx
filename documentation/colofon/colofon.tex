\documentclass[10pt, openany]{book}
\usepackage[FINAL]{../boilerplate/rexx} 
\usepackage{hyperref}
\usepackage{graphics}
\usepackage{fontspec}
% \fontspec
%      [ Path = /Users/rvjansen/Fonts/,
%        BoldFont       = MinionPro-Bold.otf ,
%        ItalicFont     = MinionPro-It.otf ,
%        BoldItalicFont = MinionPro-BoldIt.otf ]
%      {MinionPro-Regular.otf}
% \fontspec
%      [ Path = /Users/rvjansen/Fonts/,
%        BoldFont       = SourceCodePro-Bold.otf ]
%      {SoureCodePro-Regular.otf}

\setmainfont[Mapping=tex-text]{Minion Pro}
\setmonofont[Mapping=tex-text,Scale=0.80]{Source Code Pro}
\usepackage{tabularx}
\usepackage{booktabs}
\usepackage{makeidx}
\usepackage[all]{xy}
%\usepackage{lingmacros}
\usepackage{color}
\usepackage{xcolor}
\usepackage{listings}
\usepackage{caption}
\usepackage{longtable}
\usepackage{colortbl}
\usepackage{framed}
\usepackage{fancyvrb}
\definecolor{shadecolor}{rgb}{0.9,0.9,0.9}
\usepackage{alltt}
\DeclareCaptionFont{white}{\color{white}}
\DeclareCaptionFormat{listing}{\colorbox{gray}{\parbox{\textwidth}{#1#2#3}}}
\captionsetup[lstlisting]{format=listing,labelfont=white,textfont=white}
\usepackage{listings}
\usepackage[official]{eurosym}
\makeatletter
\lst@CCPutMacro\lst@ProcessOther {"2D}{\lst@ttfamily{-{}}{-{}}}
\@empty\z@\@empty
\makeatother
\lstdefinelanguage{NetRexx}
{morekeywords={abstract,adapter,binary,case,catch,class,constant,dependent,deprecated,digits,do,else,end,engineering,extends,final,finally,for,forever,if,implements,indirect,import,indirect,inheritable,interface,iterate,label,leave,loop,method,native,nop,numeric,options,otherwise,over,package,parent,parse,private,properties,protect,public,return,returns,rexx,say,scientific,set,digits,form,select,shared,signal,signals,sourceline,static,super,then,this,until,used,upper,volatile,when,where,while},
sensitive=false,
extendedchars=false,
morecomment=[s]={/*}{*/},
morecomment=[l]{--},
morecomment=[s]{/**}{*/},
morestring=[b]",
morestring=[d]",
morestring=[b]',
morestring=[d]'}

\lstset{language=NetRexx,
  captionpos=t,
  tabsize=3,
  alsolanguage=Rexx,
  keywordstyle=\color{blue},
  commentstyle=\color{cyan},
  stringstyle=\color{red},
  numbers=left,
  numberstyle=\tiny,
  numbersep=5pt,
  breaklines=true,
  showstringspaces=false,
  index=[1][keywords],
  columns=flexible,
  basicstyle=\fontsize{8}{8}\fontspec{Source Code Pro},emph={label}}

\usepackage{../boilerplate/rail}
\usepackage{pst-barcode,pstricks-add}
\usepackage{bashful}
\usepackage{metalogo}
\usepackage{marginnote}
\usepackage{pdfpages}
\usepackage{float}
\hyphenation{Net-Rexx Net-Rexx-A Net-Rexx-C Net-Rexx-R Mac-OSX
  infra-structure im-ple-men-ta-tion-de-pen-dent}
\makeindex
\DeclareGraphicsExtensions{.jpg,.png}
\setlength{\parskip}{4pt}
\setlength{\parindent}{0pt}
\usepackage{enumitem}
\newcommand{\nr}{Net\textsc{Rexx}}
\newcommand{\Rexx}{R\textsc{exx}}
\newcommand{\nrpackagename}{\splice{java GetPackageName}}
\newcommand{\minimalJVMversion}{1.6}
\newcommand{\keyword}[1]{\texttt{#1}}
\newcommand{\code}[1]{\texttt{#1}}
\newcommand{\thisyear}{\splice{java TexYear}}
\newcommand{\ecjjarname}{ecj-4.6.3.jar}
\newcommand{\msd}[1]{\msdhelper#1\relax}
\newcommand{\msdhelper}[1]
  {\ifx\relax#1\else
    \ifx-#1--{}\else#1\fi
    \expandafter\msdhelper\fi}
  \newcommand{\doublehyphen}{\mbox{\msd{``-~-''}}}
  \newcommand{\doublehyphenunquoted}{\mbox{\msd{-~-}}}
%%% Local Variables: 
%%% mode: latex
%%% TeX-master: t
%%% End: 

\begin{document}    
\renewcommand{\isbn}{978-90-819090-2-0}    
\setcounter{tocdepth}{1} 
\title{Colofon}
\author{Rene Jansen}
\date{Version 1 of \today}
\maketitle
\pagenumbering{Roman}
\pagestyle{plain}
\frontmatter
\pagenumbering{Roman}
\pagestyle{plain}
\section*{Publication Data}
\textcopyright  Copyright The Rexx Language Association, 2011-\splice{java TexYear}
%\\

All original material in this publication is published under the Creative Commons - Share Alike 3.0 License as stated at \url{http://creativecommons.org/licenses/by-nc-sa/3.0/us/legalcode}.\\[0.5cm]
The responsible publisher of this edition is identified as \emph{IBizz IT Services and Consultancy}, Amsteldijk 14, 1074 HR Amsterdam, a registered company governed by the laws of the Kingdom of The Netherlands.\\[1cm]
This edition is registered under ISBN \isbn \\[1cm]
\psset{unit=1in}
\begin{pspicture}(3.5,1in)
  \psbarcode{\isbn}{includetext guardwhitespace}{isbn}
\end{pspicture}
\newpage
%%% Local Variables:
%%% mode: latex
%%% TeX-master: t
%%% End:

\tableofcontents
\newpage
\pagenumbering{arabic}
\frontmatter
\large
\chapter*{\fontspec{IBM Plex Serif}\LARGE The \nr{} Programming Series}
This book is part of a library, the \emph{\nr{} Programming Series}, documenting the \nr{} programming language and its use and applications. This section lists the other publications in this series, and their roles. These books can be ordered in convenient hardcopy and electronic formats from the Rexx Language Association.
\newline
\newline
\begin{tabularx}{\textwidth}{>{\bfseries}lX}
\toprule
%% Quick Start Guide & This guide is meant for an audience that has done some programming and wants to start quickly. It starts with a quick tour of the language, and a section on installing the \nr{} translator and how to run it. It also contains help for troubleshooting if anything in the installation does not work as designed, and states current limits and restrictions of the open source reference implementation.
%% \\\midrule
Programming Guide & The Programming Guide is the one manual that at the same time teaches programming, shows lots of examples as they occur in the real world, and explains about the internals of the translator and how to interface with it.
\\\midrule
Language Reference & Referred to as the NRL, this is meant as the formal definition for the language, documenting its syntax and semantics, and prescribing minimal functionality for language implementers.
\\\midrule
Pipelines Guide \& Reference & The Data Flow oriented companion to \nr{}, with its CMS Pipelines compatible syntax, is documented in this manual. It discusses running Pipes for \nr{} in the command shell and the Workspace, and has ample examples of defining your own stages in \nr{}.
\\\bottomrule
\end{tabularx}
%%% Local Variables: 
%%% mode: latex
%%% TeX-master: t
%%% End: 

\chapter{Typographical conventions}
In general, the following conventions have  been observed in the NetRexx publications:
\begin{itemize}
\item Body text is in this font
\item Examples of language statements are in a \textbf{bold} type
\item Variables or strings as mentioned in source code, or things that appear on the console, are in a \texttt{typewriter} type
\item Items that are introduced, or emphasized, are in an \emph{italic} type
\item Included program fragments are listed in this fashion:
\begin{lstlisting}[label=example,caption=Example Listing]
-- salute the reader
say 'lectorem salutat'
\end{lstlisting}
\item Syntax diagrams take the form of so-called \emph{Railroad Diagrams} to convey structure, mandatory and optional items
\begin{rail}
AggregateExpression : ("AVG" |"MAX" |"MIN" |"SUM")
 (
   (
    'DISTINCT' ?  StateFieldPathExpression
   ) | 'COUNT'
   (
    'DISTINCT' ?  IdentificationVariable
                  | StateFieldPathExpression
                  | SingleValuedAssociationPathExpression
   )
 )
   ;
\end{rail}
%%% Local Variables: 
%%% mode: latex
%%% TeX-master: t
%%% End: 
\end{itemize}
\chapter{Introduction}
This document is the \emph{Quick Start Guide} for the reference implementation of
\nr{}. \nr{} is a \emph{human-oriented} programming language which makes
writing and using Java\footnote{Java is a trademark of Oracle, Inc.}
classes quicker and easier than writing in Java. It is part of the Rexx
language family, under the governance of the Rexx Language
Association.\footnote{\url{http.www.rexxla.org}} \nr{} has been
developed and was made available as a free download by IBM since 1995
and is free and open source since June 8, 2011.

In this Quick Start Guide, you’ll find information on
\begin{enumerate} 
\item How easy it is to write for the JVM: A Quick Tour of \nr{}
\item Installing \nr{} 
\item Using the \nr{} translator as a compiler, interpreter, or
  syntax checker 
\item Troubleshooting when things do not work as expected
\item Current restrictions.
\end{enumerate} 
The \nr{} documentation and software are distributed
by The Rexx Language Association under the \textsc{ICU} license. For
the terms of this license, see the included \textsc{LICENSE} file in
this package.

For details of the \nr{} language, and the latest news, downloads,
etc., please see the \nr{} documentation included with the package
or available at: \url{http://www.netrexx.org}.

\begin{shaded}\noindent
The highest Java version that is supported in this version, 3.09, is
\emph{Java 8}. Higher versions are not yet supported due to changes in
Java, including incompatibilities introduced with the Java module system.
\end{shaded}\indent

\mainmatter
\chapter{Structure}
\section{Prerequisites}
There arevery few prerequisites for running the book production
toolchain, at least in comparison to other toolchains. All executables
but one are readily available in compiled form on the net. A few
remarks.
\begin{enumerate}
\item The toolchain uses the \textbf{xelatex} version of Latex, because of
its easy integration of platform fonts and its facility to directly
produce PDF files. This program is available in the \emph{livetex}
distribution, to be found at \url{http://}.
\item The make utility (in the GNU version) must be available for the
  platform. All modern platforms have it included, there is a binary
  for Windows downloadable from \url{http://}.
\item For the utility that produces railroad diagrams, there are
  platform versions checked into the \nr source repository
\item to view the generated
\end{enumerate}

\section{The Makefile}
The directory ../netrexx/netrexxc/trunk/documentation/ contains a lot
of files. There is one subdirectory that contains the main files for
this colofon, it is name \textbf{colofon}. So in
../netrexx/netrexxc/trunk/documentation/colofon there are files we are
concentrating on here. The file to look into to find your starting
point is the file \textbf{makefile}. In the makefile the rules for the
make utility are found. Executing the command
\begin{verbatim}
make -B
\end{verbatim}
causes the book to be rebuilt from its sources. Sources are defined as
files ending in .tex, but there are also prerequisites in the form of
files that are pregenerated from *.tex files in previous runs. The
standard makefile re-runs the commands until it has determined that
all prerequisites have been met - this is at least its intention.

The main sourcefile for this book is called colofon.tex . It imports
different files from a ../boilerplate directory, which contains
materials shared by all books in the series. This is seen in the
import and include tags. It includes the file
\textbf{introduction.tex} to show how files are included, but the text
for these chapters the input is in the files colofon.txt itself. How
and why to split input text files is up to the author. The User's
Guide (Quick Start Guide) for example has a lot if imports/includes,
because the original material contained a lot of separate files. The
Programmer's Guide has less files, but uses the \emph{listings}
package a lot to import \nr source files from the examples section. 
\backmatter
\listoffigures
\listoftables
\lstlistoflistings
\printindex
\clearpage
\psset{unit=1in}
\begin{pspicture}(3.5,1in)
  \psbarcode{\isbn}{includetext guardwhitespace}{isbn}
\end{pspicture}
\end{document} 
