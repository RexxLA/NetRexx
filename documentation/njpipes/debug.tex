\chapter{Debugging Pipelines}\label{debugging-pipelines}

To find out whhat is happening in a pipeline, you can specify debug
options.

\begin{longtable}[]{@{}
  >{\raggedright\arraybackslash}p{(\linewidth - 4\tabcolsep) * \real{0.3333}}
  >{\raggedright\arraybackslash}p{(\linewidth - 4\tabcolsep) * \real{0.3333}}
  >{\raggedright\arraybackslash}p{(\linewidth - 4\tabcolsep) * \real{0.3333}}@{}}
\toprule\noalign{}
\begin{minipage}[b]{\linewidth}\raggedright
Option
\end{minipage} & \begin{minipage}[b]{\linewidth}\raggedright
Effect
\end{minipage} & \begin{minipage}[b]{\linewidth}\raggedright
\end{minipage} \\
\midrule\noalign{}
\endhead
\bottomrule\noalign{}
\endlastfoot
1 & Show all pipes starting & \\
2 & Show all pipes ending & \\
4 & Show all stages starting & \\
8 & Show all stages stopping & \\
16 & Show all Commit requests & \\
32 & Show all Commit completions & \\
64 & Show StageErrors raised via stage's Error(int,String)
method.\footnote{The stage class uses Error for all its StageError
  signals} & \\
128 & Show the argument that each stage is receiving.\footnote{Handy
  since shells have a habit of doing unexpected thing to arguments.
  (try: java findtext exit \emph{.nrx vs java findtext ``exit }.nrx'')}
& \\
\end{longtable}

To create a flag to see all stages starting and stopping you would add
8+4 and use:

\begin{verbatim}
- pipe (apipe debug 12) ...
\end{verbatim}

\section{The dump() stage}\label{the-dump-stage}

The second option is to use the invoke the dump() method in a stage.
This dumps the status of the pipe using the same format you see when a
pipe deadlocks. Using dump() does not normally cause a pipe to
terminate. Once in a while dump() will generate an exception. This
happens since dump() does not use protect or synchronize so it does not
stall.
