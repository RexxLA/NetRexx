\documentclass[10pt, openany]{book}
\usepackage[FINAL]{../boilerplate/rexx} 
\usepackage{hyperref}
\usepackage{graphics}
\usepackage{fontspec}
% \fontspec
%      [ Path = /Users/rvjansen/Fonts/,
%        BoldFont       = MinionPro-Bold.otf ,
%        ItalicFont     = MinionPro-It.otf ,
%        BoldItalicFont = MinionPro-BoldIt.otf ]
%      {MinionPro-Regular.otf}
% \fontspec
%      [ Path = /Users/rvjansen/Fonts/,
%        BoldFont       = SourceCodePro-Bold.otf ]
%      {SoureCodePro-Regular.otf}

\setmainfont[Mapping=tex-text]{Minion Pro}
\setmonofont[Mapping=tex-text,Scale=0.80]{Source Code Pro}
\usepackage{tabularx}
\usepackage{booktabs}
\usepackage{makeidx}
\usepackage[all]{xy}
%\usepackage{lingmacros}
\usepackage{color}
\usepackage{xcolor}
\usepackage{listings}
\usepackage{caption}
\usepackage{longtable}
\usepackage{colortbl}
\usepackage{framed}
\usepackage{fancyvrb}
\definecolor{shadecolor}{rgb}{0.9,0.9,0.9}
\usepackage{alltt}
\DeclareCaptionFont{white}{\color{white}}
\DeclareCaptionFormat{listing}{\colorbox{gray}{\parbox{\textwidth}{#1#2#3}}}
\captionsetup[lstlisting]{format=listing,labelfont=white,textfont=white}
\usepackage{listings}
\usepackage[official]{eurosym}
\makeatletter
\lst@CCPutMacro\lst@ProcessOther {"2D}{\lst@ttfamily{-{}}{-{}}}
\@empty\z@\@empty
\makeatother
\lstdefinelanguage{NetRexx}
{morekeywords={abstract,adapter,binary,case,catch,class,constant,dependent,deprecated,digits,do,else,end,engineering,extends,final,finally,for,forever,if,implements,indirect,import,indirect,inheritable,interface,iterate,label,leave,loop,method,native,nop,numeric,options,otherwise,over,package,parent,parse,private,properties,protect,public,return,returns,rexx,say,scientific,set,digits,form,select,shared,signal,signals,sourceline,static,super,then,this,until,used,upper,volatile,when,where,while},
sensitive=false,
extendedchars=false,
morecomment=[s]={/*}{*/},
morecomment=[l]{--},
morecomment=[s]{/**}{*/},
morestring=[b]",
morestring=[d]",
morestring=[b]',
morestring=[d]'}

\lstset{language=NetRexx,
  captionpos=t,
  tabsize=3,
  alsolanguage=Rexx,
  keywordstyle=\color{blue},
  commentstyle=\color{cyan},
  stringstyle=\color{red},
  numbers=left,
  numberstyle=\tiny,
  numbersep=5pt,
  breaklines=true,
  showstringspaces=false,
  index=[1][keywords],
  columns=flexible,
  basicstyle=\fontsize{8}{8}\fontspec{Source Code Pro},emph={label}}

\usepackage{../boilerplate/rail}
\usepackage{pst-barcode,pstricks-add}
\usepackage{bashful}
\usepackage{metalogo}
\usepackage{marginnote}
\usepackage{pdfpages}
\usepackage{float}
\hyphenation{Net-Rexx Net-Rexx-A Net-Rexx-C Net-Rexx-R Mac-OSX
  infra-structure im-ple-men-ta-tion-de-pen-dent}
\makeindex
\DeclareGraphicsExtensions{.jpg,.png}
\setlength{\parskip}{4pt}
\setlength{\parindent}{0pt}
\usepackage{enumitem}
\newcommand{\nr}{Net\textsc{Rexx}}
\newcommand{\Rexx}{R\textsc{exx}}
\newcommand{\nrpackagename}{\splice{java GetPackageName}}
\newcommand{\minimalJVMversion}{1.6}
\newcommand{\keyword}[1]{\texttt{#1}}
\newcommand{\code}[1]{\texttt{#1}}
\newcommand{\thisyear}{\splice{java TexYear}}
\newcommand{\ecjjarname}{ecj-4.6.3.jar}
\newcommand{\msd}[1]{\msdhelper#1\relax}
\newcommand{\msdhelper}[1]
  {\ifx\relax#1\else
    \ifx-#1--{}\else#1\fi
    \expandafter\msdhelper\fi}
  \newcommand{\doublehyphen}{\mbox{\msd{``-~-''}}}
  \newcommand{\doublehyphenunquoted}{\mbox{\msd{-~-}}}
%%% Local Variables: 
%%% mode: latex
%%% TeX-master: t
%%% End: 

\begin{document}
\renewcommand{\isbn}{978-90-819090-3-7}    
\setcounter{tocdepth}{1} 
\title{\fontspec{Bodoni URW Light}Pipelines Reference}
\author{Ed Tomlinson \and Jeff Hennick \and Rene Jansen}
\date{Version \splice{java org.netrexx.process.NrVersion} of \today}
\maketitle
\pagenumbering{Roman}
\pagestyle{plain}
\frontmatter
\pagenumbering{Roman}
\pagestyle{plain}
\section*{Publication Data}
\textcopyright  Copyright The Rexx Language Association, 2011-\splice{java TexYear}
%\\

All original material in this publication is published under the Creative Commons - Share Alike 3.0 License as stated at \url{http://creativecommons.org/licenses/by-nc-sa/3.0/us/legalcode}.\\[0.5cm]
The responsible publisher of this edition is identified as \emph{IBizz IT Services and Consultancy}, Amsteldijk 14, 1074 HR Amsterdam, a registered company governed by the laws of the Kingdom of The Netherlands.\\[1cm]
This edition is registered under ISBN \isbn \\[1cm]
\psset{unit=1in}
\begin{pspicture}(3.5,1in)
  \psbarcode{\isbn}{includetext guardwhitespace}{isbn}
\end{pspicture}
\newpage
%%% Local Variables:
%%% mode: latex
%%% TeX-master: t
%%% End:

\tableofcontents
\newpage
\pagenumbering{arabic}
\frontmatter
\large
\chapter*{\fontspec{IBM Plex Serif}\LARGE The \nr{} Programming Series}
This book is part of a library, the \emph{\nr{} Programming Series}, documenting the \nr{} programming language and its use and applications. This section lists the other publications in this series, and their roles. These books can be ordered in convenient hardcopy and electronic formats from the Rexx Language Association.
\newline
\newline
\begin{tabularx}{\textwidth}{>{\bfseries}lX}
\toprule
%% Quick Start Guide & This guide is meant for an audience that has done some programming and wants to start quickly. It starts with a quick tour of the language, and a section on installing the \nr{} translator and how to run it. It also contains help for troubleshooting if anything in the installation does not work as designed, and states current limits and restrictions of the open source reference implementation.
%% \\\midrule
Programming Guide & The Programming Guide is the one manual that at the same time teaches programming, shows lots of examples as they occur in the real world, and explains about the internals of the translator and how to interface with it.
\\\midrule
Language Reference & Referred to as the NRL, this is meant as the formal definition for the language, documenting its syntax and semantics, and prescribing minimal functionality for language implementers.
\\\midrule
Pipelines Guide \& Reference & The Data Flow oriented companion to \nr{}, with its CMS Pipelines compatible syntax, is documented in this manual. It discusses running Pipes for \nr{} in the command shell and the Workspace, and has ample examples of defining your own stages in \nr{}.
\\\bottomrule
\end{tabularx}
%%% Local Variables: 
%%% mode: latex
%%% TeX-master: t
%%% End: 

\chapter{Typographical conventions}
In general, the following conventions have  been observed in the NetRexx publications:
\begin{itemize}
\item Body text is in this font
\item Examples of language statements are in a \textbf{bold} type
\item Variables or strings as mentioned in source code, or things that appear on the console, are in a \texttt{typewriter} type
\item Items that are introduced, or emphasized, are in an \emph{italic} type
\item Included program fragments are listed in this fashion:
\begin{lstlisting}[label=example,caption=Example Listing]
-- salute the reader
say 'lectorem salutat'
\end{lstlisting}
\item Syntax diagrams take the form of so-called \emph{Railroad Diagrams} to convey structure, mandatory and optional items
\begin{rail}
AggregateExpression : ("AVG" |"MAX" |"MIN" |"SUM")
 (
   (
    'DISTINCT' ?  StateFieldPathExpression
   ) | 'COUNT'
   (
    'DISTINCT' ?  IdentificationVariable
                  | StateFieldPathExpression
                  | SingleValuedAssociationPathExpression
   )
 )
   ;
\end{rail}
%%% Local Variables: 
%%% mode: latex
%%% TeX-master: t
%%% End: 
\end{itemize}
\mainmatter
\chapter{Introduction}
A Pipeline, or Hartmann Pipeline\footnote{\url{http://en.wikipedia.org/wiki/Hartmann_pipeline}}, is a concept that extends and improves pipes as they are known from Unix and other operating systems. The name pipe indicates an interprocess communication mechanism, as well as the programming paradigm it has introduced. Compared to Unix pipes, Hartmann Pipelines offer multiple input- and output streams, more complex pipe topologies, and a lot more.\\

Pipelines were first implemented on VM/CMS, one of IBM's mainframe
operating systems. This version was later ported to TSO to run under
MVS and has been part of several product configurations. Pipelines are
widely used by VM users, in a symbiotic relationship with REXX, the
interpreted language that also has its origins on this platform.

Pipes for NetRexx is the implementation of Pipelines for the Java Virtual
machine.\footnote{And indeed, it can also be used from programs written
  in the Java Language.} It is written in NetRexx and pipes and stages can be defined using this
language. The resulting code can run on every platform that has a JVM
(Java Virtual Machine) installed. This portable version of Pipelines was started
by Ed Tomlinson in 1997 under the name of \emph{njPipes}, when NetRexx was
still very new, and was open sourced in 2011, soon after the NetRexx
translator itself. The included stages have always been open source. It was integrated into the NetRexx translator in
2014 and first released with version 3.04.

\chapter{The Pipeline Concept}
\section{What is a pipeline?}
The \emph{pipeline} terminology is a metaphore derived from
plumbing. Fitting two or more pipe segments together yield a
pipeline. Water flows in one direction through the pipeline.

There is a source, which could be a well or a water tower; water is
pumped through the pipe into the first segment, then through the other
segments until it reaches a tap, and most of it will end up in the
sink. A pipeline can be increased in length with more segments of
pipe, and this illustrates the modular concept of the pipeline.

When we discuss pipelines in relation to computing we have the same
basic structure, but instead of water that passes through the
pipeline, data is passed through a series of programs (\emph{stages})
that act as filters.

Data must come from some place and go to some place. Analogous to the
well or the water tower there are \emph{device drivers} that act as a
source of the data, where the tap or the \emph{sink} represents the place the
data is going to, for example to some output device as your terminal
window or a file on disk, or a network destination.

Just as water, data in a pipeline flows in one direction, by
convention from the left to the right.
\section{Stage}
A program that runs in a pipeline is called a \emph{stage}. A program
can run in more than one place in a pipeline - these occurrences
function independent of each other. 

The pipeline specification is processed by the \emph{pipeline
  compiler}, and it must be contained in a character string; on the
commandline, it needs to be between quotes, while when contained in a
file, it needs to be between the delimiters of a \nr{} string. An
exclamation mark (!) is used as \emph{stage separator}, while the
solid vertical bar | can be used as an option when specifiying the
local option for the pipe, after the pipe name.

When looking a two adjaced segments in a pipeline, we call the left
stage the \emph{producer} and the stage on the right the
\emph{consumer}, with the \emph{stage separator} as the connector.
\section{Device Driver}
A \emph{device driver} reads from a device (for instance a file, the
command prompt, a machine console or a network connection) or writes
to a device; in some cases it can both read and write. An example of a
device drivers are \texttt{diskr} for diskread and \texttt{diskw} for
diskwrite; these read and write data from and to files.

A pipeline can take data from one input device and write it to a
different device. Within the pipeline, data can be modified in almost
any way imaginable by the programmer.

The simplest process for the pipeline is to read data from the input
side and copy it unmodified to the output side. Figure X shows the
currently supported input- and output devices. The pipeline compiler
connects these programs; it uses one program for each device and
connects them together.

The inherent characteristic of the pipeline is that any program can be
connected to any other program because each obtains data and sends
data throug a device independent standard interface.

The pipeline usually processes one record (or line) at a time. The
pipeline reads a record for the input, processes it and sends it to
the output. It continues until the input source is drained.

\section{Building the pipeline}
Until now everything was just theory, but now we are going to show how
to compile and run a pipeline. The executable script \texttt{pipe} is
included in the \nr{} distribution to specify a pipeline and to compile
\nr{} source that contains pipelines. Pipelines can be specified on
the command line or in a file, but will always be compiled to a .class
file for execution in the JVM. In this case, we have the following
pipe definition in a file called \code{firstsample.njp}. We tell the
pipe compiler that the pipe is called \code{hello}. 
\lstinputlisting[label=firstpipe,caption=Hello World]{firstsample.njp} 

This specifies a pipeline consisting of a source stage
\texttt{literal} that puts a string (``hello world'') into the pipeline, and
a \texttt{console} sink, that puts the string on the screen. 

We compile this pipe with the command:
\begin{verbatim}
pipe firstsample
\end{verbatim} 
\bash[stdout]
pipe firstsample
\END
The pipe
compiler will echo the source of the pipe to the screen - or issue
messages when something was mistyped. The name of the classfile is the
name of the pipe, here specified between parentheses. Options also go there.

We call execute the pipe by typing:
\begin{verbatim}
java hello
\end{verbatim}
Its output is:
\bash[stdout]
java hello
\END
Now we
have shown the obligatory example, we can make it more interesting by
adding a \texttt{reverse} stage in between:
\lstinputlisting[label=secondpipe,caption=Hello World 2]{secondsample.njp} 
When this is executed, it dutifully types 
\bash[stdout]
pipe secondsample
java hello
\END

If we replace the string after \texttt{literal} with \texttt{arg()},
we then can start the \texttt{hello} pipeline with a an argument to
reverse:

\begin{lstlisting}[label=hellopipes3,caption=Hello World3]
pipe "(hello) literal arg() ! reverse ! console"
\end{lstlisting}
 and we run it with:
\begin{verbatim}
java hello a man a plan a canal panama
\end{verbatim}
and it will respond:
\begin{verbatim}
amanap lanac a nalp a nam a
\end{verbatim}
which goes to show that without ignoring space no palindrome is very
convincing - which we can remedy with a change to the pipeline: use the
\texttt{change} stage to take out the spaces:
\begin{lstlisting}[label=hellopipes4,caption=Hello World4]
pipe "(hello) literal arg() ! change /" "// ! console"
\end{lstlisting}

\chapter{Device Drivers}

A device driver reads from a device (for instance a file on disk, on
the network, or from the shell) or writes to a device; in some cases
it can both read from and write to the device. An example of a device
driver is \keyword{disk} which reads and writes files. You can use it
on the left-hand side of the pipeline as an input device and on the
right-hand side of the pipeline as an output device.


A pipeline can take data from one input device and write it to another
device. Within the pipeline, data can be modified in may ways.
The simplest process for the pipeline is to copy data from a device on
the input side to a device on the output side. 

Using any input driver with any output driver, Pipelines for NetRexx moves data directly from the input device to the output device. The pipeline implementation uses one program for each device and connects them together.
The inherent characteristic of the pipeline is that any program can be connected to any other one because each obtains data and sends data through a device independent standard interface.

The pipeline usually processes one record (or line) at a time. The pipeline reads a record from the input, processes it and sends it to the output. It continues until all records from the input have been processed.


\chapter{Filters}
\chapter{Record Selection}


\chapter{NetRexx Pipelines Implementation}
\nr{} Pipelines enables us to follow the usage model of CMS Pipelines
closely; in fact, the documentation for the mainframe product can be
used for most stages.

\section{Installation and verification}
To run \nr{} Pipelines a running \nr{} installation is needed.  To write your own pipes or stages you need compilers for both Java and NetRexx. 
The core classes for pipes and stages are in the archive \nr{}F.jar.  This file may
be used on the -cp option or added to your CLASSPATH, as indicated in
the \emph{NetRexx Quickstart Guide}.

To test the installation, we can run a pipeline from the command line.
Running a pipeline from the command line
 To run a pipeline from the commandline, type:
\begin{lstlisting}[label=countwords,caption=Count Words]
pipe (test) 'literal arg() ! dup 999 ! count words ! console'
\end{lstlisting}

The first time you use the pipe command in a new directory it will
create a default pipes.cnf file for you. When the \texttt{pipe}
command is not on your path, you can also use:

\begin{lstlisting}[label=testdup,caption=Test Dup]
java org.netrexx.njpipes.pipes.compiler (test) 'literal arg() ! dup 999 ! count words ! console'
\end{lstlisting}

You should see a message that the pipe compiler is processing your pipe and soon after that messages from the NetRexx compiler as it processes the pipe.
To run the pipe type:
\begin{verbatim}
java test some words
\end{verbatim}
The pipe should then output:
\begin{verbatim}
 2000
\end{verbatim}
\chapter{Differences with respect to the CMS Pipelines version}

\chapter{Developing stages in NetRexx}
Writing your own pipes or stage is simple.  Take a look at the source of the supplied stages in the stages directory.  Here are some more examples.  The first shows how to use a pipe in a NetRexx program:
    -- examples/testpipe.njp
    -- to compile use: pipe testpipe.njp
    --             or: java njp testpipe.njp
    -- to execute use: java testpipe some text
\begin{lstlisting}
 class testpipe

    method testpipe(avar=Rexx)

       F = Rexx 'abase'
       T = Rexx 1

       F[0]=5
       F[1]=222
       F[2]=3333
       F[3]=1111
       F[4]=55
       F[5]=444

       pipe (apipe stall 1000 )
         stem F ! sort ! prefix literal {avar} ! console ! stem T

       loop i=1 to T[0]
          say 'T['i']='T[i]
       end

    method main(a=String[]) static

       testpipe(Rexx(a))
\end{lstlisting}
 A couple things can be inferred from this example.  First its simple to pass rexx variables to pipes using STEM.  Also look at the  phrase {avar}. It passes the Rexx variable's value to the stage at runtime.  In CMS the pipe would be quoted and you would unquote sections to get a similiar effect.
Another thing to note is that the pipe extraction program is fairly smart. It detects when pipes takes several lines.  As long as there are stages, or the current line ends with a stagesep or stageend character, or the next line starts with a stagesep or stageend character.  It gets added to the pipe.
The arg(), arg(rexx) or arg(null) methods get the arguments passed to a stage or pipe.  To get the complete rexx string of an argument use arg(). To get the nth word of a rexx argument use arg(n).  When using pipes in netrexx or java code you can use arg('name') to get the named argument. If the class of the arguement is not rexx use arg(null) to get the object.
In .njp files you can use {avar} phrase actually just shorthand for  arg('avar').
The following example shows what has to be done in a stage to access the rexx variables passed by VAR, STEM and OVER.  The real  over stage is a bit more complete.
\begin{lstlisting}
    -- over.nrx
 class over extends stage final

    method run() public
        a = getRexx(arg())
      loop i over a
         output(a[i])
      catch StageError
         rc = rc()
      end
    
   exit(rc*(rc<>12))
\end{lstlisting}
 The getRexx method is passed the name of a string by the pipe.  In the previous example it would be passed A and would return an Object pointer to A in testpipe. If you wish to replace a stream this can be done using connectors.  For example look at the following fragment:
\begin{verbatim}
    -- examples\calltest.njp
    pipe (callt1) literal test ! calltest {} ! console
\end{verbatim}
\begin{lstlisting}
    import org.netrexx.njpipes.pipes.

    class calltest extends stage final

    method run() public

       do
          a = arg()

          callpipe (cp1) gen {a} ! *out0:

          loop forever
             line = peekto()
             output(line)
             readto()
          end

       catch StageError
          rc = rc()
       end

    exit(rc*(rc<>12))
\end{lstlisting}
Running the callt1 pipe with an argument of 10 would pass the 10 to calltest via {} and arg().  Then cp1's gen stage would be passed 'a' which is set to 10.  Since gen generate numbers in sequence, the console stage of callt1 would get the numbers from 1 to 10.  Now cp1 ends and calltest's output stream is restored and calltest unblocks and reads the the literal's data 'test' and passes it to console.

The use of {} only works when compiling from .njp files.  It will not work from the command line.
The njpipes compiler recognizes connectors as labels with the following forms:
\begin{verbatim}
    *in:
   *inN:
   *out:
   *outN
\end{verbatim}

When N is a whole number, the connector connects input or output stream N of the stage with the connector.
When the label *in or *out, the connector connects the stages's current input or output stream with the connector.  This is used instead of *: due to the way the compiler/preprocessor works.
If you do not want the stage to wait for the called pipe to complete you can use addpipe.  Here is an example.
\begin{lstlisting}
    -- similar to examples\addtest.njp

    a  = 100
    b  = 'some text for literal'

    addpipe (linktest) literal {b} ! dup {a} ! *in0:

    loop forever
       line = Rexx readto()
    catch StageError
    end
\end{lstlisting}
    readto() will get 'some text for literal' one hundred times.

A quick aside.  When writing stages remember that njPipes moves objects through pipes.  Use 'value = peekto()' instead of 'value = rexx peekto()' when ever possible.  Some of the supplied stages pass objects with classes other than rexx and forcing rexx will cause classCastExceptions. If a stage needs a rexx object try using the rexx stage modifier to attempt to convert the object.  Feel free to expand this stage, but please send me the updated version.

Serious stage writers will probably want to take a good look at the
methods defined in the \nr{} source package \texttt{org.netrexx.process.njpipes.stages}.  There you will find various methods for parsing ranges.  You will also find the stub for the stageExit compiler exit.  It can be used to produce 'on the fly' code at compile time.  You can also use it to change the topology of the unprocessed part of the pipe.  The major use is to allow implementations of stages like prefix, append or zone.  Its also used to produce better performing stages, for an example see specs.
The compiler also queries the rexxArg() and stageArg() methods.  If your stage expects objects of class Rexx as arguments rexxArg() should return the number of variables expected.  If your stage expects a stage for an argument, stageArg() should return the word position of the stage.

To get a good idea of what can be done with pipes look at the tasktest pipe in the examples directory.  It, using code from Melinda Varins 'Cramming for the Journeyman Plumber Exam' paper,  implements the shell of a  multitasking server - using about eight stages.  The file examples/tcptask.njp contains an example of this technique being used.

\chapter{Deadlocks}
Pipes for NetRexx and Java detects deadlocks and outputs information to allow you to fix the problem.  Consider the following session:
\begin{verbatim}
    java njp (deadlock) literal test ! a: fanin ! console ! a:
    Pipes for NetRexx and Java version 0.33
    Copyright (c) E. J. Tomlinson , 1998.  All rights reserved.
    Building pipe deadlock
    NetRexx portable processor, version 1.140
    Copyright (c) IBM Corporation, 1998.  All rights reserved.
    Program deadlock.nrx
    Compilation of 'deadlock.nrx' successful

    njpipes/examples]java -nojit deadlock
    test
    Deadlocked in deadlock

    Dumping deadlock Monitored by deadlock

     Flag units digit:  1=wait out, 2=wait in, 4=wait any, 8=wait commit
                     : 10=pending autocommit, 20=pending sever

     literal_1
     Running rc=0 commit=-1 Flag=201
     -> out 0 fanin_2 1 test

     fanin_2
     Running rc=0 commit=-1 Flag=101
     -> in  0 literal\_1 1 test
        in  1 console\_3 0 test
     -> out 0 console\_3 1 test

     console_3
     Running rc=0 commit=-1 Flag=101
     -> in  0 fanin\_2 1 test
     -> out 0 fanin\_2 0 test

    Dumped Pipe deadlock Flag 40F rc=16

    RC=16
\end{verbatim}

We can see that there are three stages Running.  None have any return codes set.  The Flags tell us that all the stages are waiting for an output to complete.  The '->' show which stream is selected.  From this we can see console\_3 is trying to output to fanin\_2. Unfortunately fanin\_2 is waiting for output on stream 0 to complete, it cannot read the datawaiting on in stream 1.  Hence the stall.
When a stream has data being output, there is a boolean flag following the name of the stage the stream is connected to. This tracks the peek state of the object.  For an output stream, true means the following stage has peeked at the value. With input streams, the current stage has seen the value when its true.
When a stage is multithreaded, like elastic, you can get flags of 3 or 5. This means that threads are waiting on output and read, or output and any. When using multithreaded stages, only one thread should use output unless it is serialized using protected or syncronized blocks.
When a stage has a pending sever or autocommit flag bits are set too.

\chapter{The Pipe Compiler}

% \chapter{Built-in Stages}
% This section describes the set of built in stages, i.e. the ones that are delivered with the downloadable open source package. These stages are directly executable from the njpipesC.jar file; also, the source of these stages is delivered in the package.
% General notes on the built-in stages:
% \begin{enumerate}
% \item The underlying technology, the JVM, and the chosen implementation language, NetRexx, cause the character representation to be Unicode.
% \item Most of the stages expect the objects in the pipeline to be of type Rexx
% \end{enumerate}
\chapter{Stages}
\section{abbrev}
\index{abbrev}
\begin{shaded}
\begin{rail}
  abbrev : 'abbrev'( word?|number?)'ANYCASE'?
  ;
\end{rail}
\end{shaded}

\textbf{Type:} Selection Stage

\textbf{Syntax Description:} A word, a number and a keyword are
optional.

The word specifies the characters to compare against the beginning of
input records. The default is a null word. The number specifies the
minimum count of characters that must be present to select the
record. The default is zero, which means that any abbreviation down to
a null record or a leading blank will be selected. Specify ANYCASE to
make the comparison case insensitive. 

\textbf{Operation:}  \emph{abbrev} compares the leading columns of
each record against the specified word until a blank or the end of the
record is met. The record is passed to the primary output stream if a
minimum abbreviation of the specified word is present. Otherwise, the
record is discarded (or passed to the secondary output stream if the
secondary output stream is connected).

\textbf{Example}:
\lstinputlisting[label=abbrev1,caption=abbrev example]{abbrevsample.njp} 
\bash[stdout]
pipc abbrevsample
\END
\bash[stdout]
java abbrev1
\END

\section{append}
\index{append}
\begin{shaded}
\begin{rail}
  append : 'append' string
  ;
\end{rail}
\end{shaded}

\textbf{Type:} Control

\textbf{Syntax Description:} The argument string is normally a single stage, but any pipeline
specification that can be suffixed by a connector (|*:) is acceptable.

\textbf{Operation:} All records on the primary input stream are copied
to the primary output stream. Then the string is issued as a
subroutine pipeline with CALLPIPE, using the default stage separator
(|), double quotes as the escape character ("), and the backward slash
as the end character. The beginning of the pipeline is
unconnected. The end of the pipeline is connected to append’s primary
output stream. (Do not write an explicit connector.)

\textbf{Example:}

\lstinputlisting[label=append1,caption=append example]{appendsample.njp} 
\bash[stdout]
pipc appendsample
\END
\bash[stdout]
java appendtest
\END



\section{between}
\index{between}
\begin{shaded}
\begin{rail}
  between : 'between' 'ANYcase'? delimitedString ( delimitedString?|number?)
  ;
\end{rail}
\end{shaded}

\textbf{Type:} Selection Stage

\textbf{Syntax Description:}


\textbf{Operation:}  \emph{between}

% \textbf{Example}:
% \lstinputlisting[label=between1,caption=between example]{betweensample.njp} 
% \bash[stdout]
% pipe betweensample
% \END
% \bash[stdout]
% java between1
% \END

\section{buffer}
\index{buffer}
\begin{shaded}
\begin{alltt}
/** buffer


   >>--BUFFER--+-----------------------------+----------------------------><
               +-number--+-----------------+-+
                         +-delimitedString-+

\end{alltt}
\end{shaded}

\section{casei}
\index{casei}
\begin{shaded}
\begin{alltt}
\end{alltt}
\end{shaded}

\section{change}
\index{change}
\begin{shaded}
\begin{alltt}
/** change

  >>--CHANGE--+---------+--+-----------------+---------------------------->
              +-ANYcase-+  +-inputRange------+
                           |    <-------+    |
                           +-(----range-+--)-+

  >--+-+ changeString +-----------------+--+-----------+-----------------><
     +-delimitedString--delimitedString-+  +-numorstar-+

  changeString:
  +--delimiter--string--delimiter--string--delimiter----------------------+

\end{alltt}
\end{shaded}

\section{chop}
\index{chop}
\begin{shaded}
\begin{alltt}
/** chop

                     +-80-----+
   >>--+-CHOP-----+--+--------+-------------------------------------------><
       +-TRUNCate-+  +-number-+



   >>--CHOP--+---------+--+-------------------------+--+-----+------------->
             +-ANYCase-+  |              +-BEFORE-+ |  +-NOT-+
                          +-+---------+--+--------+-+
                            +-snumber-+  +-AFTER--+

   >--+ target +----------------------------------------------------------><

   target:
   +--+-xrange----------------------+--------------------------------------+
      +-+-STRing-+--delimitedString-+
        +-ANYof--+


\end{alltt}
\end{shaded}

\section{command}
\index{command}
\begin{shaded}
\begin{alltt}
\end{alltt}
\end{shaded}

\section{compare}
\index{compare}
\begin{shaded}
\begin{alltt}
/** compare

               +-TRINARY-+  (1)           +-PAD SPACE-+
  >>--COMPARE--+---------+------------+---+-----------+----------------------><
               +-BINARY--+  (2)       |   +-PAD-Xorc--+
               |                      |
               | <-----------------+  |
               +--ANY DString------+--+   (4) (5)
               +--EQUAL DString----+      (4)
               +--LESS DString-----+  (3) (4)
               +--MORE DString-----+  (3) (4)
               +--NOTEQUAL DString-+      (4)

     (1) -1 = Primary is shorter/less, 0 = equal, 1 = Secondary is shorter/less
     (2) 0 = equal, 1 = not equal
     (3) Primary is LESS/shorter (or MORE/longer) than secondary
     (4) DStrings can use any of the following escapes (or the lowercase) for
         the unequal situation:
\begin{verbatim}
                 \C (count) for the record number,
                 \B (byte) for column number
                 \P (primary) for the primary stream record
                 \S (secondary) for the secondary stream record
                 \L (Least) for then stream number that is shortest, -1 if equal
                 \M (Most) for the stream number that is longest, -1 if equal
     (5) Equal or not, this DString precedes any of the others.
\end{verbatim}
\end{alltt}
\end{shaded}

\section{console}
\index{console}
\begin{shaded}
\begin{alltt}
/** console

                                               { CMS Only                 }
                                               {                          }
   >>--+-CONSole--+--+----------------------+- { -+----------------+----- } ----><
       +-TERMinal-+  +-EOF--delimitedString-+  {  +-DIRECT---------+      }
                     +-NOEOF----------------+  {  +-ASYNchronously-+      }
                                               {  +-DARK-----------+      }
\end{alltt}
\end{shaded}

\section{copy}
\index{copy}
\begin{shaded}
\begin{alltt}
/** copy

copy from input stream to output without delaying the stream.   See elastic
for a more generic way to do this.

\end{alltt}
\end{shaded}

\section{count}
\index{count}
\begin{shaded}
\begin{alltt}
/** count


              <--------------+
   >>--COUNT--++-CHARACTErs-++--------------------------------------------><
               +-WORDS------+
               +-LINES------+
               +-MINline----+
               +-MAXline----+

\end{alltt}
\end{shaded}

\section{dam}
\index{dam}
\begin{shaded}
\begin{alltt}
/** dam

Do no pass any objects thru secondary streams until an object apprears
on the primary input stream.  The primary output stream must not be
connected.

\end{alltt}
\end{shaded}

\section{deal}
\index{deal}
\begin{shaded}
\begin{alltt}
/** DEAL


            +-STOP--ALLEOF--------------------+
  >>--DEAL--+---------------------------------+--------------------------><
            +-STOP--+-ALLEOF-+----------------+
            |       +-ANYEOF-+                |
            |       +-number-+                |
            +-SECONDARY--+---------+----------+
            |            +-RELEASE-+          |
            +-KEY--inputRange--+-------+------+
            |                  +-STRIP-+      |
            +-STREAMid--inputRange--+-------+-+
                                    +-STRIP-+

  Since java dispatches the stage threads.  Deal may not see a sever immediatly,
  as the severing thread can get multitasked.  This can make options like 'ANYEOF' 
  work in unexpected ways. 

\end{alltt}
\end{shaded}

\section{deblock}
\index{deblock}
\begin{shaded}
\begin{alltt}
\end{alltt}
\end{shaded}

\section{dict}
\index{dict}
\begin{shaded}
\begin{alltt}
/** dict

  Call dictr at the start of a pipe segment

\end{alltt}
\end{shaded}

\section{dicta}
\index{dicta}
\begin{shaded}
\begin{alltt}
\end{alltt}
\end{shaded}

\section{dictr}
\index{dictr}
\begin{shaded}
\begin{alltt}
\end{alltt}
\end{shaded}

\section{dictw}
\index{dictw}
\begin{shaded}
\begin{alltt}
\end{alltt}
\end{shaded}

\section{disk}
\index{disk}
\begin{shaded}
\begin{alltt}
/** disk

  Call diskr at the start of a pipe segment

\end{alltt}
\end{shaded}

\section{diska}
\index{diska}
\begin{shaded}
\begin{alltt}
/** diska

Appends records on its input stream to the end of the supplied file, the file
is created if its does not exist.

\end{alltt}
\end{shaded}

\section{diskr}
\index{diskr}
\begin{shaded}
\begin{alltt}
\end{alltt}
\end{shaded}

\section{diskslow}
\index{diskslow}
\begin{shaded}
\begin{alltt}
\end{alltt}
\end{shaded}

\section{diskw}
\index{diskw}
\begin{shaded}
\begin{alltt}
\end{alltt}
\end{shaded}

\section{drop}
\index{drop}
\begin{shaded}
\begin{alltt}
/** drop

            +-FIRST-+  +-1------+
  >>--DROP--+-------+--+--------+--+-------+-----------------------------><
            +-LAST--+  +-number-+  +-BYTES-+
                       +-*------+
\end{alltt}
\end{shaded}

\section{dup}
\index{dup}
\begin{shaded}
\begin{alltt}
/** dup

Duplicate the input object.  If n is an asterix the initial input is output
forever and is never consumed.  If n is zero more, the object is output and
along with n coppies of it.  If n is -1 then dup acts like hole and just
consumes its input.

\end{alltt}
\end{shaded}

\section{elastic}
\index{elastic}
\begin{shaded}
\begin{alltt}
/** elastic

read objects from the primary input stream as soon as they arive without
delaying the stream.  Output when requested.  Note elastic always accepts
new objects - even if its output is blocked pending a write.

\end{alltt}
\end{shaded}

\section{fanin}
\index{fanin}
\begin{shaded}
\begin{alltt}
/** fanin

   >>--FANIN--+------------+----------------------------------------------><
              | <--------+ |
              +---stream-+-+
\end{alltt}
\end{shaded}

\section{faninany}
\index{faninany}
\begin{shaded}
\begin{alltt}
\end{alltt}
\end{shaded}

\section{fanout}
\index{fanout}
\begin{shaded}
\begin{alltt}
/** fanout

               +-STOP--ALLEOF-----+
   >>--FANOUT--+------------------+---------------------------------------><
               +-STOP--+-ANYEOF-+-+
                       +-number-+

\end{alltt}
\end{shaded}

\section{fblock}
\index{fblock}
\begin{shaded}
\begin{alltt}
/** fblock

   >>--FBLOCK--number--+------+-------------------------------------------><
                       +-xorc-+

\end{alltt}
\end{shaded}

\section{file}
\index{file}
\begin{shaded}
\begin{alltt}
/** file

  Call diskr at the start of a pipe segment

\end{alltt}
\end{shaded}

\section{filea}
\index{filea}
\begin{shaded}
\begin{alltt}
/** filea

  Synonym of diska

\end{alltt}
\end{shaded}

\section{filer}
\index{filer}
\begin{shaded}
\begin{alltt}
/** filer

  Synonym of diskr

\end{alltt}
\end{shaded}

\section{fileslow}
\index{fileslow}
\begin{shaded}
\begin{alltt}
/** fileslow

  Call use diskslow for writing

\end{alltt}
\end{shaded}

\section{filew}
\index{filew}
\begin{shaded}
\begin{alltt}
/** filew

  Synonym of diskw

\end{alltt}
\end{shaded}

\section{find}
\index{find}
\begin{shaded}
\begin{alltt}
/** find


   >>--+-FIND--+--------+----------------------+--------------------------><
       |       +-string-+                      |
       +-STRFIND--+---------+--delimitedString-+
                  +-ANYcase-+

\end{alltt}
\end{shaded}

\section{frlabel}
\index{frlabel}
\begin{shaded}
\begin{alltt}
/** frlabel

   >>--+-FRLABEL--+--------+----------------------+-----------------------><
       |          +-string-+                      |
       +-STRFRLABEL--+---------+--delimitedString-+
                     +-ANYcase-+

  Purpose: read records searching for a string starting in column 1. copy all records
           thereafter (including the record containing the label) to stream 0.
           copy all preceding records to stream 1 (if it exists).

           a null argument causes stream 0 to be shorted to output (0).

  Contributed by Chuck Moore.

\end{alltt}
\end{shaded}

\section{frtarget}
\index{frtarget}
\begin{shaded}
\begin{alltt}
\end{alltt}
\end{shaded}

\section{gate}
\index{gate}
\begin{shaded}
\begin{alltt}
/** Gate

pass objects thru secondary streams until an object appears on the primary
stream.  If the strict option is specified, verify that there is not a
object waiting on stream 0 before passing the selected stream's object.

\end{alltt}
\end{shaded}

\section{gen}
\index{gen}
\begin{shaded}
\begin{alltt}
\end{alltt}
\end{shaded}

\section{getfiles}
\index{getfiles}
\begin{shaded}
\begin{alltt}
/** getfiles

Expects a stream of fileids.  Output the contents of the files on stream 0, if
stream 1 is connected, it coppies the fileids there.  Fileids will be coppied
even if there is a problem with the file (ie rc=2).

rc = 2  - When a fileid does not exist

\end{alltt}
\end{shaded}

\section{getovers}
\index{getovers}
\begin{shaded}
\begin{alltt}
/** getovers

Input stream 0 should contain rexx objects.  The getovers stage will output
will output the index and contents of the stem on stream 0.  If output stream 1 is
connected, the root is placed there.  Any severed streams will cause then
stage to exit.  Passing a non rexx object will cause the stage to exit
with return code 13

\end{alltt}
\end{shaded}

\section{getstems}
\index{getstems}
\begin{shaded}
\begin{alltt}
/** getstems

Input stream 0 should contain rexx objects containing stems.  The getstems stage
will output the contents of the stem on stream 0.  If output stream 1 is
connected, the root is placed there.  Any severed streams will cause then
stage to exit.  Passing a non rexx stem object will cause the stage to exit
with return code 13

\end{alltt}
\end{shaded}

\section{hash}
\index{hash}
\begin{shaded}
\begin{alltt}
/** hash

  Call hashr at the start of a pipe segment

\end{alltt}
\end{shaded}

\section{hasha}
\index{hasha}
\begin{shaded}
\begin{alltt}
/** hasha

  Synonym of dicta

\end{alltt}
\end{shaded}

\section{hashr}
\index{hashr}
\begin{shaded}
\begin{alltt}
/** hashr

  Synonym of dictr

\end{alltt}
\end{shaded}

\section{hashw}
\index{hashw}
\begin{shaded}
\begin{alltt}
\end{alltt}
\end{shaded}

\section{hole}
\index{hole}
\begin{shaded}
\begin{alltt}
\end{alltt}
\end{shaded}

\section{insert}
\index{insert}
\begin{shaded}
\begin{alltt}
/** insert

insert a string into a record.  Will be much more effecient than specs espically
if the input is a Byte[]

                              +-BEFORE-+
 >>--INSERT--delimitedString--+--------+----->>
                              +-AFTER--+


\end{alltt}
\end{shaded}

\section{inside}
\index{inside}
\begin{shaded}
\begin{alltt}
/** inside


   >>--INSIDE---+---------+--delimitedString--+-number----------+---------><
                +-ANYcase-+                   +-delimitedString-+


  FRLABLE Contributed by Chuck Moore.

\end{alltt}
\end{shaded}

\section{join}
\index{join}
\begin{shaded}
\begin{alltt}
/** join


             +-1-----------------+
   >>--JOIN--+-------------------+--+-----------------+--+--------+-------><
             +-number------------+  +-delimitedString-+  +-number-+
             +-*-----------------+
             +-KEYLENgth--number-+

\end{alltt}
\end{shaded}

\section{joincont}
\index{joincont}
\begin{shaded}
\begin{alltt}
/** joincont


                                       +-TRAILING----------+
   >>--JOINCONT--+---------+--+-----+--+-------------------+--+-------+---->
                 +-ANYCase-+  +-NOT-+  +-RANGE--inputRange-+  +-ANYof-+
                                       +-LEADING-----------+

   >--delimitedString--+------+--+-----------------+----------------------><
                       +-KEEP-+  +-delimitedString-+


\end{alltt}
\end{shaded}

\section{juxtapose}
\index{juxtapose}
\begin{shaded}
\begin{alltt}
/** Juxtapose

prefix a record from the seconday input stream with those from records
from the primary input stream.

pipe (findtext end ? runaway 60000 debug 0 )

   command cmd /C dir /B {2} !              -- get the directory
a: not getfiles !                           -- flip streams 0/1
b: juxtapose !                              -- prefix with the file name
   specs word1 1-20 word2-* next !          -- format the output
   console ?                                -- display it
a: !
   locate /{1}/ !                           -- send lines to juxta to display
b:

\end{alltt}
\end{shaded}

\section{listzip}
\index{listzip}
\begin{shaded}
\begin{alltt}
/** locate



   >>--LISTZIP--+---------+--+--------+-zipfile-+-----+--+-------+------------->
               

\end{alltt}
\end{shaded}
The zip stage lists all the filenames contained in the zip archive
mentioned as argument, and outputs them to
the pipeline.
\section{literal}
\index{literal}
\begin{shaded}
\begin{alltt}
/** literal

                   +-{-ojbect-name-}-+
   >>--+-LITERAL---+--------+--------+----------------+-------------------><
       |           +-string-+                         |
       |             +-PREFACE-+                      |
       +-STRLITERAL--+---------+--+-----------------+-+
                     +-APPEND--+  +-delimitedString-+
\end{alltt}
\end{shaded}

\section{locate}
\index{locate}
\begin{shaded}
\begin{alltt}
/** locate



   >>--LOCATE--+---------+--+-------------+--+-------+--------------------->
               +-ANYcase-+  +-inputRanges-+  +-ANYof-+

   >--+-----------------+-------------------------------------------------><
      +-delimitedString-+

\end{alltt}
\end{shaded}

\section{lookup}
\index{lookup}
\begin{shaded}
\begin{alltt}
/** lookup

   in stream 0 are detail records
   in stream 1 are master records
   in stream 2 adds to masters
   in stream 3 delete from masters

   out stream 0 are matched records
   out stream 1 are unmatched detail records
   out stream 2 are unmatched or counted master records
   out stream 3 deleted masters
   out stream 4 duplicate masters
   out stream 5 unmatched master deletes

   lookup does not consider an unconnected output stream an error.  It
   does proprogate EOFs from output streams.

   To increase performance reorder the 'when type=' in method out so the
   type you use is first in the list and recompile the stage.


   >>--LOOKUP--+-------+--+---------+--+---------+--+--------+->
               +-COUNT-+  +-ANYCASE-+  +-AUTOADD-+  +-BEFORE-+

   >>-+---------+--+----------+--+-----------+--+------------+-------->
      +-KEYONLY-+  +-SETCOUNT-+  +-INCREMENT-+  +-TRACKCOUNT-+

   >>----------------------------+-------------------------+---------->
                                 +-inputRange-+------------+
                                              +-inputRange-+

   >--+---------------------------+---------------------------><
      +-DETAIL MASTER-------------+
      +-DETAIL ALLMASTER PAIRWISE-+
      +-DETAIL ALLMASTER----------+
      +-DETAIL--------------------+
      +-MASTER DETAIL-------------+
      +-MASTER--------------------+
      +-ALLMASTER DETAIL PAIRWISE-+
      +-ALLMASTER DETAIL----------+
      +-ALLMASTER-----------------+

\end{alltt}
\end{shaded}

\section{nfind}
\index{nfind}
\begin{shaded}
\begin{alltt}
/** nfind


   >>--+-NFIND--+--------+----------------------+--------------------------><
       |        +-string-+                      |
       +-STRNFIND--+---------+--delimitedString-+
                   +-ANYcase-+

\end{alltt}
\end{shaded}

\section{nlocate}
\index{nlocate}
\begin{shaded}
\begin{alltt}
/** nlocate



   >>--NLOCATE--+---------+--+-------------+--+-------+--------------------->
                +-ANYcase-+  +-inputRanges-+  +-ANYof-+

   >--+-----------------+-------------------------------------------------><
      +-delimitedString-+

\end{alltt}
\end{shaded}

\section{noeofback}
\index{noeofback}
\begin{shaded}
\begin{alltt}
/** noEofBack

do not proprogate eof back thru this stage.

\end{alltt}
\end{shaded}

\section{nop}
\index{nop}
\begin{shaded}
\begin{alltt}
\end{alltt}
\end{shaded}

\section{not}
\index{not}
\begin{shaded}
\begin{alltt}
\end{alltt}
\end{shaded}

\section{notinside}
\index{notinside}
\begin{shaded}
\begin{alltt}
/** notinside


   >>-+-BETWEEN----+--+---------+--delimitedString--+-number----------+---------><
      +-INSIDE-----+  +-ANYcase-+                   +-delimitedString-+
      +-NOTINSIDE--+
      +-OUTSIDE----+


  FRLABLE Contributed by Chuck Moore.

\end{alltt}
\end{shaded}

\section{outside}
\index{outside}
\begin{shaded}
\begin{alltt}
/** outside


   >>--OUTSIDE--+---------+--delimitedString--+-number----------+---------><
                +-ANYcase-+                   +-delimitedString-+


  FRLABLE Contributed by Chuck Moore.

\end{alltt}
\end{shaded}

\section{over}
\index{over}
\begin{shaded}
\begin{alltt}
/** over

Extract all the indexs of a rexx variable.

\end{alltt}
\end{shaded}

\section{overlay}
\index{overlay}
\begin{shaded}
\begin{alltt}
/** OVERLAY

              +-BLANK-+
 >>--OVERlay--+-------+---------------------------------------------->< 
              +-xorc--+

\end{alltt}
\end{shaded}

\section{pad}
\index{pad}
\begin{shaded}
\begin{alltt}
/** pad

pad a rexx record

\end{alltt}
\end{shaded}

\section{pick}
\index{pick}
\begin{shaded}
\begin{alltt}
/** pick

 >>--PICK--+---------+--+----------+--+-------------+--+-==--+--+-----------------+-->
           +-ANYcase-+  +-PAD xorc-+  +-inputRanges-+  +-^==-+  +-inputRanges-----+
                                                       +-<<--+  +-delimitedString-+
                                                       +-<<=-+
                                                       +->>--+
                                                       +->>=-+

  The performance of Pick can be enhanced by reordering the compares in the
  comp method to put your compare first.

\end{alltt}
\end{shaded}

\section{prefix}
\index{prefix}
\begin{shaded}
\begin{alltt}
/** prefix

Blocks its primary input and excutes stage supplied as an arguement.  The
output from this stage are put to the primary output stream.  When its
compete the primary input is shorted.

\end{alltt}
\end{shaded}

\section{ptimer}
\index{ptimer}
\begin{shaded}
\begin{alltt}
\end{alltt}
\end{shaded}

\section{qsort}
\index{qsort}
\begin{shaded}
\begin{alltt}
/** sort

This sort routine is very basic.  It uses sortRexx class, which implements
the sortClass interface.  To sort objects of classes other than Rexx requires
that you implement another sortClass with a name begining with 'sort'.

I implemented this class using an Java array for the objects to be sorted as
opposed to a Rexx stem.  Using an array is about 50% faster.  I choose a quick/
Bose Norman sort since I had the code, in Rexx, for it from a time before I
had pipes on CMS!

\end{alltt}
\end{shaded}

\section{random}
\index{random}
\begin{shaded}
\begin{alltt}
\end{alltt}
\end{shaded}

\section{reverse}
\index{reverse}
\begin{shaded}
\begin{alltt}
\end{alltt}
\end{shaded}

\section{serialize}
\index{serialize}
\begin{shaded}
\begin{alltt}
\end{alltt}
\end{shaded}

\section{sort}
\index{sort}
\begin{shaded}
\begin{alltt}
/** sort

This sort routine is very basic.  It uses sortRexx class, which implements
the sortClass interface.  To sort objects of classes other than Rexx requires
that you implement another sortClass with a name begining with 'sort'.

I implemented this class using an Java array for the objects to be sorted as
opposed to a Rexx stem.  Using an array is about 50% faster.  I choose a quick/
Bose Norman sort since I had the code, in Rexx, for it from a time before I
had pipes on CMS!

\end{alltt}
\end{shaded}

\section{sortClass}
\index{sortClass}
\begin{shaded}
\begin{alltt}
/**

This is an interface class to allow a generic sort routine to handle objects
of different type.  Currently it is implemented by sortRexx, which will sort
rexx objects.

\end{alltt}
\end{shaded}

\section{sortRexx}
\index{sortRexx}
\begin{shaded}
\begin{alltt}
/**

An implementation for sortClass for rexx objects.  Part of the logic to generate
sortClass stages is in sort's stageExit.

\end{alltt}
\end{shaded}

\section{specs}
\index{specs}
\begin{shaded}
\begin{alltt}
\end{alltt}
\end{shaded}

\section{split}
\index{split}
\begin{shaded}
\begin{alltt}
/** split

   >>--SPLIT--+---------+--+-----------------+----------------------------->
              +-ANYCase-+  +-MINimum--number-+

      +-AT----------------------+           +-BLANK------------------+
   >--+-------------------------+--+-----+--+------------------------+----><
      +-+---------+--+-BEFORE-+-+  +-NOT-+  +-+ target +--+--------+-+
        +-snumber-+  +-AFTER--+                           +-number-+

   target:
   +--+-xrange----------------------+--------------------------------------+
      +-+-STRing-+--delimitedString-+
        +-ANYof--+
\end{alltt}
\end{shaded}

\section{sqlselect}
\index{sqlselect}
\begin{shaded}
\begin{alltt}
\end{alltt}
\end{shaded}

\section{stem}
\index{stem}
\begin{shaded}
\begin{alltt}
/** stem

  Call stemr at the start of a pipe segment

\end{alltt}
\end{shaded}

\section{stema}
\index{stema}
\begin{shaded}
\begin{alltt}
/** stema

Append to an existing stem, exits if the arguement is not a stem.

\end{alltt}
\end{shaded}

\section{stemr}
\index{stemr}
\begin{shaded}
\begin{alltt}
\end{alltt}
\end{shaded}

\section{stemw}
\index{stemw}
\begin{shaded}
\begin{alltt}
\end{alltt}
\end{shaded}

\section{strfind}
\index{strfind}
\begin{shaded}
\begin{alltt}
/** strfind

       +-FIND--+--------+----------------------+
       |       +-string-+                      |
   >>--+-STRFIND--+---------+--delimitedString-+--------------------------><
                  +-ANYcase-+

\end{alltt}
\end{shaded}

\section{strfrlabel}
\index{strfrlabel}
\begin{shaded}
\begin{alltt}
/** strfrlable

       +-FRLABEL--+--------+----------------------+
       |          +-string-+                      |
   >>--+-STRFRLABEL--+---------+--delimitedString-+-----------------------><
                     +-ANYcase-+

  Purpose: read records searching for a string starting in column 1. copy all records
           thereafter (including the record containing the label) to stream 0.
           copy all preceding records to stream 1 (if it exists).

           a null argument causes stream 0 to be shorted to output (0).

  FRLABLE Contributed by Chuck Moore.

\end{alltt}
\end{shaded}

\section{strliteral}
\index{strliteral}
\begin{shaded}
\begin{alltt}
/**  Distributed under the ICU 1.8.1 Licence with NO WARRANTIES of ANY kind.
 *  See LICENSE for the licence and information on using, copying, modifying,
/** strliteral

       +-LITERAL--+--------+--------------------------+
       |          +-string-+                          |
       |             +-PREFACE-+                      |
   >>--+-STRLITERAL--+---------+--+-----------------+-+-------------------><
                     +-APPEND--+  +-delimitedString-+

\end{alltt}
\end{shaded}

\section{strnfind}
\index{strnfind}
\begin{shaded}
\begin{alltt}
/** strnfind

       +-NFIND--+--------+----------------------+
       |        +-string-+                      |
   >>--+-STRNFIND--+---------+--delimitedString-+--------------------------><
                   +-ANYcase-+

\end{alltt}
\end{shaded}

\section{strtolabel}
\index{strtolabel}
\begin{shaded}
\begin{alltt}
/** strtolabel

       +-TOLABel--+--------+----------------------+
       |          +-string-+                      |
   >>--+-STRTOLABel--+---------+--delimitedString-+-----------------------><
                     +-ANYcase-+

  Purpose: read records up to a matching label (i.e. a string starting in column 1).
           copy all preceding records to stream 0.
           copy all following records (including the matched record) to stream 1
             (if it exists).

           a null argument causes all records to be shorted to output stream 1.

\end{alltt}
\end{shaded}

\section{take}
\index{take}
\begin{shaded}
\begin{alltt}
/** take


             +-FIRST-+  +-1------+
   >>--TAKE--+-------+--+--------+--+-------+-----------------------------><
             +-LAST--+  +-number-+  +-BYTES-+
                        +-*------+

\end{alltt}
\end{shaded}

\section{task}
\index{task}
\begin{shaded}
\begin{alltt}
\end{alltt}
\end{shaded}

\section{tcpclient}
\index{tcpclient}
\begin{shaded}
\begin{alltt}
/** tcpclient

Simple tcpclient implementation.  The options implemented are similiar to the
cms definition.

linger      - wait a bit before terminating the last read (units SECONDS)
timeout     - wait this long before timing reads out (units MS)
deblock     - If deblock is ommited a copy stage is used.
group       - similiar to cms.  A delimited string containing a stage is
              expected.  You can use a run of stages, but its is dangerous
              since you to know the stage sep character being used...
greeting    - expect a greeting message and discard it
nodelay     - use the nodelay option
keepalive   - enable keep alive socket option
oneresponse - synchronize cmds/replys

\end{alltt}
\end{shaded}

\section{tcpdata}
\index{tcpdata}
\begin{shaded}
\begin{alltt}
/** tcpdata

Simple tcpdata implementation.

linger      - wait a bit before terminating the last read (units SECONDS)
timeout     - wait this long before timing reads out (units MS)
deblock     - If deblock is ommited a copy stage is used.
group       - similiar to cms.  A delimited string containing a stage is
              expected.  You can use a run of stages, but its is dangerous
              since you to know the stage sep character being used...
nodelay     - use the nodelay option
oneresponse - synchronize requests/replies

\end{alltt}
\end{shaded}

\section{tcplisten}
\index{tcplisten}
\begin{shaded}
\begin{alltt}
/** tcplisten

Simple tcplisten implementation.  You can only supply the port and a timeout
value.  Its ignored unless tcplisten's output stream has been severed, in
which case tcplisten terminates.

If input stream 0 is connected, tcplisten does a peekto before calling the
accept method.  The object is consumed after the output of the socket object
returns.

\end{alltt}
\end{shaded}

\section{timestamp}
\index{timestamp}
\begin{shaded}
\begin{alltt}
/**
timestamp--Prefix the Date and Time to Records

timestamp  prefixes  each  record with a timestamp showing when the record was read by timestamp.

+------------------------------------------------------------------------+
|                                                                        |
|                +-8------+                                              |
| >>--TIMEstamp--+--------+------------------------------------------->< |
|                +-number-+                                              |
|                                                                        |
+------------------------------------------------------------------------+
Type:  Filter.


Syntax Description:  The arguments are optional.  The default is 8.


Operation:  A 16-character timestamp is developed whenever  a  record  has been  read.
It consists of the year (including the century), the month, day, hour, minute, second,
and hundredth  of  a  second.  Unlike CMS, the java interface reports hundredths of a
second.  The  specified number of characters are taken from the right-hand side of this
number and prefixed to the input record.
\end{alltt}
\end{shaded}

\section{tokenize}
\index{tokenize}
\begin{shaded}
\begin{alltt}
/** tokenise

   >>--+-TOKENISE-+--delimitedString--+-----------------+-----------------><
       +-TOKENIZE-+                   +-delimitedString-+

\end{alltt}
\end{shaded}

\section{tolabel}
\index{tolabel}
\begin{shaded}
\begin{alltt}
/** tolabel

  Purpose: read records up to a matching label (i.e. a string starting in column 1).
           copy all preceding records to stream 0.
           copy all following records (including the matched record) to stream 1
             (if it exists).

           a null argument causes all records to be shorted to output stream 1.

\end{alltt}
\end{shaded}

\section{totarget}
\index{totarget}
\begin{shaded}
\begin{alltt}
/** totarget



\end{alltt}
\end{shaded}

\section{unique}
\index{unique}
\begin{shaded}
\begin{alltt}
/** unique.nrx

                          +-NOPAD-----+
   >>--UNIQue--+-------+--+-----------+--+---------+----------------------->
               +-COUNT-+  +-PAD--xorc-+  +-ANYcase-+

                            +-LAST-----+
   >--+------------------+--+----------+----------------------------------><
      +-+ uniqueRanges +-+  +-SINGLEs--+
                            +-FIRST----+
                            +-MULTiple-+
                            +-PAIRwise-+

   uniqueRanges:
   +--+-inputRange--------------------------+------------------------------+
      |    <---------------------------+    |
      +-(----inputRange--+-----------+-+--)-+
                         +-NOPAD-----+
                         +-PAD--xorc-+

\end{alltt}
\end{shaded}

\section{unzip}
\index{unzip}
\begin{shaded}
\begin{alltt}
/** locate



   >>--UNZIP--+---------+--+--------+-zipfile-+-----+--+-------+------------->
               

\end{alltt}
\end{shaded}
The unzip stage lists all the filenames fed to it in the pipeline,
unzips them from  the zip archive
mentioned as argument, and outputs them to the pipeline, as they are
uncompressed and written to the filesystem.
\section{var}
\index{var}
\begin{shaded}
\begin{alltt}
\end{alltt}
\end{shaded}

\section{vector}
\index{vector}
\begin{shaded}
\begin{alltt}
/** vector

  Call vectorr at the start of a pipe segment

\end{alltt}
\end{shaded}

\section{vectora}
\index{vectora}
\begin{shaded}
\begin{alltt}
\end{alltt}
\end{shaded}

\section{vectorr}
\index{vectorr}
\begin{shaded}
\begin{alltt}
\end{alltt}
\end{shaded}

\section{vectorw}
\index{vectorw}
\begin{shaded}
\begin{alltt}
\end{alltt}
\end{shaded}

\section{watchp}
\index{watchp}
\begin{shaded}
\begin{alltt}
\end{alltt}
\end{shaded}

\section{xconsole}
\index{xconsole}
\begin{shaded}
\begin{alltt}
\end{alltt}
\end{shaded}

\section{xlate}
\index{xlate}
\begin{shaded}
\begin{alltt}
/** xlate

                                                <---------------------+
   >>--+-XLATE-----+--+----------------------+---+-------------------++---->
       +-TRANSlate-+  +-inputRange-----------+   +-+ default-table +-+
                      | <------------------+ |
                      +---(--inputRange--)-+-+

      <------------------+
   >---+----------------++------------------------------------------------><
       +-xrange--xrange-+

   default-table:
   +--+-UPper--------------------------+-----------------------------------+
      +-LOWer--------------------------+
      +-INput--------------------------+
   {  +-OUTput-------------------------+  }
   {  +-+-TO---+--+----------+--number-+  }
   {    +-FROM-+  +-CODEPAGE-+            }
   {                                      }
   { Not yet in njPipes                   }

  XLATE Contributed by Rene Jansen

\end{alltt}
\end{shaded}

\section{zip}
\index{zip}
\begin{shaded}
\begin{alltt}
/** locate



   >>--ZIP--+---------+--+--------+-zipfile-+-----+--+-------+------------->
               

\end{alltt}
\end{shaded}
The zip stage makes a zipfile consisting of the filenames specified in
the pipeline, with the zipfile's name as in the argument.
\section{zone}
\index{zone}
\begin{shaded}
\begin{alltt}
\end{alltt}
\end{shaded}


\chapter{Appendix A}
\begin{verbatim}
.50  - Released May 30, 1999
         - Fixed a stall occurring when interrupted threads, with the interrupt
           caught by ThreadPool, were reused.
         - Fixed a thread safety problem in ELASTIC
         - Improved the timeout options in TCPDATA and TCPCLIENT, they also
           byte[] instead of strings.  This was done since converting to and
           from strings sometimes scrambles binary data (more research on
           encodings...)
         - Changed DELBLOCK it now handles byte[] to help keep tcpdata and
           tcpclient efficient.  The EOF option was broken, its fixed now.
         - Changed DISKR, DISKW and DISKA to handle byte[] when using streams.
         - Added INSERT which handles byte[].  This should be used instead of
           SPECS to add LF or CR .
         - Changes SERIALIZE to use byte[].
   0.49  - Released May 21, 1999
         - compiled with 1.2.1 and NetRexx 1.148
         - Added preliminary support added to .njp compiler for files containing
           java source!  See the (some what messy) java samples in vectort1.njp,
           overtest.njp and addtest4.njp
         - Added code to generate a dummy .nrx file containing the public class
           in a .java file.  This allows NetRexx to compile class that depend on
           the java source.
         - Modified sort to accept arguements in the same order as CMS
         - Fixed rc logic in drop stage
         - Fixed shortcut code for {n} where n is numeric.
   0.48  - Released May 16, 1999
         - Fixed a (nasty) bug involving reusing pipe objects.
         - Added the reuse() method to the stage class.  To use it override
           it in your stage.  It was added so there was a foolproof way to
           reset a stage when its pipe object is reused.  (doSetup is intended
           for use with dynamic arguements in call or added pipes)
         - Added the cont option and defaulted it to comma.
         - fixed return code logic in some stages and in selectInput/Output
         - Added the Emsg methods
         - Added arguement debug option (128)
         - There are no more final methods
         - Much improved error reporting from stages via new Emsg method
   0.47  - Released Jan 3, 1999
         - recompiled with 1.1.7A and netrexx 1.148
         - UNIQUE repaired?
         - Added stages to acess java objects easily
           VECTOR, VECTORR, VECTORW, VECTORA for java.util.vector
           ARRAY, ARRAYR, ARRAYW, ARRAYA for Object[]
           HASH, HASHR, HASHW, HASHA for java.util.Hashtable
           DICT, DICTR, DICTW, DICTA for java.util.Dictionary
           The hash stages mostly map directly to DICT stages.  The exception
           is HASHW which uses the clear() method of Hashtable.
         - Modified LITERAL to be able to put any object into a pipe
         - Modified pipe package to store arguements in a hashtable instead of
           a rexx stem - arguements can now be of any class.  Use the arg(null)
           method to get an object arguement.
   0.46a - Released Oct 14, 1998
         - recompiled with 1.1.7
         - TCPLISTEN now supports an input stream to be used to pace accepts
   0.46  - Released Sept 20, 1998
         - COMMAND, CHANGE, FILE, LOCATE, DROP, LOOKUP, TCPCLIENT, TCPLISTEN
           SQLSELECT, CONSOLE, TCPDATA, NOEOFBACK improved.
         - Jeff improved the testing process with the addition of the COMPARE
           stage, he also upgraded many of the tests.
         - Added the buildtests pipe, it builds a test script to be run with:
           test > output < console.data
         - Unexpected exceptions should no longer hang pipes
   0.45  - Released Sept 9, 1998
         * Recompile all your stages.  To fix a commit problem I had to
           change the _stage interface class...
         - tcpclient restart problems with oneresp active fixed.
         - commit now returns the current return code of the pipe.
         - fixed minor errors in tcpclient and diska.
   0.44  - Released Sept 8, 1998
         * a recompile of pipes using STEM is required
         - smart DISK, FILE and STEM stages now exist.
         - Made to and from synonoms for in and out in REXX and STRING stages.
         - Added stream option to DISKR and DISKW to read raw streams.
         - Added DISKSLOW and SERIALIZE stages.
         - Now DISK, DISKR, DISKW, DISKA and DISKSLOW have FILE synonyms.
         - Deadlock detection improvements.
         - TCPDATA & TCPCLIENT optimized once again.
         - selectAnyInput could deadlock - fixed.
         - interrupting a pipe now kills it - use this with care (ie. kill -9)
         - Pseudo methods njpRC() and njpObject() are reconized by the pipes
           compiler and return the pipe's RC or object respectivily.
   0.43  - Released August 30, 1998
         - Fixed deadlock dection to see commit deadlocks.
         - Added rest of code to handle improved StageError logic.
         - Added stage templates (template*.nrx) in the njpipes directory.
         - Added a debug flag (64) to trace all StageError rasied by the
           stage class.
   0.42  - Not released
         * A recompile of pipes using TCPCLIENT, TCPDATA is required.
         * A recompile of pipes using REXX, STRING, ZONE, CASEI is recommended.
         - Updated the comments in _stage to reflect the possible StageError
           and return codes that can be issued.
         - Added the DEBLOCK stage and reworked TCPDATA, TCPCLIENT & GATE.
         - Improved eofReport processing and added a new option 'either' that
           will trigger a StageError when any stream, input or output, severs.
         - Fixed variable subsitution so multiple variables passed to a stage
           will work.
         - Added the ability to pass thru arguements to callpipe and addpipe.
         - Fixed a problem with some StageExits requiring stage_reset methods.
         - Added a function to utils to help assign smarter name to classes
           generated by StageExits.
         - Added counter method to stage.  use to count external waits so
           deadlock/stall detection is not fooled.
   0.41  - Released August 23, 1998
         * removed OBJ2REXX, OBJ2STRING stages, use REXX and STRING stage
           modifiers.
         * pipes using TCPDATA, TCPCLIENT & LOOKUP should be recompiled
         - exhanced REXX stage modifier via an object2rexx improvement in
           pipes/utils.nrx
         - optimized ThreadPool startup times.  No setName and only use
           setPriority when its required.
         - made it possible to optimized stage startup time when arguements
           are static.  See TCPDATA, TCPCLIENT & LOOKUP
         - faq.txt enhanced
   0.40  - Released August 14, 1998
         * All pipes MUST be recompiled.  Old pipe class files will stall.
         - OBJ2REXX is depreciated and will be removed, use the REXX stage.
         - added REXX and STRING stages to convert objects entering and leaving
           a stage to rexx or string.  Inorder to avoid nasty class conflicts,
           REXX and STRING are implemented in _rexx and _string.  The compler
           adds the '_' when necessary (any stage can use this feature).
         - fixed an intermitant stall in callpipe (was completing too fast :-)
         - fixed a stall occuring between shortStreams and COMMAND
         - optimized pipe startup time in pipe.class and via the compiler.
         - optimized rc, commit, deadlock, threadpool code
   0.39  - Released August 9, 1998
         - WAIT_COMMIT and WAIT_ANY are now used in the call/addpipe logic
         - callpipe was not notifiing its pipe when ending leading to an
           very intermitant hang.
   0.38  - Released August 3, 1998
         * All your stages must be recompiled.  Recompile your pipes to
           exploit the pipe & thread pool performance improvements.
         - fixed and optimized commit logic.
         - implement a pool for pipes to decrease overhead.
         - implement a pool for threads to decrease overhead.
         - compiler fix to proprogate return codes from stageExits (thanks Jeff).
         - signal StageError('...  in all stageExits modified to
           signal StageError(13,'Error - 'pInfo' - ...
         - UNIQUE stage added by Jeff.  It exploits stageExit.
         - COMMAND stage was not starting its threads correctly.
         - SORTs in different pipes could corrupt each other.  Thanks René‚
   0.37  - Released July 25, 1998
         * A recompile of pipes using SORT is required
         - added NOEOFBACK, TOTARGET and FRTARGET.
         - removed a protected method from dump(), added arg() to the dump
         - upgraded SORT, sortRexx to exploit IRange and stageExit, optimized
           use, and factored the sort algorithm out of sort/sortRexx.
         - multiple sort stages no longer try to share static variables...
         - the compiler just uses the stage name (not args) when naming stages
   0.36  - Released July 19, 1998
         * A recompile of ALL pipes with stages using IRANGE is required.
           (CHANGE, DEAL, JOINCONT, LOCATE, LOOKUP, PICK, XLATE & ZONE)
         * pipes using NFIND, NLOCATE, STRNFIND or SORT also need to be
           recompiled
         - Added BuildIRangeExit and other methods to an updated IRange
           class.  Using 'zone range stage ...' will be faster than
           'stage range ...' when the range consists of n.c or n-c (s).
         - NFIND, NLOCATE, STRNFIND implemented via stageExit and NOT
         - Fixed bugs in, JUXTAPOSE, FIND, STRFIND, SORT, COMMAND, CHANGE
         - The compiler was not calling stageExit in the correct order when
           several calls were needed to build the stage.  (zone w1 nfind..)
   0.35  - Released July 16, 1998
         - Jeff Hennick pointed out a bugglet that effected LOOKUP, ZONE and
           PICK that could occur with complex ranges, I found another bug in
           strliteral
         - Jeff Hennick updated this doc with information on IRange and DString
   0.35  - Released July 15, 1998
         * A recompile of ALL pipes using ZONE, TCPCLIENT, TCPDATA, PREFIX
           and APPEND is required.
         - prefix and append can now be labeled, tcpclient and tcpdata
           now use a stage, instead of a pipe, to group data.
         - added compiler support for negitive stream numbers.  This is
           intended to be used by stageExit.  See append, prefix, tcpdata
           and tcpclient.
         - Redefined rexxArg() and stageArg() to simplify the compiler.
         - selection stages are no longer defined as final.
         - SelectInput(0) and selectOutput(0) are always called by the
           stage implementation so they can be overridden...
         - Reimplemented ZONE using stageExit, added CASEI using the same
           technique.  In theory NOT could be done the same way but, to
           avoid some recursion problems NOT is staying in the compiler.
         - StageExit modified to allow it to pass back another stage to
           call.  see ZONE, CASEI and NOT.
   0.34  - Released July 11, 1998
         - minor reportEOF(any) logic fix
         - improved command stage, threads used to process stdout and stderr.
           added zone, pad, lookup, pick, upgraded juxtapose, fixed bugs in
           specs & buffer.
         - added pad option to setIRange method
   0.33  - Released July 5, 1998
         - added rexxArg() and stageArg() methods to utils.nrx for use by the
$          compiler to query stages about what they expect their arguments to
           contain.  This allowed the compiler to be simplified.
$        - locate now handles null arguments correctly.  literals now include
           leading blanks.  Thanks for pointing out the problem René.
         - René Jansen contributed the timestamp stage.
         - logic modified to stop output() from getting an EOF when the output
           object has been peeked.  The peek status is also displayed by the
           dump() method and hense by deadlocks.
         - minor specs bug fixes (next.n and nextw.n output specs now work)
         - modified the compiler to invoke stageExit(rexx,rexx) method.  This
           allows stages to generate code and/or change the pipe topology.  See
           specs, append, prefix, change and xnop, in the stages directory.
         - modified StageError in preparation for usage changes.
         - removed the Range class - Jeff's code is better and anything that
           could be done with Range can be done using stageExit.
         - Jeff fixed bugs in change and join and added:
           fblock          joincont        notinside       outside
           inside
   0.32  - Released June 20, 1998
           Jeff updated these stages adding a few new ones too:
           abbrev          between         split           locate
           nlocate         strnfind        strfind         nfind
           find            chop
         - minor docuementation updates
         - the Range class is depreciated and will be removed.  Use the
           replacements Jeff created (see pipes\utils.nrx and stages\).
   0.31  - Released June 17, 1998
         - modified count, drop, take and deal to handle non rexx objects
           when possible
   0.31  - Released June 16, 1998
         - improved eofReport(ANY) logic to trigger when waiting on output
           and a different output stream severs.
         - factored the source for utils.class out of stages so there is
           a class to add (probably static) shared methods for all stages
         - fixed a deadlock that occured between shortStreams and exit
           (severInput)
         - Jeff Hennick updated many stages to work at CMS or near CMS levels.
           append          deal            join            strfrlabel      xlate
           buffer          drop            literal         strliteral
           change          fanin           locate          strtolabel
           console         fanout          split           take
           count           frlabel         strfind         tokenize
           All of Jeff's changes are GNUed.  See CopyLeft.txt in the njpipes
           directory.
   0.30  - Released May 24, 1998
         - fixed logic in core classes to post all pending severs and not
           clear them too early either, this corrects a problem seen on
           Multiprocessor machines.
   0.29  - www page update (docuemention) May 20
         - deadlock section updated
         - installation verification example corrected!
   0.29  - Released May 17, 1998
         - added obj2rexx stage, tolabel stage courtesy of Chuck Moore.
         - enhanced change to support a single range
         - Added setJITCache(Hashtable) method to pipes.  This can be used
           to build a global object cache in programs calling pipes.  The name
           of the Hasttable is passed to pipe/callpipe/addpipe via a cache
           parameter.
         - Added support for reportEof options.  This support is not too
           well tested - some good testcases are needed.
   0.28  - Released May 9, 1998
         - Enhanced parsing in specs (word2.1 would work, word 2.1 would not)
         - Fixed COPY for a NT jit bug, fixed locate so NOT LOCATE would
           work, updated LITERAL not to use more than one exit(rc)
         - Fixed a compiler problem that would hit multistreamed pipes using
           append or prefix.
         - Any options not consumed by njp are passed on to nrc
           and java.  Mainly for use from the command line, use with care
           in .njp files...
         - Fixed shortStreams() so it works correctly when shorting streams
           in a stage with multiple streams.
         - Tested all 8 addpipe forms and fixed runtime to work with all
           test cases
         - modified filternjp to accept *in and *out without additional labels
         - reenabled stop() in exit code...
         - added gate, dam, tokenize, juxtapose and courtesy of Chuck Moore,
           frlabel stages
   0.27  - Released May 3, 1998
         - Automated the generation of in/outStream calls.  For this to work
           the labels need to be of the form *in0: or *out0: where the '0' is
           replaced by the input or output stream to connect to.
         - Fixed compiler/filter problems with stema
         - Tighted range checking code in specs, fixed problem with delimited
           ranges.  Specs was compiling the NetRexx EXIT command...
         - Fixed a problem where output was not see that objects were
           consumed when using sipping pipes...
         - Fixed a problem where severing an output stream did not cause the
           stages stacked on the node's outlist to see the sever
         - Fixed a problem where the stage issuing a callpipe was not seeing
           the called pipe end
         - Added debug option to pipes compiler
         - Repaired commit and added commit levels to dump() method
         - Fixed problems with callpipe servering several outputs, unstacking
           the saved stream was selecting it...
         - Modified tcpclient and tcpdata to use a secondary thread to
           recieve the tcpip inputs.
         - Now keep a referenced object for each pipe/stage so the JIT does
           not throw away its work and call/addpipes in loops work faster.
         - in/outStreamState now return -1 when autocommit is enabled and
           the stream is unused.
   0.26  - Released April 26, 1998
   - Added selection methods to compiler (see getRange in section 4 and
           the locate stage an example#
         - Added the specs stage.  The compiler builds a stage to process the
           specs, reducing overhead.
         - Added tcp/ip stages
         - Fixed problems with severs using addpipe
   0.25  - Optimized some stages using jinsight from www.alphaworks.ibm.com.
           This more than doubled the speed of some stages.
         - fixed bugs in fanin, diskw
         - Added netrexx filters to extract pipes, extended the functions
           of .njp files (multiple pipes in a file and .njp files can now
           contain netrexx code with pipe/callpipe/addpipe)
         - fixed a timing bug in deadlock detection.
         - xlate and sqlselect stages contributed by René Jansen added
   0.24  - Release Feb 98
         - modified the compiler so the syntax of pipes from the command line
           is the same as pipes from .njp files
         - added the sort stage, the sortClass interface and the sortRexx
           example implementation
         - added the timer stage
   0.23  - fixed minor compiler errors (20 Dec 97)
         - not stage modifier added.
         - errors in this page corrected, NT install information added.
         - modified diskr/diskw to use Buffered Streams.
   0.22  - second public release
   0.21  - enabled auto commit, stages start at a commit level of -2 and
           commit to a level of -1 at the first readto, peekto or output.
           nocommit disables the auto commit.  This feature has not been
           completely tested (yet).
         - fixed compiler not to call netrexx if one of its pipes deadlocks
   0.20  - Upgraded to May version of the NetRexx compiler (Thanks Mike!)
           this changed the compiler interface.  NetRexx from May 10 or
           later is now required.
         - nocommit added to _stages, though its a nop for now
         - modified the compiler class to use the May 10th NetRexx compiler
   0.19  - initial public release (4 May 97)
\end{verbatim}
\backmatter
\listoffigures
\listoftables
\lstlistoflistings
\printindex
\clearpage
\psset{unit=1in}
\begin{pspicture}(3.5,1in)
  \psbarcode{\isbn}{includetext guardwhitespace}{isbn}
\end{pspicture}
\end{document} 
